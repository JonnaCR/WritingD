
In order to solve the optimal control problems \eqref{AdvDiff} and \eqref{AdvDiff_Linear} some inputs must be provided. The desired state $\widehat \rho$, the PDE source term $f$, and the external potential $V_{ext}$ must be given. Furthermore, an initial condition for $\rho$, the final time condition for $\adj$ and an initial guess for the control $\vec{w}$ have to be be specified. 
The interaction kernel (++ terminology? ++) is of the form:
\begin{align*}
\vec{K} = \nabla V_2, \qquad V_2 = e^{-x^2}.
\end{align*}
Three interaction strengths are considered in this section. Firstly, each problem is solved without an interaction term present ($\gamma = 0$). Then, the considered problem is solved with an order one attractive interaction term ($\gamma = -1$) and an order one repulsive interaction term ($\gamma = 1$), respectively. Initially, the control $\vec{w}$ is set to zero. It is then investigated how the control changes from this baseline, influenced by the different interaction strengths. This is considered for different values of the regularization parameter $\beta$ and it is expected that the control will increase with decreasing $\beta$, since the cost functionals in problems \eqref{AdvDiff} and \eqref{AdvDiff_Linear} allow for a larger control with smaller $\beta$.
In the following examples, the domain considered is $\Omega \times [0,T] = [-1,1] \times [0,1]$. The number of spatial points is $N=30$, and the number of time points is $n=20$, unless stated otherwise. The tolerances in the ODE solver are set to $10^{-8}$ and the tolerance for the convergence of the optimization algorithm is $10^{-4}$. The mixing parameter $\lambda$ is $0.01$, unless stated otherwise.
\subsection{Nonlinear control problems with an additional nonlocal integral term 1D} \label{sec:Examples1d}
Examples of solving Problem \eqref{AdvDiff}, with 'no-flux type' boundary conditions \eqref{NoFlux} and Dirichlet boundary conditions \eqref{Dirichlet} are given in this section. 
 
\subsubsection{Neumann boundary conditions, Example 1}	 
The chosen inputs for this example are:
\begin{align*}
&\widehat \rho = \frac{1}{2}(1-t) + t\bigg(\frac{1}{2}\sin(\pi (y - 2)/2) + \frac{1}{2}\bigg),\\
&\rho_{0} = \frac{1}{2}, \ \
\adj_{T} = 0, \ \
\vec{w} = 0, \ \ 
f =0,\ \
V_{ext} =0.
\end{align*}	
Table \ref{TabS5:Prob1} displays the results for this example. The value of the cost functional for the uncontrolled case ($J_{uc}$), where $\vec{w} =0$, is compared with the controlled case ($J_{c}$) for different values of $\beta$ and for each of the interaction strengths. It can be observed that in all cases $J_{c}$ is lower or equal value to $J_{uc}$. The lowest values of $J_{c}$ will be observed for the smallest $\beta$ value considered. At large values of $\beta$, applying control is heavily penalised and the optimal control approaches zero, which coincides with the uncontrolled case. Furthermore, Table \ref{TabS5:Prob1} displays the number of iterations for each of these examples. The desired state $\widehat \rho$, and the uncontrolled state $\rho$ for $\gamma =1$ and $\gamma = -1$, with $\beta =10^{-3}$ are shown in Figure \ref{Ex12DN1}. The desired state $\hat \rho$ and the uncontrolled $\rho$ are independent of $\beta$. However, $\rho$ changes considerably with the choice of interaction strength $\gamma$. The optimal states $\rho$ for $\gamma = 1,0,-1$ and corresponding optimal controls, with $\beta = 10^{-3}$ are shown in Figure \ref{Ex12DN2}. 
It can be observed that in the case of $\beta = 10^{-3}$, the optimal state $\rho$ is very similar to $\hat \rho$, regardless of the choice of interaction. However, the corresponding control plot reviles that the control has to be applied differently in each case to account for the interaction effects. In general, the control is largely applied on the right half of the spatial domain, to carry mass to the left, where the desired state dictates it to be, as can be seen when $\gamma = 0$. However, when the particle interaction is repulsive, the control is moving some of the particle mass from the boundary at $x=-1$ to correct for the repulsive particles accumulating there in the uncontrolled state, as illustrated in Figure \ref{Ex12DN1}. In the attractive case, the control corrects by carrying some mass to the boundary at $x=1$, since the uncontrolled particle density is clustered in the middle of the domain in this case, compare to Figure \ref{Ex12DN1}.
\begin{figure}[h]
	\includegraphics[scale=0.05]{Figure1.png}
	\caption{Example 1, desired state $\widehat \rho$ and uncontrolled state $\rho$ at $\gamma =1$ and $\gamma =-1$, $\beta = 10^{-3}$}
	\label{Ex12DN1}
\end{figure}
\begin{figure}[h]
	\includegraphics[scale=0.05]{Figure2.png}
	\caption{Example 1, optimal state $\rho$ and the corresponding optimal control $\vec{w}$ for $\gamma = 1,0,-1$, $\beta = 10^{-3}$.}
	\label{Ex12DN2}
\end{figure}

\begin{table}
\begin{tabular}{ | c | c || c | c | c | c ||}
\hline
\multicolumn{2}{|c||}{}& $\beta = 10^{-3}$ & $\beta = 10^{-1}$ & $\beta = 10^{1}$ & $\beta = 10^{3}$  \\
\hline
\hline
 & $\mathcal{J}_{uc}$ & $\numprint{0.0438}$ & $\numprint{0.0438}$ & $\numprint{0.0438}$ & $\numprint{0.0438}$ \\
$\kappa= \numprint{-1}$  & $\mathcal{J}_c$ & $\numprint{0.0011}$ & $\numprint{0.0267}$ & $\numprint{0.0435}$ & $\numprint{0.0438}$ \\
& \texttt{Iter} & $\numprint{670}$ & $\numprint{650}$ & $\numprint{449}$ & $\numprint{1}$ \\
\hline
 & $\mathcal{J}_{uc}$ & $\numprint{0.0417}$ & $\numprint{0.0417}$ & $\numprint{0.0417}$ & $\numprint{0.0417}$ \\
$\kappa= \numprint{0}$  & $\mathcal{J}_c$ & $\numprint{0.0014}$ & $\numprint{0.0283}$ & $\numprint{0.0415}$ & $\numprint{0.0417}$ \\
& \texttt{Iter} & $\numprint{665}$ & $\numprint{656}$ & $\numprint{434}$ & $\numprint{1}$ \\
\hline
 & $\mathcal{J}_{uc}$ & $\numprint{0.0434}$ & $\numprint{0.0434}$ & $\numprint{0.0434}$ & $\numprint{0.0434}$ \\
$\kappa= \numprint{1}$  & $\mathcal{J}_c$ & $\numprint{0.0020}$ & $\numprint{0.0322}$ & $\numprint{0.0432}$ & $\numprint{0.0434}$ \\
& \texttt{Iter} & $\numprint{654}$ & $\numprint{682}$ & $\numprint{422}$ & $\numprint{1}$ \\
\hline
\end{tabular}
\caption{Example 1: Cost $\mathcal{J}_{uc}$ of applying no control (i.e., $\vec{w} = \vec{0}$), optimal control cost $\mathcal{J}_{c}$, and number of iterations (PDE solves) \emph{\texttt{Iter}} required, for a range of values of the interaction strength $\kappa$ and regularization parameter $\beta$.}
\label{TabS5:Prob1}
\end{table} %\label{TabS5:Prob1}
%\begin{table}[h]
%	\begin{tabular}{ ||c|| c | c |c | c ||}
%		\hline
%		$\beta$ / $\gamma$ & $10^{-3}$  & $10^{-1}$  & $10$ & $10^3$ \\ 
%		\hline 
%		      & $J_{uc} = 0.0438$ & $J_{uc} = 0.0438$  & $J_{uc} = 0.0438$ & $J_{uc} = 0.0438$\\ 
%		 $-1$ & $J_c = 0.0011$ & $J_c = 0.0270$ & $J_c = 0.0435$ & $J_c = 0.0438$\\ 
%		      & Iter. $= 667$ & Iter. $= 649$  & Iter. $= 468$ & Iter. $= 13$\\ 
%		 \hline
%		      & $J_{uc} = 0.0417$ & $J_{uc} = 0.0417$   & $J_{uc} = 0.0417$& $J_{uc} = 0.0417$\\
%		 $0$  & $J_c = 0.0014$ & $J_c = 0.0283$  & $J_c = 0.0415$ & $J_c = 0.0417$\\ 
%		      & Iter. $= 671$ & Iter. $= 656$  & Iter. $= 434$ & Iter. $= 1$\\ 
%		 \hline
%		      & $J_{uc} = 0.0434$ & $J_{uc} = 0.0434$  & $J_{uc} = 0.0434$ & $J_{uc} = 0.0434$\\
%		 $1$  & $J_c = 0.0020$ & $J_c = 0.0324$  & $J_c = 0.0432$ & $J_c = 0.0434$\\ 
%		      & Iter. $= 674$ & Iter. $= 686$  & Iter. $= 411$ & Iter. $= 1$\\ 
%		 \hline 
%	\end{tabular}
%    \caption{}
%    \label{TabNFlowEx1}
%\end{table}

\subsubsection{Neumann boundary conditions, Example 2} 
The chosen inputs for Example 2 are:
\begin{align*}
&\widehat \rho = \bigg(\frac{1}{2}\cos(\pi y) + \frac{1}{2}\bigg)(1-t) + t\bigg(-\frac{1}{2}\cos(2 \pi y) + \frac{1}{2}\bigg),\\
&\rho_{0} = \frac{1}{2}\cos(\pi y) + \frac{1}{2},\ \
\adj_{T} = 0,\ \
\vec{w} = 0,\ \
f =0,\ \
V_{ext} =0.
\end{align*}
In Table \ref{TabS5:Prob2a} the results for Example 2 are displayed. These are comparable with the results for Example 1, in the effect of $\beta$ and the number of iterations. In all three configurations of the interaction term, the control is focussed on transporting the mass from the middle of the domain onto two piles centred at $x=-0.5$ and $x=0.5$. In Figure \ref{Ex22DN1}, the desired state $\widehat \rho$, the optimal state $\rho$ and the optimal control $\vec{w}$ are displayed for $\beta = 10^{-3}$, and compared to Example 3 below. 
\begin{figure}[h]
	\includegraphics[scale=0.05]{Figure3.png}
	\caption{Example 2/ Example 3, desired state $\widehat \rho$, optimal state $\rho$ and corresponding optimal control $\vec{w}$, $\beta = 10^{-3}$, $\gamma = 1$.}
	\label{Ex22DN1}
\end{figure}

\begin{table}
\begin{tabular}{ ||c|| c | c | c | c | c ||}
\hline
& & $\beta = 10^{-3}$ & $\beta = 10^{-1}$ & $\beta = 10^{1}$ & $\beta = 10^{3}$  \\
\hline
 & $J_{uc}$ & $\numprint{5.3559e-2}$ & $\numprint{5.3559e-2}$ & $\numprint{5.3559e-2}$ & $\numprint{5.3559e-2}$ \\
$\gamma= \numprint{-1}$  & $J_c$ & $\numprint{9.6557e-3}$ & $\numprint{4.9268e-2}$ & $\numprint{5.3511e-2}$ & $\numprint{5.3559e-2}$ \\
& $Iter.$ & $\numprint{520}$ & $\numprint{768}$ & $\numprint{378}$ & $\numprint{1}$ \\
\hline
 & $J_{uc}$ & $\numprint{6.6902e-2}$ & $\numprint{6.6902e-2}$ & $\numprint{6.6902e-2}$ & $\numprint{6.6902e-2}$ \\
$\gamma= \numprint{0}$  & $J_c$ & $\numprint{1.0920e-2}$ & $\numprint{6.0339e-2}$ & $\numprint{6.6826e-2}$ & $\numprint{6.6903e-2}$ \\
& $Iter.$ & $\numprint{679}$ & $\numprint{770}$ & $\numprint{390}$ & $\numprint{1}$ \\
\hline
 & $J_{uc}$ & $\numprint{8.3948e-2}$ & $\numprint{8.3948e-2}$ & $\numprint{8.3948e-2}$ & $\numprint{8.3948e-2}$ \\
$\gamma= \numprint{1}$  & $J_c$ & $\numprint{1.2510e-2}$ & $\numprint{7.4874e-2}$ & $\numprint{8.3842e-2}$ & $\numprint{8.3949e-2}$ \\
& $Iter.$ & $\numprint{681}$ & $\numprint{771}$ & $\numprint{396}$ & $\numprint{1}$ \\
\hline
\end{tabular}
\caption{Problem 2}
\label{TabS5:Prob2}
\end{table} %\label{TabS5:Prob2a}
%
%\begin{table}
%	\begin{tabular}{ ||c|| c | c |c | c ||}
%		\hline
%		$\beta$ / $\gamma$ & $10^{-3}$  & $10^{-1}$  & $10$ & $10^3$ \\ 
%		\hline 
%		& $J_{uc} = 0.0536$ & $J_{uc} = 0.0536$  & $J_{uc} = 0.0536$ & $J_{uc} = 0.0536$\\ 
%		$-1$ & $J_c = 0.0096$ & $J_c = 0.0493$ & $J_c = 0.0535$ & $J_c = 0.0536$\\ 
%		& Iter. $= 724$ & Iter. $= 769$  & Iter. $= 379$ & Iter. $= 1$\\ 
%		\hline
%		& $J_{uc} = 0.0669$ & $J_{uc} = 0.0669$   & $J_{uc} = 0.0669$& $J_{uc} = 0.0669$\\
%		$0$  & $J_c = 0.0109$ & $J_c = 0.0603$  & $J_c = 0.0668$ & $J_c = 0.0669$\\ 
%		& Iter. $= 726$ & Iter. $= 770$  & Iter. $= 390$ & Iter. $= 1$\\ 
%		\hline
%		& $J_{uc} = 0.0839$ & $J_{uc} = 0.0839$  & $J_{uc} = 0.0839$ & $J_{uc} = 0.0839$\\
%		$1$  & $J_c = 0.0125$ & $J_c = 0.0749$  & $J_c = 0.0838$ & $J_c = 0.0839$\\ 
%		& Iter. $= 728$ & Iter. $= 772$  & Iter. $= 396$ & Iter. $= 1$\\ 
%		\hline 
%	\end{tabular}
%    \caption{}
%    \label{TabNFlowEx2}
%\end{table}

\subsubsection{Dirichlet boundary conditions, Example 3} 
The inputs for this example are:
\begin{align*}
&\widehat \rho = \bigg(\frac{1}{2}\cos(\pi y) + \frac{1}{2}\bigg)(1-t) + t\bigg(-\frac{1}{2}\cos(2 \pi y) + \frac{1}{2}\bigg),\\
&\rho_{0} = \frac{1}{2}\cos(\pi y) + \frac{1}{2},\ \
\adj_{T} = 0,\ \
\vec{w} = 0,\ \
f =0,\ \
V_{ext} =0.
\end{align*}
Table \ref{TabS5:Prob3} presents the results for this example for a range of $\beta$ values and different interaction strengths. The observations are in line with those in Example 1 and 2. In particular, $ \widehat \rho$ and $\rho_0$ coincide with those of the problem with Neumann boundary conditions in Example 2. A comparison between the two examples is illustrated in Figure \ref{Ex22DN1}. Both the optimal state $\rho$ and the optimal control are qualitatively different when considering Dirichlet boundary conditions over Neumann conditions. The numerical result for this example was achieved with $N=40$ and $n = 30$, rather than with $N=30$ and $n=20$. This indicates that the Dirichlet boundary conditions are harder to apply in this problem, due to the steep shape of the desired state. This steepness is somewhat less impactful in Example 2, where the desired state is not closely matched at the boundaries. In Example 3, while the desired state is matched perfectly at the boundary, the peaks of the desired state are matched less closely. In Figure \ref{Ex22DN1}, this can be confirmed by considering the control plots. The optimal control for Example 3 is larger than for Example 2, specifically between the boundaries of the domain and the peaks of the desired state.
\begin{table}
\begin{tabular}{ ||c|| c | c | c | c | c ||}
\hline
& & $\beta = 10^{-3}$ & $\beta = 10^{-1}$ & $\beta = 10^{1}$ & $\beta = 10^{3}$  \\
\hline
 & $J_{uc}$ & $\numprint{1.4165e-1}$ & $\numprint{1.4165e-1}$ & $\numprint{1.4165e-1}$ & $\numprint{1.4165e-1}$ \\
$\gamma= \numprint{-1}$  & $J_c$ & $\numprint{3.5594e-2}$ & $\numprint{1.3270e-1}$ & $\numprint{1.4155e-1}$ & $\numprint{1.4165e-1}$ \\
& $Iter.$ & $\numprint{944}$ & $\numprint{816}$ & $\numprint{437}$ & $\numprint{1}$ \\
\hline
 & $J_{uc}$ & $\numprint{1.5452e-1}$ & $\numprint{1.5452e-1}$ & $\numprint{1.5452e-1}$ & $\numprint{1.5452e-1}$ \\
$\gamma= \numprint{0}$  & $J_c$ & $\numprint{3.8023e-2}$ & $\numprint{1.4549e-1}$ & $\numprint{1.5442e-1}$ & $\numprint{1.5452e-1}$ \\
& $Iter.$ & $\numprint{940}$ & $\numprint{825}$ & $\numprint{440}$ & $\numprint{1}$ \\
\hline
 & $J_{uc}$ & $\numprint{1.6610e-1}$ & $\numprint{1.6610e-1}$ & $\numprint{1.6610e-1}$ & $\numprint{1.6610e-1}$ \\
$\gamma= \numprint{1}$  & $J_c$ & $\numprint{4.1143e-2}$ & $\numprint{1.5751e-1}$ & $\numprint{1.6601e-1}$ & $\numprint{1.6610e-1}$ \\
& $Iter.$ & $\numprint{932}$ & $\numprint{827}$ & $\numprint{440}$ & $\numprint{1}$ \\
\hline
\end{tabular}
\caption{Problem 3 ($n = 30,N = 40$)}
\label{TabS5:Prob3}
\end{table} %\label{TabS5:Prob3}
%\begin{table}
%	\begin{tabular}{ ||c|| c | c |c | c ||}
%		\hline
%		$\beta$ / $\gamma$ & $10^{-3}$  & $10^{-1}$  & $10$ & $10^3$ \\ 
%		\hline 
%		& $J_{uc} = 0.0510$ & $J_{uc} = 0.0510$  & $J_{uc} = 0.0510$ & $J_{uc} = 0.0510$\\ 
%		$-1$ & $J_c = 0.0026$ & $J_c = 0.0365$ & $J_c = 0.0508$ & $J_c = 0.0510$\\ 
%		& Iter. $= 690$ & Iter. $= 696$  & Iter. $= 696$ & Iter. $= 696$\\ 
%		\hline
%		& $J_{uc} = 0.0417$ & $J_{uc} = 0.0417$  & $J_{uc} = 0.0417$& $J_{uc} = 0.0417$\\
%		$0$  & $J_c = 0.0027$ & $J_c = 0.0343$  & $J_c = 0.0416$ & $J_c = 0.0417$\\ 
%		& Iter. $= 696$ & Iter. $= 742$  & Iter. $= 409$ & Iter. $= 1$\\ 
%		\hline
%		& $J_{uc} = 0.0452$ & $J_{uc} = 0.0452$  & $J_{uc} = 0.0452$ & $J_{uc} = 0.0452$\\
%		$1$  & $J_c = 0.0030$ & $J_c = 0.0388$  & $J_c = 0.0452$ & $J_c = 0.0452$\\ 
%		& Iter. $= 703$ & Iter. $= 779$  & Iter. $= 397$ & Iter. $= 1$\\ 
%		\hline 
%	\end{tabular}
%	\caption{Update table! Is for different example}
%	\label{TabNFlowEx3}
%\end{table}


\subsection{Linear control problems with an additional nonlocal integral term}
In this section, examples of solving Problem \eqref{AdvDiff_Linear} with both 'no-flux type' boundary conditions \eqref{NoFlux_Linear} and Dirichlet boundary conditions \eqref{Dirichlet}.
\subsubsection{Dirichlet boundary conditions, Example 4}
The inputs for this example are:
\begin{align*}
&\widehat \rho = (1 - t)\bigg(\frac{1}{2}\cos(\pi y) + \frac{1}{2}\bigg)  + t\bigg(-\frac{1}{2}\cos(\pi y) + \frac{1}{2}\bigg),\\
&\rho_{0} = \frac{1}{2}\cos(\pi y) + \frac{1}{2},\ \
\adj_{T} = 0,\ \
{w} = 0,\ \
f =0, \ \
V_{ext} =0.
\end{align*}
In Table \ref{TabS5:Prob4} the results for Example 4 for a range of parameter values can be found. The results are qualitatively similar to the previous examples, the only difference is that the control is applied linearly in this example.
\begin{table}
\begin{tabular}{ ||c|| c | c | c | c | c ||}
\hline
& & $\beta = 10^{-3}$ & $\beta = 10^{-1}$ & $\beta = 10^{1}$ & $\beta = 10^{3}$  \\
\hline
 & $J_{uc}$ & $\numprint{1.4165e-1}$ & $\numprint{1.4165e-1}$ & $\numprint{1.4165e-1}$ & $\numprint{1.4165e-1}$ \\
$\gamma= \numprint{-1}$  & $J_c$ & $\numprint{2.0286e-2}$ & $\numprint{9.0267e-2}$ & $\numprint{1.4067e-1}$ & $\numprint{1.4166e-1}$ \\
& $Iter.$ & $\numprint{784}$ & $\numprint{740}$ & $\numprint{503}$ & $\numprint{49}$ \\
\hline
 & $J_{uc}$ & $\numprint{1.5452e-1}$ & $\numprint{1.5452e-1}$ & $\numprint{1.5452e-1}$ & $\numprint{1.5452e-1}$ \\
$\gamma= \numprint{0}$  & $J_c$ & $\numprint{2.0037e-2}$ & $\numprint{1.0154e-1}$ & $\numprint{1.5356e-1}$ & $\numprint{1.5452e-1}$ \\
& $Iter.$ & $\numprint{788}$ & $\numprint{739}$ & $\numprint{509}$ & $\numprint{56}$ \\
\hline
 & $J_{uc}$ & $\numprint{1.6610e-1}$ & $\numprint{1.6610e-1}$ & $\numprint{1.6610e-1}$ & $\numprint{1.6610e-1}$ \\
$\gamma= \numprint{1}$  & $J_c$ & $\numprint{2.0448e-2}$ & $\numprint{1.1354e-1}$ & $\numprint{1.6518e-1}$ & $\numprint{1.6610e-1}$ \\
& $Iter.$ & $\numprint{792}$ & $\numprint{740}$ & $\numprint{515}$ & $\numprint{61}$ \\
\hline
\end{tabular}
\caption{Problem 4}
\label{TabS5:Prob4}
\end{table} %\label{TabS5:Prob4}

%\begin{table}[h]
%	\begin{tabular}{ ||c|| c | c |c | c ||}
%		\hline
%		$\beta$ / $\gamma$ & $10^{-3}$  & $10^{-1}$  & $10$ & $10^3$ \\ 
%		\hline 
%		& $J_{uc} = 0.1417$ & $J_{uc} = 0.1417$  & $J_{uc} = 0.1417$ & $J_{uc} = 0.1417$\\ 
%		$-1$ & $J_c = 0.0203$ & $J_c = 0.0903$ & $J_c = 0.1407$ & $J_c = 0.1417$\\ 
%		& Iter. $= 787$ & Iter. $= 740$  & Iter. $= 503$ & Iter. $= 49$\\ 
%		\hline
%		& $J_{uc} = 0.1545$ & $J_{uc} = 0.1545$   & $J_{uc} = 0.1545$& $J_{uc} = 0.1545$\\
%		$0$  & $J_c = 0.0200$ & $J_c = 0.1015$  & $J_c = 0.1536$ & $J_c = 0.1545$\\ 
%		& Iter. $= 791$ & Iter. $= 740$  & Iter. $= 510$ & Iter. $= 56$\\ 
%		\hline
%		& $J_{uc} = 0.1661$ & $J_{uc} = 0.1661$  & $J_{uc} = 0.1661$ & $J_{uc} = 0.1661$\\
%		$1$  & $J_c = 0.0204$ & $J_c = 0.1135$  & $J_c = 0.1652$ & $J_c = 0.1661$\\ 
%		& Iter. $= 795$ & Iter. $= 741$  & Iter. $= 515$ & Iter. $= 61$\\ 
%		\hline 
%	\end{tabular}
%	\caption{}
%	\label{TabNFlowEx4}
%\end{table}


\subsubsection{Neumann boundary conditions, Example 5}
The inputs for this example are:
\begin{align*}
&\widehat \rho = \frac{1}{2}(1-t) + t\frac{1}{2}(-\cos(\pi y) + 1),\\
&\rho_{0} = \frac{1}{2},\ \
\adj_{T} = 0,\ \
{w} = 0,\ \
f =0,\ \
V_{ext} =0.
\end{align*}
Table \ref{TabS5:Prob5} shows the results for Example 5. Note that for this example, when $\beta = 10^{-3}$, the mixing parameter $\lambda$ had to be set to $0.001$ (why? explanation needed?).
Again, the only qualitative difference to interpreting the results is that the control is applied linearly.
\begin{table}
\begin{tabular}{ | c | c || c | c | c | c ||}
\hline
\multicolumn{2}{|c||}{}& $\beta = 10^{-3}$ & $\beta = 10^{-1}$ & $\beta = 10^{1}$ & $\beta = 10^{3}$  \\
\hline
\hline
 & $\mathcal{J}_{uc}$ & $\numprint{0.0606}$ & $\numprint{0.0606}$ & $\numprint{0.0606}$ & $\numprint{0.0606}$ \\
$\kappa= \numprint{-1}$  & $\mathcal{J}_c$ & $\numprint{0.0060}$ & $\numprint{0.0541}$ & $\numprint{0.0605}$ & $\numprint{0.0606}$ \\
& \texttt{Iter} & $\numprint{7024}$ & $\numprint{7731}$ & $\numprint{3961}$ & $\numprint{1}$ \\
\hline
 & $\mathcal{J}_{uc}$ & $\numprint{0.0417}$ & $\numprint{0.0417}$ & $\numprint{0.0417}$ & $\numprint{0.0417}$ \\
$\kappa= \numprint{0}$  & $\mathcal{J}_c$ & $\numprint{0.0045}$ & $\numprint{0.0383}$ & $\numprint{0.0416}$ & $\numprint{0.0417}$ \\
& \texttt{Iter} & $\numprint{7003}$ & $\numprint{7618}$ & $\numprint{3642}$ & $\numprint{1}$ \\
\hline
 & $\mathcal{J}_{uc}$ & $\numprint{0.0286}$ & $\numprint{0.0286}$ & $\numprint{0.0286}$ & $\numprint{0.0286}$ \\
$\kappa= \numprint{1}$  & $\mathcal{J}_c$ & $\numprint{0.0036}$ & $\numprint{0.0261}$ & $\numprint{0.0285}$ & $\numprint{0.0286}$ \\
& \texttt{Iter} & $\numprint{7052}$ & $\numprint{7490}$ & $\numprint{3474}$ & $\numprint{1}$ \\
\hline
\end{tabular}
\caption{Example 5: Uncontrolled cost $\mathcal{J}_{uc}$, optimal cost $\mathcal{J}_{c}$, and number of iterations, for a range of $\kappa$ and $\beta$ values.}
\label{TabS5:Prob5}
\end{table} %\label{TabS5:Prob5}
%\begin{table}[h]
%	\begin{tabular}{ ||c|| c | c |c | c ||}
%		\hline
%		$\beta$ / $\gamma$ & $10^{-3}$  & $10^{-1}$  & $10$ & $10^3$ \\ 
%		\hline 
%		& $J_{uc} = 0.0606$ & $J_{uc} = 0.0606$  & $J_{uc} = 0.0606$ & $J_{uc} = 0.0606$\\ 
%		$-1$ & $J_c = 0.0060$ & $J_c = 0.0554$ & $J_c = 0.0606$ & $J_c = 0.0606$\\ 
%		& Iter. $= 7311$ & Iter. $= 771$  & Iter. $= 389$ & Iter. $= 1$\\ 
%		\hline
%		& $J_{uc} = 0.0417$ & $J_{uc} = 0.0417$   & $J_{uc} = 0.0417$& $J_{uc} = 0.0417$\\
%		$0$  & $J_c = 0.0045$ & $J_c = 0.0383$  & $J_c = 0.0416$ & $J_c = 0.0417$\\ 
%		& Iter. $= 7227$ & Iter. $= 759$  & Iter. $= 364$ & Iter. $= 1$\\ 
%		\hline
%		& $J_{uc} = 0.0286$ & $J_{uc} = 0.0286$  & $J_{uc} = 0.0286$ & $J_{uc} = 0.0286$\\
%		$1$  & $J_c = 0.0036$ & $J_c = 0.0265$  & $J_c = 0.0285$ & $J_c = 0.0286$\\ 
%		& Iter. $= 7205$ & Iter. $= 746$  & Iter. $= 341$ & Iter. $= 1$\\ 
%		\hline 
%	\end{tabular}
%	\caption{}
%	\label{TabNFlowEx5}
%\end{table}


\subsection{Nonlinear control problems with an additional nonlocal integral term 2D}

\subsubsection{Neumann boundary conditions, Example 1}	
We have the following set up:
\begin{align*}
&\widehat \rho = \frac{1}{4}(1-t) + t\bigg(\frac{1}{4}\sin \bigg(\frac{\pi}{2}(x_1 - 2)\bigg)\sin \bigg(\frac{\pi}{2}(x_2 - 2)\bigg) + \frac{1}{4}\bigg),\\
&\rho_0 = \frac{1}{4},\ \
q_{T} = 0,\ \
\vec{w} = 0,\ \
f =0,\ \
V_{ext} =0.
\end{align*}
The results for this example are displayed in Table \ref{TabS5:Prob12D}. In figures \ref{rhoHat2dEx2} and \ref{rhoOpt2dEx2} the results for this example are shown at different time points, for $\beta = 10^{-3}$ and $\gamma = -1$. In Figure \ref{rhoOpt2dEx2} the control through a vector field illustrates why nonlinear control is called 'flow control'. This example is the two dimensional version of Example 1 in Section \ref{sec:Examples1d}. 

\begin{table}
\begin{tabular}{ ||c|| c | c | c | c | c ||}
\hline
& & $\beta = 10^{-3}$ & $\beta = 10^{-1}$ & $\beta = 10^{1}$ & $\beta = 10^{3}$  \\
\hline
 & $\mathcal J_{\vec w = \vec 0}$ & $0.0113$ & $0.0113$ & $0.0113$ & $0.0113$ \\
$\kappa= -1$  & $\mathcal J_{Opt}$ & $0.0013$ & $0.0104$ & $0.0113$ & $0.0113$ \\
& $\text{Iterations}$ & $676$ & $700$ & $290$ & $1$ \\
\hline
\end{tabular}
\caption{Results for the test problem, with different $\beta$}
\label{TabS5:Prob12D}
\end{table} % \label{TabS5:Prob12D}

\begin{figure}[h]
	\includegraphics[scale=0.044]{Figure12D.png}
	\caption{2D Example 1, uncontrolled $\rho$ and $\widehat \rho$, $\beta = 10^{-3}$, $\gamma = -1$.}
	\label{rhoHat2dEx2}
\end{figure}
\begin{figure}[h]
	\includegraphics[scale=0.04]{Figure22D.png}
	\caption{2D Example 1, controlled $\rho$ and optimal control $\vec{w}$, $\beta = 10^{-3}$, $\gamma = -1$.}
	\label{rhoOpt2dEx2}
\end{figure}


\subsubsection{Neumann boundary conditions, Example 2}	
Here, we have:
\begin{align*}
&\widehat \rho = \frac{1}{4}(1-t) + t\frac{1}{0.9921}e^{-3((y_1+0.2)^2 + (y_2+0.2)^2))},\\
&\rho_0 = \frac{1}{4},\ \
q_{T} = 0,\ \
\vec{w} = 0,\ \
f =0,\ \
V_{ext} =0.
\end{align*}
The numerical results for this example are displayed in Table \ref{TabS5:Prob22D}. In figures \ref{rhoHat2dEx4} and \ref{rhoOpt2dEx4} the results are illustrated. It can be observed very clearly that the control is driving to the desired state. It is noticeable that the peak of the desired state does not have to be supported as much as the slopes. This is due to the attractive interactions of the particles in this configuration and cannot be observed for repulsive particles.


\begin{table}
\begin{tabular}{ | c | c || c | c | c | c ||}
\hline
\multicolumn{2}{|c||}{}& $\beta = 10^{-3}$ & $\beta = 10^{-1}$ & $\beta = 10^{1}$ & $\beta = 10^{3}$  \\
\hline
\hline
 & $\mathcal{J}_{uc}$ & $\numprint{0.0378}$ & $\numprint{0.0378}$ & $\numprint{0.0378}$ & $\numprint{0.0378}$ \\
$\kappa= \numprint{-1}$  & $\mathcal{J}_c$ & $\numprint{0.0017}$ & $\numprint{0.0312}$ & $\numprint{0.0377}$ & $\numprint{0.0378}$ \\
& \texttt{Iter} & $\numprint{691}$ & $\numprint{736}$ & $\numprint{347}$ & $\numprint{1}$ \\
\hline
 & $\mathcal{J}_{uc}$ & $\numprint{0.0478}$ & $\numprint{0.0478}$ & $\numprint{0.0478}$ & $\numprint{0.0478}$ \\
$\kappa= \numprint{0}$  & $\mathcal{J}_c$ & $\numprint{0.0064}$ & $\numprint{0.0450}$ & $\numprint{0.0478}$ & $\numprint{0.0478}$ \\
& \texttt{Iter} & $\numprint{718}$ & $\numprint{784}$ & $\numprint{343}$ & $\numprint{1}$ \\
\hline
 & $\mathcal{J}_{uc}$ & $\numprint{0.0526}$ & $\numprint{0.0526}$ & $\numprint{0.0526}$ & $\numprint{0.0526}$ \\
$\kappa= \numprint{1}$  & $\mathcal{J}_c$ & $\numprint{0.0137}$ & $\numprint{0.0514}$ & $\numprint{0.0526}$ & $\numprint{0.0526}$ \\
& \texttt{Iter} & $\numprint{735}$ & $\numprint{790}$ & $\numprint{338}$ & $\numprint{1}$ \\
\hline
\end{tabular}
\caption{2D Ex. 2: Cost when $\vec{w}=\vec{0}$, optimal control cost, and iterations required, for a range of $\kappa$, $\beta$.}
\label{TabS5:Prob22D}
\end{table} %\label{TabS5:Prob22D}

\begin{figure}[h]
	\includegraphics[scale=0.3]{Figure32D.png}
	\caption{2D Example 4, uncontrolled $\rho$ and $\widehat \rho$, $\beta = 10^{-3}$, $\gamma = -1$. }
	\label{rhoHat2dEx4}
\end{figure}
\begin{figure}[h]
	\includegraphics[scale=0.3]{Figure42D.png}
	\caption{2D Example 4, controlled $\rho$ and optimal control $\vec{w}$, $\beta = 10^{-3}$, $\gamma = -1$.}
	\label{rhoOpt2dEx4}
\end{figure}







