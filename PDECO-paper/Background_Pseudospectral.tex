There are a number of standard methods for solving DDFT-like problems.  The two most common are the finite element method (FEM)
and pseudospectral methods.  Here we focus on the latter, but note that the algorithm presented below [[REF]] is general and may
be easily adapted to other numerical methods.  The main challenge in using FEM for DDFT problems lies in their non-locality.
Heuristically, the principal benefits of FEM are that it (i) produces large, but sparse matrices, leading to systems which may be efficiently
solved, for example through the implementation of standard timestepping schemes and carefully-chosen preconditioners [[REF PRECONDITIONERS]]; and
(ii) may be applied to complex domains through standard triangulation/meshing routines.
In contrast, for non-local problems such as DDFT the corresponding matrices are not only large, but they are also dense.  This prevents
the use of standard numerical schemes and significantly increases the computational cost.

Recently, accurate and efficient pseudospectral methods have been developed to tackle these non-local, non-linear DDFTs~\cite{NGYSK17}.
Some details of the implementation will be discussed in Section [[REF]]; here we highlight the benefits and challenges.  As is widely 
known~\cite{T00,B01}, pseudospectral methods are extremely accurate for problems with smooth solutions on `nice' 
domains; here `nice' roughly corresponds to domains which may be mapped to the unit square in a simple (e.g.\ conformal) manner.
They are more challenging to apply on complex domains (although spectral elements can be seen as  a compromise between FEM
and pseudospectral methods), and are also of poor accuracy when the solutions are not smooth (heuristically, the accuracy is 
order $(1/N)^p$ where the solution is $p$-times differentiable.

Their use to treat DDFT problems stems from three main observations: (i) at least morally, the diffusion term present in
all DDFTs should lead to smoothing of solutions for sufficiently smooth particle interactions; (ii) the matrices involved are always
dense and, as such, treating non-local terms does not formally affect the numerical cost; (iii) the implementation of non-local 
boundary conditions may be treated via standard algebraic-differential equations solver, thus removing the need for bespoke
treatments of different boundary conditions.
