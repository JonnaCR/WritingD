We are interested in modelling sedimentation processes. In order to achieve this, the advection-diffusion equation with mean-field interaction term has to be modified to include an approximation to volume exclusion. Archer and Malijevs\'y \cite{ArcherSed1} have achieved this using the following model to describe sedimentation processes. 
The modelling equations are:
\begin{align*}
	&\frac{\partial \rho}{\partial t^*} = \Gamma\nabla \cdot \left(  \rho \nabla \frac{\delta F[\rho]}{\delta \rho} \right) ,
\end{align*}
where $\Gamma$ is the diffusion coefficient. 	
We can rescale this equation as done in \cite{ArcherSed1} using the relationship $t = t^*/ \tau_B$, where $\tau_B = \beta \sigma^2 / \Gamma$ is the Brownian time scale.
Applying this rescaling we get:
\begin{align}\label{Eq1}
	&\frac{\partial \rho}{\partial t} = \beta \sigma^2\nabla \cdot \left(  \rho \nabla \frac{\delta F[\rho]}{\delta \rho} \right).
\end{align}
The free energy functional considered in \cite{ArcherSed1} is:
\begin{align*}
	F[\rho] &= \frac{1}{\beta} \int \rho (\ln \Lambda^2 \rho - 1) + f_{HDA} dr + \frac{1}{2}\int \int \rho(r) \rho(r') V_2(|r - r'|) dr dr' + \int \rho V_{ext} dr,
\end{align*}
where $f_{HDA}$ the approximate free energy density describing the volume exclusion through hard disks. The external potential is defined as:
\begin{align*}
	V_{ext} &= c y, \quad \text{for } \quad 0 < y < L,
\end{align*}
where $c$ a constant and $L$ is the height of a rectangular domain. Outside these bounds $V_{ext} = \infty$. 
Furthermore, we have the pair potential:
\begin{align*}
	V_2 = exp(-r/\sigma),
\end{align*}
where $\sigma$ is the particle diameter of the hard sphere particle.


\subsection{The Hard Disk Approximation}
The part of the free energy functional, which accounts for the hard disk approximation, is:
\begin{align*}
	F_{HDA}[\rho] &= \frac{1}{\beta} \int  f_{HDA} dr = \frac{1}{\beta} \int   -  \rho - \rho \ln(1 - \eta) + \frac{\rho}{1 - \eta} dr,
\end{align*}
where $\eta = a \rho = \frac{\pi \sigma^2}{4} \rho$.
This can be thought of as the bulk fluid, one species, two dimensional approximation of Fundamental Measure Theory (FMT)  \cite{RosenfeldFMT}, which is a Density Functional Theory for hard sphere mixtures. The basis of this theory is that the excess free energy functional is of the form:
\begin{align*}
	\beta F_{ex}[{\rho_i}] = \int \Phi({n_\alpha(r')})d^3r',
\end{align*} 
where $i$ is the species count and $\Phi$ is a function of the weighted densities $n_\alpha$. By now there are many different versions of $\Phi$, yielding approximations of $F_{ex}$ with different limitations, see \cite{Roth_2010FMTReview}. Rosenfeld's original version is defined as:
	\begin{align*}
	\Phi = -n_0 \ln(1-n_3) + \frac{n_1 n_2 - \mathbf{n_1} \cdot \mathbf{n_2}}{1-n_3} + \frac{n_2^3 - 3n_2 \mathbf{n_2} \cdot \mathbf{n_2}}{24 \pi (1-n_3)^2}.
	\end{align*}
The weighted densities for $\nu$ species are:
	\begin{align} \label{eqn:WeightedDensities}
		n_\alpha (r) = \sum_{i=1}^\nu \int \rho_i(r') \omega_\alpha^i(r -r').
	\end{align}
The weight functions chosen by Rosenfeld are:
\begin{align*}
	&\omega_3^i = \Theta(R_i -r),\quad \omega_2^i = \delta(R_i -r),\quad \mathbf{\omega_2^i} = \frac{\mathbf r}{r}\delta (R_i -r),\quad \\
	&\omega_1^i = \omega_2^i/(4 \pi R_i), \quad \omega_0^i = \omega_2^i/(4\pi R_i^2), \quad \mathbf{\omega_1^i} = \mathbf{\omega_2^i}/(4 \pi R_i),
\end{align*}
where $R_i$ is the radius of the excluded volume, $\Theta$ is the Heaviside function and $\delta$ is the delta function. Integrating over $\omega_\alpha$, with $\alpha = 0,1,2,3$, we get the fundamental measures of a sphere: volume, surface area, radius and the Euler characteristic \cite{Roth_2010FMTReview} \cite{RosenfeldFMT}.
\\
Based on this theory for three dimensional spheres and the fact that the theory for hard rods is known exactly \cite{Percus1976}, Rosenfeld derived a version of this approach for two dimensional hard disks \cite{Rosenfeld2DInterp}. However, some additional approximations have to be made when choosing the weighted densities, which is not necessary in one and three dimensions. The resulting equation is:
\begin{align*}
	\Phi = - n_0 \ln(1-n_3) + \frac{1}{4 \pi} \frac{n_2 n_2}{1-n_3} + \frac{1}{4 \pi} \frac{\mathbf{n_2} \cdot \mathbf{n_2}}{1-n_3}.
\end{align*}
\\
In the uniform limit, for one particle species, we get that:
\begin{align*}
	n_0 = \rho, \quad n_2 = 2 \pi R \rho, \quad n_3 = \pi R^2 \rho,
\end{align*}
by solving the integrals in \eqref{eqn:WeightedDensities}, using spherical polar coordinates, with $\rho = \rho_{\text{bulk}}$, a constant. 
Substituting this in the 2D version of $\Phi$ gives:
\begin{align*}
	\Phi = - \rho \ln (1- \pi R^2 \rho) + \frac{1}{4 \pi} \frac{4\pi^2 R^2 \rho^2}{1 - \pi R^2 \rho} + \frac{1}{4 \pi}\frac{\mathbf{n_2} \cdot \mathbf{n_2}}{1 - \pi R^2 \rho},
\end{align*}
where $\mathbf{n_2} = \mathbf 0$ in the uniform limit, since the corresponding equation in \eqref{eqn:WeightedDensities} is an integral over an odd function.
Noting that $R = \sigma/2$ and $\eta = \pi \sigma^2 \rho /4$, we get that:
\begin{align} \label{eqn:SPTFunctional2D}
	\Phi = - \rho \ln (1- \eta) + \frac{\rho \eta}{1 - \eta} = \rho \left(-\ln(1-\eta) + \frac{1}{1- \eta} -1 \right).
\end{align} 
This expression for the free energy for the bulk fluid is the same as derived by scaled particle theory (SPT) \cite{Reiss1959}, \cite{Reiss1960}, \cite{Helfand1961}, which also coincides with the Percus-Yevic compressibility equation \cite{PercusYevick1}, as detailed in \cite{RosenfeldSPT}.
While the SPT approximation \eqref{eqn:SPTFunctional2D} and its three-dimensional equivalent are used in classical DFT, see \cite{DFTWinkelmann2001}, \cite{DFTRoth1}, \cite{DFTRoth2}, \cite{DFTGonzalez1997}, \cite{DFTCuesta2008}, \cite{DFTLoewen2002}, and other statistical mechanics approaches, see \cite{GrafLoewen1999}, \cite{DuBois2002}, \cite{Chamoux1998}, \cite{Chamoux1996}, in dynamical DFT it is not commonly applied and only the work of Archer et al. \cite{ArcherSed1}, \cite{ArcherSed2008}, \cite{ArcherSed2011}, \cite{ArcherSed2013}, is known to us in this context.

\subsection{Deriving the equation of motion}
Since we are interested in the equation of motion, we need to calculate $ \nabla \cdot \left(\rho \nabla \frac{\delta F_{HDA}[\rho]}{\delta \rho} \right)$. We combine $F_{HDA}$ and $F_{ID} = \int_\Omega \rho (\ln \Lambda^2 \rho - 1) dr$ here so that we have:
\begin{align*}
	F_{N} = F_{HDA} + F_{ID}.
\end{align*}
Taking the functional derivative of $F_{N}$ gives:
\begin{align*}
	\frac{\delta F_{N}[\rho]}{\delta \rho} &= \frac{1}{\beta} \bigg(1 + \ln \rho + \Lambda^2 -2 - \ln(1-\eta) + a \frac{\rho}{1 - \eta} + \frac{1}{1 - \eta} + a \frac{\rho}{(1 - \eta)^2}  \bigg)\\
	&= \frac{1}{\beta} \bigg(1 + \ln \rho + \Lambda^2 -2 - \ln(1-\eta) + \frac{1}{(\eta - 1)^2} - \frac{1}{\eta - 1}  - 1\bigg)\\
	&= \frac{1}{\beta} \bigg( \ln \rho + \Lambda^2 - 2 - \ln(1-\eta) - \frac{\eta - 2}{(\eta - 1)^2}  \bigg),
\end{align*}
using partial fractions. Then:
\begin{align*}
	\nabla \frac{\delta F_{N}[\rho]}{\delta \rho} &= \frac{1}{\beta} \bigg( \nabla\ln \rho + \nabla(\Lambda^2 - 2) - \nabla\ln(1-\eta) - \nabla\frac{\eta - 2}{(\eta - 1)^2}  \bigg)\\
	& = \frac{1}{\beta} \bigg( \frac{\nabla \rho}{\rho} - \frac{\nabla( 1- \eta)}{1 - \eta} - \nabla\frac{\eta - 2}{(\eta - 1)^2}  \bigg)\\
	& = \frac{1}{\beta} \bigg( \frac{\nabla \rho}{\rho} + \frac{\nabla \eta}{1 - \eta} - \nabla\frac{\eta - 2}{(\eta - 1)^2}  \bigg).
\end{align*}
Then multiplying by $\rho$ gives:
\begin{align*}
	\rho \nabla \frac{\delta F_{N}[\rho]}{\delta \rho} &= \frac{1}{\beta} \bigg( \nabla \rho +   \frac{\rho \nabla \eta}{1 - \eta} - \rho \nabla\frac{\eta - 2}{(\eta - 1)^2}  \bigg)\\
	&= \frac{1}{\beta} \bigg( \nabla \rho +   \frac{\eta\nabla \rho}{1 - \eta} - \rho \nabla\frac{\eta - 2}{(\eta - 1)^2}  \bigg)\\
	&= \frac{1}{\beta} \bigg( \nabla \rho + \frac{\nabla \rho}{1 - \eta} - \nabla \rho - \rho \nabla\frac{\eta - 2}{(\eta - 1)^2}  \bigg)\\
	&= \frac{1}{\beta} \bigg(  \frac{\nabla \rho}{1 - \eta}  - \rho \nabla\frac{\eta - 2}{(\eta - 1)^2}  \bigg).
\end{align*}
Finally we take the divergence:
\begin{align*}
	\nabla \cdot \bigg(\rho \nabla \frac{\delta F_{N}[\rho]}{\delta \rho}\bigg) &= \frac{1}{\beta} \bigg( \nabla  \cdot \left(  \frac{\nabla \rho}{1 - \eta} \right) - \nabla  \cdot \left(\rho \nabla\frac{\eta - 2}{(\eta - 1)^2} \right) \bigg)\\
	&= \frac{1}{\beta} \bigg( \frac{\nabla^2 \rho}{1 - \eta} +  \nabla \rho \cdot \nabla \frac{1}{1 - \eta} - \nabla \rho \cdot \nabla \frac{\eta - 2}{(\eta - 1)^2} - \rho \nabla^2\frac{\eta - 2}{(\eta - 1)^2} \bigg)\\
	&= \frac{1}{\beta} \bigg( \frac{\nabla^2 \rho}{1 - \eta} +  \nabla \rho \cdot \nabla \frac{(3- 2 \eta)}{(1 - \eta)^2}  - \rho \nabla^2\frac{\eta - 2}{(\eta - 1)^2} \bigg).
\end{align*}