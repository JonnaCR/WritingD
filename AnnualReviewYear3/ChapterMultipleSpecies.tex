	In this section we are interested in deriving the optimality conditions for two species, which are interacting through a non-local interaction term.
	We have the following set of forward equations:
	\begin{align*}
	\frac{\partial \ra}{\partial t} =& D_a\nabla^2 \ra - D_a\nabla \cdot(\ra F_a(\w)) + D_a \nabla \cdot (\ra \nabla V_{ext,a}) + D_a\kappa \nabla \cdot \int_\Omega \ra(r) \ra (r') \K_{aa}(r,r')dr' \\
	&+  D_a\tilde{\kappa}\nabla \cdot \int_\Omega \ra(r) \rb (r') \K_{ab}(r,r')dr'\\
	\frac{\partial \rb}{\partial t} =& D_b\nabla^2 \rb - D_b\nabla \cdot(\rb F_b(\w)) + D_b \nabla \cdot (\rb \nabla V_{ext,b}) + D_b\kappa \nabla \cdot \int_\Omega \rb(r) \rb (r') \K_{bb}(r,r')dr' \\
	&+  D_b\tilde{\kappa} \nabla \cdot \int_\Omega \rb(r) \ra (r') \K_{ba}(r,r')dr',
	\end{align*}
	where $D = \frac{1}{\gamma m}$. The interaction kernel $\mathbf K_{ij}$ are describing the effect of species $i$ on species $j$. The function $F_i$ describes a function of the control, which may be different for species $i = a$ and $i = b$. This generalization allows for effects such as that species $a$ gets advected faster than species $b$, for example, due to size differences. We have the interaction strength $\kappa$, describing the effects within the species and the interaction strength $\tilde \kappa$, describing interaction strengths between species.
	The no-flux boundary conditions are:
	\begin{align*}
	&\bigg( D_a \nabla \ra - D_a \ra F_a(\w) + D_a \ra \nabla V_{ext,a} + D_a\kappa \int_\Omega \ra(r) \ra (r') \K_{aa}(r,r')dr' \\
	&+  D_a\tilde{\kappa} \int_\Omega \ra(r) \rb (r') \K_{ab}(r,r')dr' \bigg) \cdot \n = 0\\
	&\bigg( D_b \nabla \rb - D_b \rb F_b(\w) + D_b \rb \nabla V_{ext,b} + D_b\kappa \int_\Omega \rb(r) \rb (r')\K_{bb}(r,r')dr' \\
	&+  D_b\tilde{\kappa} \int_\Omega \rb(r) \ra (r') \K_{ba}(r,r')dr' \bigg) \cdot \n = 0
	\end{align*}
	The cost functional is:
	\begin{align*}
	J(\ra,\rb, \w) \coloneqq \frac{1}{2}|| \ra - \widehat{\ra} ||^2_{L_2(\Sigma)} + \frac{\alpha}{2}|| \rb - \widehat {\rb} ||^2_{L_2(\Sigma)} + \frac{\beta}{2}||\w||^2_{L_2(\Sigma)}.
	\end{align*}
	Therefore, the Lagrangian is:
	\begin{align*}
	\mathcal{L}(\ra,\rb, \w, \adja, \adjb) =& \frac{1}{2}\int_0^T \int_\Omega ( \ra - \widehat{\ra})^2 dr dt + \frac{\alpha}{2}\int_0^T \int_\Omega ( \rb - \widehat{\rb})^2 dr dt + \frac{\beta}{2}\int_0^T \int_\Omega \w^2 dr dt\\
	& - \int_0^T \int_\Omega \bigg(\frac{\partial \ra}{\partial t} - D_a\nabla^2 \ra + D_a\nabla \cdot(\ra F_a(\w)) - D_a \nabla \cdot (\ra \nabla V_{ext,a})\\
	& - D_a\kappa \nabla \cdot \int_\Omega \ra(r) \ra (r') \K_{aa}(r,r')dr' - D_a\tilde{\kappa} \nabla \cdot \int_\Omega \ra(r) \rb (r') \K_{ab}(r,r')dr\bigg)\adja dr dt \\
	& - \int_0^T \int_\Omega \bigg(
	\frac{\partial \rb}{\partial t} - D_b\nabla^2 \rb + D_b\nabla \cdot(\rb F_b(\w)) - D_b \nabla \cdot (\rb \nabla V_{ext,b}) \\
	&- D_b\kappa \nabla \cdot \int_\Omega \rb(r) \rb (r') \K_{bb}(r,r')dr'
	-  D_b\tilde{\kappa} \nabla \cdot \int_\Omega \rb(r) \ra (r')\K_{ba}(r,r')dr'\bigg) \adjb dr dt\\
	&- \int_0^T \int_{\partial \Omega} \bigg( D_a \nabla \ra - D_a \ra F_a(\w) + D_a \ra \nabla V_{ext,a} + D_a\kappa \int_\Omega \ra(r) \ra (r') \K_{aa}(r,r')dr' \\
	&+  D_a\tilde{\kappa} \int_\Omega \ra(r) \rb (r') \K_{ab}(r,r')dr' \bigg) \cdot \n \adjaB dr dt\\
	& - \int_0^T \int_{\partial \Omega} \bigg( D_b \nabla \rb - D_b \rb F_b(\w) + D_b \rb \nabla V_{ext,b} + D_b\kappa \int_\Omega \rb(r) \rb (r')\K_{bb}(r,r')dr' \\
	&+  D_b\tilde{\kappa} \int_\Omega \rb(r) \ra (r') \K_{ba}(r,r')dr' \bigg) \cdot \n \adjbB dr dt.
	\end{align*}
	\subsection{Adjoint 1}
	Taking the derivative with respect to $\ra$ gives
	\begin{align*}
		\mathcal{L}_{\ra}(\ra,\rb, \w, \adja, \adjb) h &= \int_0^T \int_\Omega ( \ra - \widehat{\ra})h dr dt 
       + \int_0^T \int_\Omega \bigg(-\frac{\partial h}{\partial t}\adja + D_a\nabla^2 h \adja - D_a\nabla \cdot(h F_a(\w)) \adja\\
		&  + D_a \nabla \cdot (h \nabla V_{ext,a}) \adja + D_a\kappa\adja \nabla \cdot \int_\Omega  h(r) \ra (r') \K_{aa}(r,r')dr' \\
		&+ D_a\kappa \adja \nabla \cdot \int_\Omega  \ra (r) h (r') \K_{aa}(r,r')dr' + D_a \tilde{\kappa}\adja \nabla \cdot \int_\Omega  h(r) \rb (r') \K_{ab} (r,r')dr  \\
		& + D_b\tilde{\kappa}\adjb \nabla \cdot \int_\Omega \rb(r) h (r')\K_{ba}(r,r')dr'\bigg)  dr dt\\
		&- \int_0^T \int_{\partial \Omega} \bigg( D_a \nabla h - D_a h F_a(\w) + D_a h \nabla V_{ext,a} + D_a\kappa \int_\Omega h(r) \ra (r') \K_{aa}(r,r')dr' \\
		&+ D_a\kappa \int_\Omega \ra(r) h(r') \K_{aa}(r,r')dr'+  D_a\tilde{\kappa} \int_\Omega h(r) \rb (r') \K_{ab}(r,r')dr' \bigg) \cdot \n \adjaB dr dt\\
		&- \int_0^T \int_{\partial \Omega}  D_b\tilde{\kappa} \int_\Omega \rb(r) h(r') \K_{ba}(r,r')dr'  \cdot \n \adjbB dr dt,
	\end{align*}
	and so:
	\begin{align*}
	\mathcal{L}_{\ra}(\ra,\rb, \w, \adja, \adjb) h &= \int_0^T \int_\Omega ( \ra - \widehat{\ra})h dr dt 
	+ \int_0^T \int_\Omega \bigg(\frac{\partial \adja}{\partial t}h + D_a\nabla^2 \adja h + D_a\nabla \adja \cdot(h F_a(\w)) \\
	&  - D_a \nabla \adja \cdot (h \nabla V_{ext,a})  - D_a\kappa \nabla \adja(r) h(r)  \int_\Omega \ra (r') \K_{aa}(r,r')dr' \\
	&- D_a\kappa h (r)\int_\Omega \nabla \adja(r') \ra (r') \K_{aa}(r',r)dr' - D_a \tilde{\kappa} \nabla \adja h(r) \int_\Omega  \rb (r') \K_{ab} (r,r')dr \\
	&- D_b \tilde \kappa h (r)\int_\Omega \nabla \adjb(r') \rb (r') \K_{ba}(r',r)dr'\bigg)  dr dt \\
	 &+\int_{ \Omega} \adja(T)h(T) - \adja(0)h(0) dr \\
	 &+ \int_0^T \int_{\Omega} \bigg(D_a \kappa h(r) \int_{\partial \Omega} \adja(r') \ra(r') \K_{aa}(r',r)  dr' \cdot \n \\
	 &+ D_b \tilde \kappa h(r) \int_{\partial \Omega} \adjb(r') \rb(r') \K_{ba}(r',r)dr' \cdot \n \bigg) dr dt\\
	 &+ \int_0^T \int_{\partial \Omega} D_a \frac{\partial h}{\partial n}\adja - D_a \frac{\partial \adja}{\partial n}h - D_a F_a(\w) h \adja \cdot \n + D_a \nabla V_{ext,a} h \adja \cdot \n dr dt\\ 
	 & + \int_0^T \int_{\partial \Omega} \bigg(D_a \kappa h(r) \adja(r) \int_\Omega \ra(r') \K_{aa}(r,r') \cdot \n dr' \\
	 &+ D_a \tilde{\kappa} \adja(r) h(r) \int_\Omega \rb(r') \K_{ab}(r,r') \cdot \n dr' \bigg) dr dt\\
	 &- \int_0^T \int_{\partial \Omega} \bigg( D_a \nabla h \adjaB- D_a h F_a(\w)\adjaB + D_a h \nabla V_{ext,a}\adjaB\\
	 & + D_a\kappa \adjaB(r) h(r)  \int_\Omega \ra (r') \K_{aa}(r,r')dr' +  D_a\tilde{\kappa}\adjaB  h(r)\int_\Omega \rb (r') \K_{ab}(r,r')dr' \bigg) \cdot \n dr dt \\
	 &- \int_0^T \int_{\Omega} \bigg(D_a\kappa h(r) \int_{\partial \Omega} \adjaB(r') \ra(r')   \K_{aa}(r',r)dr'\\
	 & +D_b\tilde{\kappa} h(r)  \int_\Omega \adjbB(r') \rb(r') \K_{ba}(r',r)dr'  \bigg) \cdot \n  dr dt.\\
	\end{align*}
	Then for $\frac{\partial h}{\partial n} \neq 0$ we get:
	\begin{align*}
	&(D_a \adja - D_a \adjaB) \n = \mathbf 0\\
	&\adja = \adjaB.
	\end{align*}
	And all boundary terms cancel so that we get:
	\begin{align*}
	\frac{\partial \adja}{\partial n} = 0 \quad \text{on} \quad \partial \Omega.
	\end{align*}
	And we also get $\adja(T) = 0$.
	
	We get:
	\begin{align*}
	 \frac{\partial \adja}{\partial t} = &- D_a\nabla^2\adja - \ra + \widehat{\ra}    - D_a\nabla \adja \cdot F_a(\w) + D_a \nabla \adja \cdot  \nabla V_{ext,a} \\
	 &+ D_a\kappa \nabla \adja(r) \int_\Omega \ra (r') \K_{aa}(r,r')dr' + D_a\kappa \int_\Omega \nabla \adja(r') \ra (r') \K_{aa}(r',r)dr' \\
	 & + D_a \tilde{\kappa} \nabla \adja(r) \int_\Omega  \rb (r') \K_{ab} (r,r')dr' + D_b \tilde \kappa\int_\Omega \nabla \adjb(r') \rb (r') \K_{ba}(r',r)dr' .
	\end{align*}
	\subsection{Adjoint 2}
	The second adjoint equation is derived in an equivalent manner to the first:
	\begin{align*}
	\frac{\partial \adjb}{\partial t} = &- D_b\nabla^2\adjb - \alpha \rb + \alpha \widehat{\rb}    - D_b\nabla \adjb \cdot F_b(\w) + D_b \nabla \adjb \cdot  \nabla V_{ext,b} \\
	&+ D_b\kappa \nabla \adjb(r) \int_\Omega \rb (r') \K_{bb}(r,r')dr' + D_b\kappa \int_\Omega \nabla \adjb(r') \rb (r') \K_{bb}(r',r)dr' \\
	& + D_b \tilde{\kappa} \nabla \adjb \int_\Omega  \ra (r') \K_{ba} (r,r')dr' + D_a \tilde \kappa\int_\Omega \nabla \adja(r') \ra (r') \K_{ab}(r',r)dr'.
	\end{align*}
	The boundary condition is:
	\begin{align*}
	\frac{\partial \adjb}{\partial n} = 0 \quad \text{on} \quad \partial \Omega,
	\end{align*} 
	and we also get $\adjb(T) = 0$.
	
	\subsection{Gradient Equation}
	We consider the derivative of the Lagrangian with respect to $\w$. However, we will need to consider the Frech\'et derivative of terms involving $F(\w)$ first. If $F$ is a function of $\w$ only and not of the position variable $r$, we can do the following. Otherwise, we will have to work with the definition of the Frech\'et derivative and derive the gradient equation like that.
	We consider the first order term of the Taylor expansion, so that we have:
	\begin{align*}
	F(\w + \h) - F(\w) =  \left(\nabla_{\w} F(\w)^T\right) \h 
	\end{align*}

    Then:
	\begin{align*}
	\mathcal{L}_{\w}(\ra,\rb, \w, \adja, \adjb) \h  &= \int_0^T \int_\Omega \bigg( \beta \w \cdot \h - D_a \nabla \cdot (\ra \left(\nabla_{\w} F_a(\w)^T\right) \h)  \adja - D_b \nabla \cdot (\rb \left(\nabla_{\w} F_b(\w)^T \right)\h) \adjb \bigg)dr dt \\
	&+ \int_0^T \int_{\partial \Omega} \bigg( D_a \ra \left(\nabla_{\w} F_a(\w)^T\right) \h \adjaB   + D_b \rb \left(\nabla_{\w} F_b(\w)^T\right) \h\adjbB     \bigg) \cdot \n dr dt\\
	&= \int_0^T \int_\Omega \bigg( \beta \w \cdot \h + D_a \ra \left(\left(\nabla_{\w} F_a(\w)^T\right) \h \right)\cdot\nabla  \adja \\
	&+ D_b \rb \left(\left(\nabla_{\w} F_b(\w)^T\right) \h \right)\cdot \nabla \adjb  \bigg)dr dt \\
	&- \int_0^T \int_{\partial \Omega} \bigg( D_a \ra \left(\nabla_{\w} F_a(\w)^T\right) \h \adja   + D_b \rb \left(\nabla_{\w} F_b(\w)^T\right) \h\adjb     \bigg) \cdot \n dr dt\\
	&+ \int_0^T \int_{\partial \Omega} \bigg( D_a \ra \left(\nabla_{\w} F_a(\w)^T\right) \h \adjaB   + D_b \rb \left(\nabla_{\w} F_b(\w)^T\right) \h\adjbB     \bigg) \cdot \n dr dt\\
	&=\int_0^T \int_\Omega \bigg( \beta \w \cdot \h + D_a \ra \left(\left(\nabla_{\w} F_a(\w)^T\right) \h \right)\cdot\nabla  \adja \\
	&+ D_b \rb \left(\left(\nabla_{\w} F_b(\w)^T\right) \h \right)\cdot \nabla \adjb  \bigg)dr dt,
	\end{align*}
	since $\adja = \adjaB$ and $\adjb = \adjbB$ from the adjoint derivation.\\
	Now we use the relation $((\nabla \mathbf a)^T)\mathbf b) \cdot \mathbf c= (( \mathbf c \cdot \nabla) \mathbf a ) \cdot \mathbf b$ (from second year review) to find that:
	\begin{align*}
	\mathcal{L}_{\w}(\ra,\rb, \w, \adja, \adjb) \h  &= \int_0^T \int_\Omega \bigg( \beta \w \cdot \h + D_a \ra \left( \left(\nabla_r \adja \cdot \nabla_{\w} \right) F_a(\w) \right) \cdot \h         \\
	&+ D_b \rb \left( \left(\nabla_r \adjb \cdot \nabla_{\w} \right) F_b(\w) \right) \cdot \h       \bigg)dr dt.
	\end{align*}
	Setting this to zero and since this holds for all permissible $\h$, we get:
	\begin{align*}
	\beta \w  + D_a \ra \left( \left(\nabla_r \adja \cdot \nabla_{\w} \right) F_a(\w) \right) 
	+ D_b \rb \left( \left(\nabla_r \adjb \cdot \nabla_{\w} \right) F_b(\w) \right) = 0.
	\end{align*} 
	Using that $\nabla \cdot (\mathbf{b a}^T) = \mathbf a (\nabla \cdot \mathbf b) + (\mathbf b \cdot \nabla) \mathbf a$, and observing that $\nabla_{\w} \cdot (\nabla_r q) = 0$, we get:
	\begin{align*}
	\beta \w  + D_a \ra \nabla_{\w} \cdot \left(\nabla \adja F_a(\w)^T \right) 
	+ D_b \rb \nabla_{\w} \cdot \left(\nabla \adjb F_b(\w)^T \right) = 0.
	\end{align*} 
	Since $\nabla_r q$ does not depend on $\w$, we can rearrange this to get:
	\begin{align*}
	\beta \w  + D_a \ra \left(\nabla_\w F_a(\w)\right)^T \nabla \adja  
	+ D_b \rb \left(\nabla_\w F_b(\w)\right)^T \nabla \adjb = 0.
	\end{align*}
	And finally we have:
	\begin{align*}
	\w = - \frac{1}{\beta} \left( D_a \ra \left(\nabla_\w F_a(\w)\right)^T \nabla \adja  
	+ D_b \rb \left(\nabla_\w F_b(\w)\right)^T \nabla \adjb \right).
	\end{align*}
    As an example, take $F_a(\w) = c_a \w$ and $F_b(\w) = c_b \w$. We get:
	\begin{align*}
	\w  = - \frac{1}{\beta}\bigg( D_a  \ra c_a \mathbf 1 \nabla \adja + D_b \rb c_b \mathbf 1 \nabla \adjb \bigg).
	\end{align*}