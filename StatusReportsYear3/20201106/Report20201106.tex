\documentclass[11pt, a4paper]{article}
%\usepackage{proj1}
\usepackage{natbib}
\usepackage{fancyhdr}  
\usepackage{subcaption}
\usepackage{caption}
\usepackage{graphicx}
\usepackage{numprint}
\usepackage{multirow}
\linespread{1.25} 
\setlength{\parindent}{0cm}
\graphicspath{{Images/}}
\usepackage{hyperref}
\usepackage{amsmath}
\usepackage{amsfonts}
\usepackage{amssymb}
\usepackage{amsthm}
\usepackage{mathtools}
\usepackage{commath}
\usepackage{bbm}

%\usepackage[sc,osf]{mathpazo}
\usepackage{subcaption}
\usepackage[a4paper, top=1in, left=1.0in, right=1.0in, bottom=1in, includehead, includefoot]{geometry} %Usually have top as 1in

\usepackage{listings}
\usepackage{color} %red, green, blue, yellow, cyan, magenta, black, white
\definecolor{mygreen}{RGB}{28,172,0} % color values Red, Green, Blue
\definecolor{mylilas}{RGB}{170,55,241}


\hypersetup{colorlinks,linkcolor={black},citecolor={blue},urlcolor={black}}
\usepackage{color}
\urlstyle{same}


\theoremstyle{definition}
\newtheorem{definition}{Definition}[section]

%\newcommand{\Sta}{\rho}
\newcommand{\adja}{q_a}
\newcommand{\adjb}{q_b}
\newcommand{\adjaB}{q_{a,\partial \Omega}}
\newcommand{\adjbB}{q_{b,\partial \Omega}}
%\newcommand{\Con}{u}
\newcommand{\ra}{\rho_a}
\newcommand{\rb}{\rho_b}
\newcommand{\w}{\mathbf{w}}
\newcommand{\Stav}{\mathbf{v}}
\newcommand{\Adja}{\mathbf{p}}
\newcommand{\Adjb}{q}
\newcommand{\Adjc}{{p}_{\partial \Sigma}}
\newcommand{\Con}{\mathbf{f}}
\newcommand{\n}{\mathbf{n}}
\newcommand{\h}{\mathbf{h}}
\newcommand{\K}{\mathbf{K}}


\pagenumbering{gobble}
\begin{document}
	
	
\section{The Multiple Species Gradient Equation}
We consider the derivative of the Lagrangian with respect to $\w$. However, we will need to consider the Frech\'et derivative of terms involving $F(\w)$ first. If $F$ is a function of $\w$ only and not of the position variable $r$, we can do the following. Otherwise, we will have to work with the definition of the Frech\'et derivative and derive the gradient equation like that.
We consider the first order term of the Taylor expansion, so that we have:
\begin{align*}
F(\w + \h) - F(\w) =  \left(\nabla_{\w} F(\w)^T\right) \h 
\end{align*}

Then:
\begin{align*}
\mathcal{L}_{\w}(\ra,\rb, \w, \adja, \adjb) \h  &= \int_0^T \int_\Omega \bigg( \beta \w \cdot \h - D_a \nabla \cdot (\ra \left(\nabla_{\w} F_a(\w)^T\right) \h)  \adja - D_b \nabla \cdot (\rb \left(\nabla_{\w} F_b(\w)^T \right)\h) \adjb \bigg)dr dt \\
&+ \int_0^T \int_{\partial \Omega} \bigg( D_a \ra \left(\nabla_{\w} F_a(\w)^T\right) \h \adjaB   + D_b \rb \left(\nabla_{\w} F_b(\w)^T\right) \h\adjbB     \bigg) \cdot \n dr dt\\
&= \int_0^T \int_\Omega \bigg( \beta \w \cdot \h + D_a \ra \left(\left(\nabla_{\w} F_a(\w)^T\right) \h \right)\cdot\nabla  \adja \\
&+ D_b \rb \left(\left(\nabla_{\w} F_b(\w)^T\right) \h \right)\cdot \nabla \adjb  \bigg)dr dt \\
&- \int_0^T \int_{\partial \Omega} \bigg( D_a \ra \left(\nabla_{\w} F_a(\w)^T\right) \h \adja   + D_b \rb \left(\nabla_{\w} F_b(\w)^T\right) \h\adjb     \bigg) \cdot \n dr dt\\
&+ \int_0^T \int_{\partial \Omega} \bigg( D_a \ra \left(\nabla_{\w} F_a(\w)^T\right) \h \adjaB   + D_b \rb \left(\nabla_{\w} F_b(\w)^T\right) \h\adjbB     \bigg) \cdot \n dr dt\\
&=\int_0^T \int_\Omega \bigg( \beta \w \cdot \h + D_a \ra \left(\left(\nabla_{\w} F_a(\w)^T\right) \h \right)\cdot\nabla  \adja \\
&+ D_b \rb \left(\left(\nabla_{\w} F_b(\w)^T\right) \h \right)\cdot \nabla \adjb  \bigg)dr dt,
\end{align*}
since $\adja = \adjaB$ and $\adjb = \adjbB$ from the adjoint derivation.\\
Now we use the relation $((\nabla \mathbf a)^T)\mathbf b) \cdot \mathbf c= (( \mathbf c \cdot \nabla) \mathbf a ) \cdot \mathbf b$ (from year end review) to find that:
\begin{align*}
\mathcal{L}_{\w}(\ra,\rb, \w, \adja, \adjb) \h  &= \int_0^T \int_\Omega \bigg( \beta \w \cdot \h + D_a \ra \left( \left(\nabla_r \adja \cdot \nabla_{\w} \right) F_a(\w) \right) \cdot \h         \\
&+ D_b \rb \left( \left(\nabla_r \adjb \cdot \nabla_{\w} \right) F_b(\w) \right) \cdot \h       \bigg)dr dt,
\end{align*}
Setting this to zero and since this holds for all permissible $\h$, we get:
\begin{align*}
\beta \w  + D_a \ra \left( \left(\nabla_r \adja \cdot \nabla_{\w} \right) F_a(\w) \right) 
+ D_b \rb \left( \left(\nabla_r \adjb \cdot \nabla_{\w} \right) F_b(\w) \right) = 0.
\end{align*} 
Using that $\nabla \cdot (\mathbf{b a}^T) = \mathbf a (\nabla \cdot \mathbf b) + (\mathbf b \cdot \nabla) \mathbf a$, and observing that $\nabla_{\w} \cdot (\nabla_r q) = 0$, we get:
\begin{align*}
\beta \w  + D_a \ra \nabla_{\w} \cdot \left(\nabla \adja F_a(\w)^T \right) 
+ D_b \rb \nabla_{\w} \cdot \left(\nabla \adjb F_b(\w)^T \right) = 0.
\end{align*} 
Since $\nabla_r q$ does not depend on $\w$ we can rearrange this to get:
\begin{align*}
\beta \w  + D_a \ra \left(\nabla_\w F_a(\w)\right)^T \nabla \adja  
+ D_b \rb \left(\nabla_\w F_b(\w)\right)^T \nabla \adjb = 0.
\end{align*}
And finally we have:
\begin{align*}
\w = - \frac{1}{\beta} \left( D_a \ra \left(\nabla_\w F_a(\w)\right)^T \nabla \adja  
+ D_b \rb \left(\nabla_\w F_b(\w)\right)^T \nabla \adjb \right).
\end{align*}
As an example, take $F_a(\w) = c_a \w$ and $F_b(\w) = c_b \w$. We get:
\begin{align*}
\w  = - \frac{1}{\beta}\bigg( D_a  \ra c_a \mathbf 1 \nabla \adja + D_b \rb c_b \mathbf 1 \nabla \adjb \bigg).
\end{align*}









	
\section{Sedimentation}	

\subsection{Free Energy Frech\'et Derivative}
We have the general expression: 
\begin{align*}
\nabla \cdot \bigg(\rho \nabla \frac{\delta F[\rho]}{\delta \rho}\bigg) &= \frac{1}{\beta} \bigg( \nabla  \cdot \left(  \frac{\nabla \rho}{1 - \eta} \right) - \nabla  \cdot \left(\rho \nabla\frac{\eta - 2}{(\eta - 1)^2} \right) \bigg)\\
&= \frac{1}{\beta} \bigg( \frac{\nabla^2 \rho}{1 - \eta} +  \nabla \rho \cdot \nabla \frac{1}{1 - \eta} - \nabla \rho \cdot \nabla \frac{\eta - 2}{(\eta - 1)^2} - \rho \nabla^2\frac{\eta - 2}{(\eta - 1)^2} \bigg)\\
&= \frac{1}{\beta} \bigg( \frac{\nabla^2 \rho}{1 - \eta} +  \nabla \rho \cdot \nabla \frac{(3- 2 \eta)}{(1 - \eta)^2}  - \rho \nabla^2\frac{\eta - 2}{(\eta - 1)^2} \bigg)
\end{align*}
We want to take the Frech\'et derivative of these terms.
We set:
\begin{align*}
F_1 &= \frac{\nabla^2 \rho}{1 - \eta}= \frac{\nabla^2 \rho}{1 - a \rho}\\
F_2 &= \nabla \rho \cdot \nabla \frac{(3- 2 \eta)}{(1 - \eta)^2} = \nabla \rho \cdot \nabla \frac{(3- 2 a \rho)}{(1 - a \rho)^2} \\
F_3 &= \rho \nabla^2\frac{\eta - 2}{(\eta - 1)^2}= \rho \nabla^2\frac{a \rho - 2}{(a \rho - 1)^2}.
\end{align*}
We are looking at $F(\rho + h) - F(\rho)$. We use the expansions:
\begin{align*}
	\frac{1}{1-x} &= 1 + x + O(x^2)\\
	\frac{1}{(1-x)^2} & = 1 + 2x + O(x^2).
\end{align*}

For $F_1$ we get:
\begin{align*}
F_1(\rho+ h) - F_1(\rho) &= \frac{\nabla^2 (\rho + h)}{1 - a (\rho + h)} - \frac{\nabla^2 \rho}{1 - a \rho}\\
&= \nabla^2 (\rho + h) ( 1 + a(\rho + h)) - \nabla^2 \rho (1 + a \rho)\\
&= (\nabla^2 \rho) ( 1 + a \rho + a h - 1 - a \rho ) + (\nabla^2 h )(1 + a \rho + a h)\\
&= (\nabla^2 \rho) ( a h ) + (\nabla^2 h )(1 + a \rho)\\
\end{align*}
For $F_2$ we have:
\begin{align*}
F_{2}(\rho + h) - F_2(\rho) &= \nabla (\rho + h) \cdot \nabla \frac{(3- 2 a (\rho+h))}{(1 - a (\rho+h))^2} - \nabla \rho \cdot \nabla \frac{(3- 2 a \rho)}{(1 - a \rho)^2}\\
&= \nabla (\rho + h) \cdot \nabla \left( (3- 2 a (\rho+h)) (1 + 2a (\rho+ h))\right) - \nabla \rho \cdot \nabla \left( (3- 2 a \rho) (1 + 2a \rho) \right)\\
&= \nabla (\rho + h) \cdot \nabla \left( 3  + 6a (\rho + h) - 2a (\rho + h) - 4 a^2 (\rho+ h)^2\right) \\
&- \nabla \rho \cdot \nabla \left(3  + 6a \rho - 2a \rho - 4 a^2 \rho^2\right)\\
& = \nabla \rho \cdot \nabla \left(  3  + 4a (\rho + h)  - 4 a^2 (\rho+ h)^2 -  \left(3  + 4a \rho  - 4 a^2 \rho^2\right)           \right)\\
&+ \nabla h \cdot \nabla \left( 3  + 6a (\rho + h) - 2a (\rho + h) - 4 a^2 (\rho+ h)^2\right)\\
& = \nabla \rho \cdot \nabla \left( 4a h  - 8a^2 \rho h \right) + \nabla h \cdot \nabla \left( 3  + 6a \rho  - 2a \rho - 4 a^2\rho^2 \right)\\
&= \nabla \rho \cdot \nabla \left( (4a   - 8a^2 \rho) h \right) + \nabla h \cdot \nabla \left( 4a \rho - 4 a^2\rho^2 \right)\\
&= \nabla \rho \cdot h \nabla \left(4a   - 8a^2 \rho\right) + \nabla \rho \cdot (4a - 8a^2 \rho) \nabla h + \nabla h \cdot \nabla \left( 4a \rho - 4 a^2\rho^2 \right)\\
&= - 8a^2 h \nabla \rho \cdot \nabla \rho  + \nabla h \cdot \left(\nabla \rho (4a - 8a^2 \rho)           + \nabla \left( 4a \rho - 4 a^2\rho^2 \right) \right)\\
&= - 8a^2 h \left(\nabla \rho\right)^2   + \nabla h \cdot \left( 8a \nabla \rho - 16 a^2 \rho \nabla \rho  \right)
\end{align*}
Finally $F_3$ is:
\begin{align*}
F_3(\rho + h) - F_3 (\rho) &= (\rho + h) \nabla^2 \left(\frac{a (\rho + h) - 2}{(a (\rho + h) - 1)^2}\right) - \rho \nabla^2 \left(\frac{a \rho - 2}{(a \rho - 1)^2}\right)\\
&= (\rho + h) \nabla^2 \left( (a (\rho + h) - 2) ( 1 + 2a(\rho +h)) \right) - \rho \nabla^2 \left((a \rho - 2) (1 + 2a \rho) \right)\\
&= (\rho + h) \nabla^2 \left( -2 -3a(\rho + h) + 2a^2 (\rho + h)^2 \right) - \rho \nabla^2 \left(-2 -3a \rho + 2a^2 \rho^2\right)\\
&= \rho \nabla^2 \left( - 3ah + 4a^2 \rho h  \right) + h \nabla^2 \left(-3a \rho  + 2a^2 \rho^2 \right)\\
&= - 3a \rho \nabla ^2 h + 4a^2 \rho \nabla^2(\rho h) - 3ah \nabla^2 \rho + 2a^2 h \nabla^2 \rho^2
\end{align*}


\subsection{Lagrangian}
We consider the part of the Lagrangian that is relevant:
\begin{align*}
\mathcal{L}(\rho, \w, q) = - \int_0^T \int_\Omega \left(\frac{1}{\beta} \bigg( \frac{\nabla^2 \rho}{1 - \eta} +  \nabla \rho \cdot \nabla \frac{(3- 2 \eta)}{(1 - \eta)^2}  - \rho \nabla^2\frac{\eta - 2}{(\eta - 1)^2} \bigg) q \right) dr dt
\end{align*}
Taking the derivatives with respect to $\rho$ gives:
\begin{align*}
\mathcal{L}_\rho(\rho, \w, q) h &= -\frac{1}{\beta}  \int_0^T \int_\Omega \bigg( (\nabla^2 \rho) ( a h ) + (\nabla^2 h )(1 + a \rho) - 8a^2 h \left(\nabla \rho\right)^2   + \nabla h \cdot \left( 8a \nabla \rho - 16 a^2 \rho \nabla \rho  \right) \\
&+a \rho \nabla ^2 h - 2a^2 \rho \nabla^2(\rho h) + ah \nabla^2 \rho - a^2 h \nabla^2 \rho^2 \bigg) q dr dt
\end{align*}
Integrate by parts the term involving $\nabla^2 (\rho h)$:
\begin{align*}
\int_0^T \int_\Omega q \rho \nabla^2(\rho h) dr dt &= \int_0^T \int_{\partial \Omega} q\rho \nabla(\rho h)\cdot \n dr dt - \int_0^T \int_\Omega \nabla (q \rho) \cdot \nabla (\rho h) dr dt\\
&= \int_0^T \int_{\partial \Omega} q \rho \left(\rho \nabla h + h \nabla \rho \right) \cdot \n dr dt - \int_0^T \int_{\partial \Omega} \rho h \nabla (q\rho) \cdot \n dr dt +
\int_0^T \int_\Omega \rho h \nabla^2 (q\rho)  dr dt\\
&= \int_0^T \int_{\partial \Omega} \left(q \rho^2 \nabla h + q \rho h \nabla \rho  -  \rho^2 h \nabla q - q \rho h \nabla \rho \right)\cdot \n dr dt +
\int_0^T \int_\Omega \rho h \nabla^2 (q\rho)  dr dt\\
&= \int_0^T \int_{\partial \Omega} \left(q \rho^2 \nabla h   -  \rho^2 h \nabla q  \right)\cdot \n dr dt +
\int_0^T \int_\Omega \rho h \nabla^2 (q\rho)  dr dt\\
\end{align*}
Then we have the terms involving $\nabla^2 h$:
\begin{align*}
\int_0^T \int_\Omega  (\nabla^2 h) (q + 2 a q \rho )  dr dt &= \int_0^T \int_{\partial \Omega}(\nabla h) (q + 2 a q \rho ) \cdot \n dr dt - \int_0^T \int_\Omega (\nabla h) \cdot \nabla (q + 2 a q \rho )dr dt\\
&= \int_0^T \int_{\partial \Omega} \left((\nabla h) (q + 2 a q \rho ) -  h \nabla q -2 ah \nabla (q \rho ) \right) \cdot \n  dr dt \\
&+ \int_0^T \int_\Omega  h\nabla^2 q + 2 a h \nabla^2(q \rho ) dr dt\\
&= \int_0^T \int_{\partial \Omega} \left((\nabla h) (q + 2 a q \rho ) -  h \nabla q -2 ah \nabla (q \rho ) \right) \cdot \n  dr dt \\
&+ \int_0^T \int_\Omega  h\nabla^2 q + 2 a h q \nabla^2 \rho + 2 a h  \rho \nabla^2 q + 2 a h \nabla \rho \cdot \nabla q dr dt
\end{align*}
Finally, the terms involving $\nabla h$:
\begin{align*}
\int_0^T \int_\Omega \nabla h \cdot \left( 8a q\nabla \rho - 16 a^2 q \rho \nabla \rho  \right) dr dt &= \int_0^T \int_{\partial \Omega}  h \left( 8a q \nabla \rho - 16 a^2 q\rho \nabla \rho  \right) \cdot \n dr dt \\
&- \int_0^T \int_\Omega h \nabla \cdot\left( 8a q \nabla \rho - 16 a^2 q\rho \nabla \rho  \right) dr dt\\
& = \int_0^T \int_{\partial \Omega}  h \left( 8a q\nabla \rho - 16 a^2 q \rho \nabla \rho  \right) \cdot \n dr dt \\
&- \int_0^T \int_\Omega h \left( 8a \nabla \cdot( q \nabla \rho) - 16 a^2 \nabla \cdot (q\rho \nabla \rho ) \right) dr dt\\
& = \int_0^T \int_{\partial \Omega}  h \left( 8a q\nabla \rho - 16 a^2 q \rho \nabla \rho  \right) \cdot \n dr dt \\
&- \int_0^T \int_\Omega h \bigg( 8a \nabla q \cdot \nabla \rho + 8aq \nabla^2 \rho  \\
&- 16 a^2 q (\nabla \rho)^2 - 16 a^2 \rho \nabla \rho \cdot \nabla q - 16 a^2 q \rho \nabla^2 \rho \bigg) dr dt\\
\end{align*}
Combining all of these gives:
\begin{align*}
\mathcal{L}_\rho(\rho, \w, q) h &= -\frac{1}{\beta}  \int_0^T \int_\Omega \bigg( (\nabla^2 \rho) ( aq h ) + h\nabla^2 q + 2 a h q \nabla^2 \rho + 2 a h  \rho \nabla^2 q + 2 a h \nabla \rho \cdot \nabla q
 - 8a^2 hq \left(\nabla \rho\right)^2   \\
&- h \bigg( 8a \nabla q \cdot \nabla \rho + 8aq \nabla^2 \rho  
- 16 a^2 q (\nabla \rho)^2 - 16 a^2 \rho \nabla \rho \cdot \nabla q - 16 a^2 q \rho \nabla^2 \rho \bigg)\\
& - 2a^2 \rho h \nabla^2 (q\rho)  + qah \nabla^2 \rho - qa^2 h \nabla^2 \rho^2 \bigg)  dr dt
\end{align*}
Rearranging and cancelling results in:

\begin{align*}
\mathcal{L}_\rho(\rho, \w, q) h &= -\frac{1}{\beta}  \int_0^T \int_\Omega q \bigg(a \nabla^2 \rho + 2 a   \nabla^2 \rho - 8a^2  \left(\nabla \rho\right)^2  - 8a \nabla^2 \rho\\
& + 16 a^2  (\nabla \rho)^2 + 16a^2 \rho \nabla^2 \rho + a \nabla^2 \rho - a^2  \nabla^2 \rho^2  - 2a^2 \rho  \nabla^2 \rho \bigg)\\
&+ \nabla q \cdot \bigg( 2 a  \nabla \rho -  8a  \nabla \rho  + 16 a^2 \rho \nabla \rho  - 4a^2 \rho  \nabla \rho \bigg)\\
&+ \nabla^2 q \bigg(1 + 2 a   \rho   - 2a^2 \rho^2        \bigg) dr dt
\end{align*}
\begin{align*}
\mathcal{L}_\rho(\rho, \w, q) h &= -\frac{1}{\beta}  \int_0^T \int_\Omega q \bigg( 6a^2  \left(\nabla \rho\right)^2  - 4a \nabla^2 \rho + 13a^2 \rho \nabla^2 \rho \bigg)\\
&+ \nabla q \cdot \bigg( -  6a  \nabla \rho  + 12 a^2 \rho \nabla \rho  \bigg)+ \nabla^2 q \bigg(1 + 2 a   \rho   - 2a^2 \rho^2        \bigg) dr dt
\end{align*}














\end{document}

