
\documentclass[11pt, a4paper]{article}
%\usepackage{proj1}
\usepackage{natbib}
\usepackage{fancyhdr}  
\usepackage{subcaption}
\usepackage{caption}
\usepackage{graphicx}
\usepackage{numprint}
\usepackage{multirow}
\linespread{1.25} 
\setlength{\parindent}{0cm}
\graphicspath{{Images/}}
\usepackage{hyperref}
\usepackage{amsmath}
\usepackage{amsfonts}
\usepackage{amssymb}
\usepackage{amsthm}
\usepackage{mathtools}
\usepackage{commath}
\usepackage{bbm}

%\usepackage[sc,osf]{mathpazo}
\usepackage{subcaption}
\usepackage[a4paper, top=1in, left=1.0in, right=1.0in, bottom=1in, includehead, includefoot]{geometry} %Usually have top as 1in

\usepackage{listings}
\usepackage{color} %red, green, blue, yellow, cyan, magenta, black, white
\definecolor{mygreen}{RGB}{28,172,0} % color values Red, Green, Blue
\definecolor{mylilas}{RGB}{170,55,241}


\hypersetup{colorlinks,linkcolor={black},citecolor={blue},urlcolor={black}}
\usepackage{color}
\urlstyle{same}


\theoremstyle{definition}
\newtheorem{definition}{Definition}[section]

\newcommand{\adja}{q_a}
\newcommand{\adjb}{q_b}
\newcommand{\adjaB}{q_{a,\partial \Omega}}
\newcommand{\adjbB}{q_{b,\partial \Omega}}
\newcommand{\adjB}{q_{\partial \Omega}}
\newcommand{\Adja}{\mathbf{p}}
\newcommand{\Adjb}{q}
\newcommand{\adj}{q}
\newcommand{\Adjc}{{q}_{\partial \Omega}}
\newcommand{\ra}{\rho_a}
\newcommand{\rb}{\rho_b}
\newcommand{\w}{\mathbf{w}}
\newcommand{\x}{\mathbf{x}}
\newcommand{\f}{\mathbf{f}}
\newcommand{\ve}{\mathbf{v}}
\newcommand{\n}{\mathbf{n}}
\newcommand{\h}{\mathbf{h}}
\newcommand{\K}{\mathbf{K}}
\newcommand{\hr}{\widehat \rho}
\newcommand{\jf}{\mathbf j}

\DeclareMathOperator{\sgn}{sgn}
\DeclareMathOperator{\Grad}{Grad}
\DeclareMathOperator{\Div}{Div}
\DeclareMathOperator{\Lap}{Lap}
%	\begin{figure}[h]
%		\centering
%		\includegraphics[scale=0.35]{F1.png}
%		\caption{Forward $\rho$ for $a = 0.01$} 
%		\label{F1}
%	\end{figure}

\begin{document}
++ Typo in Volume?: Interp 2D ++
++ go through Control Plotting Function ++ streamlines not working ++
	
	\section*{Paper Example 3D}
	We choose 
	\begin{align*}
		\rho_0 &= 0.125\\
		V_{ext} &= ((x_1 + 0.3)^2 - 1)((x_1 - 0.4)^2 - 0.5)\\
		&\quad ((x_2 + 0.3)^2 - 1)((x_2 - 0.4)^2 - 0.5)((x_3 + 0.3)^2 - 1)((x_3 - 0.4)^2 - 0.5)\\
		\hr &= 0.125(1-t) + t\left(\frac{\pi}{4}\right)^3\cos\left(\frac{\pi x_1}{2}\right)\cos\left(\frac{\pi x_2}{2}\right)\cos\left(\frac{\pi x_3}{2}\right)
	\end{align*}
	The external potential is shown in Figure \ref{F1}.
	\begin{figure}[h]
		\centering
		\includegraphics[scale=0.7]{Vext3D.png}
		\caption{External Potential $V_{ext}$} 
		\label{F1}
	\end{figure}
	
	For $N = 20$ and $n = 11$, with $\beta = 10^{-3}$ we get for $\kappa = 0$, $\mathcal J_c = 0.0078$, with $\mathcal J_1 = 0.0071$ and $\mathcal J_2 = 8.5034$.
	This can be compared to $\mathcal J_{uc} = 0.0195$ from the computed forward problem with $\w = \vec 0$.\\
	For $\kappa = 1$, we get that $\mathcal J_c = 0.0102$, with $\mathcal J_1 = 0.0097$, $\mathcal J_2 = 10.7306$. Compare to $\mathcal J_{uc} = 0.0232$. \\
	For $\kappa = -1$ we have $\mathcal J_c = 0.0059$, $\mathcal J_1 = 0.0054$, $\mathcal J_2 = 6.4039$. Compare to $\mathcal J_{uc} = 0.0477$.
	While the forward problem takes around $12$ minutes to solve, the optimal control problem with Newton-Krylov takes about 35 hours for 10 outer iterations, which is enough for convergence. Mass is conserved to $10^{-4}$. The results can be seen in Figures \ref{F2}, \ref{F3} and \ref{F4}. 
	\\
	\\
	The controls are plotted in Figures \ref{F5}, \ref{F6} and \ref{F7}. They are all normalized to the maximum over all three controls and scaled by a factor of $2$ for visibility. The figures are still not good though. I am not sure how the scaling works -- $\kappa = 0$ should have way smaller arrows by that logic.
	\\
	\\
	The norm of the control  (in Euclidean norm) can be seen in Figures \ref{F4a}, \ref{F4b} and \ref{F4c}. Between these there is not much of a qualitative difference. However, the control for $\kappa = 0$ is of order $10^{-4}$, while the control for $\kappa = 1$ and $\kappa = -1$ is of order $1$.
	
	
	
	\begin{figure}[h]
		\centering
		\includegraphics[scale=0.35]{rho3Dk1n.png}
		\caption{Optimal state $\rho$ for $\kappa = 1$.} 
		\label{F2}
	\end{figure}
	\begin{figure}[h]
		\centering
		\includegraphics[scale=0.35]{rho3Dk0n.png}
		\caption{Optimal state $\rho$ for $\kappa = 0$.} 
		\label{F3}
	\end{figure}
	\begin{figure}[h]
		\centering
		\includegraphics[scale=0.35]{rho3Dkn1n.png}
		\caption{Optimal state $\rho$ for $\kappa = -1$.} 
		\label{F4}
	\end{figure}
	
	\begin{figure}[h]
		\centering
		\includegraphics[scale=0.35]{Controlk1.png}
		\caption{Optimal control $\w$ for $\kappa = 1$.} 
		\label{F5}
	\end{figure}
	\begin{figure}[h]
		\centering
		\includegraphics[scale=0.35]{Controlk0.png}
		\caption{Optimal control $\w$ for $\kappa = 0$.} 
		\label{F6}
	\end{figure}
	\begin{figure}[h]
		\centering
		\includegraphics[scale=0.35]{Controlkn1.png}
		\caption{Optimal control $\w$ for $\kappa = -1$.} 
		\label{F7}
	\end{figure}
	
	\begin{figure}[h]
		\centering
		\includegraphics[scale=0.35]{wNormk1.png}
		\caption{Norm of the optimal control $\w$, at times 1, 3, 5, 7 and 9 for $\kappa = 1$.} 
		\label{F4a}
	\end{figure}
	
	\begin{figure}[h]
		\centering
		\includegraphics[scale=0.35]{wNormk0.png}
		\caption{Norm of the optimal control $\w$, at times 1, 3, 5, 7 and 9 for $\kappa = 0$.} 
		\label{F4b}
	\end{figure}
	\begin{figure}[h]
		\centering
		\includegraphics[scale=0.35]{wNormkn1.png}
		\caption{Norm of the optimal control $\w$, at times 1, 3, 5, 7 and 9 for $\kappa = -1$.} 
		\label{F4c}
	\end{figure}
	
	
\end{document}