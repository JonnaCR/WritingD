
The multiple shooting algorithm, introduced in the first year report, has been extended by employing a Picard mixing scheme to replace the {\scshape MATLAB} inbuilt solver \texttt{fsolve}. In the following, this is briefly outlined.
The idea of the updating scheme is similar to the one presented for the fixed point algorithm. However, while the fixed point algorithm updates through the control variable, the fixed point algorithm updates via the variables $\rho$ and $q$.
The multiple shooting method consists of discretizing the time interval and solving the optimality system on each interval individually. This is done because of the non-local time coupling of the forward and adjoint equations. It requires the input of an initial guess at each discretized time point for each of the variables. The aim of the optimization solver is then to minimize the distance between the initial guesses and numerical solutions of the variables at each of the time points. \\
The Picard mixing scheme is a fixed point type algorithm. At each iteration $i$ it takes a set of guesses at the discretized time points, denoted by $Y_i$. The matrix $Y = [P,Q]$ contains the discretized values for the variables $\rho$ and $q$, denoted by $P$ and $Q$, analogously to the previous section.  
The system of PDEs is solved on each of the discretized intervals and a new set of variable values at the time points is created, denoted by $Y_{out}$. Then, the mixing scheme provides a new guess for the iteration $i+1$:
\begin{align*}
Y_{i+1} = (1 - \lambda)Y_i + \lambda Y_{out},
\end{align*}
where $\lambda$ is the mixing rate. It typically takes values between $0.1$ and $0.01$, depending on the complexity of the system to solve. Choosing a relatively small value of $\lambda$ stabilizes the algorithm. 
The algorithm terminates when the system of PDEs is solved self-consistently, i.e. when the distance between $Y_i$ and $Y_{out}$ is small, as measured in a chosen norm. The most frequently applied norm is discussed in Section \ref{sec:ErrorMeasure}.
This algorithm is working very well for examples involving the overdamped equations. However, the fixed point algorithm provides an even simpler method, which does not require the solution of the optimality system on small time intervals and is therefore even quicker. Since we will apply the numerical optimization method to increasingly difficult optimal control problems in the future, the multiple shooting algorithm may provide more numerical stability for numerically harder problems and is therefore a relevant tool to consider in the future. Changing the optimization solver in the implementation is straightforward and only requires changing a flag in the input file.
\\
A challenge with this solver is, that it needs to be provided with good initial guesses for the variables $\rho$ and $\adj$ at the discretized time points. The guess for $\rho$ can be obtained by solving the associated forward problem and using the result as a first guess. However, a good guess for $\adj$ is trickier to obtain. One way of doing so is by using the gradient equation, which relates $\rho$, $\adj$ and $\mathbf w$, the control. Since the input for the forward control is known, one can use this information, together with the initial guess for $\rho$, to construct an initial guess for $\adj$. 
One challenge however arises when considering the flow control problem involving the overdamped equations. The gradient equation is $\mathbf{w} = - \frac{1}{\beta} \rho\nabla \adj$. In order to derive the value of $\adj$ from this equation, we need to divide by $\rho$, making use of the assumption that $\rho$ is strictly positive, and integrate over the whole space. The issue here is that integration introduces an indeterminable constant. Furthermore, if Dirichlet boundary conditions are applied, the strict positivity of $\rho$ is in question.\\
An alternative method of obtaining an initial guess for $\adj$ is to perform one step of the fixed point method.
