\documentclass[11pt, a4paper]{article}
%\usepackage{proj1}
\usepackage{natbib}
\usepackage{fancyhdr}  
\usepackage{subcaption}
\usepackage{caption}
\usepackage{graphicx}
\usepackage{numprint}
\usepackage{multirow}
\linespread{1.25} 
\setlength{\parindent}{0cm}
\graphicspath{{Images/}}
\usepackage{hyperref}
\usepackage{amsmath}
\usepackage{amsfonts}
\usepackage{amssymb}
\usepackage{amsthm}
\usepackage{mathtools}
\usepackage{commath}
\usepackage{bbm}

%\usepackage[sc,osf]{mathpazo}
\usepackage{subcaption}
\usepackage[a4paper, top=1in, left=1.0in, right=1.0in, bottom=1in, includehead, includefoot]{geometry} %Usually have top as 1in

\usepackage{listings}
\usepackage{color} %red, green, blue, yellow, cyan, magenta, black, white
\definecolor{mygreen}{RGB}{28,172,0} % color values Red, Green, Blue
\definecolor{mylilas}{RGB}{170,55,241}


\hypersetup{colorlinks,linkcolor={black},citecolor={blue},urlcolor={black}}
\usepackage{color}
\urlstyle{same}


\theoremstyle{definition}
\newtheorem{definition}{Definition}[section]

\newcommand{\adja}{q_a}
\newcommand{\adjb}{q_b}
\newcommand{\adjaB}{q_{a,\partial \Omega}}
\newcommand{\adjbB}{q_{b,\partial \Omega}}
\newcommand{\adjB}{q_{\partial \Omega}}
\newcommand{\Adja}{\mathbf{p}}
\newcommand{\Adjb}{q}
\newcommand{\adj}{q}
\newcommand{\Adjc}{{q}_{\partial \Omega}}
\newcommand{\ra}{\rho_a}
\newcommand{\rb}{\rho_b}
\newcommand{\w}{\mathbf{w}}
\newcommand{\f}{\mathbf{f}}
\newcommand{\ve}{\mathbf{v}}
\newcommand{\n}{\mathbf{n}}
\newcommand{\h}{\mathbf{h}}
\newcommand{\K}{\mathbf{K}}
\newcommand{\hr}{\widehat \rho}

\pagenumbering{gobble}
\begin{document}
\section*{Thesis Draft}	
	
	
\section*{Year 1}	
\section{Introduction}
PDE-constrained optimization and multiscale particle dynamics are both fields of growing interest to academia and industry. Applying methods of PDE-constrained optimization to industrial processes is a highly relevant topic, for example in the oil and gas industry \cite{Brandman2018}, the beer industry \cite{RamirezW.F.2007Obf} and the wine industry \cite{MergerJuri2017Ocoa}. 
There are two industrial partners affiliated with this project. WEST brewery is interested in optimizing the yeast sedimentation in the beer brewing process. The company ufraction8 works on cell separation and nano filtration devices, which separate particles based on their sizes. They are interested in optimizing this process.
\\
\\
Many processes, including these two examples, can be described as interacting particle systems, using Density Functional Theory (DFT). Further examples include processes in biology and nanotechnology \cite{FrinkDFT}, as well as in chemical engineering \cite{WuJianzhong2006Dftf}.
Therefore, developing a numerical framework for PDE-constrained optimization problems, where the PDE constraint describes particle dynamics, is highly relevant. This task is challenging, because the PDEs involved are non-local, nonlinear integro-PDEs. This makes the application of standard methods, such as finite element methods, hard. Pseudospectral methods can be used to tackle the numerical challenges, such as non-local boundary conditions and dense matrices in the discretized problem.
In this report, steps towards developing this numerical method are taken by deriving a method for one- and two-dimensional test problems, which can be extended to the full problem, including the PDE describing the particle dynamics.
\\
\\
The report is structured as follows. In the next section, the PDE for the particle dynamics is derived from the Smoluchowski equation. In the third section, the PDE-constrained optimization framework is set up and first-order optimality conditions are derived for two different test problems, one diffusion type problem and one problem including an integral particle interaction term.
In the fourth section, some exact solutions and the numerical methods are introduced, such as pseudospectral methods and the multiple shooting method.
In the fifth section, some numerical experiments are presented for both introduced test problems in one and two dimensions.
In the final section, some conclusions are drawn and opportunities for future work are pointed out.
\section{Multiscale Particle Dynamics}


The aim of this section is to model the dynamics of particles, which are suspended in a bath. The particles evolve in time and in some domain $\Omega$ in space. Instead of modelling the movement of a particular particle, a space coordinate $r \in \Omega$ is fixed. The probability of a particle being at the position $r$ at time $t$ is modelled as a  distribution $\rho(r,t)$, the one-body particle density.
The model equations are:
\begin{align}\label{sysParticleModel1}
\partial_t \rho(r,t) &= \nabla \cdot \bigg(\big(\nabla  + \nabla V_{ext} + \int_\Omega \rho(r') \nabla V_2(|r-r'|)dr' - \mathbf{w} \big) \rho(r,t) \bigg),\\
&:=\nabla \cdot \mathbf{j},\notag \\
\notag  \\
\mathbf{j} \cdot \mathbf{n}&=0 \quad  \ \text{on   } \quad  \partial \Omega, \notag  \\
\rho(0) &=\rho_0 \quad \text{at} \quad t=t_0, \notag 
\end{align}
where $\mathbf{j}=\big(\nabla  + \nabla V_{ext} + \int_\Omega \rho(r') \nabla V_2(|r-r'|)dr' - \mathbf{w} \big) \rho(r,t) $ and $\mathbf{n}$ denotes the outward normal to $\partial \Omega$, compare to \cite{RexLoewen1}.
 The equation describes the particle dynamics, including a time derivative, a diffusion term, and an external potential $V_{ext}$, whose negative gradient is a body force such as gravity. Furthermore, there is a term describing the background flow field of the bath, $\mathbf{w}$, and a particle interaction term.\\
 This particle interaction term describes the interactions of two particles at the position $r$ and some other position $r' \in \Omega$, where $r' \neq r$. The forces between the particles at $r$ and  $r'$ are represented by $-\nabla V_2(|r-r'|)$. This is multiplied by the particle density at $r'$, $\rho(r')$, and the particle density at $r$, $\rho(r)$. Since each position $r' \in \Omega$ needs to be considered, the resulting expression is integrated over all $r'$.\\
The particle interaction in (\ref{sysParticleModel1}) is restricted to two-body interactions and can be extended to three- or more body interaction terms, which makes the model equations more accurate and complex. The inclusion of higher-order terms is dependent on the application. Furthermore, the forces between the particles included here only describe interactions such as attraction and repulsion. The model neglects hydrodynamic interactions, which are the effects of the particles moving in the bath. This movement causes a change in the flow field, which in turn influences the surrounding particles. This is a non-local phenomenon, which is more complex to model, however, significant to include in various applications. 

\subsubsection{Deriving Model Equations}
The model equation (\ref{sysParticleModel1}) can be derived from the full $N$-body particle distribution. This has been done by M. Rex and H. Loewen \cite{RexLoewen1}, whose derivation is followed in this section.	
On the microscopic level, a probability distribution $P(r^N,t)$ of the total number $N$ particles in the bath at time $t$ is considered, where $r^N= (r_1, r_2,...,r_N)$, and $r_i \in \mathbf{R}^3$, $i=1,2,...,N$. 
The dynamics of the probability distribution $P(r^N,t)$ can be described by the Smoluchowski equation, as presented in \cite{RexLoewen1}:
\begin{align} \label{eqnPpde}
\partial_t P(r^N,t)= \sum_{i,j}^N \nabla_i \cdot \bigg( D_{i,j}(r^N) \bigg( \nabla_j + \frac{\nabla_j U(r^N,t)}{k_BT} - \mathbf{w}(r_i) \bigg) P(r^N,t) \bigg),
\end{align}
where $\nabla_i$ refers to the operation with respect to the coordinate $r_i$. The term $D_{i,j}(r^N)$ is the diffusion tensor, which describes the hydrodynamic interactions of the particles at $r_i$ and $r_j$, $U(r^N,t)$ is a potential, which could include an external potential and particle interactions, $T$ is the temperature, and $k_B$ is the Boltzmann constant. A background flow is defined by $\mathbf{w} \in \mathbf{R}^3$, describing the flow of the bath fluid.
 Note that this version of the equation is more general than the representation in the paper, due to the extra term $\mathbf{w}$. The aim is to derive a three dimensional approximation $\rho^{(1)}(r_1,t)$ to (\ref{eqnPpde}). The following $n$-body densities are defined as in \cite{RexLoewen1}:
\begin{align*}
\rho^{(1)}(r_1,t) &= N \int_\Omega dr_2.... dr_N P(r^N,t):= \rho(r,t),\\
\vdots\\
\rho^{(n)}(r_n,t) &= \frac{N!}{(N-i)!} \int_\Omega dr_{n+1}.... dr_N P(r^N,t),
\end{align*} 
where $\Omega=\mathbf{R}^{3N}$.
The $n$-body densities are derived by integrating the full $N$ particle probability distribution over $r_{n+1},...r_N$, and multiplying it by a prefactor. This definition is chosen to suit the computations, which are detailed below.
In order to derive a three-dimensional approximation in terms of the one-body density $\rho(r,t)$, initially a simplification of (\ref{eqnPpde}) is considered and then extended in later sections to derive the approximation for the full system. 

\subsubsection*{The Diffusion Equation}
Considering $D_{i,j}=\delta_{i,j}$, $U=0$ and $\mathbf{w}=0$ in (\ref{eqnPpde}), the diffusion equation is recovered:
\begin{align*} 
\partial_t P(r^N,t) &= \sum_{i}^N \nabla_i \cdot \bigg(\nabla_i   P(r^N,t) \bigg)= \sum_{i}^N \Delta_i P(r^N,t)\\
&= \Delta_{r^N} P(r^N,t),
\end{align*}
where $\sum_{i}^N \Delta_i := \Delta_{r^N}$.

In order to derive the one-body density $\rho(r,t)$ for the diffusion equation, the equation is multiplied by $N$ and integrated over all other positions $r_2,...,r_N$:
\begin{align*} 
\int_\Omega N \partial_t P(r^N,t)dr_2...dr_N &= \int_\Omega N \sum_{i}^N \nabla_i \cdot \bigg(\nabla_i   P(r^N,t) \bigg)dr_2...dr_N.
\end{align*}
The integration is only dependent on space, not on time, so that the time derivative can be taken out of the integral. Furthermore, the sum on the right-hand side of the equation, as well as the integration, is finite. Therefore, Fubini's Theorem can be used to take the sum out of the integral. The equation is then:
\begin{align*} 
N \partial_t \int_\Omega P(r^N,t)dr_2...dr_N &=N \sum_{i}^N \int_\Omega  \nabla_i \cdot \bigg(\nabla_i   P(r^N,t) \bigg)dr_2...dr_N.
\end{align*}
The Divergence Theorem can be applied to $i=2,...,N$, while for $i=1$ the integral remains unchanged, since the integration is independent of $r_1$. The equation is now:
\begin{align*} 
N \partial_t \int_\Omega P(r^N,t)dr_2...dr_N &=N \sum_{i=2}^N \int_{\partial \Omega} \frac{\partial P(r^N,t)}{\partial n} dr_2...dr_N + N\int_\Omega \nabla_1 \cdot \bigg(\nabla_1   P(r^N,t) \bigg)dr_2...dr_N,\\
\end{align*}
where $n$ is the outward normal. 
Now, the boundary condition for $P(r^N,t)$ can be employed. Since $P$ is a probability distribution over a finite number of positions $r_i$, on an infinite domain $\Omega= \mathbf{R}^{3N}$, the natural boundary condition is that $P$ and its derivatives vanish on the boundary $\partial \Omega$, i.e. at infinity.
Furthermore, considering the fact that $\nabla_1$ is constant with respect to the integration variables, it can be taken out of the integral and the following result is found, where the definition $\rho(r,t)= N \int_\Omega P(r^N,t)dr_2...dr_N$ is used:
\begin{align*} 
\partial_t \rho(r,t) &= \nabla_1 \cdot \bigg(\nabla_1 N \int_\Omega  P(r^N,t) \bigg)dr_2...dr_N,\\
&=  \nabla_1 \cdot(\nabla_1  \rho(r,t) ).
\end{align*}
This is a one-body diffusion equation in $\mathbf{R}^3$, which is an approximation to the diffusion equation of the full $N$ particle probability distribution $P(r^N,t)$. In a last step, the subscript of $\nabla_1$ can be dropped, since the only position considered in this equation is $r_1$. The final equation is
\begin{align} \label{eqnPMDiffTerm}
\partial_t \rho(r,t) &=  \nabla \cdot(\nabla \rho(r,t) ).
\end{align}


\subsubsection*{Adding Pairwise Interactions}
After deriving the one-body equation for the diffusion term, a more complex version of (\ref{eqnPpde}) can be considered. Let $D_{i,i}= \delta_{i,j}$, as before, but define $U$ as: 
\begin{align*}
U= \sum_{m=1}^N V_{ext}(r_m,t) + \frac{1}{2} \sum_{m \neq n}^N V_2(|r_m - r_n|),
\end{align*}
where $V_{ext}$ is an external potential and $V_2$ is the potential due to forces between two particles.
The PDE considered is again a simplified version of (\ref{eqnPpde}) and has the form:
\begin{align} \label{eqnPpde2}
 \partial_t P(r^N,t)= \sum_{i}^N \nabla_i \cdot \bigg(\bigg( \nabla_i + \nabla_i \bigg( \sum_{m=1}^N V_{ext}(r_m,t) + \frac{1}{2} \sum_{m \neq n}^N V_2(|r_m - r_n|)\bigg)  \bigg) P(r^N,t) \bigg),
 \end{align}
where $k_B T=1$ for simplicity.
The diffusion term in the equation can be treated equivalently to the procedure in the previous section.
The two new terms are treated similarly. The equation is multiplied by $N$ and integrated over $r_2...r_N$.
This gives:
\begin{align*} 
&N\int_\Omega \partial_t P(r^N,t) dr_2...dr_N\\
=& N\int_\Omega \sum_{i}^N \nabla_i \cdot \bigg(\bigg( \nabla_i + \nabla_i \bigg( \sum_{m=1}^N V_{ext}(r_m,t) + \frac{1}{2} \sum_{m \neq n}^N V_2(|r_m - r_n|)\bigg)  \bigg) P(r^N,t) \bigg) dr_2...dr_N,\\
=& N\int_\Omega \sum_{i}^N \nabla_i \cdot \nabla_i P(r^N,t)dr_2...dr_N + N\int_\Omega \sum_{i}^N \nabla_i \cdot \bigg( P(r^N,t) \nabla_i \sum_{m=1}^N V_{ext}(r_m,t) \bigg) dr_2...dr_N\\ +& N\int_\Omega \sum_{i}^N \nabla_i \cdot \bigg( P(r^N,t)\nabla_i \frac{1}{2} \sum_{m \neq n}^N V_2(|r_m - r_n|) \bigg) dr_2...dr_N\\
:=& I_1 + I_2 +I_3.
\end{align*}
The left-hand side satisfies $N\int_\Omega \partial_t P(r^N,t) dr_2...dr_N= \partial_t \rho(r,t)$ from the previous section.
The integral $I_1$, by (\ref{eqnPMDiffTerm}), satisfies:
\begin{align} \label{eqnPInt1}
I_1=  N\int_\Omega \sum_{i}^N \nabla_i \cdot \nabla_i P(r^N,t)dr_2...dr_N = \Delta \rho(r,t).
\end{align}
Next, the integral $I_2$ is considered:
\begin{align*}
I_2&=N\int_\Omega \sum_{i}^N \nabla_i \cdot \bigg( P(r^N,t) \nabla_i \sum_{m=1}^N V_{ext}(r_m,t) \bigg) dr_2...dr_N. 
\end{align*}
By the same argument as in the previous section, the integration and summation can be swapped. Furthermore, $\nabla_i \sum_{m=1}^N V_{ext}(r_m,t) =\nabla_i V_{ext}(r_i,t)$, since all other terms in the sum are zero, when $\nabla_i$ is applied to a term independent of $r_i$. The resulting equation is:
\begin{align*}
I_2&=N\sum_{i}^N\int_\Omega  \nabla_i \cdot \bigg( P(r^N,t) \nabla_i V_{ext}(r_i,t) \bigg) dr_2...dr_N. 
\end{align*}
As before, the Divergence Theorem can be used for all variables $r_2,...r_N$, while the equation for $r_1$ remains unchanged. This gives:
\begin{align*}
I_2&=N\sum_{i=2 }^N\int_{ \partial \Omega}  P(r^N,t) \frac{\partial V_{ext}(r_i,t)}{\partial {n}}  dr_2...dr_N + N\int_\Omega  \nabla_1 \cdot \bigg( P(r^N,t) \nabla_1 V_{ext}(r_i,t) \bigg) dr_2...dr_N, 
\end{align*}
where ${n}$ is again the outward normal.
Then, applying the boundary conditions for $P(r^N,t)$, as discussed above, and realising that $\nabla_1 V_{ext}(r_i,t)=\nabla_1 V_{ext}(r_1,t)$, since this expression is zero for all $r_i \neq r_1$, the following equations is derived:
\begin{align*}
I_2&= N \int_\Omega  \nabla_1 \cdot \bigg( P(r^N,t) \nabla_1 V_{ext}(r_1,t) \bigg) dr_2...dr_N. 
\end{align*}
Since $r_1$ is constant with respect to the integration variables, all terms only depending on $r_1$ can be taken out of the integral to give:
\begin{align*}
I_2&= N \nabla_1 \cdot \bigg(\nabla_1 V_{ext}(r_1,t)\int_\Omega  P(r^N,t) dr_2...dr_N\bigg)\\
&=   \nabla_1 \cdot \bigg( (\nabla_1 V_{ext}(r_1,t)) \rho(r,t)\bigg).
\end{align*}
Then, dropping the subscripts, $I_2$ is:
\begin{align}\label{eqnPInt2}
I_2= \nabla \cdot \bigg( \rho(r,t)\nabla V_{ext}(r_1,t) \bigg).
\end{align}

The final term in the PDE that has to be considered is $I_3$:
\begin{align*}
I_3 = N\int_\Omega \sum_{i}^N \nabla_i \cdot \bigg( P(r^N,t)\nabla_i \frac{1}{2} \sum_{m \neq n}^N V_2(|r_m - r_n|) \bigg) dr_2...dr_N.
\end{align*}
As for $I_1$ and $I_2$, the integration and summation operations can be swapped, and the Divergence Theorem can be applied to $r_2,...r_N$ to give:
\begin{align*}
I_3 =& \frac{1}{2}N \sum_{i}^N \int_\Omega  \nabla_i \cdot \bigg( P(r^N,t) \nabla_i \sum_{m \neq n}^N V_2(|r_m - r_n|) \bigg) dr_2...dr_N\\
=&\frac{1}{2}N\int_{\partial \Omega} \sum_{i=2}^N  \bigg( P(r^N,t) \sum_{m \neq n}^N \frac{ \partial V_2(|r_m - r_n|)}{\partial n} \bigg) dr_2...dr_N\\
 +& \frac{1}{2}N\int_\Omega  \nabla_1 \cdot \bigg( P(r^N,t)\nabla_1 \sum_{m \neq n}^N  V_2(|r_m - r_n|) \bigg) dr_2...dr_N.
\end{align*}
The boundary conditions for $P$ are applied to set the first term to zero. \\
The term $\nabla_1 \sum_{m \neq n}^N  V_2(|r_m - r_n|)$ has to be examined in more detail. Since the gradient is applied with respect to $r_1$, one of the $r_m,r_n$ in the double sum has to be $r_1$, since all other terms will be zero when the gradient is applied.
Therefore:
\begin{align*}
\nabla_1 \sum_{m \neq n}^N  V_2(|r_m - r_n|)=\nabla_1 \sum_{n=2}^N  V_2(|r_1 - r_n|) + \nabla_1 \sum_{m=2}^N  V_2(|r_m - r_1|).
\end{align*}
 Since $m$ and $n$ are dummy variables and  $|r_m - r_n|=|r_n - r_m|$ is symmetric, the previous equation reduces to:
 \begin{align*}
 \nabla_1 \sum_{m \neq n}^N  V_2(|r_m - r_n|)=2 \nabla_1 \sum_{n=2}^N  V_2(|r_1 - r_n|) .
 \end{align*} 
Then $I_3$ becomes:
\begin{align*}
I_3 
=& N\int_\Omega  \nabla_1 \cdot \bigg( P(r^N,t)\nabla_1 \sum_{n=2}^N  V_2(|r_1 - r_n|) \bigg) dr_2...dr_N.
\end{align*}
Writing out the sum in $N$ explicitly gives:
\begin{align*}
I_3 
=& N\int_\Omega  \nabla_1 \cdot \bigg( P(r^N,t)\nabla_1 V_2(|r_1 - r_2|) \bigg) dr_2...dr_N\\
+& N\int_\Omega  \nabla_1 \cdot \bigg( P(r^N,t)\nabla_1 V_2(|r_1 - r_3|) \bigg) dr_2...dr_N\\
\vdots \\
+& N\int_\Omega  \nabla_1 \cdot \bigg( P(r^N,t)\nabla_1 V_2(|r_1 - r_N|) \bigg) dr_2...dr_N.
\end{align*}
Since the particles are indistinguishable, a permutation argument can be employed and the indices $r_i$ in the terms $V_2(|r_1 - r_i|)$ can be relabelled, such that $r_i=r_2$, $i=3,...,n$, for each term in the sum. Since the integration is symmetric, the integration order can be permuted arbitrarily and hence does not have to be adapted. This results in $N-1$ identical equations, and so $I_3$ is:
\begin{align*}
 I_3 
 =& N(N-1) \int_\Omega  \nabla_1 \cdot \bigg( P(r^N,t)\nabla_1 V_2(|r_1 - r_2|) \bigg) dr_2...dr_N.
\end{align*}
Considering now the definition of the $n$-body particle distributions, $I_3$ can be written in terms of the two-body distribution $\rho^{(2)}(r_1,r_2,t)= N(N-1) \int_\Omega dr_3...dr_N P(r^N,t)$. Terms that only depend on $r_1$ can be taken out of the integral. Then $I_3$ is:
\begin{align*}
I_3 
=& N(N-1)  \nabla_1 \cdot \bigg( \bigg(\nabla_1 \int_\Omega  V_2(|r_1 - r_2|) \bigg) P(r^N,t) \bigg) dr_2...dr_N.
\end{align*}
Since $V_2$ only depends on $r_2$ and the order of integration is arbitrary, the integral can be rewritten as follows:
\begin{align*}
I_3 
=&  \nabla_1 \cdot \bigg( \nabla_1 \int_\Omega  V_2(|r_1 - r_2|) \bigg(N(N-1) \int_\Omega  P(r^N,t)  dr_3...dr_N \bigg) dr_2 \bigg)\\
=&  \nabla_1 \cdot \bigg( \nabla_1 \int_\Omega  V_2(|r_1 - r_2|) \rho^{(2)}(r_1,r_2,t) dr_2 \bigg).
\end{align*}
Dropping the indices on $\nabla_1$, the equation is:
\begin{align} \label{eqnPInt3}
I_3 
=&  \nabla\cdot \bigg( \nabla \int_\Omega  V_2(|r - r_2|) \rho^{(2)}(r,r_2,t) dr_2 \bigg).
\end{align}
The full three dimensional approximation of (\ref{eqnPpde2}) is found by combining (\ref{eqnPInt1}), (\ref{eqnPInt2}) and (\ref{eqnPInt3}), to give:
\begin{align*}
\partial_t \rho(r,t) &=
 \nabla\cdot \bigg(\nabla \rho(r,t)
+ \rho(r,t)\nabla V_{ext}(r_1,t) 
+ \int_\Omega \nabla  V_2(|r - r_2|) \rho^{(2)}(r,r_2,t) dr_2 \bigg).
\end{align*}
This equation is not closed, since it depends on $\rho^{(2)}(r,r_2,t)$. 
There are different approaches to closing the equation, and two of them are discussed below.

Note that the derivation for $\mathbf{w}$ term follows the same steps as above and is therefore omitted.
\subsubsection{Approximating the Two-Body Density}
There are different ways to approximate the two-body density $\rho^{(2)}$. In this section, two of these are presented; the Mean Field Approximation and Density Functional Theory.
\subsubsection*{The Mean Field Approximation}
In order to use a mean field approach, a modelling assumption has to be made. It is assumed that the particles in the bath are only weakly interacting and the resulting approximation is that of independence. It is assumed that the two-body density of particles $r_1$ and $r_2$ is approximately the product of the individual one-body densities of $r_1$ and $r_2$, that is:
\begin{align*}
\rho^{(2)}(r,r_2,t)\approx \rho(r_1,t)\rho(r_2,t),
\end{align*}
as in \cite{RexLoewen1}.
The resulting closed PDE is:
\begin{align}\label{eqnMeanFieldApprox1}
\partial_t \rho(r,t) &=
\nabla\cdot \bigg( \bigg(\nabla 
+ \nabla V_{ext}(r_1,t) 
+\int_\Omega  \nabla  V_2(|r - r_2|) \rho(r_2,t) dr_2 \bigg) \rho(r,t) \bigg).
\end{align}
In the context of the mean field approximation, the integral term can be interpreted as the sum of forces between a particle at a position $r$ and all other particles in $\Omega$.
However, the independence approximation is often not practical in modelling industrial phenomena and so an alternative approach needs to be considered, which includes the effects of the two-body density.
\subsubsection*{The Adiabatic Approximation}
Another approach for approximating $\rho^{(2)}$ is using Density Functional Theory (DFT). DFT shows that at equilibrium, when $\partial_t \rho=0$, there exists a free energy functional $\mathcal{F}$, such that:
\begin{align*}
&\nabla \rho(r,t)
+ \rho(r,t)\nabla V_{ext}(r_1,t) 
+ \int_\Omega \nabla  V_2(|r - r_2|) \rho^{(2)}(r,r_2,t) dr_2 \\
 &= \rho(r,t) \nabla \frac{\delta \mathcal{F[\rho]}}{\delta \rho},
\end{align*}
where $\frac{\delta}{\delta \rho}$ is the functional derivative with respect to $\rho$.
While in most cases $\mathcal{F}$ is not known explicitly, it is known that at equilibrium $\nabla \frac{\delta \mathcal{F[\rho]}}{\delta \rho}=0$, and therefore it can be assumed that the PDE can be rewritten as:
\begin{align}\label{eqnAdiaAprox1}
\partial_t \rho(r,t) = \nabla \cdot \bigg( \rho(r,t) \nabla \frac{\delta \mathcal{F[\rho]}}{\delta \rho} \bigg), 
\end{align}
as discussed in \cite{GoddardPseudospectralCode1}.
This is called the Adiabatic Approximation, and $\mathcal{F}$ contains all of the information about particle correlations, if it is known.
For non-interacting particles, the explicit form for $\mathcal{F}[\rho]$ is known to be:
\begin{align}\label{eqnFhardrod}
\mathcal{F}[\rho] = \int \rho(\log(\rho)-1) dr,
\end{align}
see \cite{Tarazona2008}.

To show that $\mathcal{F}[\rho]$ satisfies the adiabatic approximation (\ref{eqnAdiaAprox1}), the functional derivative can be computed and substituted into (\ref{eqnAdiaAprox1}). 
Since $\mathcal{F}$ is of the form $\mathcal{F}[\rho] = \int f(r,\rho(r), \nabla \rho(r))dr$, where $f(r,\rho(r), \nabla \rho(r))= \rho(\log(\rho)-1)$, a function $\phi$ of compact support can be defined, as discussed in \cite{CalculusofVariations1}, such that:
\begin{align*}
\int \frac{\delta \mathcal{F}[\rho]}{\delta \rho} \phi(r) dr &= \bigg[ \frac{d}{d \epsilon} \int f(r,\rho(r) + \epsilon \phi, \nabla \rho(r) + \epsilon \nabla \phi) dr \bigg]_{\epsilon=0}\\
&= \int \bigg( \frac{\partial f}{ \partial \rho} - \bigg( \nabla \cdot \frac{\partial f}{\partial \nabla \rho} \bigg) \bigg) \phi(r) dr.
\end{align*}
Since this holds for all functions $\phi \in C_0^\infty (\Omega)$, the following holds:
\begin{align*}
\frac{\delta \mathcal{F}[\rho]}{\delta \rho} =\bigg( \frac{\partial f}{ \partial \rho} - \bigg( \nabla \cdot \frac{\partial f}{\partial \nabla \rho} \bigg) \bigg).
\end{align*}
Applying this result to (\ref{eqnFhardrod}) results in:
\begin{align*}
\frac{\delta \mathcal{F}[\rho]}{\delta \rho} &= \frac{\delta}{\delta \rho} \bigg(\int \rho(\log(\rho)-1) dr \bigg)\\
&=  \frac{\partial}{ \partial \rho}(\rho(\log(\rho)-1)) - \nabla \cdot \bigg(\frac{\partial}{\partial \nabla \rho} (\rho(\log(\rho)-1) \bigg)\\
&=  \frac{\partial}{ \partial \rho}(\rho(\log(\rho)-1))\\
&= \log \rho. 
\end{align*}
Applying the gradient to this result gives:
\begin{align*}
\nabla \frac{\delta \mathcal{F}[\rho]}{\delta \rho} = \nabla \log \rho = \frac{\nabla \rho}{\rho}.
\end{align*}
Substituting this into (\ref{eqnAdiaAprox1}) results in:
\begin{align*}
\partial_t \rho(r,t) &= \nabla \cdot \bigg( \rho(r,t) \nabla \frac{\delta \mathcal{F[\rho]}}{\delta \rho} \bigg)\\
&= \nabla \cdot \bigg( \rho(r,t)  \frac{\nabla \rho(r,t) }{\rho(r,t)}  \bigg)\\
&= \nabla \cdot \nabla \rho(r,t) \\
&= \Delta \rho(r,t).
\end{align*}
This shows that the particular choice of $\mathcal{F}$, defined by (\ref{eqnFhardrod}), recovers the diffusion equation when substituted into (\ref{eqnAdiaAprox1}). This is to be expected, since $\mathcal{F}$ represents non-interacting particles.
	

\section{PDE-Constrained Optimization} 

The aim of this project is to work towards using the particle model derived in the previous section to describe an industrial process and optimize this process with minimal cost involved.
It is of interest to achieve a particle distribution $\hat{\rho}$ in some time over some domain $\Omega$.  
In the context of PDE-constrained optimization, the aim is to minimize the distance between a state variable $\rho$ and a desired state $\hat\rho$, in some norm, while also minimizing the cost involved in reaching the desired state. This minimization is constrained by the underlying physics of the particle system. The PDE describing the particle dynamics is called the state equation.
\newline
Achieving the desired state $\hat\rho$ as close as possible can be of interest either for all times, as in this report, or only at some times, such as the final time $T$. In order to achieve $\hat \rho$, the particle distribution $\rho$ can be controlled through a so-called control variable, denoted by $u$. The control can be applied in various ways, which is dependent on the application. Since the background flow influences the particle distribution $\rho$, $\hat\rho$ can try to be reached by changing the flow field. Then the flow field is the control $u$. Alternatively, $u$ could represent the temperature or the geometry of the boundaries of the bath. Moreover, $u$ could be a parameter in the body force or in the particle interaction term, influencing the particle distribution through the forces involved. Note that $u$ cannot always influence the system enough to reach the desired state $\hat \rho$. This highly depends on the choice of $\hat \rho$, the physics of the problem and on the choice of the parameter $\beta$, which is discussed below.
Since controlling $\rho$ requires energy, $u$ can be thought of as the cost involved in reaching $\hat\rho$. 
\\ 
The weight of the control is determined by the so-called regularization parameter, $\beta$. If $\beta$ is small, the desired state will be reached, however, at a high cost. If $\beta$ is large, the control will be minimized, but the desired state might not be reached. The choice of $\beta$ depends on the application involved. It is generally of interest to find a range of $\beta$ values, for which the solution to the optimization problem is robust. 
The PDE-constrained optimization problem of interest in this report is of the form:
\begin{align} \label{sysPDEconOpti1}
&\min_{\rho,\mathbf{w}} \quad \frac{1}{2}\norm{\rho- \hat{\rho}}_{L^2}^2 + \frac{\beta}{2} \norm{\mathbf{w}}_{L^2}^2,\\
\notag\\
&\textbf{subject to:}\notag\\ 
&\partial_t \rho =\nabla^2 \rho - \nabla \cdot (\rho \mathbf{w}) + \nabla \cdot (\rho \nabla V_{ext}) \quad \text{in} \quad \Omega,\notag\\
\notag\\
&\dfrac{\partial \rho}{\partial {n}} - \rho \mathbf{w} \cdot \mathbf{n} + \rho \dfrac{\partial V_{ext}}{\partial {n}}  =0 \quad \text{on} \quad \partial \Omega,\notag \\
& \rho = \rho_0 \quad \text{at} \quad t=0,  \notag
\end{align}
where the state equation is a simplified version of (\ref{sysParticleModel1}), which neglects the particle interaction term.
Note that the norms are the $L^2$ norms with respect to $\Omega$ and time. Other norms could be used, depending on the type of application. In the following it is assumed, as described in \cite{TroeltzschFredi2010OCoP}, that $\rho \in H^1$ and $\mathbf{w} \in L^2$.
In order to solve this optimization problem, continuous optimality conditions can be derived, which can then be discretized and solved numerically. This is known as the optimize-then-discretize approach.
Another approach, discretize-then-optimize, would be to first discretize (\ref{sysPDEconOpti1}) and then derive the discrete optimality conditions that need to be solved.
A good introduction to PDE-constrained optimization, can be found in \cite{PearsonThesis} and a more detailed introduction to numerical PDE-constrained optimization is provided in \cite{DeLosReyesOptimization}. Both of these texts provided the basis for the above discussion.


\section{Numerical Methods}	

In order to solve the PDE-constrained optimization problem (\ref{sysPDEconstroptiAndNonlocal1}), it is necessary to solve the first-order optimality conditions (\ref{sysFirstOderOptimalityNonLocal1}). Therefore, methods of time and space discretization, as well as a method for solving the system of PDEs are needed. One challenge specific to this problem is the final time condition in the adjoint equation, which means that it is a boundary value problem in time as well as in space. 	
The numerical methods that are needed to solve (\ref{sysFirstOderOptimalityNonLocal1}) are introduced in this section.
A lot of research has been done on numerical methods for solving linear PDE-constrained optimization problems, as demonstrated in \cite{DeLosReyesOptimization}, \cite{CarraroDirectIndirectMultipleShooting} and \cite{TroeltzschFredi2010OCoP}.
New approaches to solving the optimality system (\ref{sysPDEconstroptiAndNonlocal1}) are needed because of the non-linear, non-local nature of the particle interaction term in the PDE-constraint. Standard methods are no longer sufficient to solve this type of problem, as discussed in this section.

\subsection{Pseudospectral Methods} \label{secPSMTheory1}

Pseudospectral methods on non-periodic domains are based on polynomial interpolation on non-equispaced points.  
Typically, Chebyshev points $\{x_j\}$ are chosen as collocation points on $[-1,1]$, which are defined as:
\begin{align}\label{defChebyshevPoints}
x_j= \cos\bigg(\frac{j \pi}{N}\bigg), \quad j=0,1,...,N,
\end{align}	
see \cite{bibTrefethen}.
These points are clustered at the endpoints of the interval, and sparse around $0$. Using this approach, the points are distributed from $1$ to $-1$, which is counter-intuitive. Therefore, in the code library \cite{GoddardPseudospectralCode1}, which is used in producing the results of this report, the Chebyshev points are automatically flipped back to run from $-1$ to $1$. Moreover, a linear map takes the points from the computational domain $[-1,1]$ to any domain $[a,b]$ of interest.
Interpolation on the Chebyshev grid is done using barycentric Lagrange interpolation, derived in \cite{bibTrefethenBerrut1}. The barycentric formula is:
\begin{align*}
p_N(x)= \frac{\displaystyle \sum_{k=0}^N \frac{\tilde w_k}{x-x_k}f(x_k)}{\displaystyle \sum_{k=0}^N \frac{\tilde w_k}{x-x_k}},
\end{align*}
where the weights are defined as:
\begin{align*}
\tilde w_j = (-1)^j d_j, \quad d_j = 
\left \{
\begin{tabular}{c}
$\frac{1}{2}$ \text{for} $j=0$, $j=N$, \\
$1$ \text{otherwise}, \phantom{abksla} 
\end{tabular}
\right .
\end{align*}
see \cite{bibTrefethenBerrut1} and \cite{GoddardPseudospectralCode1}.

The derivation of the Chebyshev differentiation matrices is described below, following the presentation in \cite{bibTrefethen}.
The function of interest, $f$, is evaluated at the Chebyshev points $\{x_j\}$ and a grid function, $f_j := f(x_j)$, is defined.
There exists a unique polynomial of degree $\leq N$ that can be used to interpolate $f$ on the grid points $x_j$. The polynomial $p$ satisfies, by definition, the following relationship:
\begin{align}\label{eqnptov1}
p(x_j)=f_j,
\end{align}
so that the residual $p(x_j) -f_j$ is zero at these points. Therefore, this method is called a collocation method, see \cite{Boyd1}. 
Once such a polynomial $p$ is found, it can be differentiated and the following relationship is defined:
\begin{align*}
w_j = p'(x_j).
\end{align*} 	
This relationship can be rewritten as a multiplication of $f_j$ by a $(N+1) \times (N+1)$ matrix, denoted by $D$, as follows:
\begin{align*}
w_j= D f_j,
\end{align*}
using (\ref{eqnptov1}).
A $(N+1) \times (N+1)$ differentiation matrix $D$ has the following entries, compare with \cite{bibTrefethen}:
\begin{align*}
(D)_{00}&= \frac{2N^2 +1}{6},\\
(D)_{NN}&= -\frac{2N^2 +1}{6},\\
(D)_{jj}&= -\frac{x_j}{2(1-x_j^2)}, \quad j=1,...,N-1,\\
(D)_{ij}&= \frac{c_i}{c_j} \frac{(-1)^{i+j}}{(x_i-x_j)}, \quad i \neq j, \quad i,j=0,...,N,
\end{align*} 	
where 
\begin{align*}
c_i =\left\{\begin{array}{l} 2, \quad i=0 \text{   or   }N, \\1, \quad \text{otherwise.}\end{array}\right.
\end{align*}	
The second derivative is represented by the second differentiation matrix $D_2$, which can be found by squaring the first differentiation matrix; $D_2=D^2$, and more generally the $j^{th}$ differentiation matrix is found as follows:
\begin{align*}
D_j=D^j.
\end{align*}
However, in \cite{GoddardPseudospectralCode1}, the exact coefficients, derived in a similar way as above for $D$, are used to compute $D_2$, since it is more accurate than squaring $D$.
\\
In order to extend the definition of the differentiation matrices to two-dimensional grids, a so-called tensor product grid has to be defined. First, Chebyshev points $x_j$, for $j=1,...,n$, on the $x$-axis and another set of Chebyshev points $y_i$, for $i=1,...,m$ on the $y$-axis are taken, both between $[-1,1]$. Then the following two vectors are defined:
\begin{align*}
\mathbf{x}_{K}=(x_1,x_1,...,x_1,x_2,x_2,...,x_2,...,x_n,x_n,...,x_n)^T,\\
\mathbf{y}_{K}=(y_1,y_2,...,y_m,y_1,y_2...,y_m,.....,y_1,y_2,...,y_m)^T.
\end{align*} 
In $\mathbf{x}_K$, each $x_j$ is repeated $m$ times, while $\mathbf{y}_K$, each sequence $y_1,y_2,...,y_m$ is repeated $n$ times. The total length of each vector is $n \times m$. 
These vectors are defined, so that the set $(\mathbf{x}_K,\mathbf{y}_K)$ is a full set of all Chebyshev points on the two-dimensional tensor grid.
Note that the points are clustered around the boundary of the two-dimensional grid and sparse in the middle of the grid.
These Kronecker vectors can be used to find the Chebyshev differentiation matrices for two-dimensional problems as follows, compare to \cite{bibTrefethen}. For a first derivative $D$ in the $x$ direction, a Kronecker product is taken of the one-dimensional Chebyshev differentiation matrix with the identity, as demonstrated here with three points:
\begin{align*}
D_x&=I \otimes D = 
\begin{pmatrix}
1 & 0 & 0\\
0 & 1 & 0 \\
0 & 0 & 1
\end{pmatrix}
\otimes
\begin{pmatrix}
d_{11} & d_{12} & d_{13}\\
d_{21} & d_{22} & d_{23} \\
d_{31} & d_{32} & d_{33}
\end{pmatrix}
\\&=
\begin{pmatrix}
d_{11} & d_{12} & d_{13} & & &  & & &\\
d_{21} & d_{22} & d_{23} & & & & & & \\
d_{31} & d_{32} & d_{33} & & & & & & \\
& & &d_{11} & d_{12} & d_{13} & & &\\
& & &d_{21} & d_{22} & d_{23}  & & &\\
& & &d_{31} & d_{32} & d_{33} & & &\\
 & & & & & &d_{11} & d_{12} & d_{13}\\
& & & & & &d_{21} & d_{22} & d_{23}  \\
& & & & & &d_{31} & d_{32} & d_{33} 
\end{pmatrix},
\end{align*}
where the block structure matches the repetition of each $x_j$ in $\mathbf{x}_K$.
The second derivative with respect to $x$, $D_{xx}$ can be found by using $D_2$ instead of $D$ in this calculation. 
The derivative with respect to $y$ is found by taking the Kronecker product the other way around:
\begin{align*}
D_y&=D \otimes I = 
\begin{pmatrix}
d_{11} & d_{12} & d_{13}\\
d_{21} & d_{22} & d_{23} \\
d_{31} & d_{32} & d_{33}
\end{pmatrix}
\otimes
\begin{pmatrix}
1 & 0 & 0\\
0 & 1 & 0 \\
0 & 0 & 1
\end{pmatrix}
\\&=
\begin{pmatrix}
d_{11} & & & d_{12} & & & d_{13} & & \\
& d_{11} & & & d_{12} & & &  d_{13} & \\
& & d_{11} & & &  d_{12} &  & & d_{13}\\
d_{21} & & & d_{22} & & & d_{23} & & \\
& d_{21} & & & d_{22} & & &  d_{23} & \\
& & d_{21} & & &  d_{22} &  & & d_{23}\\
d_{31} & & & d_{32} & & & d_{33} & & \\
& d_{31} & & & d_{32} & & &  d_{33} & \\
& & d_{31} & & &  d_{32} &  & & d_{33}\\
\end{pmatrix},
\end{align*}
which now matches the repetition of each $y_1,...y_m$ in $\mathbf{y}_K$.
The Chebyshev differentiation of the Laplacian is given by: $L=I  \otimes D_2 + D_2 \otimes I$.
\\
\\
In order to evaluate integrals in a similar way, the so-called Clenshaw--Curtis quadrature is used, which is derived in \cite{ClenCurt1}.
This is, for the integral over a smooth function $f$:
\begin{align}\label{eqnClenCurtQuad}
\int_{-1}^1 f(x)dx = \sum_{k=0}^N w_kf(x_k),
\end{align}
where the weights are defined as:
\begin{align*}
w_j = \frac{2d_j}{N}
\left \{
\begin{tabular}{c}
$1- \displaystyle \sum_{k=1}^{(N-2)/2} \frac{2 \cos(2kt_j)}{4k^2-1} - \frac{\cos(\pi j)}{N^2 -1} \quad \quad\text{for $N$ even}$,\\
$1- \displaystyle \sum_{k=1}^{(N-2)/2} \frac{2 \cos(2kt_j)}{4k^2-1} \quad \quad \quad \quad \quad \quad \ \ \ \text{for $N$ odd}$,
\end{tabular}
\right .
\end{align*}
see \cite{GoddardPseudospectralCode1}.

The advantage of Spectral Methods is that, for smooth functions, the convergence is exponential, see \cite{Boyd1}:
\begin{align*}
\text{Pseudospectral Error} \approxeq O \bigg[ \bigg( \frac{1}{N} \bigg)^N \bigg].
\end{align*}


A good overview on spectral methods is given in \cite{bibTrefethen} and a more in depth discussion can be found in \cite{Boyd1}.


\subsection{Comparison with FEM and FDM} \label{secCompareFEMFDMPDM}

In this section the advantages and disadvantages of pseudospectral methods over finite element methods (FEM) and finite difference methods (FDM) are discussed, compare to \cite{Boyd1}.
The main difference between pseudospectral methods (PSM) and the other two methods is that PSM uses global basis functions on all Chebyshev points, while FEM uses local basis functions and FDM uses local, low order polynomials.
\\
Finite difference methods use overlapping sequences of polynomials to approximate the solution of the problem. They are easy to implement and less costly per degree of freedom. However, they are also less accurate than PSM.
\\
The basis functions used in FEM methods are generally of fixed, low degree and more accuracy is achieved through refinement of the elements; either in the whole domain or in regions where the problem is more difficult to solve.
In comparison, the global basis function used in PSM are generally of higher degree than the ones in FEM. Furthermore, in order to refine the method, the degree of the polynomial can be increased.
\\
In general, FEM results in large sparse matrix systems, which in many cases can be solved by exploiting their structure. It is also easily applied to complex domains, due to the shape of the elements.
However, the disadvantage of FEM is low accuracy of solutions, due to the low degree of the polynomial basis functions. Furthermore, the sparsity property of the matrix systems is compromised when the PDEs involve nonlocal terms. Therefore, PSM are advantageous in this type of problems, since small dense matrices are utilized. 
 As discussed in \cite{FEMIntegroPaper}, an adaptive FEM method can be used to solve an integro-differential PDE-constrained optimal control problem. However, the accuracy achieved is mainly of order $O(10^{-2})$ and maximal $O(10^{-4})$, for a two dimensional problem with $N$ nodes, where $N$ is between $N=3549$ and $50421$. The time step is $dt=0.05$. Furthermore, Dirichlet boundary conditions are used, which avoids applying more realistic no-flux boundary conditions. These no-flux boundary conditions are difficult to apply in the FEM context, because they are nonlocal as well. Within the existing code framework \cite{GoddardPseudospectralCode1}, these boundary conditions are straightforward to apply.
The disadvantages of PSM are that they are more computationally expensive and the domain is required to be smooth.
However, as discussed in \cite{Boyd1}, if the accuracy of PSM with $N=10$ is to be achieved by FEM or FDM, a 10th order method has to be chosen with an error of $O(h^{10})$.
All in all, PSM is the best method for solving PDE-contrained optimization problems involving integro-PDEs.

\subsection{Exact Solutions} \label{secExactSolsDiffusion1}

For some relatively simple PDE-constrained optimization problems it is possible to construct exact solutions to the first-order optimality system. This is an important aspect in the development of new numerical methods, since these problems can be used as test problems for the numerical method and the exact error can be measured. The considered test problem is a simplified version of (\ref{sysPDEconOpti1}), and therefore testing the numerical method on this problem is a step towards solving (\ref{sysPDEconOpti1}). In this section, the construction of an exact solution for the following problem is derived, where the PDE constraint is the forced heat equation on $\Omega =[-1,1]$:
\begin{align}\label{sysOptimalHeating1}
&\min_{\rho,u} \quad \frac{1}{2}\norm{\rho- \hat{\rho}}_{L_2}^2 + \frac{\beta}{2} \norm{u}_{L_2}^2,\\
&\text{subject to:}\notag 
\notag \\
&\partial_t \rho - \Delta \rho - u=0,  \quad \text{in} \quad \Omega,\notag 
\notag \\
&\rho(r,0)=\rho_0(r),\notag 
\notag \\
& \rho(r,t)=0, \quad  \quad \quad \quad \ \text{on} \quad \partial \Omega. \notag 
\end{align}
Note that $u$ is now the control variable, comparable to $\mathbf{w}$ in (\ref{sysPDEconOpti1}).
The first-order optimality system for this PDE-constrained optimization problem is:
\begin{align} \label{sysPotimaltiyheatequn1}
\textbf{Adjoint Equation} \notag \\
\partial_t p + \Delta p -\rho +\hat{\rho} &=0 \ \ \ \quad \quad \quad \text{in} \quad \Omega,  \\
p(r,T) &= 0 \notag\\
p(r,t) &=0 \quad \quad\quad\quad \text{on} \quad \partial\Omega, \notag\\
\textbf{Gradient Equation} \notag \\
\beta u  - p  &=0 \quad \quad\quad\quad \ \text{in} \quad \Omega, \notag \\
\textbf{Forward Problem} \notag \\
\partial_t \rho - \Delta \rho - u &=0 \ \quad \quad\quad\quad \text{in} \quad \Omega, \notag \\ 
\rho(r,0)&=\rho_0(r) \notag \\
\rho(r,t) &=0 \quad \quad\quad\quad \ \text{on} \quad \partial \Omega \notag. 
\end{align}
This follows almost directly from taking the relevant terms from the optimality system (\ref{sysFirstOderOptimality1}).
The solution to this system is derived in one and two dimensions, as well as for Dirichlet and Neumann boundary conditions.
\newline
\newline
In order to construct a full solution to the optimality system, the following steps have to be taken. At first, an expression for $p$ has to be chosen, such that the boundary conditions for the adjoint equation, as well as the final-time condition, are satisfied.
In a second step, this is substituted into the gradient equation to find $u$. The resulting expression can be used in the state equation to solve for $\rho$. Finally, once all three variables are defined, the adjoint equation can be used to solve for $\hat \rho$.
A functional form for $p$, satisfying Dirichlet boundary conditions and the final time condition is:
\begin{align*}
p(r,t) = \bigg( e^T -e^t \bigg) \cos(\pi r /2).
\end{align*}
Substituting this into the gradient equation gives:
\begin{align*}
u(r,t) = \frac{1}{\beta}\bigg( e^T -e^t \bigg) \cos(\pi r /2).
\end{align*}
Substituting $u$ into the state equation results in a decoupled PDE for $\rho$ that can be solved:
\begin{align}\label{eqnStateEqnDiff1DExact1}
\partial_t \rho - \partial_{r} \rho=\frac{1}{\beta}\bigg( e^T -e^t \bigg) \cos(\pi r /2).
\end{align}
Assuming a solution of the form: 
\begin{align*}
\rho(r,t)= \bigg(a +be^t\bigg)\cos(\pi r /2),
\end{align*}
and substituting it into (\ref{eqnStateEqnDiff1DExact1}) gives:
\begin{align*}
be^t\cos(\pi r /2) + \frac{\pi^2}{4}\bigg(a +be^t\bigg)\cos(\pi r /2)=\frac{1}{\beta}\bigg( e^T -e^t \bigg) \cos(\pi r /2),
\end{align*}
which results in:
\begin{align*}
\bigg(b+\frac{\pi^2}{4}b + \frac{1}{\beta} \bigg)e^t + \frac{\pi^2}{4}a-\frac{1}{\beta} e^T  =0.
\end{align*}
Therefore, $\displaystyle b=-\frac{1}{(1+\frac{\pi^2}{4}) \beta}$ and $ \displaystyle a=\frac{4}{\beta \pi^2}e^T$, and so $\rho$ becomes:
\begin{align*}
\rho(r,t)= \bigg(\frac{4}{\beta \pi^2}e^T -\frac{1}{(1+\frac{\pi^2}{4}) \beta}e^t\bigg)\cos(\pi r /2).
\end{align*}
The expressions for $\rho$ and $p$ can be substituted into the adjoint equation, to solve for $\hat \rho$:
\begin{align*}
e^t \cos(\pi r/2) + \frac{\pi^2}{4}\bigg( e^T -e^t \bigg) \cos(\pi r /2) = \hat \rho - \bigg(\bigg(\frac{4}{\beta \pi^2}e^T -\frac{1}{(1+\frac{\pi^2}{4}) \beta}\bigg)\cos(\pi r /2) \bigg).
\end{align*}
This gives:
\begin{align*}
\hat \rho(r,t)=\bigg( \bigg( \frac{4}{\beta \pi^2}+ \frac{\pi^2}{4} \bigg) e^T + \bigg(1- \frac{\pi^2}{4}  -\frac{1}{(1+\frac{\pi^2}{4}) \beta} \bigg) e^t\bigg)  \cos(\pi r /2) .
\end{align*}
This solution can be used for Neumann boundary conditions as well, when considered on an interval $[-2,2]$. This is due to the fact that $\cos(\pi r/2)$ evaluated at $-2,2$ is equal to $-1$ and $1$ respectively, which is exactly at its stationary points. Therefore, the Neumann boundary conditions are satisfied at these points.
Instead, the approach used in the numerical experiments below is to slightly change the calculations above to derive the following exact solutions for $\rho$, $p$ and $\hat \rho$ for solving (\ref{sysOptimalHeating1}) with Neumann boundary conditions:
\begin{align*}
p(r,t) &=\bigg( e^T -e^t \bigg) \cos(\pi r),\\
\rho(r,t) &= \bigg( \frac{1}{\pi^2 \beta}e^T - \frac{1}{(1+\pi^2)\beta}e^t\bigg)\cos(\pi r),\\
\hat \rho(r,t) &= \bigg( \bigg(\pi^2 + \frac{1}{\pi^2 \beta}\bigg)e^T + \bigg( 1- \pi^2 - \frac{1}{(1+\pi^2)\beta}\bigg)e^t \bigg) \cos(\pi r).
\end{align*}
Furthermore, these calculations can be done equivalently for two or more dimensional problems. 
The exact solution to the two-dimensional problem (\ref{sysOptimalHeating1}) with Dirichlet boundary conditions is:
\begin{align*}
p(r,t) &= \bigg( e^T -e^t \bigg) \cos(\pi x /2)\cos(\pi y /2),\\
\rho(r,t) &= \bigg(\frac{2}{\beta \pi^2}e^T -\frac{4}{(4+2\pi^2) \beta}e^t\bigg)\cos(\pi x /2)\cos(\pi y /2),\\
\hat \rho (r,t) &=\bigg( \bigg( \frac{2}{\beta \pi^2}+ \frac{\pi^2}{2} \bigg) e^T + \bigg(1- \frac{\pi^2}{2}  -\frac{4}{(4+2\pi^2) \beta} \bigg) e^t\bigg)  \cos(\pi x /2)\cos(\pi y /2),
\end{align*}
where $r=(x,y)$.
Finally, the exact solution to the two-dimensional problem (\ref{sysOptimalHeating1}) with Neumann boundary conditions is:
\begin{align*}
p(r,t)&= \bigg( e^T -e^t \bigg) \cos(\pi x )\cos(\pi y),\\
\rho(r,t) &= \bigg(\frac{1}{2\beta \pi^2}e^T -\frac{1}{(1+2\pi^2) \beta}e^t\bigg)\cos(\pi x )\cos(\pi y ),\\
\hat \rho (r,t) &=\bigg( \bigg( \frac{1}{\beta \pi^2}+ 2\pi^2 \bigg) e^T + \bigg(1- 2\pi^2  -\frac{1}{(1+2\pi^2) \beta} \bigg) e^t\bigg)  \cos(\pi x)\cos(\pi y).
\end{align*}
The exact solutions presented here for Dirichlet boundary conditions, for $d$ dimensions, can be found in \cite{GuettelPearson1}.


\subsection{Multiple Shooting Method}	

Boundary value problem (BVP) solvers, such as \texttt{bvp4c}, are designed to treat BVPs in time, see \cite{bvp4cPaper1}. Therefore, they are not equipped to deal with BVPs in both space and time. A method has to be developed that circumvents using BVP solvers and uses initial value problem (IVP) solvers instead. This strategy is called multiple shooting and the theoretical derivation of a multiple shooting approach for PDE-constrained optimization problems can be found in \cite{CarraroIndMultipleShooting} and \cite{CarraroDirectIndirectMultipleShooting}.
\\
In this section, the numerical method is described at the different stages of its development. 
The PDE constraint considered has either Dirichlet or Neumann boundary conditions in space. The problem is treated in one and two dimensions. In order to initiate the development of the method, a simpler PDE constrained optimization problem is considered at first. Once the method is established for the simpler problem, the non-local term is added. After that, two dimensional problems are considered.

\subsubsection*{One-Dimensional Diffusion} \label{secNumericsOneDDiffusion1}
One of the simplest cases of a PDE-constrained optimization problem is heat control in one dimension. The PDE constraint involved is the forced heat equation, and the PDE-constrained optimisation problem (\ref{sysOptimalHeating1}), derived in Section \ref{secExactSolsDiffusion1}, is used, and the derived optimality system is treated below. 

The first step in solving this optimality system is to substitute the gradient equation into the heat equation for $u$ and rearranging the equations to only have the time derivative on the left-hand side.
This results in a coupled system of PDEs: 
\begin{align}\label{sysOneDimHeatEqunOptisys1}
\partial_t \rho(r,t) &= \partial_{rr}\rho(r,t) + \frac{1}{\beta}p(r,t),\\
\partial_t p(r,t) &= - \partial_{rr} p(r,t) + \rho(r,t) -\hat{\rho}(r,t), \notag \\
\text{Initial } & \text{and  Final-Time Conditions:}\notag\\
\rho(r,0)&=\rho_0(r),\notag\\
p(r,T) &=0,\notag\\
\text{Dirich} &\text{let  Boundary Conditions:}\notag\\
\rho(r,t) &=0,\quad \text{on} \quad \partial \Omega,\notag\\
p(r,t) &=0,\quad \text{on} \quad \partial \Omega,\notag
\end{align}
where $r \in [a,b]$ and $t \in [0,T]$.
This system is considered as a test problem, since exact solutions for $\rho$ and $p$ are known. Therefore, the exact error of the numerical method can be determined at all points in space and time. The derivation of the exact solutions are discussed in Section \ref{secExactSolsDiffusion1}.

The method that is used to solve this system of PDEs is called the shooting method.
The procedure is to first discretize the problem in space, that is to replace the space derivatives with the appropriate Chebyshev differentiation matrices, as defined above. Then the problem reduces to a coupled system of ODEs, which can be solved using an ODE solver, such as the Matlab solver \texttt{ode15s}. The challenge is that the optimality system is a boundary value problem in time, since the adjoint equation has a final time condition in $p$. 
Therefore, the first idea is to create a guess for the initial condition $p_0(r)$, solve the coupled ODE system using this guess, extracting the computed $p$ value at the final time, $p_{co}(T)$ and measuring the error between the computed $p_{co}$ and the exact $p_{ex}$:
\begin{align*}
e= \norm{p_{co}(T) - p_{ex}(T)}.
\end{align*}
The final step in this procedure is to minimize this error by varying $p_0(r)$, using an in-built Matlab optimization routine, such as \texttt{fsolve}.
This is easily implemented in Matlab, see Appendix \ref{ShootingTest1DiffusionLineBlowsUp}.
However, the problem with this approach is that the adjoint equation, written in this form, is not well posed. The solution to this system blows up in finite time, which is caused by the negative diffusion term in the PDE for $p$.\\

Therefore, the adjoint equation has to be rewritten. This is done by rescaling time as
\begin{align*}
\tau=T+t_0-t,
\end{align*}
which causes the adjoint equation to run backwards in time from $T$ to $t_0=0$. This changes the final time condition for $p$ at time $t=T$, $p(T)=0$, into an initial condition at time $\tau =0$, ${p}(0)=0$.
The optimality system is then:
\begin{align} \label{sysOptimDiffAdjBW1}
\partial_t \rho(r,t) &= \partial_{rr} \rho(r,t) + \frac{1}{\beta}{p}(r,t), \\
\partial_t {p}(r,\tau) &=\partial_{rr} {p}(r,\tau) - \rho(r,\tau) +\hat{\rho}(r,\tau), \notag \\
\text{Initial   } & \text{ Conditions:} \notag  \\
 \rho(r,0)&=\rho_0(r),  \quad \text{for} \quad t=0,\notag \\
{p}(r,0) &=0, \quad \text{for} \quad \tau=0,\notag \\
\text{Dirich} &\text{let Boundary Conditions:} \notag \\
\rho(a,t) &=\rho(b,t)=0, \notag \\
{p}(a,\tau) &={p}(b,\tau)=0, \notag 
\end{align} 
where $t \in [t_0,T]$ and $\tau \in [T,t_0]$.
This is now well posed. However, the issue with this rewritten system is that information about $\rho$ at later times is needed to solve the adjoint equation and ${p}$ values for earlier times are needed to solve the state equation, while neither of these information is available. Figure \ref{ShootingMethod1} visualises this problem. The initial conditions for $\rho$ and $p$ are represented as a green and blue dot respectively. Time $t$ is represented by the green arrow, while time $\tau$, the backwards time, is represented by a blue arrow. In order to test whether this approach works, the exact solution for $\rho$ and $q$ can be used, where information is missing. Then the problem is a decoupled system of PDEs, which is straightforward to solve.  


%\begin{figure}[h] 
%		\vspace{-20pt}
%	\centering
%	\includegraphics[scale=0.85]{FullSol.png}
%		\vspace{-10pt}
%    \caption{Solving a coupled system of PDEs on $[0,T]$.} 
%    \label{ShootingMethod1}
%    	\vspace{-10pt}
%\end{figure}


In order to replace the missing information, as illustrated above, interpolation is used. Since interpolation using only the endpoints of the interval $[t_0,T]$ would be highly inaccurate, a strategy, called multiple shooting, is exploited in this section. 
The time interval is divided into a number of $n$ subintervals, such that $t_0 < t_1<t_2<...<t_n=T$. The subintervals are denoted by $I_i$, where $i=0,1,...,n-1$. The values for $\rho$ and ${p}$ at these times constitute the initial guess. The discretization of the time interval and initial guesses for $\rho$ and $p$ are illustrated in Figure \ref{ShootingMethod2}. The initial guess can be obtained by different methods, which will be discussed in a later section.

%\begin{figure}[h] 
%	\centering
%	\vspace{-10pt}
%	\includegraphics[scale=0.9]{FullDiscr.png}
%		\vspace{-10pt}
%	\caption{Discretizing the time interval and obtain initial guesses for $\rho$ and $q$.} 
%	\label{ShootingMethod2}
%	\vspace{-10pt}
%\end{figure}

In a first step, these initial guesses are chosen to be the known exact solutions to $\rho$ and ${p}$ at the specified times $t_i$. 
The optimality system (\ref{sysOptimDiffAdjBW1}) is solved on each of the $I_i$, by considering the upper and lower bound of the subinterval, $t_i$ and $t_{i+1}$ instead of the global bounds $t_0$ and $T$. The new backward running time is defined, equivalently to above, as $\tilde\tau =t_{i+1}+t_i-t$, and the system becomes:
\begin{align} \label{sysOptimDiffAdjBW2}
\partial_t \rho(r,t) &= \partial_{rr}\rho(r,t) + \frac{1}{\beta}{p}(r,t), \\
\partial_t {p}(r,\tilde\tau) &= \partial_{rr}{p}(r,\tilde\tau) - \rho(r,\tilde\tau) +\hat{\rho}(r,\tilde\tau), \notag \\
\text{Initial   } & \text{ Conditions:} \notag  \\
\rho(r,t_i)&=\rho_{t_i}, \quad \ \ \text{for} \ \quad t=t_i, \notag \\
{p}(r,t_{i+1}) &=p_{t_{i+1}}, \quad \text{for} \quad \tilde t=t_{i+1}, \notag \\
\text{Dirich} &\text{let  Boundary Conditions:} \notag \\
\rho(a,t) &=\rho(b,t)=0, \notag \\
{p}(a,\tilde\tau) &={p}(b,\tilde\tau)=0, \notag 
\end{align} 
where $t \in I_{i}=[t_i,t_{i+1}]$.
On each subinterval, both $\rho$ and ${p}$ are interpolated between their known values at $t_i$ and $t_{i+1}$, and the result is used to provide $\rho$ at a later time step, to solve the adjoint equation, as well as ${p}$ at an earlier time step, to solve the state equation. 
\newline
\newline
In order to implement the strategy, (\ref{sysOptimDiffAdjBW2}) is evaluated on each time interval $I_i=[t_i,t_{i+1}]$, using interpolation for $\rho$ in the adjoint equation and for $ {p}$ in the state equation to provide the missing information.

%\begin{figure}[h] 
%	\centering
%	\vspace{-10pt}
%	\includegraphics[scale=0.9]{rhoSol.png}
%	\caption{Solution strategy for $\rho$.} 
%	\label{ShootingMethod3}
%\end{figure}
%
%\begin{figure}[h] 
%	\centering
%	\vspace{-20pt}
%	\includegraphics[scale=0.9]{pSol.png}
%	\caption{Solution strategy for $p$.} 
%	\label{ShootingMethod4}
%	\vspace{-10pt}
%\end{figure}

As can be observed in Figure \ref{ShootingMethod3}, the value of $\rho$, taken from the ODE solver, at $t_{i+1}$ is compared to $\rho$ at $t_{i+1}$, which is the initial guess for solving (\ref{sysOptimDiffAdjBW2}) on the next interval $I_{i+1}=[t_{i+1},t_{i+2}]$:
\begin{align}\label{eqnErrorINitoptiguess}
e= \norm{g_{init}-g_{sol}},
\end{align} 
where $g_{init}=(g_1,g_2,...g_n)$ is a vector of initial guesses for $\rho$ on all $n$ time points and $g_{sol}$ is the vector of PDE solutions associated with the initial guesses on all time points.
This provides an error measure of the quality of the set of initial guesses $g_{init}$ for $\rho$. This error calculation is repeated for ${p}$, as illustrated in Figure \ref{ShootingMethod4}. However, since ${p}$ is running backwards in time, the solution to the ODE solver provides the value for ${p}$ at $t_i$, the lower bound on the considered interval $I_i$, which is then compared with the initial guess for ${p}$ for the previous interval $I_{i-1}=[t_{i-1},t_i]$.
Note that for this solution strategy any ODE solver can be used to solve the discretized PDE on each subinterval. Furthermore, while pseudospectral methods are the best method for PDE-constrained optimization problems involving integro-PDE constraints, as discussed in Section \ref{secCompareFEMFDMPDM}, it is in principle possible to use other time or space discretization methods.
\subsubsection*{One-Dimensional Diffusion with a Non-Local Term}\label{secOneDDiffusionNonlocalOptim1}
The problem (\ref{sysOptimalHeating1}) is extended by adding a non-local term to the forced heat equation. This non-local term is the two body interaction term that is introduced in Section \ref{secOptimalitySysNonLocal1}.
The one dimensional PDE-constrained optimization problem is:
\begin{align}
&\min_{\rho,u} \quad \frac{1}{2}\norm{\rho- \hat{\rho}}_{L_2}^2 + \frac{\beta}{2} \norm{u}_{L_2}^2,\\
&\text{subject to:}\notag 
\notag \\
&\partial_t \rho - \Delta \rho - u - \nabla \cdot \rho(r) \int_\Omega \nabla V_2(|r-r'|) \rho(r')dr'=0,  \quad \text{in} \quad \Omega,\notag 
\notag \\
&\rho(r,0)=\rho_0(r),\notag 
\notag \\
& \rho(r,t)=0, \quad \text{on} \quad \partial \Omega. \notag 
\end{align}
The first-order optimality system, including the non-local term, is:
\begin{align}\label{sysoptinonlocandheat1D}
\textbf{Adjoint Equation}  \\
\partial_t  p  + \partial_{rr} p + \int_\Omega \bigg(\partial_r  p(r)+\partial_{r'}  p(r')\bigg) \rho(r') \partial_r V_2(|r-r'|) dr' &=(\rho- \hat{\rho})  \quad \text{in} \quad \Omega, \notag \\
p(T) &= 0 \notag\\
p(r,t) &=0 \quad \quad \quad \quad \text{on}\quad \partial\Omega, \notag\\
\textbf{Gradient Equation} \notag \\
\beta u  - \rho  &=0 \quad\quad\quad\quad \text{in} \quad \Omega, \notag \\
\textbf{Forward Problem} \notag \\
\partial_t \rho - \partial_{rr} \rho-u - \partial_r  \rho(r) \int_\Omega \partial_r V_2(|r-r'|) \rho(r')dr' &=0 \quad\quad\quad\quad \text{in} \quad \Omega, \notag \\ 
\rho(r,0)&=\rho_0(r), \notag \\
\rho(r,t) &=0 \quad\quad\quad \quad \text{on} \quad \partial \Omega. \notag
\end{align}
This is directly derived from taking the relevant terms in (\ref{sysFirstOderOptimalityNonLocal1}).
The system that has to be solved on each interval $[t_i,t_{i+1}]$ is, equivalent to (\ref{sysOptimDiffAdjBW2}):
\begin{align*}
\partial_t \rho(r,t) &= \partial_{rr}\rho(r,t) + \frac{1}{\beta}{p}(r,t) + \partial_r  \rho(r,t) \int_\Omega \partial_r V_2(|r-r'|) \rho(r',t)dr', \\
\partial_t {p}(r,\tilde\tau) &= \partial_{rr}{p}(r,\tilde\tau) - \rho(r,\tilde\tau) +\hat{\rho}(r,\tilde\tau) -\int_\Omega \bigg(\partial_r  p(r,\tilde\tau)+\partial_{r'}  p(r',\tilde\tau)\bigg) \rho(r',\tilde\tau) \partial_r V_2(|r-r'|) dr', \notag \\
\text{Initial } & \text{Conditions:} \notag  \\
\rho(r,t_i)&=\rho_{t_i}(r), \quad \quad\text{for} \ \quad t=t_i, \notag \\
{p}(r,t_{i+1}) &=p_{t_{i+1}}(r), \  \quad \text{for} \quad t=t_{i+1}, \notag \\
\text{Dirichl}&\text{et Boundary Conditions:} \notag \\
\rho(a,t) &=\rho(b,t)=0, \notag \\
{p}(a,\tilde\tau) &={p}(b,\tilde\tau)=0, \notag 
\end{align*} 
where $\tilde\tau=t_{i+1} +t_i -t$, as before.
As discussed above, $V_2$ is the force between two particles at positions $r$ and $r'$. It is defined depending on the physical problem involved. For a first numerical test problem, the choice of $V_2$ is a Gaussian:
\begin{align}\label{eqn1Dgaussian}
V_2(x)= \alpha e^{-x^2}.
\end{align}
Then $\partial_r V_2$ satisfies:
\begin{align*}
\partial_r V_2(|r-r'|)= -2\alpha|r-r'| e^{-|r-r'|^2}.
\end{align*}
The specific problem that is solved numerically is:
\begin{align}\label{OptmSysNonloc1alpha}
\partial_t \rho(r,t) &= \partial_{rr}\rho(r,t) + \frac{1}{\beta}{p}(r,t) + \alpha \partial_r  \rho(r,t) \int_\Omega \partial_r  e^{-|r-r'|^2} \rho(r',t)dr', \\
\partial_t {p}(r,\tilde\tau) &= \partial_{rr}{p}(r,\tilde\tau) - \rho(r,\tilde\tau) +\hat{\rho}(r,\tilde\tau) - \alpha\int_\Omega \bigg(\partial_r  p(r,\tilde\tau)+\partial_{r'}  p(r',\tilde\tau)\bigg) \rho(r',\tilde\tau) \partial_r  e^{-|r-r'|^2} dr', \notag \\
\text{Initial } & \text{Conditions:} \notag  \\
\rho(r,0)&=\rho_0(r), \quad \text{for}  \quad t=0, \notag \\
{p}(r,0) &=0,\quad \ \ \quad \text{for} \quad \tilde \tau=0, \notag \\
\notag \\
\text{Dirichl}&\text{et Boundary Conditions:} \notag \\
\rho(a,t) &=\rho(b,t)=0, \notag \\
{p}(a,\tilde\tau) &={p}(b,\tilde\tau)=0. \notag 
\end{align} 

 While the solution method is similar to the approach in Section \ref{secNumericsOneDDiffusion1}, there are two key differences. Firstly, quadrature has to be used, employing (\ref{eqnClenCurtQuad}), to evaluate the integral terms in the optimality system. The other difficulty is that no exact solutions exist to this problem because of the complexity of the non-local term. Therefore, the main issue in solving (\ref{OptmSysNonloc1alpha}) is finding an initial guess on the Chebyshev time points $t_i$, which is close enough to the solution of the system, so that convergence to a continuous solution on the whole interval is possible. The initial guess is found by using a technique called continuation.
 \\
The parameter $\alpha$ in (\ref{OptmSysNonloc1alpha}) represents the strength of the contribution of the integral term to the system of PDEs. 
It can be varied to choose the strength of the particle interactions on the PDE solution. If $\alpha_0=0$, there is no particle interaction and the optimality system for the forced heat equation is recovered, compare to (\ref{sysOptimDiffAdjBW2}).
 Since the solution to that problem is known, the idea is to use the exact solution to the problem where $\alpha_0=0$ as an initial guess to the problem where $\alpha_1$ is non-zero but small.
The optimal initial guess for the problem involving $\alpha_1$ is found by multiple shooting and used as an initial guess for the problem with $\alpha_2$, where $\alpha_2>\alpha_1$. This process is repeated iteratively until a certain contribution of the integral term is reached, for example $\alpha=1$. 
\newline
Generally, the step size in $\alpha$ is not determined linearly, but chosen adaptively. This is because some parts of the problems may be easier to solve than others. If the change in $\alpha$ is chosen small, then the optimization function only needs a few iteration to find the optimal initial guess, based on the result of the previous step in $\alpha$. This is because the problem with $\alpha_{i+1}$ is similar to the problem involving $\alpha_i$ and therefore the optimal initial guesses for the two problems are close. The downside of this approach is that the problem has to be re-evaluated for many different values of $\alpha$, which is computationally expensive. If $\alpha_{i+1}$ is chosen to be considerably larger than $\alpha_i$ at each step, the problem has to be solved less times. However, the risk is that the optimal initial guess for the problem with $\alpha_i$ is not a suitable initial guess for the problem with $\alpha_{i+1}$, and that no solution is found.
There are many ways to change $\alpha$ adaptively and it depends greatly on the problem that is to be solved. 

\subsubsection*{Two-Dimensional Problems}\label{sec2Dprobsnum1}
The two dimensional version of the PDE constrained optimization problem (\ref{sysOptimalHeating1}) is treated with numerical methods equivalent to the ones used in one-dimensional case, as discussed in Section \ref{secNumericsOneDDiffusion1}. 
The corresponding optimality system is (\ref{sysPotimaltiyheatequn1}).
All the solution methods follow directly from the one-dimensional method presented in Section \ref{secNumericsOneDDiffusion1}. The only difference is that, instead of having a one-dimensional set of spatial Chebyshev points, a two-dimensional Chebyshev grid has to be used, making use of Kronecker products. This has been introduced in Section \ref{secPSMTheory1}.
One of the things to note is that the computational effort in two dimensions is much higher than for one dimensional calculations. This is because instead of $N$ spatial points, $N_1 \times N_2$ spatial points have to be evaluated.

\subsubsection*{Two-Dimensional Problems with a Non-Local Term}
Adding a non-local term to (\ref{sysOptimalHeating1}), the PDE-constrained optimization problem becomes:
\begin{align*}
&\min_{\rho,u} \quad \frac{1}{2}\norm{\rho- \hat{\rho}}_{L_2}^2 + \frac{\beta}{2} \norm{u}_{L_2}^2,\\
\\
&\text{subject to:}\\
&\partial_t \rho = \Delta \rho + {u} + \nabla \cdot \int_\Omega \rho(r) \rho(r') \nabla V_2(|r-r'|)dr'  \quad \text{in} \quad \Omega,
\\
&\rho(r,0)=\rho_0(r),
\\
& \rho(r,t)=0, \quad\quad\quad\quad\quad\quad\quad\quad\quad\quad\quad\quad\quad\quad\quad\quad \quad  \text{on} \quad \partial \Omega,
\end{align*}
where $r=(x,y) \in \mathbf{R}^2$.
The corresponding optimality system is:
\begin{align*}
\textbf{Adjoint Equation}  \\
\partial_t p(r,\tau) &= \Delta p(r,\tau) -\rho(r,\tau) +\hat{\rho}(r,\tau)\\
&-\int_\Omega \bigg( \nabla_r p(r,\tau) + \nabla_{r'} p(r',\tau) \bigg) \rho(r',t) \nabla_r V_2(|r-r'|)dr'  \quad\quad\quad   \text{in} \quad \Omega,\\
p(T) &= 0 \notag\\
p(r,t) &=0 \quad \quad \quad  \quad\quad\quad\quad\quad\quad\quad\quad\quad\quad\quad\quad\quad\quad\quad\quad\quad\quad\quad\quad \quad\ \ \ \text{on}\quad \partial\Omega, \notag\\
\textbf{Gradient Equation} \notag \\
\beta u  - \rho  &=0 \quad\quad\quad\quad \quad\quad\quad\quad\quad\quad\quad\quad\quad\quad\quad\quad\quad\quad\quad\quad\quad\quad\quad\quad\quad\text{in} \quad \Omega, \notag \\
\textbf{Forward Problem} \notag \\
\partial_t \rho(r,t) &= \Delta \rho(r,t) + u(r,t)+  \nabla_r \cdot \int_\Omega \rho(r,t) \rho(r',t) \nabla_r V_2(|r-r'|)dr' \quad \ \text{in} \quad \Omega,\\  
\rho(r,0)&=\rho_0(r), \notag \\
\rho(r,t) &=0 \quad\quad\quad \quad\quad\quad\quad\quad\quad\quad\quad\quad\quad\quad\quad\quad\quad \quad\quad\quad\quad\quad\quad\quad \ \ \ \text{on} \quad \partial \Omega, \notag
\end{align*}
compare with the one-dimensional system (\ref{sysoptinonlocandheat1D}).
Equivalently to the one-dimensional particle interaction term, (\ref{eqn1Dgaussian}), $V_2$ is the two-dimensional Gaussian:
\begin{align*}
V_2(x,y)= \alpha e^{-(x^2 +y^2)}.
\end{align*}
When substituting the gradient equation into the state equation, the optimality system becomes:
\begin{align}\label{OptmSysNonloc1alpha2D}
\partial_t \rho(r,t) &= \Delta\rho(r,t) + \frac{1}{\beta}{p}(r,t) + \alpha \nabla_r \cdot  \int_\Omega \nabla_r\bigg(  e^{-|r-r'|^2} \bigg)\rho(r',t)\rho(r,t)dr', \\
\partial_t {p}(r,\tilde\tau) &=\Delta {p}(r,\tilde\tau) - \rho(r,\tilde\tau) +\hat{\rho}(r,\tilde\tau) - \alpha\int_\Omega \bigg(\nabla_r  p(r,\tilde\tau)+\nabla_{r'}  p(r',\tilde\tau)\bigg) \cdot \nabla_r \bigg(  e^{-|r-r'|^2} \bigg)\rho(r',\tilde\tau)  dr', \notag \\
\text{Initial } & \text{Conditions:} \notag  \\
\rho(r,0)&=\rho_0(r), \quad \text{for}  \quad t=0, \notag \\
{p}(r,0) &=0,\quad \ \ \quad \text{for} \quad \tilde \tau=0, \notag \\
\text{Dirichl}&\text{et Boundary Conditions:} \notag \\
\rho(a,t) &=\rho(b,t)=0, \notag \\
{p}(a,\tilde\tau) &={p}(b,\tilde\tau)=0, \notag 
\end{align}
where $\tilde \tau= T+t_0 -t$. 
The method for solving this system, including multiple shooting and continuation, follows exactly from the one-dimensional approach discussed in Section \ref{secOneDDiffusionNonlocalOptim1}.



	
\section*{Year 2}
	
\section{Literature Review on Mean Field Optimal Control}
\input{MeanFieldOCPReview.tex}

\section{Relevant Equations and their Optimality Conditions}
This section is concerned with discussing the different equations that have been studied in the past year and the derivation of the optimality conditions for optimal control problems with constrains involving these equations. The discussion of the different PDE models and the connections between them is heavily based on \cite{Archer1}.

\subsection{The Equations}

\input{ArcherDerivation1.tex}


\subsection{Optimality Conditions for the Inertial Eqations} \label{sec:INOptimalityConditions}

\input{ArcherOptimalityConditionsIN1.tex}

	
\subsection{Optimality Conditions for the Overdamped Equations}	
	
The optimality conditions for the optimal control problems involving the overdamped equations \eqref{eqn:ADeqn1} are stated here for completion. The details of their derivation can be found in either the year one report or in our paper `PDE-Constrained Optimization Models and Pseudospectral Methods for Multiscale Particle Dynamics'.
Two optimal control problems are considered; one which applies the control through the flow field, as in Section \ref{sec:INOptimalityConditions}, and another, where the control is an added source term in the PDE. The latter case is less physically relevant in applications, however, it is often a simpler problem to study, because the control is applied linearly, while the flow control problem a non-linear control is used. For each problem, no-flux and Dirichlet boundary conditions are considered. Note that, for ease of notation, we set $\tilde t = t$.
The flow control problem is:
\begin{align}
\label{eqn:ADFlowOCP}
&\min_{\rho,\mathbf{w} } \mathcal J(\rho,\mathbf{w} ) =\ \ \frac{1}{2}||\rho - \widehat \rho||_{L_2(\Sigma)}^2  +\frac{\beta}{2}||\mathbf{w}||_{L_2(\Sigma)}^2\\
& \text{subject to:}: \notag\\
&\frac{\partial \rho}{\partial t} = \nabla \cdot (\rho\f) - \nabla \cdot (\rho \mathbf{w})  + \nabla \cdot (\rho\nabla V_{ext}) + \nabla \cdot (\nabla \rho) +\nabla \cdot \int_\Omega \rho(r)\rho(r') \nabla V_2(|r-r'|)dr'. \notag
\end{align}
The adjoint and gradient equations are:
\begin{align*}
\frac{\partial \Adjb}{\partial t} =& - \nabla^2\Adjb - \mathbf{w} \cdot \nabla \Adjb + \nabla V_{ext} \cdot \nabla \Adjb - \rho + \widehat \rho+\int_\Omega (\nabla_r \Adjb(r) - \nabla_{r'} \Adjb(r') ) \rho(r') \mathbf{K}(r,r') dr'\\
\mathbf{w} =& - \frac{1}{\beta} \rho \nabla \Adjb,
\end{align*}
where $\Adjb$ is the adjoint variable. The condition $\frac{\partial \Adjb}{\partial n} = 0$ on $\partial \Omega$ corresponds to a no-flux boundary condition, while $\Adjb = 0$ on $\partial \Omega$ corresponds to a Dirichlet boundary condition.

The source control problem is:
\begin{align}
\label{eqn:AdvDiffLinear}
&\min_{\rho,{w} } \mathcal J(\rho,{w}) = \ \ \frac{1}{2}||\rho - \widehat \rho||_{L_2(\Sigma)}^2  +\frac{\beta}{2}||{w}||_{L_2(\Sigma)}^2\\
& \text{subject to:}: \notag \\
&\frac{\partial \rho}{\partial  t} = \nabla \cdot (\rho\f) + \nabla \cdot (\rho\nabla V_{ext}) + \nabla \cdot (\nabla \rho) + w \notag +\nabla \cdot \int_\Omega \rho(r)\rho(r') \nabla V_2(|r-r'|)dr' \notag
\end{align}

The adjoint and gradient equations are:
\begin{align*}
\frac{\partial \Adjb}{\partial t} =& - \nabla^2 \Adjb + \nabla V_{ext} \cdot \nabla \Adjb - \rho + \widehat \rho +\int_\Omega (\nabla_r \Adjb(r) - \nabla_{r'} \Adjb(r') ) \rho(r') \mathbf{K}(r,r') dr'\\
\mathbf{w} =& - \frac{1}{\beta} \Adjb.
\end{align*}
Boundary conditions for the adjoint equation are applied analogously to the flow control problem.

	

\subsection{Subdomain and Boundary Observation with Non-Constant Flux}

The first problem of interest is of the form:



In this section, two optimal control problems involving the overdamped equations are discussed briefly. The differences to the standard optimal control problem, considered in the previous section, are that a non-constant flux is considered instead of a no-flux boundary condition and that observations are made on a subdomain $\Sigma_{Ob}$, or on parts of the boundary, instead of the whole space-time domain $\Sigma$. For illustration, the control is only applied linearly through a source term, but the results follow analogously for the flow control problem. 
\begin{figure}[h]
	\includegraphics[scale=0.8]{observation.png}
	\caption{Domain of Interest}
	\label{Observation1}
\end{figure}
The first problem of interest is of the form:
\begin{align*}
&\min_{\Sta, w} \quad \frac{1}{2}|| \Sta -\widehat \Sta||^2_{L_2( \Sigma_{Ob})} + \frac{\beta}{2}||w||^2_{L_2(\Sigma)}\\
&\text{subject to:}\\
&\partial_t \rho = \nabla^2 \rho - \nabla \cdot (\rho \mathbf{w}) +\nabla \cdot (\rho \nabla V_{ext}) + \nabla \cdot \int_\Omega \rho(r) \rho(r') \nabla V_2(|r-r'|) dr' + w \quad  \text{in} \quad \Sigma\notag\\
& \rho = \rho_0 \quad \text{at} \quad t=0 \notag\\
& - \mathbf{j} \cdot \nor = \mathbbm{1}_{\partial \Omega_L}( C_{L1}  + C_{L2}\Sta) +\mathbbm{1}_{\partial \Omega_R} ( C_{R1}  + C_{R2}\Sta) +\mathbbm{1}_{\partial \Omega_I} 0 \ \quad \quad\qquad\qquad  \qquad \text{on} \quad \partial \Omega, 
\end{align*}
where $C_{L1}, C_{L2}, C_{R1}$, $C_{R2}$ are constants and $\mathbbm{1}$ is the indicator function of the set of interest. Considering Figure \ref{Observation1}, the stated non-constant flux boundary condition provides the option of describing a non-constant inflow on boundary $\partial \Omega_L$ and a non-constant outflow on $\partial \Omega_R$, while keeping a no flux condition on the rest of the boundary, denoted by $\partial \Omega_I$.
Furthermore, $\mathbf{j}$ satisfies:
\begin{align*}
\mathbf{j}=\nabla \rho - (\rho \mathbf{w}) +(\rho \nabla V_{ext}) +  \int_\Omega \rho(r) \rho(r') \nabla V_2(|r-r'|) dr'.
\end{align*}
Moreover, let $\widehat \Sta$ be defined such that:
\begin{align*}
\widehat \Sta = \mathbbm{1}_{ \Omega_{Ob1}} \tilde \Sta  +\mathbbm{1}_{ \Omega_{Ob2}} 0.
\end{align*}
This describes a desired state where the particle mass accumulates in the observation domain $\Omega_{Ob1}$ and no particles are found in $\Omega_{Ob1}$. Since observations are only taken on $\Omega_{Ob}$, there is no prescribed desired state on $\Omega / \Omega_{Ob}$.\\
The Lagrangian is of the form:
\begin{align*}
\mathcal{L}(\Sta,w,\Adjb,\Adjc ) &=\frac{1}{2} \int_0^T \int_{\Omega_{Ob}} (\Sta - \widehat \Sta)^2 dr dt + \frac{\beta}{2}\int_0^T \int_\Omega w^2 drdt \\
&+ \int_0^T \int_\Omega \bigg( \partial_t \rho - \nabla^2 \rho + \nabla \cdot (\rho \mathbf{w}) -\nabla \cdot (\rho \nabla V_{ext}) \\
&+ \nabla \cdot \int_\Omega \rho(r) \rho(r') \nabla V_2(|r-r'|) -w \bigg) \Adjb dr dt\\
&+ \int_0^T \int_{\partial \Omega} \bigg(  \bigg(-\nabla \rho+ (\rho \mathbf{w}) -(\rho \nabla V_{ext}) -  \int_\Omega \rho(r) \rho(r') \nabla V_2(|r-r'|) dr' \bigg)\cdot \nor\\
&  -\mathbbm{1}_{\partial \Omega_L}( C_{L1}  + C_{L2}\Sta) -\mathbbm{1}_{\partial \Omega_R} ( C_{R1}  + C_{R2}\Sta) -\mathbbm{1}_{\partial \Omega_I} 0 \bigg) \Adjc dr dt.
\end{align*}
The derivative of $\mathcal{L}$ with respect to $\rho$ is, as taken from the extended project, is:
\begin{align*}
&\mathcal{L}_\rho (\rho,{w},\Adjb,\Adjc)h=
\int_\Omega h(T) \Adjb(T) dr\\
&+ \int_0^T \int_\Omega \bigg( \mathbbm{1}_{ \Omega_{Ob}} (\rho- \widehat{\rho})  - \partial_t \Adjb  - \nabla \Adjb \cdot \mathbf{w}  - \nabla^2 \Adjb \notag 
+  \nabla \Adjb \cdot \nabla V_{ext}  \notag \\
&+ \int_\Omega (\nabla  \Adjb(r)+\nabla  \Adjb(r')) \rho(r') \nabla V_2(|r-r'|) dr'+ \int_{\partial \Omega} ( \Adjc(r') - \Adjb(r')) \rho(r')   \frac{\partial V_2(|r-r'|)}{\partial n} dr' \bigg) h dr dt \\
&+  \int_0^T\int_{\partial \Omega}  \bigg(
\bigg(\frac{\partial \Adjb }{\partial n} + \Adjb  \mathbf{w} \cdot \mathbf{n} - \Adjc \mathbf{w} \cdot \mathbf{n}  +  \Adjc \dfrac{\partial V_{ext}}{\partial n} - \Adjb \frac{\partial V_{ext}}{\partial n} + ( \Adjc - \Adjb)  \int_\Omega \rho(r') \frac{\partial V_2(|r-r'|)}{\partial n} dr'\\
& -\mathbbm{1}_{\partial \Omega_L} C_{L2} \Adjc   -\mathbbm{1}_{\partial \Omega_R} C_{R2} \Adjc \bigg)h + \bigg( \Adjc- \Adjb \bigg) \frac{\partial h}{\partial n} \bigg) dr dt =0.
\end{align*}
Then, from appropriate analysis we find that:
\begin{align*}
\Adjc = \Adjb,
\end{align*}
and therefore we get:
\begin{align*}
\mathbbm{1}_{\Omega_{Ob}}(\rho- \widehat{\rho})   - \partial_t  \Adjb  - \nabla \Adjb \cdot \mathbf{w}  - \nabla^2 \Adjb \notag 
+  \nabla \Adjb \cdot \nabla V_{ext}  \notag \\
+ \int_\Omega (\nabla  \Adjb(r)+\nabla  \Adjb(r')) \rho(r') \nabla V_2(|r-r'|) dr' &=0, \quad \text{in} \quad \Sigma, \\
\frac{\partial \Adjb }{\partial n}  -\mathbbm{1}_{\partial \Omega_L} C_{L2} \Adjb   -\mathbbm{1}_{\partial \Omega_R} C_{R2} \Adjb&=0, \quad \text{on} \quad \partial \Omega.
\end{align*}
In particular, this is:
\begin{align*}
\mathbbm{1}_{\Omega_{Ob1}}(\rho- \widehat{\rho}) +\mathbbm{1}_{\Omega_{Ob2}}\rho  - \partial_t  \Adjb  - \nabla \Adjb \cdot \mathbf{w}  - \nabla^2 \Adjb \notag 
+  \nabla \Adjb \cdot \nabla V_{ext}  \notag \\
+ \int_\Omega (\nabla  \Adjb(r)+\nabla  \Adjb(r')) \rho(r') \nabla V_2(|r-r'|) dr' &=0, \quad \text{in} \quad \Sigma, \\
\frac{\partial \Adjb }{\partial n}  -\mathbbm{1}_{\partial \Omega_L} C_{L2} \Adjb   -\mathbbm{1}_{\partial \Omega_R} C_{R2} \Adjb&=0, \quad \text{on} \quad \partial \Omega.
\end{align*}
The gradient equation is:
\begin{align*}
w = \frac{1}{\beta}\Adjb.
\end{align*}
Comparing this to the previous section, it can be observed that the gradient equations have opposite signs. This is due to a different construction of the Lagrangian.





The problem of interest is of the form:
\begin{align*}
&\min_{\Sta, \Con} \quad \frac{1}{2}|| \Sta -\hat \Sta||^2_{L_2(\partial Q_R)} + \frac{\beta}{2}|| \Con||^2_{L_2(Q)}\\
\text{subject to:}\\
&\partial_t \rho = \nabla^2 \rho - \nabla \cdot (\rho \mathbf{w}) +\nabla \cdot (\rho \nabla V_{ext}) + \nabla \cdot \int_\Omega \rho(r) \rho(r') \nabla V_2(|r-r'|) dr' + \Con \quad  \quad\text{in} \quad Q,\notag\\
& \rho = \rho_0 \quad \text{at} \quad t=0 \notag\\
& - \mathbf{j} \cdot \nor = \mathbbm{1}_{\partial \Omega_L}( C_{L1}  + C_{L2}\Sta) +\mathbbm{1}_{\partial \Omega_R} ( C_{R1}  + C_{R2}\Sta) +\mathbbm{1}_{\partial \Omega_I} 0, \quad  \quad\text{on} \quad \partial Q, 
\end{align*}
where $C_{L1}, C_{L2}, C_{R1}$, $C_{R2}$ are constants and $\mathbbm{1}$ is the indicator function of the set (the parts of the boundary) of interest.
Furthermore, $\mathbf{j}$ satisfies:
\begin{align*}
\mathbf{j}=\nabla \rho - (\rho \mathbf{w}) +(\rho \nabla V_{ext}) +  \int_\Omega \rho(r) \rho(r') \nabla V_2(|r-r'|) dr'.
\end{align*}
Moreover, let $\hat \Sta$ be defined such that:
\begin{align*}
\hat \Sta = \mathbbm{1}_{\partial \Omega_{R1}} \tilde \Sta  +\mathbbm{1}_{\partial \Omega_{R2}} 0.
\end{align*}

\subsection*{The Lagrangian}
The Lagrangian is of the form:
\begin{align*}
\mathcal{L}(\Sta,\Con,\Adja,\Adjc ) &= \int_0^T \int_{\partial \Omega_R} \frac{1}{2}(\Sta - \hat \Sta)^2 dr dt + \frac{\beta}{2}\int_0^T \int_\Omega \Con^2 drdt \\
&+ \int_0^T \int_\Omega \bigg( \partial_t \rho - \nabla^2 \rho + \nabla \cdot (\rho \mathbf{w}) -\nabla \cdot (\rho \nabla V_{ext}) + \nabla \cdot \int_\Omega \rho(r) \rho(r') \nabla V_2(|r-r'|) \bigg) \Adja dr dt\\
&+ \int_0^T \int_{\partial \Omega} \bigg(  \bigg(-\nabla \rho+ (\rho \mathbf{w}) -(\rho \nabla V_{ext}) -  \int_\Omega \rho(r) \rho(r') \nabla V_2(|r-r'|) dr' \bigg)\cdot \nor\\
&  -\mathbbm{1}_{\partial \Omega_L}( C_{L1}  + C_{L2}\Sta) -\mathbbm{1}_{\partial \Omega_R} ( C_{R1}  + C_{R2}\Sta) -\mathbbm{1}_{\partial \Omega_I} 0 \bigg) \Adjc dr dt.
\end{align*}

\subsection*{The Adjoint Equation}
The derivative of $\mathcal{L}$ with respect to $\rho$ is, as taken from the extended project:
\begin{align*}
&\mathcal{L}_\rho (\rho,\mathbf{w},p_\Omega,p_{\partial \Omega})h=
\int_\Omega h(T) \Adja(T) dr\\
&+ \int_0^T \int_\Omega \bigg(   - \partial_t \Adja  - \nabla \Adja \cdot \mathbf{w}  - \nabla^2 \Adja \notag 
+  \nabla \Adja \cdot \nabla V_{ext}  \notag \\
&+ \int_\Omega (\nabla  \Adja(r)+\nabla  \Adja(r')) \rho(r') \nabla V_2(|r-r'|) dr'+ \int_{\partial \Omega} ( \Adjc(r') - \Adja(r')) \rho(r')   \frac{\partial V_2(|r-r'|)}{\partial n} dr' \bigg) h dr dt \\
&+  \int_0^T\int_{\partial \Omega}  \bigg(
\bigg(\frac{\partial \Adja }{\partial n} + \Adja  \mathbf{w} \cdot \mathbf{n} - \Adjc \mathbf{w} \cdot \mathbf{n}  +  \Adjc \dfrac{\partial V_{ext}}{\partial n} - \Adja \frac{\partial V_{ext}}{\partial n} + ( \Adjc - \Adja)  \int_\Omega \rho(r') \frac{\partial V_2(|r-r'|)}{\partial n} dr'\\
&\mathbbm{1}_{\partial \Omega_R} (\rho- \hat{\rho}) -\mathbbm{1}_{\partial \Omega_L} C_{L2} \Adjc   -\mathbbm{1}_{\partial \Omega_R} C_{R2} \Adjc \bigg)h + \bigg( \Adjc- \Adja \bigg) \frac{\partial h}{\partial n} \bigg) dr dt =0.
\end{align*}
Then, from appropriate analysis we find that:
\begin{align*}
\Adjc = \Adja,
\end{align*}
and therefore we get:
\begin{align*}
- \partial_t  \Adja  - \nabla \Adja \cdot \mathbf{w}  - \nabla^2 \Adja \notag 
+  \nabla \Adja \cdot \nabla V_{ext}  \notag \\
+ \int_\Omega (\nabla  \Adja(r)+\nabla  \Adja(r')) \rho(r') \nabla V_2(|r-r'|) dr' &=0, \quad \text{in} \quad Q, \\
\frac{\partial \Adja }{\partial n}+ \mathbbm{1}_{\partial \Omega_R} (\rho- \hat{\rho}) -\mathbbm{1}_{\partial \Omega_L} C_{L2} \Adja   -\mathbbm{1}_{\partial \Omega_R} C_{R2} \Adja&=0, \quad \text{on} \quad \partial Q.
\end{align*}
Again, in particular the boundary condition is:
\begin{align*}
\frac{\partial \Adja }{\partial n}+ \mathbbm{1}_{\partial \Omega_{R1}}(\rho- \tilde{\rho} -C_{R2} \Adja) + \mathbbm{1}_{\partial \Omega_{R2}} (\Sta-C_{R2} \Adja) - \mathbbm{1}_{\partial \Omega_L} C_{L2} \Adja   &=0, \quad \text{on} \quad \partial Q.
\end{align*}

	
	
	
\section{Numerical Methods} \label{sec:NumericalMethods}	
	
	In this section, the numerical methods used in the computational implementation are discussed. Methods which have been covered in the year one report are omitted.
	In general it is necessary to change the time variable in the adjoint equation, as demonstrated in Section \ref{sec:INImplementation}, for numerical stability. This is necessary because the forward and adjoint equations contain Laplacians of opposite sign. Running the adjoint equation with a negative Laplacian leads to a blow up of the solution at the first time step. The reversal of time, using $\tau = T-t$, remedies this issue, however, this causes a non-local coupling in time between the two PDEs.
	The following algorithms provide methods of treating this non-local coupling.
	
	\subsection{Fixed Point Algorithm}\label{sec:Method_SolverFP}
	
	In this section we describe the fixed point algorithm, which is an efficient and stable optimization method for the optimal control problems considered above. 
We denote the discretized versions of the variables $\rho$, $\adj$ and $\mathbf{w}$ with $P$, $Q$ and $W$, respectively. Each of these matrices is of the form $A = [\boldsymbol{a_0}, \boldsymbol{a_1}, ... ,\boldsymbol{a_n}]$, where the vectors $\boldsymbol{a_k}$ represent the solutions at the discretized times $k \in 0,1,....,n$, where $n$ is the number of time points. In particular, the first column of $P$, denoted by $\boldsymbol{\rho_0}$, corresponds to the initial condition $\rho(r,0)$. If the spatial domain is one-dimensional, $P$, $Q$ and $W$ are of size $N \times (n + 1)$, where $N$ is the number of spatial points. In the two-dimensional case, $P$ and $Q$ are of size $(N_1N_2) \times (n + 1)$, where $N_1$ is the number of spatial points in the direction of $x_1$ and $N_2$ the points along the $x_2$ axis. Generally, $N_1 = N_2$. The discretized control $W$ for linear control problems is also $(N_1N_2) \times (n + 1)$ dimensional, while it is $(2N_1N_2) \times (n + 1)$ dimensional for nonlinear control problems. This is due to the fact that the control is represented by a vector field, when applied nonlinearly.
\\
\\
The optimization algorithm is initialized with a guess for the control, $W^{(0)}$. Then, in each iteration, denoted by $i$, the following steps are computed:
\vspace{0.1cm}
\begin{enumerate}
	\item Starting with a guess for the control $W^{(i)}$ as input variable, the corresponding state $P^{(i)}$ is found by solving the forward equation.
	\item In a next step, the value of the adjoint, $Q^{(i)}$, is established by computing the adjoint equation, using $W^{(i)}$ and $P^{(i)}$ as inputs. Since $P^{(i)}$ contains the solution for all discretized times $k \in 0,1,...,n$, this circumvents issues resulting from the non-local coupling in time, resulting from reversing time in the adjoint equation. As illustrated in the same section, time is reversed in the adjoint equation, so that the result is a matrix $\tilde{Q}^{(i)} =  [\boldsymbol{\adj_n},\boldsymbol{\adj_{n-1}}, ..., \boldsymbol{\adj_1} ]$. The columns of $\tilde{Q}^{(i)}$ are permuted to obtain the solution  $Q^{(i)}$.
	\item The gradient equation is solved, given the solutions $P^{(i)}$ and $Q^{(i)}$. This results in a new value for the control, $W^{(i)}_g$.
	\item  The convergence of the optimization scheme is measured by computing the error between $W^{(i)}$ and $W^{(i)}_{g}$. The error measure, $\mathcal{E}$, is discussed in detail in Section \ref{sec:ErrorMeasure}. 
	\begin{itemize}
		\item  If this error is lower than a set tolerance, the optimality system is self-consistent. This implies that the solution triplet ($\bar{P},\bar{W},\bar{Q}$) solves the (discretized) optimality system, and is therefore an optimal solution to the PDE-constrained optimization problem of interest.
		\item If the error is above the optimality tolerance, Step 5 is executed.
	\end{itemize}
	\item Finally, the update $W^{(i+1)}$ is a linear combination of the current guess $W^{(i)}$, and the value obtained in Step 3, $W^{(i)}_{g}$, employing a mixing rate $\lambda \in [0,1]$:
	\begin{align*}
	W^{(i+1)} = (1-\lambda)W^{(i)} + \lambda W^{(i)}_{g}.
	\end{align*}
	The guess for the control is updated from $W^{(i)} $ to $W^{(i+1)} $ and Steps 1-5 are repeated until the method converges. 
\end{enumerate}
\vspace{0.3cm}
The update scheme in Step 5, with mixing rate $\lambda$, is known to stabilise fixed point methods, see \cite{Roth1}. Typical values of $\lambda$, which provide stable convergence, lie between $0.1$ and $0.001$. Throughout this paper, $\lambda =0.01$, unless stated otherwise. This mixing scheme is similar to the updating scheme presented in~\cite{Burger1}. 
Note that, while the solutions $P^{(i)}$ and $Q^{(i)}$ change in each iteration, the initial condition $\boldsymbol{\rho_0}$ and final time condition $\boldsymbol{\adj_n}$ remain unchanged throughout the process. Therefore, the only variable inducing a change in the solution is $W^{(i)}$.
	
	\subsection{Picard Multiple Shooting}
	
	
The multiple shooting algorithm, introduced in the first year report, has been extended by employing a Picard mixing scheme to replace the {\scshape MATLAB} inbuilt solver \texttt{fsolve}. In the following, this is briefly outlined.
The idea of the updating scheme is similar to the one presented for the fixed point algorithm. However, while the fixed point algorithm updates through the control variable, the fixed point algorithm updates via the variables $\rho$ and $q$.
The multiple shooting method consists of discretizing the time interval and solving the optimality system on each interval individually. This is done because of the non-local time coupling of the forward and adjoint equations. It requires the input of an initial guess at each discretized time point for each of the variables. The aim of the optimization solver is then to minimize the distance between the initial guesses and numerical solutions of the variables at each of the time points. \\
The Picard mixing scheme is a fixed point type algorithm. At each iteration $i$ it takes a set of guesses at the discretized time points, denoted by $Y_i$. The matrix $Y = [P,Q]$ contains the discretized values for the variables $\rho$ and $q$, denoted by $P$ and $Q$, analogously to the previous section.  
The system of PDEs is solved on each of the discretized intervals and a new set of variable values at the time points is created, denoted by $Y_{out}$. Then, the mixing scheme provides a new guess for the iteration $i+1$:
\begin{align*}
Y_{i+1} = (1 - \lambda)Y_i + \lambda Y_{out},
\end{align*}
where $\lambda$ is the mixing rate. It typically takes values between $0.1$ and $0.01$, depending on the complexity of the system to solve. Choosing a relatively small value of $\lambda$ stabilizes the algorithm. 
The algorithm terminates when the system of PDEs is solved self-consistently, i.e. when the distance between $Y_i$ and $Y_{out}$ is small, as measured in a chosen norm. The most frequently applied norm is discussed in Section \ref{sec:ErrorMeasure}.
This algorithm is working very well for examples involving the overdamped equations. However, the fixed point algorithm provides an even simpler method, which does not require the solution of the optimality system on small time intervals and is therefore even quicker. Since we will apply the numerical optimization method to increasingly difficult optimal control problems in the future, the multiple shooting algorithm may provide more numerical stability for numerically harder problems and is therefore a relevant tool to consider in the future. Changing the optimization solver in the implementation is straightforward and only requires changing a flag in the input file.
\\
A challenge with this solver is, that it needs to be provided with good initial guesses for the variables $\rho$ and $\adj$ at the discretized time points. The guess for $\rho$ can be obtained by solving the associated forward problem and using the result as a first guess. However, a good guess for $\adj$ is trickier to obtain. One way of doing so is by using the gradient equation, which relates $\rho$, $\adj$ and $\mathbf w$, the control. Since the input for the forward control is known, one can use this information, together with the initial guess for $\rho$, to construct an initial guess for $\adj$. 
One challenge however arises when considering the flow control problem involving the overdamped equations. The gradient equation is $\mathbf{w} = - \frac{1}{\beta} \rho\nabla \adj$. In order to derive the value of $\adj$ from this equation, we need to divide by $\rho$, making use of the assumption that $\rho$ is strictly positive, and integrate over the whole space. The issue here is that integration introduces an indeterminable constant. Furthermore, if Dirichlet boundary conditions are applied, the strict positivity of $\rho$ is in question.\\
An alternative method of obtaining an initial guess for $\adj$ is to perform one step of the fixed point method.

	
	\subsection{{\scshape MATLAB}'s Inbuilt Optimization Solver \texttt{fsolve}} \label{sec:fsolvedescription}
	
	Another option of solving the optimality system is using the inbuilt {\scshape MATLAB} solver \texttt{fsolve}, in combination with the multiple shooting method, briefly described in the previous section. The optimization solver tries to minimize the error in the variables $\rho$ and $\adj$ at the discretized time points. 
\\
In general, for the set of non-linear equations, $F(x) =0$, that are supposed to be solved, \texttt{fsolve} tries to find an input vector $x$, such that we minimize the sum of squares $\sum_i f_i(x)^2$, where $f_i$ are the components of $F$. 
While \texttt{fsolve} has three different algorithm options, the default algorithm, used in solving our optimality systems, is the trust region dogleg algorithm, a variant of Powell's dogleg algorithm, see \cite{Powell1}.   
The general idea of trust-region algorithms is to consider a so-called trust-region $N$, in which the function $F$ is approximated by a simpler function. Then, a search direction $s$ is defined and it is checked whether $F(x+s) < F(x)$. If that is the case, the position $x$ is updated to the position $x+s$. Otherwise, we remain at the position $x$ and the trust region $N$ is made smaller. Convergence is achieved when $F(x)$ and $F(x+s)$ are close.
The main questions are (i) how to approximate the function in the trust region, and (ii) how to determine the search direction $s$ reliably.\\
In the case of the dogleg algorithm, the choice for (i) is to minimize the linear approximation:
\begin{align}
\label{eqn:trustregionsubprob1}
\min_s m(s) &= \frac{1}{2}|| F(x_k) + J(x_k)s||_2^2 \\
&= \frac{1}{2}F(x_k)^T F(x_k) + s^T J(x_k)^T F(x_k) + \frac{1}{2}s^T J(x_k)^TJ(x_k)s, \notag
\end{align}
where $J$ is the Jacobian.
In order to minimize $m$, we choose, answering (ii), the appropriate search direction $s$. In the dogleg method this is done by combining a Gauss-Newton step $s_{GN}$ with a Cauchy step $s_C$.
If $J(x_k)$ is singular, $s = s_C$. Otherwise, $s$ is chosen as a linear combination of these two steps:
\begin{align*}
s = s_C + \lambda(s_{GN} - s_C),
\end{align*}
where $\lambda \in [0,1]$ is the largest value such that $||s|| \leq \Delta$. The positive scalar $\Delta$ is the trust region dimension, and is adjusted at each iteration. The algorithm converges when $F(x)$ and $F(x+s)$ are close, as measured by a certain norm. 
This method is more stable than a Newton method, and therefore the initial guess for $x$ does not have to be as good. Furthermore, it is cheaper to compute. However, it is also more prone to converging to local minima, since we do not consider the whole domain on which the problem is posed.
This section is based on \cite{Powell1} and \cite{fsolve1}.
	
	\section{Validation of the Optimization Algorithm} \label{sec:Validation}
	In this section, the measure of accuracy, used in the numerical experiments, is discussed, some validation methods and results are presented and further comments are made on general observations regarding the functionality of the numerical algorithm.
	
	\subsection{Error Measure}\label{sec:ErrorMeasure}
	
	While other norms such as an $L_1$ norm or a pointwise error measure have been considered, the main measure employed in this work is described in the following.

All errors in Sections \ref{sec:Validation} and \ref{sec:Examples} are calculated between a variable of interest, $y$, and $y_R$, the reference value that $y$ is compared to. When measuring convergence of the fixed point scheme, described in Section \ref{sec:Method_SolverFP}, $y = W^{(i)}_g$ and $y_R = W^{(i)}_i$. Alternatively, when investigating a known test problem, $y$ is a numerical solution and $y_R$ is an exact solution. The error measure $\mathcal{E}$ is composed of an $L^2$ error in space and an $L^\infty$ error in time. The relative $L^2$ error in the spatial direction is:
\begin{align*}
\mathcal{E}_{Rel}(t) = \frac{|| y(x,t) - y_{R}(x,t)||_{L^2(\Omega)} }{||y_R(x,t) ||_{L^2(\Omega)}+ 10^{-10}},
\end{align*}
where the small additional term on the denominator prevents division by zero.
Furthermore, the absolute $L^2$ error is:
\begin{align*}
\mathcal{E}_{Abs}(t) = || y(x,t) - y_R(x,t)||_{L^2(\Omega)}.
\end{align*}
Then, an $L^\infty$ error in time is taken of the minimum of $\mathcal{E}_{Rel}$ and $\mathcal{E}_{Abs}$, to obtain the error of interest:
\begin{align*}
\mathcal{E} = \max_{t \in [0,T]}\left[\min\left(\mathcal{E}_{Rel}(t), \mathcal{E}_{Abs}(t)\right)\right].
\end{align*}
The minimum between absolute and relative spatial error is taken to avoid choosing an erogenously large relative error, caused by division of one small term by another.


	
	\subsection{Validation Against \texttt{fsolve}}
	As a benchmark, we compared the fixed point scheme to Matlab's inbuilt \texttt{fsolve} function. It uses the trust-region-dogleg algorithm, see Section \ref{sec:fsolvedescription} and \cite{Powell1}, to solve the optimality system of interest. While it is very robust, it is also much slower than the fixed point method, which works reliably for the types of problems we set out to solve. 
	
Example 1 in Section \ref{sec:Examples1d} is considered to compare the computational time taken of the fixed point algorithm and the inbuilt Matlab function \texttt{fsolve}. Note that the comparison is slightly impacted by the fact that convergence is measured differently in these two numerical methods. However, a general comparison can be made regarding the efficiency of the two approaches.
We choose $n=20$, $N=30$, the ODE solver tolerance is set to be $10^{-8}$, the optimality tolerance is $10^{-4}$ and $\beta = 10^{-3}$. 
As can be seen in Table \ref{TabA3:Prob1}, the running time of the fixed point algorithm is considerably faster than for \texttt{fsolve}, while the resulting values of the cost functional remain the same. This can be confirmed by comparing the number of function evaluations for each method, which is an important measure when dealing with large systems, such as the two-dimensional problems discussed in this paper, since each iteration is costly for large problems. The differences in $\rho$ and $\adj$ are broadly in line with the optimality tolerance set, however the control differs more because $\vec{w}$ is updated using the optimal values of $\rho$ and $\adj$. 
%
%\begin{table}
\begin{tabular}{ | c | c || c | c | c ||}
\hline
\multicolumn{2}{|c||}{} & Fixed Point & \texttt{fsolve} & Difference   \\
\hline
\hline
 & $\mathcal{J}_{uc}$ & $\numprint{0.0438}$ & $\numprint{0.0438}$ &   \\
 & $\mathcal{J}_{c}$ & $\numprint{0.0011}$ & $\numprint{0.0011}$ &   \\
 & \texttt{Iter} (\texttt{funcEval}) & $\numprint{670}$ ($\numprint{670}$)  & $\numprint{38}$ ($\numprint{31959}$)  &   \\
$\kappa =-1$ & Time taken (s) & $\numprint{2.4939e+2}$ & $\numprint{9.1546e+3}$ &   \\
 & $\mathcal{E}_{\rho_{Diff}}$ & & &$\numprint{1.1348e-3}$  \\
 & $\mathcal{E}_{\adj_{Diff}}$ & & &$\numprint{7.2742e-5}$  \\
 & $\mathcal{E}_{\vec{w}_{Diff}}$ & & & $\numprint{7.6725e-2}$  \\
\hline
 & $\mathcal{J}_{uc}$ & $\numprint{0.0434}$ & $\numprint{0.0434}$ &   \\
 & $\mathcal{J}_{c}$ & $\numprint{0.0020}$ & $\numprint{0.0020}$ &   \\
 & \texttt{Iter} (\texttt{funcEval}) & $\numprint{654}$ ($\numprint{654}$)  & $\numprint{38}$ ($\numprint{34239}$)  &   \\
$\kappa =1$ & Time taken (s) & $\numprint{3.3794e+2}$ & $\numprint{1.0167e+4}$ &   \\
 & $\mathcal{E}_{\rho_{Diff}}$ & & &$\numprint{3.0610e-4}$  \\
 & $\mathcal{E}_{\adj_{Diff}}$ & & &$\numprint{4.8701e-5}$  \\
 & $\mathcal{E}_{\vec{w}_{Diff}}$ & & & $\numprint{8.9056e-3}$  \\
\hline
\end{tabular}
\caption{Comparison of the outputs of the fixed point method, with those obtained using \texttt{fsolve}.}
\label{TabA3:Prob1}
\end{table} %\label{TabA3:Prob1}

	
	
	\subsection{Perturbing $w$}
	+++ need a bit of smoothing since it's copied from the paper +++
As detailed in Section \ref{sec:Method_SolverFP}, it is necessary to provide an initial guess for the control $\vec{w}$ to start the optimization routine. Therefore, one way of validating the numerical method is to perturb the exact solution for $\vec{w}$ taken from a test problem with analytic solution and use this as an initial guess in the optimization solver. In the first iteration, the solutions for $\rho$ and $\adj$ differ from the exact solution. The optimization method then converges to the exact, optimal solution. We consider Test Problem 2 from Appendix \ref{app:TestProblems}, which is an exact solution for th overdamped flow control problem \eqref{eqn:ADFlowOCP}, with no-flux boundary conditions, and no particle interaction term. 
The following two perturbation functions are considered. The first perturbation is in time only and is defined as:
\begin{align*}
g(t) &= \frac{1}{2} f(t-t_0, a) \times f(t-t_0, -a)\\
&= \frac{1}{2} \frac{e^{-a/(t-t_0)}}{e^{-a/(t-t_0)} + e^{-a/(1-t -t_0)}} \times \frac{e^{a/(t-t_0)}}{e^{a/(t-t_0)} + e^{a/(1-t - t_0)}},
\end{align*}
and normalised by:
\begin{align*}
\tilde g(t) = \frac{g(t)}{\max{|{g(t)}|}}.
\end{align*}
A similar perturbation can be done in space, taking into account the difference in length of spatial and time domains:
\begin{align*}
h(x) &= \frac{1}{2} f(x-x_0, 2a) \times f(x-x_0, -2a)\\
&= \frac{1}{2} \frac{e^{-2a/(x-x_0)}}{e^{-2a/(x-x_0)} + e^{-2a/(1-x-x_0)}} \times \frac{e^{2a/(x-x_0)}}{e^{2a/(x-x_0)} + e^{2a/(1-x-x_0)}}.
\end{align*}
Again, this is normalised:
\begin{align*}
\tilde h(x) = \frac{h(x)}{\max{|{h(x)}|}}.
\end{align*}
These perturbation functions are chosen such that the perturbation is smooth and respects the initial condition for $\rho$, as well as the final time condition for $\adj$, by not changing the first or final time point. If this is not respected, the algorithm converges up to a point and then diverges, since the boundary conditions in time cannot be matched. (++ I know that's what we said back then but I am a little confused now given how $w$ is applied to problems... I think it's worth looking at this again at some point!!++)
 The considered perturbations are applied to the exact solution of the control, $\vec{w}_{ex}$, as follows:
\begin{align*}
\vec{w}_{pert1} &= \vec{w}_{ex}(1+ \epsilon \tilde g(t))\\
\vec{w}_{pert2} &= \vec{w}_{ex}(1+ \epsilon \tilde g(t) \tilde h(x)),
\end{align*}
where $a = 0.7$, $x_0 = t_0 = -0.01$ and the perturbation strength is either $\epsilon = 0.1$ or $\epsilon = 0.5$.
The chosen number of points is $N =30$ and $n=20$, the ODE tolerances are $10^{-8}$ and the optimality tolerance is $10^{-4}$. The mixing rate for the optimization solver is $\lambda = 0.01$.
The results presented in Table \ref{TabA2:Prob1} show the initial error in $\vec{w}$, $\mathcal{E}_{\vec{w}_{uc}}$, and the final errors in $\vec{w}$, $\rho$ and $\adj$, measured in the norm presented in Section \ref{sec:ErrorMeasure}, with respect to the exact solution. The initial error $\mathcal{E}_{\vec{w}_{uc}}$ is proportional to the perturbation strength $\epsilon$. The final errors for $\vec{w}$ and $\rho$ and $\adj$ are mostly within the specified optimality tolerance regardless of the perturbation strength and location. 

%\begin{table}
\begin{tabular}{ | c | c || c | c | c | c ||}
\hline
  \multicolumn{2}{|c||}{} & $\beta = 10^{-3}$ & $\beta = 10^{-1}$ & $\beta = 10^{1}$ & $\beta = 10^{3}$  \\
\hline
\hline
\multirow{4}{*}{$0.1 \tilde g(t)$} & $\mathcal{E}_{\vec{w}_{uc}}$ & $\numprint{1.0000e-1}$ & $\numprint{1.0000e-1}$ & $\numprint{1.0000e-1}$ & $\numprint{1.0000e-1}$ \\
 & $\mathcal{E}_{\vec{w}_c}$ & $\numprint{5.3770e-5}$ & $\numprint{5.2340e-5}$ & $\numprint{5.2201e-5}$ & $\numprint{5.2203e-5}$ \\
 & $\mathcal{E}_{\rho}$ & $\numprint{1.1396e-5}$ & $\numprint{7.8597e-5}$ & $\numprint{7.8595e-5}$ & $\numprint{7.8597e-5}$ \\
 & $\mathcal{E}_{\adj}$ & $\numprint{2.7854e-5}$ & $\numprint{2.7836e-4}$ & $\numprint{5.7043e-4}$ & $\numprint{5.7045e-4}$ \\
\hline
\multirow{4}{*}{$0.5 \tilde g(t)$} & $\mathcal{E}_{\vec{w}_{uc}}$ & $\numprint{5.0000e-1}$ & $\numprint{5.0000e-1}$ & $\numprint{5.0000e-1}$ & $\numprint{5.0000e-1}$ \\
 & $\mathcal{E}_{\vec{w}_c}$ & $\numprint{2.1970e-4}$ & $\numprint{2.1747e-4}$ & $\numprint{2.1735e-4}$ & $\numprint{2.1735e-4}$ \\
 & $\mathcal{E}_{\rho}$ & $\numprint{2.4256e-5}$ & $\numprint{2.2878e-4}$ & $\numprint{2.2878e-4}$ & $\numprint{2.2879e-4}$ \\
 & $\mathcal{E}_{\adj}$ & $\numprint{3.3247e-5}$ & $\numprint{3.3227e-4}$ & $\numprint{6.8088e-4}$ & $\numprint{6.8090e-4}$ \\
\hline
\multirow{4}{*}{$0.1 \tilde h(x)$} & $\mathcal{E}_{\vec{w}_{uc}}$ & $\numprint{8.5568e-2}$ & $\numprint{8.5568e-2}$ & $\numprint{8.5568e-2}$ & $\numprint{8.5568e-2}$ \\
 & $\mathcal{E}_{\vec{w}_c}$ & $\numprint{5.3700e-5}$ & $\numprint{5.2250e-5}$ & $\numprint{5.2100e-5}$ & $\numprint{5.2103e-5}$ \\
 & $\mathcal{E}_{\rho}$ & $\numprint{1.1704e-5}$ & $\numprint{7.7973e-5}$ & $\numprint{7.7969e-5}$ & $\numprint{7.7968e-5}$ \\
 & $\mathcal{E}_{\adj}$ & $\numprint{2.6426e-5}$ & $\numprint{2.6387e-4}$ & $\numprint{5.6982e-4}$ & $\numprint{5.6984e-4}$ \\
\hline
\multirow{4}{*}{$0.5 \tilde h(x)$} & $\mathcal{E}_{\vec{w}_{uc}}$ & $\numprint{4.2784e-1}$ & $\numprint{4.2784e-1}$ & $\numprint{4.2784e-1}$ & $\numprint{4.2784e-1}$ \\
 & $\mathcal{E}_{\vec{w}_c}$ & $\numprint{2.1203e-4}$ & $\numprint{2.0982e-4}$ & $\numprint{2.0967e-4}$ & $\numprint{2.0968e-4}$ \\
 & $\mathcal{E}_{\rho}$ & $\numprint{2.2565e-5}$ & $\numprint{2.1275e-4}$ & $\numprint{2.1274e-4}$ & $\numprint{2.1275e-4}$ \\
 & $\mathcal{E}_{\adj}$ & $\numprint{3.0225e-5}$ & $\numprint{3.0219e-4}$ & $\numprint{6.1920e-4}$ & $\numprint{6.1923e-4}$ \\
\hline
\end{tabular}
\caption{Test Problem 2: Error measures for $\vec{w}_{uc}$, $\vec{w}_{c}$, $\rho$, and $\adj$, for four perturbation strategies for $\vec{w}$, and a range of $\beta$.}
\label{TabA2:Prob1}
\end{table} %\label{TabA2:Prob1}

	
	\subsection{Additional Observations}
	In the following, a few further observations are stated that were made when applying the optimization solver to problems involving the overdamped model. Demonstrations of these points are omitted, due to time constraints, and will be provided in future work.
	During the investigation of different perturbed exact problems and other test problems, it could be observed that the weakness of the optimization method lies in solving advection dominant problems. 
	This became apparent when considering different analytic exact solutions to the overdamped flow control problem \eqref{eqn:ADFlowOCP}. Depending on the magnitude of the control in each problem, the algorithm could either converge, for small controls, or not, for large control values. Scaling the size of the control down, by scaling the exact solutions accordingly, it is possible to achieve convergence for problems that were previously too difficult to solve for the optimization solver. Another way of achieving convergence is to introduce a diffusion coefficient into the problem. A large advection term can then be offset with a large diffusion coefficient and the optimization solver is able to converge.
	The issue of advection dominance is especially prevalent when applying no-flux boundary conditions. This is because in order to match the boundary conditions in an advection dominated problem, the gradients of the particle distribution become steep at the boundary. Since steep gradients are difficult to treat numerically, this is an exacerbation of the problem at hand.
	It is important to point out that these issues are encountered with any optimization and forward solver and is not unique to our choice of methods. 
	\\
	\\
	During the work on the overdamped equations it was found that one limiting factor in the convergence of the method is interpolation errors. The error made during interpolation is of order $10^{-8}$ to $10^{-9}$. The ODE solver cannot be more accurate than that, since variables are interpolated in time during each ODE solve, and consequently the optimization tolerance has to be adapted to this finding as well.
	Furthermore, the optimization tolerance has to be chosen in such a way that it takes into account the accumulation of error during each ODE solve and with each iteration of the optimization algorithm. This results in the optimization tolerance having to be at least three orders larger than the ODE solver tolerance, which is bounded by the interpolation error. We found that, in general, choosing the ODE solver tolerance to be $10^{-8}$ and the optimization tolerance to be $10^{-4}$, we get reliable convergence for most test problems.
	\\
	Another aspect to take into consideration is that the problem becomes numerically harder with decreasing values of $\beta$. In general, small $\beta$ may need more points to be solved to the same accuracy as larger values of $\beta$, or may not reach the same accuracy at all.  Finally, it is worth investigating how interpolation, forward solution and optimization are affected by exponential changes in time of the quantity of interest, as opposed to it showing polynomial behaviour. It is expected that quantities which change exponentially in time are harder to compute numerically, and this therefore could have an effect on the accuracy of the method. This is particularly relevant given that many test problems with exact solutions were considered with $\rho$ and $\adj$ changing exponentially in time.
	
	
	\section{Numerical Experiments} \label{sec:Examples}
	% add later... question is whether these wouldn't have to be rerun anyway.
	%
In order to solve the optimal control problems \eqref{AdvDiff} and \eqref{AdvDiff_Linear} some inputs must be provided. The desired state $\widehat \rho$, the PDE source term $f$, and the external potential $V_{ext}$ must be given. Furthermore, an initial condition for $\rho$, the final time condition for $\adj$ and an initial guess for the control $\vec{w}$ have to be be specified. 
The interaction kernel (++ terminology? ++) is of the form:
\begin{align*}
\vec{K} = \nabla V_2, \qquad V_2 = e^{-x^2}.
\end{align*}
Three interaction strengths are considered in this section. Firstly, each problem is solved without an interaction term present ($\gamma = 0$). Then, the considered problem is solved with an order one attractive interaction term ($\gamma = -1$) and an order one repulsive interaction term ($\gamma = 1$), respectively. Initially, the control $\vec{w}$ is set to zero. It is then investigated how the control changes from this baseline, influenced by the different interaction strengths. This is considered for different values of the regularization parameter $\beta$ and it is expected that the control will increase with decreasing $\beta$, since the cost functionals in problems \eqref{AdvDiff} and \eqref{AdvDiff_Linear} allow for a larger control with smaller $\beta$.
In the following examples, the domain considered is $\Omega \times [0,T] = [-1,1] \times [0,1]$. The number of spatial points is $N=30$, and the number of time points is $n=20$, unless stated otherwise. The tolerances in the ODE solver are set to $10^{-8}$ and the tolerance for the convergence of the optimization algorithm is $10^{-4}$. The mixing parameter $\lambda$ is $0.01$, unless stated otherwise.
\subsection{Nonlinear control problems with an additional nonlocal integral term 1D} \label{sec:Examples1d}
Examples of solving Problem \eqref{AdvDiff}, with 'no-flux type' boundary conditions \eqref{NoFlux} and Dirichlet boundary conditions \eqref{Dirichlet} are given in this section. 
 
\subsubsection{Neumann boundary conditions, Example 1}	 
The chosen inputs for this example are:
\begin{align*}
&\widehat \rho = \frac{1}{2}(1-t) + t\bigg(\frac{1}{2}\sin(\pi (y - 2)/2) + \frac{1}{2}\bigg),\\
&\rho_{0} = \frac{1}{2}, \ \
\adj_{T} = 0, \ \
\vec{w} = 0, \ \ 
f =0,\ \
V_{ext} =0.
\end{align*}	
Table \ref{TabS5:Prob1} displays the results for this example. The value of the cost functional for the uncontrolled case ($J_{uc}$), where $\vec{w} =0$, is compared with the controlled case ($J_{c}$) for different values of $\beta$ and for each of the interaction strengths. It can be observed that in all cases $J_{c}$ is lower or equal value to $J_{uc}$. The lowest values of $J_{c}$ will be observed for the smallest $\beta$ value considered. At large values of $\beta$, applying control is heavily penalised and the optimal control approaches zero, which coincides with the uncontrolled case. Furthermore, Table \ref{TabS5:Prob1} displays the number of iterations for each of these examples. The desired state $\widehat \rho$, and the uncontrolled state $\rho$ for $\gamma =1$ and $\gamma = -1$, with $\beta =10^{-3}$ are shown in Figure \ref{Ex12DN1}. The desired state $\hat \rho$ and the uncontrolled $\rho$ are independent of $\beta$. However, $\rho$ changes considerably with the choice of interaction strength $\gamma$. The optimal states $\rho$ for $\gamma = 1,0,-1$ and corresponding optimal controls, with $\beta = 10^{-3}$ are shown in Figure \ref{Ex12DN2}. 
It can be observed that in the case of $\beta = 10^{-3}$, the optimal state $\rho$ is very similar to $\hat \rho$, regardless of the choice of interaction. However, the corresponding control plot reviles that the control has to be applied differently in each case to account for the interaction effects. In general, the control is largely applied on the right half of the spatial domain, to carry mass to the left, where the desired state dictates it to be, as can be seen when $\gamma = 0$. However, when the particle interaction is repulsive, the control is moving some of the particle mass from the boundary at $x=-1$ to correct for the repulsive particles accumulating there in the uncontrolled state, as illustrated in Figure \ref{Ex12DN1}. In the attractive case, the control corrects by carrying some mass to the boundary at $x=1$, since the uncontrolled particle density is clustered in the middle of the domain in this case, compare to Figure \ref{Ex12DN1}.
\begin{figure}[h]
	\includegraphics[scale=0.05]{Figure1.png}
	\caption{Example 1, desired state $\widehat \rho$ and uncontrolled state $\rho$ at $\gamma =1$ and $\gamma =-1$, $\beta = 10^{-3}$}
	\label{Ex12DN1}
\end{figure}
\begin{figure}[h]
	\includegraphics[scale=0.05]{Figure2.png}
	\caption{Example 1, optimal state $\rho$ and the corresponding optimal control $\vec{w}$ for $\gamma = 1,0,-1$, $\beta = 10^{-3}$.}
	\label{Ex12DN2}
\end{figure}

\begin{table}
\begin{tabular}{ | c | c || c | c | c | c ||}
\hline
\multicolumn{2}{|c||}{}& $\beta = 10^{-3}$ & $\beta = 10^{-1}$ & $\beta = 10^{1}$ & $\beta = 10^{3}$  \\
\hline
\hline
 & $\mathcal{J}_{uc}$ & $\numprint{0.0438}$ & $\numprint{0.0438}$ & $\numprint{0.0438}$ & $\numprint{0.0438}$ \\
$\kappa= \numprint{-1}$  & $\mathcal{J}_c$ & $\numprint{0.0011}$ & $\numprint{0.0267}$ & $\numprint{0.0435}$ & $\numprint{0.0438}$ \\
& \texttt{Iter} & $\numprint{670}$ & $\numprint{650}$ & $\numprint{449}$ & $\numprint{1}$ \\
\hline
 & $\mathcal{J}_{uc}$ & $\numprint{0.0417}$ & $\numprint{0.0417}$ & $\numprint{0.0417}$ & $\numprint{0.0417}$ \\
$\kappa= \numprint{0}$  & $\mathcal{J}_c$ & $\numprint{0.0014}$ & $\numprint{0.0283}$ & $\numprint{0.0415}$ & $\numprint{0.0417}$ \\
& \texttt{Iter} & $\numprint{665}$ & $\numprint{656}$ & $\numprint{434}$ & $\numprint{1}$ \\
\hline
 & $\mathcal{J}_{uc}$ & $\numprint{0.0434}$ & $\numprint{0.0434}$ & $\numprint{0.0434}$ & $\numprint{0.0434}$ \\
$\kappa= \numprint{1}$  & $\mathcal{J}_c$ & $\numprint{0.0020}$ & $\numprint{0.0322}$ & $\numprint{0.0432}$ & $\numprint{0.0434}$ \\
& \texttt{Iter} & $\numprint{654}$ & $\numprint{682}$ & $\numprint{422}$ & $\numprint{1}$ \\
\hline
\end{tabular}
\caption{Example 1: Cost $\mathcal{J}_{uc}$ of applying no control (i.e., $\vec{w} = \vec{0}$), optimal control cost $\mathcal{J}_{c}$, and number of iterations (PDE solves) \emph{\texttt{Iter}} required, for a range of values of the interaction strength $\kappa$ and regularization parameter $\beta$.}
\label{TabS5:Prob1}
\end{table} %\label{TabS5:Prob1}
%\begin{table}[h]
%	\begin{tabular}{ ||c|| c | c |c | c ||}
%		\hline
%		$\beta$ / $\gamma$ & $10^{-3}$  & $10^{-1}$  & $10$ & $10^3$ \\ 
%		\hline 
%		      & $J_{uc} = 0.0438$ & $J_{uc} = 0.0438$  & $J_{uc} = 0.0438$ & $J_{uc} = 0.0438$\\ 
%		 $-1$ & $J_c = 0.0011$ & $J_c = 0.0270$ & $J_c = 0.0435$ & $J_c = 0.0438$\\ 
%		      & Iter. $= 667$ & Iter. $= 649$  & Iter. $= 468$ & Iter. $= 13$\\ 
%		 \hline
%		      & $J_{uc} = 0.0417$ & $J_{uc} = 0.0417$   & $J_{uc} = 0.0417$& $J_{uc} = 0.0417$\\
%		 $0$  & $J_c = 0.0014$ & $J_c = 0.0283$  & $J_c = 0.0415$ & $J_c = 0.0417$\\ 
%		      & Iter. $= 671$ & Iter. $= 656$  & Iter. $= 434$ & Iter. $= 1$\\ 
%		 \hline
%		      & $J_{uc} = 0.0434$ & $J_{uc} = 0.0434$  & $J_{uc} = 0.0434$ & $J_{uc} = 0.0434$\\
%		 $1$  & $J_c = 0.0020$ & $J_c = 0.0324$  & $J_c = 0.0432$ & $J_c = 0.0434$\\ 
%		      & Iter. $= 674$ & Iter. $= 686$  & Iter. $= 411$ & Iter. $= 1$\\ 
%		 \hline 
%	\end{tabular}
%    \caption{}
%    \label{TabNFlowEx1}
%\end{table}

\subsubsection{Neumann boundary conditions, Example 2} 
The chosen inputs for Example 2 are:
\begin{align*}
&\widehat \rho = \bigg(\frac{1}{2}\cos(\pi y) + \frac{1}{2}\bigg)(1-t) + t\bigg(-\frac{1}{2}\cos(2 \pi y) + \frac{1}{2}\bigg),\\
&\rho_{0} = \frac{1}{2}\cos(\pi y) + \frac{1}{2},\ \
\adj_{T} = 0,\ \
\vec{w} = 0,\ \
f =0,\ \
V_{ext} =0.
\end{align*}
In Table \ref{TabS5:Prob2a} the results for Example 2 are displayed. These are comparable with the results for Example 1, in the effect of $\beta$ and the number of iterations. In all three configurations of the interaction term, the control is focussed on transporting the mass from the middle of the domain onto two piles centred at $x=-0.5$ and $x=0.5$. In Figure \ref{Ex22DN1}, the desired state $\widehat \rho$, the optimal state $\rho$ and the optimal control $\vec{w}$ are displayed for $\beta = 10^{-3}$, and compared to Example 3 below. 
\begin{figure}[h]
	\includegraphics[scale=0.05]{Figure3.png}
	\caption{Example 2/ Example 3, desired state $\widehat \rho$, optimal state $\rho$ and corresponding optimal control $\vec{w}$, $\beta = 10^{-3}$, $\gamma = 1$.}
	\label{Ex22DN1}
\end{figure}

\begin{table}
\begin{tabular}{ ||c|| c | c | c | c | c ||}
\hline
& & $\beta = 10^{-3}$ & $\beta = 10^{-1}$ & $\beta = 10^{1}$ & $\beta = 10^{3}$  \\
\hline
 & $J_{uc}$ & $\numprint{5.3559e-2}$ & $\numprint{5.3559e-2}$ & $\numprint{5.3559e-2}$ & $\numprint{5.3559e-2}$ \\
$\gamma= \numprint{-1}$  & $J_c$ & $\numprint{9.6557e-3}$ & $\numprint{4.9268e-2}$ & $\numprint{5.3511e-2}$ & $\numprint{5.3559e-2}$ \\
& $Iter.$ & $\numprint{520}$ & $\numprint{768}$ & $\numprint{378}$ & $\numprint{1}$ \\
\hline
 & $J_{uc}$ & $\numprint{6.6902e-2}$ & $\numprint{6.6902e-2}$ & $\numprint{6.6902e-2}$ & $\numprint{6.6902e-2}$ \\
$\gamma= \numprint{0}$  & $J_c$ & $\numprint{1.0920e-2}$ & $\numprint{6.0339e-2}$ & $\numprint{6.6826e-2}$ & $\numprint{6.6903e-2}$ \\
& $Iter.$ & $\numprint{679}$ & $\numprint{770}$ & $\numprint{390}$ & $\numprint{1}$ \\
\hline
 & $J_{uc}$ & $\numprint{8.3948e-2}$ & $\numprint{8.3948e-2}$ & $\numprint{8.3948e-2}$ & $\numprint{8.3948e-2}$ \\
$\gamma= \numprint{1}$  & $J_c$ & $\numprint{1.2510e-2}$ & $\numprint{7.4874e-2}$ & $\numprint{8.3842e-2}$ & $\numprint{8.3949e-2}$ \\
& $Iter.$ & $\numprint{681}$ & $\numprint{771}$ & $\numprint{396}$ & $\numprint{1}$ \\
\hline
\end{tabular}
\caption{Problem 2}
\label{TabS5:Prob2}
\end{table} %\label{TabS5:Prob2a}
%
%\begin{table}
%	\begin{tabular}{ ||c|| c | c |c | c ||}
%		\hline
%		$\beta$ / $\gamma$ & $10^{-3}$  & $10^{-1}$  & $10$ & $10^3$ \\ 
%		\hline 
%		& $J_{uc} = 0.0536$ & $J_{uc} = 0.0536$  & $J_{uc} = 0.0536$ & $J_{uc} = 0.0536$\\ 
%		$-1$ & $J_c = 0.0096$ & $J_c = 0.0493$ & $J_c = 0.0535$ & $J_c = 0.0536$\\ 
%		& Iter. $= 724$ & Iter. $= 769$  & Iter. $= 379$ & Iter. $= 1$\\ 
%		\hline
%		& $J_{uc} = 0.0669$ & $J_{uc} = 0.0669$   & $J_{uc} = 0.0669$& $J_{uc} = 0.0669$\\
%		$0$  & $J_c = 0.0109$ & $J_c = 0.0603$  & $J_c = 0.0668$ & $J_c = 0.0669$\\ 
%		& Iter. $= 726$ & Iter. $= 770$  & Iter. $= 390$ & Iter. $= 1$\\ 
%		\hline
%		& $J_{uc} = 0.0839$ & $J_{uc} = 0.0839$  & $J_{uc} = 0.0839$ & $J_{uc} = 0.0839$\\
%		$1$  & $J_c = 0.0125$ & $J_c = 0.0749$  & $J_c = 0.0838$ & $J_c = 0.0839$\\ 
%		& Iter. $= 728$ & Iter. $= 772$  & Iter. $= 396$ & Iter. $= 1$\\ 
%		\hline 
%	\end{tabular}
%    \caption{}
%    \label{TabNFlowEx2}
%\end{table}

\subsubsection{Dirichlet boundary conditions, Example 3} 
The inputs for this example are:
\begin{align*}
&\widehat \rho = \bigg(\frac{1}{2}\cos(\pi y) + \frac{1}{2}\bigg)(1-t) + t\bigg(-\frac{1}{2}\cos(2 \pi y) + \frac{1}{2}\bigg),\\
&\rho_{0} = \frac{1}{2}\cos(\pi y) + \frac{1}{2},\ \
\adj_{T} = 0,\ \
\vec{w} = 0,\ \
f =0,\ \
V_{ext} =0.
\end{align*}
Table \ref{TabS5:Prob3} presents the results for this example for a range of $\beta$ values and different interaction strengths. The observations are in line with those in Example 1 and 2. In particular, $ \widehat \rho$ and $\rho_0$ coincide with those of the problem with Neumann boundary conditions in Example 2. A comparison between the two examples is illustrated in Figure \ref{Ex22DN1}. Both the optimal state $\rho$ and the optimal control are qualitatively different when considering Dirichlet boundary conditions over Neumann conditions. The numerical result for this example was achieved with $N=40$ and $n = 30$, rather than with $N=30$ and $n=20$. This indicates that the Dirichlet boundary conditions are harder to apply in this problem, due to the steep shape of the desired state. This steepness is somewhat less impactful in Example 2, where the desired state is not closely matched at the boundaries. In Example 3, while the desired state is matched perfectly at the boundary, the peaks of the desired state are matched less closely. In Figure \ref{Ex22DN1}, this can be confirmed by considering the control plots. The optimal control for Example 3 is larger than for Example 2, specifically between the boundaries of the domain and the peaks of the desired state.
\begin{table}
\begin{tabular}{ ||c|| c | c | c | c | c ||}
\hline
& & $\beta = 10^{-3}$ & $\beta = 10^{-1}$ & $\beta = 10^{1}$ & $\beta = 10^{3}$  \\
\hline
 & $J_{uc}$ & $\numprint{1.4165e-1}$ & $\numprint{1.4165e-1}$ & $\numprint{1.4165e-1}$ & $\numprint{1.4165e-1}$ \\
$\gamma= \numprint{-1}$  & $J_c$ & $\numprint{3.5594e-2}$ & $\numprint{1.3270e-1}$ & $\numprint{1.4155e-1}$ & $\numprint{1.4165e-1}$ \\
& $Iter.$ & $\numprint{944}$ & $\numprint{816}$ & $\numprint{437}$ & $\numprint{1}$ \\
\hline
 & $J_{uc}$ & $\numprint{1.5452e-1}$ & $\numprint{1.5452e-1}$ & $\numprint{1.5452e-1}$ & $\numprint{1.5452e-1}$ \\
$\gamma= \numprint{0}$  & $J_c$ & $\numprint{3.8023e-2}$ & $\numprint{1.4549e-1}$ & $\numprint{1.5442e-1}$ & $\numprint{1.5452e-1}$ \\
& $Iter.$ & $\numprint{940}$ & $\numprint{825}$ & $\numprint{440}$ & $\numprint{1}$ \\
\hline
 & $J_{uc}$ & $\numprint{1.6610e-1}$ & $\numprint{1.6610e-1}$ & $\numprint{1.6610e-1}$ & $\numprint{1.6610e-1}$ \\
$\gamma= \numprint{1}$  & $J_c$ & $\numprint{4.1143e-2}$ & $\numprint{1.5751e-1}$ & $\numprint{1.6601e-1}$ & $\numprint{1.6610e-1}$ \\
& $Iter.$ & $\numprint{932}$ & $\numprint{827}$ & $\numprint{440}$ & $\numprint{1}$ \\
\hline
\end{tabular}
\caption{Problem 3 ($n = 30,N = 40$)}
\label{TabS5:Prob3}
\end{table} %\label{TabS5:Prob3}
%\begin{table}
%	\begin{tabular}{ ||c|| c | c |c | c ||}
%		\hline
%		$\beta$ / $\gamma$ & $10^{-3}$  & $10^{-1}$  & $10$ & $10^3$ \\ 
%		\hline 
%		& $J_{uc} = 0.0510$ & $J_{uc} = 0.0510$  & $J_{uc} = 0.0510$ & $J_{uc} = 0.0510$\\ 
%		$-1$ & $J_c = 0.0026$ & $J_c = 0.0365$ & $J_c = 0.0508$ & $J_c = 0.0510$\\ 
%		& Iter. $= 690$ & Iter. $= 696$  & Iter. $= 696$ & Iter. $= 696$\\ 
%		\hline
%		& $J_{uc} = 0.0417$ & $J_{uc} = 0.0417$  & $J_{uc} = 0.0417$& $J_{uc} = 0.0417$\\
%		$0$  & $J_c = 0.0027$ & $J_c = 0.0343$  & $J_c = 0.0416$ & $J_c = 0.0417$\\ 
%		& Iter. $= 696$ & Iter. $= 742$  & Iter. $= 409$ & Iter. $= 1$\\ 
%		\hline
%		& $J_{uc} = 0.0452$ & $J_{uc} = 0.0452$  & $J_{uc} = 0.0452$ & $J_{uc} = 0.0452$\\
%		$1$  & $J_c = 0.0030$ & $J_c = 0.0388$  & $J_c = 0.0452$ & $J_c = 0.0452$\\ 
%		& Iter. $= 703$ & Iter. $= 779$  & Iter. $= 397$ & Iter. $= 1$\\ 
%		\hline 
%	\end{tabular}
%	\caption{Update table! Is for different example}
%	\label{TabNFlowEx3}
%\end{table}


\subsection{Linear control problems with an additional nonlocal integral term}
In this section, examples of solving Problem \eqref{AdvDiff_Linear} with both 'no-flux type' boundary conditions \eqref{NoFlux_Linear} and Dirichlet boundary conditions \eqref{Dirichlet}.
\subsubsection{Dirichlet boundary conditions, Example 4}
The inputs for this example are:
\begin{align*}
&\widehat \rho = (1 - t)\bigg(\frac{1}{2}\cos(\pi y) + \frac{1}{2}\bigg)  + t\bigg(-\frac{1}{2}\cos(\pi y) + \frac{1}{2}\bigg),\\
&\rho_{0} = \frac{1}{2}\cos(\pi y) + \frac{1}{2},\ \
\adj_{T} = 0,\ \
{w} = 0,\ \
f =0, \ \
V_{ext} =0.
\end{align*}
In Table \ref{TabS5:Prob4} the results for Example 4 for a range of parameter values can be found. The results are qualitatively similar to the previous examples, the only difference is that the control is applied linearly in this example.
\begin{table}
\begin{tabular}{ ||c|| c | c | c | c | c ||}
\hline
& & $\beta = 10^{-3}$ & $\beta = 10^{-1}$ & $\beta = 10^{1}$ & $\beta = 10^{3}$  \\
\hline
 & $J_{uc}$ & $\numprint{1.4165e-1}$ & $\numprint{1.4165e-1}$ & $\numprint{1.4165e-1}$ & $\numprint{1.4165e-1}$ \\
$\gamma= \numprint{-1}$  & $J_c$ & $\numprint{2.0286e-2}$ & $\numprint{9.0267e-2}$ & $\numprint{1.4067e-1}$ & $\numprint{1.4166e-1}$ \\
& $Iter.$ & $\numprint{784}$ & $\numprint{740}$ & $\numprint{503}$ & $\numprint{49}$ \\
\hline
 & $J_{uc}$ & $\numprint{1.5452e-1}$ & $\numprint{1.5452e-1}$ & $\numprint{1.5452e-1}$ & $\numprint{1.5452e-1}$ \\
$\gamma= \numprint{0}$  & $J_c$ & $\numprint{2.0037e-2}$ & $\numprint{1.0154e-1}$ & $\numprint{1.5356e-1}$ & $\numprint{1.5452e-1}$ \\
& $Iter.$ & $\numprint{788}$ & $\numprint{739}$ & $\numprint{509}$ & $\numprint{56}$ \\
\hline
 & $J_{uc}$ & $\numprint{1.6610e-1}$ & $\numprint{1.6610e-1}$ & $\numprint{1.6610e-1}$ & $\numprint{1.6610e-1}$ \\
$\gamma= \numprint{1}$  & $J_c$ & $\numprint{2.0448e-2}$ & $\numprint{1.1354e-1}$ & $\numprint{1.6518e-1}$ & $\numprint{1.6610e-1}$ \\
& $Iter.$ & $\numprint{792}$ & $\numprint{740}$ & $\numprint{515}$ & $\numprint{61}$ \\
\hline
\end{tabular}
\caption{Problem 4}
\label{TabS5:Prob4}
\end{table} %\label{TabS5:Prob4}

%\begin{table}[h]
%	\begin{tabular}{ ||c|| c | c |c | c ||}
%		\hline
%		$\beta$ / $\gamma$ & $10^{-3}$  & $10^{-1}$  & $10$ & $10^3$ \\ 
%		\hline 
%		& $J_{uc} = 0.1417$ & $J_{uc} = 0.1417$  & $J_{uc} = 0.1417$ & $J_{uc} = 0.1417$\\ 
%		$-1$ & $J_c = 0.0203$ & $J_c = 0.0903$ & $J_c = 0.1407$ & $J_c = 0.1417$\\ 
%		& Iter. $= 787$ & Iter. $= 740$  & Iter. $= 503$ & Iter. $= 49$\\ 
%		\hline
%		& $J_{uc} = 0.1545$ & $J_{uc} = 0.1545$   & $J_{uc} = 0.1545$& $J_{uc} = 0.1545$\\
%		$0$  & $J_c = 0.0200$ & $J_c = 0.1015$  & $J_c = 0.1536$ & $J_c = 0.1545$\\ 
%		& Iter. $= 791$ & Iter. $= 740$  & Iter. $= 510$ & Iter. $= 56$\\ 
%		\hline
%		& $J_{uc} = 0.1661$ & $J_{uc} = 0.1661$  & $J_{uc} = 0.1661$ & $J_{uc} = 0.1661$\\
%		$1$  & $J_c = 0.0204$ & $J_c = 0.1135$  & $J_c = 0.1652$ & $J_c = 0.1661$\\ 
%		& Iter. $= 795$ & Iter. $= 741$  & Iter. $= 515$ & Iter. $= 61$\\ 
%		\hline 
%	\end{tabular}
%	\caption{}
%	\label{TabNFlowEx4}
%\end{table}


\subsubsection{Neumann boundary conditions, Example 5}
The inputs for this example are:
\begin{align*}
&\widehat \rho = \frac{1}{2}(1-t) + t\frac{1}{2}(-\cos(\pi y) + 1),\\
&\rho_{0} = \frac{1}{2},\ \
\adj_{T} = 0,\ \
{w} = 0,\ \
f =0,\ \
V_{ext} =0.
\end{align*}
Table \ref{TabS5:Prob5} shows the results for Example 5. Note that for this example, when $\beta = 10^{-3}$, the mixing parameter $\lambda$ had to be set to $0.001$ (why? explanation needed?).
Again, the only qualitative difference to interpreting the results is that the control is applied linearly.
\begin{table}
\begin{tabular}{ | c | c || c | c | c | c ||}
\hline
\multicolumn{2}{|c||}{}& $\beta = 10^{-3}$ & $\beta = 10^{-1}$ & $\beta = 10^{1}$ & $\beta = 10^{3}$  \\
\hline
\hline
 & $\mathcal{J}_{uc}$ & $\numprint{0.0606}$ & $\numprint{0.0606}$ & $\numprint{0.0606}$ & $\numprint{0.0606}$ \\
$\kappa= \numprint{-1}$  & $\mathcal{J}_c$ & $\numprint{0.0060}$ & $\numprint{0.0541}$ & $\numprint{0.0605}$ & $\numprint{0.0606}$ \\
& \texttt{Iter} & $\numprint{7024}$ & $\numprint{7731}$ & $\numprint{3961}$ & $\numprint{1}$ \\
\hline
 & $\mathcal{J}_{uc}$ & $\numprint{0.0417}$ & $\numprint{0.0417}$ & $\numprint{0.0417}$ & $\numprint{0.0417}$ \\
$\kappa= \numprint{0}$  & $\mathcal{J}_c$ & $\numprint{0.0045}$ & $\numprint{0.0383}$ & $\numprint{0.0416}$ & $\numprint{0.0417}$ \\
& \texttt{Iter} & $\numprint{7003}$ & $\numprint{7618}$ & $\numprint{3642}$ & $\numprint{1}$ \\
\hline
 & $\mathcal{J}_{uc}$ & $\numprint{0.0286}$ & $\numprint{0.0286}$ & $\numprint{0.0286}$ & $\numprint{0.0286}$ \\
$\kappa= \numprint{1}$  & $\mathcal{J}_c$ & $\numprint{0.0036}$ & $\numprint{0.0261}$ & $\numprint{0.0285}$ & $\numprint{0.0286}$ \\
& \texttt{Iter} & $\numprint{7052}$ & $\numprint{7490}$ & $\numprint{3474}$ & $\numprint{1}$ \\
\hline
\end{tabular}
\caption{Example 5: Uncontrolled cost $\mathcal{J}_{uc}$, optimal cost $\mathcal{J}_{c}$, and number of iterations, for a range of $\kappa$ and $\beta$ values.}
\label{TabS5:Prob5}
\end{table} %\label{TabS5:Prob5}
%\begin{table}[h]
%	\begin{tabular}{ ||c|| c | c |c | c ||}
%		\hline
%		$\beta$ / $\gamma$ & $10^{-3}$  & $10^{-1}$  & $10$ & $10^3$ \\ 
%		\hline 
%		& $J_{uc} = 0.0606$ & $J_{uc} = 0.0606$  & $J_{uc} = 0.0606$ & $J_{uc} = 0.0606$\\ 
%		$-1$ & $J_c = 0.0060$ & $J_c = 0.0554$ & $J_c = 0.0606$ & $J_c = 0.0606$\\ 
%		& Iter. $= 7311$ & Iter. $= 771$  & Iter. $= 389$ & Iter. $= 1$\\ 
%		\hline
%		& $J_{uc} = 0.0417$ & $J_{uc} = 0.0417$   & $J_{uc} = 0.0417$& $J_{uc} = 0.0417$\\
%		$0$  & $J_c = 0.0045$ & $J_c = 0.0383$  & $J_c = 0.0416$ & $J_c = 0.0417$\\ 
%		& Iter. $= 7227$ & Iter. $= 759$  & Iter. $= 364$ & Iter. $= 1$\\ 
%		\hline
%		& $J_{uc} = 0.0286$ & $J_{uc} = 0.0286$  & $J_{uc} = 0.0286$ & $J_{uc} = 0.0286$\\
%		$1$  & $J_c = 0.0036$ & $J_c = 0.0265$  & $J_c = 0.0285$ & $J_c = 0.0286$\\ 
%		& Iter. $= 7205$ & Iter. $= 746$  & Iter. $= 341$ & Iter. $= 1$\\ 
%		\hline 
%	\end{tabular}
%	\caption{}
%	\label{TabNFlowEx5}
%\end{table}


\subsection{Nonlinear control problems with an additional nonlocal integral term 2D}

\subsubsection{Neumann boundary conditions, Example 1}	
We have the following set up:
\begin{align*}
&\widehat \rho = \frac{1}{4}(1-t) + t\bigg(\frac{1}{4}\sin \bigg(\frac{\pi}{2}(x_1 - 2)\bigg)\sin \bigg(\frac{\pi}{2}(x_2 - 2)\bigg) + \frac{1}{4}\bigg),\\
&\rho_0 = \frac{1}{4},\ \
q_{T} = 0,\ \
\vec{w} = 0,\ \
f =0,\ \
V_{ext} =0.
\end{align*}
The results for this example are displayed in Table \ref{TabS5:Prob12D}. In figures \ref{rhoHat2dEx2} and \ref{rhoOpt2dEx2} the results for this example are shown at different time points, for $\beta = 10^{-3}$ and $\gamma = -1$. In Figure \ref{rhoOpt2dEx2} the control through a vector field illustrates why nonlinear control is called 'flow control'. This example is the two dimensional version of Example 1 in Section \ref{sec:Examples1d}. 

\begin{table}
\begin{tabular}{ ||c|| c | c | c | c | c ||}
\hline
& & $\beta = 10^{-3}$ & $\beta = 10^{-1}$ & $\beta = 10^{1}$ & $\beta = 10^{3}$  \\
\hline
 & $\mathcal J_{\vec w = \vec 0}$ & $0.0113$ & $0.0113$ & $0.0113$ & $0.0113$ \\
$\kappa= -1$  & $\mathcal J_{Opt}$ & $0.0013$ & $0.0104$ & $0.0113$ & $0.0113$ \\
& $\text{Iterations}$ & $676$ & $700$ & $290$ & $1$ \\
\hline
\end{tabular}
\caption{Results for the test problem, with different $\beta$}
\label{TabS5:Prob12D}
\end{table} % \label{TabS5:Prob12D}

\begin{figure}[h]
	\includegraphics[scale=0.044]{Figure12D.png}
	\caption{2D Example 1, uncontrolled $\rho$ and $\widehat \rho$, $\beta = 10^{-3}$, $\gamma = -1$.}
	\label{rhoHat2dEx2}
\end{figure}
\begin{figure}[h]
	\includegraphics[scale=0.04]{Figure22D.png}
	\caption{2D Example 1, controlled $\rho$ and optimal control $\vec{w}$, $\beta = 10^{-3}$, $\gamma = -1$.}
	\label{rhoOpt2dEx2}
\end{figure}


\subsubsection{Neumann boundary conditions, Example 2}	
Here, we have:
\begin{align*}
&\widehat \rho = \frac{1}{4}(1-t) + t\frac{1}{0.9921}e^{-3((y_1+0.2)^2 + (y_2+0.2)^2))},\\
&\rho_0 = \frac{1}{4},\ \
q_{T} = 0,\ \
\vec{w} = 0,\ \
f =0,\ \
V_{ext} =0.
\end{align*}
The numerical results for this example are displayed in Table \ref{TabS5:Prob22D}. In figures \ref{rhoHat2dEx4} and \ref{rhoOpt2dEx4} the results are illustrated. It can be observed very clearly that the control is driving to the desired state. It is noticeable that the peak of the desired state does not have to be supported as much as the slopes. This is due to the attractive interactions of the particles in this configuration and cannot be observed for repulsive particles.


\begin{table}
\begin{tabular}{ | c | c || c | c | c | c ||}
\hline
\multicolumn{2}{|c||}{}& $\beta = 10^{-3}$ & $\beta = 10^{-1}$ & $\beta = 10^{1}$ & $\beta = 10^{3}$  \\
\hline
\hline
 & $\mathcal{J}_{uc}$ & $\numprint{0.0378}$ & $\numprint{0.0378}$ & $\numprint{0.0378}$ & $\numprint{0.0378}$ \\
$\kappa= \numprint{-1}$  & $\mathcal{J}_c$ & $\numprint{0.0017}$ & $\numprint{0.0312}$ & $\numprint{0.0377}$ & $\numprint{0.0378}$ \\
& \texttt{Iter} & $\numprint{691}$ & $\numprint{736}$ & $\numprint{347}$ & $\numprint{1}$ \\
\hline
 & $\mathcal{J}_{uc}$ & $\numprint{0.0478}$ & $\numprint{0.0478}$ & $\numprint{0.0478}$ & $\numprint{0.0478}$ \\
$\kappa= \numprint{0}$  & $\mathcal{J}_c$ & $\numprint{0.0064}$ & $\numprint{0.0450}$ & $\numprint{0.0478}$ & $\numprint{0.0478}$ \\
& \texttt{Iter} & $\numprint{718}$ & $\numprint{784}$ & $\numprint{343}$ & $\numprint{1}$ \\
\hline
 & $\mathcal{J}_{uc}$ & $\numprint{0.0526}$ & $\numprint{0.0526}$ & $\numprint{0.0526}$ & $\numprint{0.0526}$ \\
$\kappa= \numprint{1}$  & $\mathcal{J}_c$ & $\numprint{0.0137}$ & $\numprint{0.0514}$ & $\numprint{0.0526}$ & $\numprint{0.0526}$ \\
& \texttt{Iter} & $\numprint{735}$ & $\numprint{790}$ & $\numprint{338}$ & $\numprint{1}$ \\
\hline
\end{tabular}
\caption{2D Ex. 2: Cost when $\vec{w}=\vec{0}$, optimal control cost, and iterations required, for a range of $\kappa$, $\beta$.}
\label{TabS5:Prob22D}
\end{table} %\label{TabS5:Prob22D}

\begin{figure}[h]
	\includegraphics[scale=0.3]{Figure32D.png}
	\caption{2D Example 4, uncontrolled $\rho$ and $\widehat \rho$, $\beta = 10^{-3}$, $\gamma = -1$. }
	\label{rhoHat2dEx4}
\end{figure}
\begin{figure}[h]
	\includegraphics[scale=0.3]{Figure42D.png}
	\caption{2D Example 4, controlled $\rho$ and optimal control $\vec{w}$, $\beta = 10^{-3}$, $\gamma = -1$.}
	\label{rhoOpt2dEx4}
\end{figure}








	
	
	\section{Conclusion}
	During the past year, a fast and accurate optimization solver has been developed, which reliably solves various optimal control problems. In the course of the next year, this will be applied to different model problems, such as models with inertial effects. Furthermore, the numerical method is extended to be applied to different domains. At the present time, only rectangular domains are considered. However, in the following year, the optimal control problems will be solved on more complex domains, which are composed of quadrilateral and circular shapes. 
	Other possibilities to be considered are models including multiple species and different control types other than flow and source control.
	The main aim is to extend the model and the numerical method to industrial applications.
	

\section*{Year 3}	
	
	
	
	
	
	
	
	
	
	
	
	\pagebreak	
	\bibliography{GeneralBib}
	\bibliographystyle{unsrt}
	
	
\end{document}