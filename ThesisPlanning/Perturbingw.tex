
As detailed in Section \ref{sec:Method_SolverFP}, it is necessary to provide an initial guess for the control $\mathbf{w}$ to start the optimization routine. Therefore, one way of validating the numerical method is to perturb the exact solution for $\mathbf{w}$, taken from a test problem with analytic solution, and use this as an initial guess in the optimization solver. In the first iteration, the solutions for $\rho$ and $\adj$ differ from the exact solution. The optimization method then converges to the exact, optimal solution. We consider an exact solution for the overdamped flow control problem \eqref{eqn:ADFlowOCP}, with no-flux boundary conditions, and no particle interaction term. This specific exact solution can be found in our paper's supplementary material (Section A, Test Problem 2). 
The following two perturbation functions are considered. The first perturbation is in time only and is defined as:
\begin{align*}
g(t) &= \frac{1}{2} f(t-t_0, a) \times f(t-t_0, -a)\\
&= \frac{1}{2} \frac{e^{-a/(t-t_0)}}{e^{-a/(t-t_0)} + e^{-a/(1-t -t_0)}} \times \frac{e^{a/(t-t_0)}}{e^{a/(t-t_0)} + e^{a/(1-t - t_0)}},
\end{align*}
and normalised by:
\begin{align*}
\tilde g(t) = \frac{g(t)}{\max{|{g(t)}|}}.
\end{align*}
A similar perturbation can be done in space, taking into account the difference in length of spatial and time domains:
\begin{align*}
h(x) &= \frac{1}{2} f(x-x_0, 2a) \times f(x-x_0, -2a)\\
&= \frac{1}{2} \frac{e^{-2a/(x-x_0)}}{e^{-2a/(x-x_0)} + e^{-2a/(1-x-x_0)}} \times \frac{e^{2a/(x-x_0)}}{e^{2a/(x-x_0)} + e^{2a/(1-x-x_0)}}.
\end{align*}
Again, this is normalised:
\begin{align*}
\tilde h(x) = \frac{h(x)}{\max{|{h(x)}|}}.
\end{align*}
These perturbation functions are chosen such that the perturbation is smooth and respects the initial condition for $\rho$, as well as the final time condition for $\adj$, by not changing the first or final time point. If this is not respected, the algorithm converges up to a point and then diverges, since the boundary conditions in time cannot be matched.
The considered perturbations are applied to the exact solution of the control, $\mathbf{w}_{ex}$, as follows:
\begin{align*}
\mathbf{w}_{pert1} &= \mathbf{w}_{ex}(1+ \epsilon \tilde g(t))\\
\mathbf{w}_{pert2} &= \mathbf{w}_{ex}(1+ \epsilon \tilde g(t) \tilde h(x)),
\end{align*}
where $a = 0.7$, $x_0 = t_0 = -0.01$ and the perturbation strength is either $\epsilon = 0.1$ or $\epsilon = 0.5$.
The chosen number of points is $N =30$ and $n=20$, the ODE tolerances are $10^{-8}$ and the optimality tolerance is $10^{-4}$. The mixing rate for the optimization solver is $\lambda = 0.01$.
The results presented in Table \ref{TabA2:Prob11} show the initial error in $\mathbf{w}$, $\mathcal{E}_{\mathbf{w}_{uc}}$, and the final errors in $\mathbf{w}$, $\rho$ and $\adj$, measured in the norm presented in Section \ref{sec:ErrorMeasure}, with respect to the exact solution. The initial error $\mathcal{E}_{\mathbf{w}_{uc}}$ is proportional to the perturbation strength $\epsilon$. The final errors for $\mathbf{w}$ and $\rho$ and $\adj$ are mostly within the specified optimality tolerance regardless of the perturbation strength and location. 

\begin{table}
\begin{tabular}{ | c | c || c | c | c | c ||}
\hline
  \multicolumn{2}{|c||}{} & $\beta = 10^{-3}$ & $\beta = 10^{-1}$ & $\beta = 10^{1}$ & $\beta = 10^{3}$  \\
\hline
\hline
\multirow{4}{*}{$0.1 \tilde g(t)$} & $\mathcal{E}_{\mathbf{w}_{uc}}$ & $\numprint{1.0000e-1}$ & $\numprint{1.0000e-1}$ & $\numprint{1.0000e-1}$ & $\numprint{1.0000e-1}$ \\
 & $\mathcal{E}_{\mathbf{w}_c}$ & $\numprint{5.3770e-5}$ & $\numprint{5.2340e-5}$ & $\numprint{5.2201e-5}$ & $\numprint{5.2203e-5}$ \\
 & $\mathcal{E}_{\rho}$ & $\numprint{1.1396e-5}$ & $\numprint{7.8597e-5}$ & $\numprint{7.8595e-5}$ & $\numprint{7.8597e-5}$ \\
 & $\mathcal{E}_{\adj}$ & $\numprint{2.7854e-5}$ & $\numprint{2.7836e-4}$ & $\numprint{5.7043e-4}$ & $\numprint{5.7045e-4}$ \\
\hline
\multirow{4}{*}{$0.5 \tilde g(t)$} & $\mathcal{E}_{\mathbf{w}_{uc}}$ & $\numprint{5.0000e-1}$ & $\numprint{5.0000e-1}$ & $\numprint{5.0000e-1}$ & $\numprint{5.0000e-1}$ \\
 & $\mathcal{E}_{\mathbf{w}_c}$ & $\numprint{2.1970e-4}$ & $\numprint{2.1747e-4}$ & $\numprint{2.1735e-4}$ & $\numprint{2.1735e-4}$ \\
 & $\mathcal{E}_{\rho}$ & $\numprint{2.4256e-5}$ & $\numprint{2.2878e-4}$ & $\numprint{2.2878e-4}$ & $\numprint{2.2879e-4}$ \\
 & $\mathcal{E}_{\adj}$ & $\numprint{3.3247e-5}$ & $\numprint{3.3227e-4}$ & $\numprint{6.8088e-4}$ & $\numprint{6.8090e-4}$ \\
\hline
\multirow{4}{*}{$0.1 \tilde h(x)$} & $\mathcal{E}_{\mathbf{w}_{uc}}$ & $\numprint{8.5568e-2}$ & $\numprint{8.5568e-2}$ & $\numprint{8.5568e-2}$ & $\numprint{8.5568e-2}$ \\
 & $\mathcal{E}_{\mathbf{w}_c}$ & $\numprint{5.3700e-5}$ & $\numprint{5.2250e-5}$ & $\numprint{5.2100e-5}$ & $\numprint{5.2103e-5}$ \\
 & $\mathcal{E}_{\rho}$ & $\numprint{1.1704e-5}$ & $\numprint{7.7973e-5}$ & $\numprint{7.7969e-5}$ & $\numprint{7.7968e-5}$ \\
 & $\mathcal{E}_{\adj}$ & $\numprint{2.6426e-5}$ & $\numprint{2.6387e-4}$ & $\numprint{5.6982e-4}$ & $\numprint{5.6984e-4}$ \\
\hline
\multirow{4}{*}{$0.5 \tilde h(x)$} & $\mathcal{E}_{\mathbf{w}_{uc}}$ & $\numprint{4.2784e-1}$ & $\numprint{4.2784e-1}$ & $\numprint{4.2784e-1}$ & $\numprint{4.2784e-1}$ \\
 & $\mathcal{E}_{\mathbf{w}_c}$ & $\numprint{2.1203e-4}$ & $\numprint{2.0982e-4}$ & $\numprint{2.0967e-4}$ & $\numprint{2.0968e-4}$ \\
 & $\mathcal{E}_{\rho}$ & $\numprint{2.2565e-5}$ & $\numprint{2.1275e-4}$ & $\numprint{2.1274e-4}$ & $\numprint{2.1275e-4}$ \\
 & $\mathcal{E}_{\adj}$ & $\numprint{3.0225e-5}$ & $\numprint{3.0219e-4}$ & $\numprint{6.1920e-4}$ & $\numprint{6.1923e-4}$ \\
\hline
\end{tabular}
\caption{Error measures for $\mathbf{w}_{uc}$, $\mathbf{w}_{c}$, $\rho$, and $\adj$, for four perturbation strategies for $\mathbf{w}$, and a range of $\beta$.}
\label{TabA2:Prob1}
\end{table} \label{TabA2:Prob11}
