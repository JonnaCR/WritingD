While mean-field games were first introduced by Lasry and Lions, \cite{LASRY2006619}, \cite{LASRY2006679},\cite{LASRY4} and \cite{Lasry2007}, and independently by Huang, Caines and Malham\'e,  \cite{Huang1}, under the name Nash certainty equivalence, the optimal control side of this class of problems is quite a new area of research. The main difficulty is a non-linear, non-local particle interaction term. Therefore, standard results in optimal control theory cannot readily be applied, and new approaches have to be developed to address theoretical and numerical challenges.
\\
\\
There are two types of models that recent work has focussed on. The most popular model is a deterministic microscopic model, which is a generalization of the well-known Cucker-Smale model, see \cite{CuckerSmale1}, \cite{CuckerSmale2}. In the mean-field limit, a Vlasov-type PDE arises. For control problems involving this class of models, the work by Fornasier et al. provides a range of theoretical results on the convergence of the microscopic optimal control problem to a corresponding macroscopic problem, using methods of optimal transport and a $\Gamma$-limit argument, proving existence of optimal controls in the mean-field setting, see \cite{Fornasier_2014},
\cite{Fornasier_2014no2}
and \cite{fornasier_lisini_orrieri_savare_2019}. The work focusses on sparse control strategies, where one or more agents influence a larger crowd.
Additional work on sparse control strategies can be found in \cite{piccoli2014no1}, as well as in the review paper \cite{Fornasier_20161no1}.
In \cite{burger2019meanfield}, an alternative method, an $L_2$ calculus, is developed, and convergence results are proved. The control in this work is applied through external agents. 
\\
Numerical advances have been made in \cite{burger2019instantaneous} and \cite{burger2016controlling}, where sparse and other control strategies through the external agents are considered. In both papers a Strang-Splitting scheme, \cite{ChengC.Z1976Tiot}, is applied to solve the optimal control problem. The numerical results verify the convergence of the microscopic control problem to its mean-field limit.
Furthermore, in \cite{albi2016selective}, different selective control strategies are considered, and an iterative numerical method is chosen, where the interaction term is approximated stochastically.
\\
\\
Fewer work has been done on the optimal control of the Fokker-Planck PDE, which arises as the mean-field limit of a stochastic microscopic model. Some theoretical results on this model are published. In \cite{albi2016mean}, the existence of optimal controls for microscopic and macroscopic versions of a class of problems is proved and a rigorous derivation of first-order optimality conditions is given. 
Following this, \cite{carrillo2019mean} discusses the existence and regularity of an optimal control problem of this type on periodic domains, including the well-posedness of the Fokker-Planck equation. In \cite{Pinnau_2017} and  \cite{carrillo2018no1}, the convergence of the microscopic optimal control problem to its mean-field limit is proved.
Numerical results on the model include those presented in \cite{Pinnau_2017}, where a Strang-Splitting scheme, \cite{gilbertstrang1}, is applied, and in which convergence to the mean field optimal control problem is shown numerically. Furthermore, in \cite{albi2016mean}, an optimal control hierarchy, including instantaneous and Boltzmann-type controls, is proposed. The mean-field first-order optimality system in \cite{albi2016mean} is solved using a Chang-Cooper scheme for the forward equation, finite differences for the adjoint equation, while approximating the integrals using a Monte-Carlo scheme. This is coupled by a sweeping algorithm, where updates are made through the gradient equation.
Some numerical results on a porous media version of the Fokker-Planck equation are presented in \cite{carrillo2018no1}. In \cite{Albi_2014no1} and \cite{albi2014kinetic}, steady state solutions to a Fokker-Planck-type PDE are considered, however, the main focus  are Boltzmann-type approaches to solving the optimal control problem.
\\
\\
The most common control types in the literature are flow control, e.g. \cite{albi2016mean}, control through the interaction term, e.g. \cite{Fornasier_2014no2}, as well as control through external agents, e.g. \cite{burger2016controlling}. 
Most papers do not consider boundary conditions, because it is assumed that the particle distribution is of compact support, see \cite{fornasier_lisini_orrieri_savare_2019}, \cite{burger2019meanfield}, or \cite{burger2016controlling}. No-flux boundary conditions, which are of high relevance in applications, are not often found in the literature, but are considered in \cite{albi2016mean} and \cite{carrillo2018no1}.
Our work considers the mean-field equation of Fokker-Planck type, flow-type control or control through a source term and no-flux or Dirichlet boundary conditions, in order to address a broad range of test problems and real world applications. 
\\
\\
As described above, some numerical methods have been developed for solving optimal control problems involving non-local, non-linear PDEs. Most of these papers however focus on other methods and use the mean-field optimal control as verification tool, see \cite{albi2016mean}, \cite{Pinnau_2017}. It takes large computational effort to solve these problems, which increases with dimensionality, see \cite{burger2019instantaneous}, \cite{burger2016controlling}. 
We are proposing a new numerical framework for PDE-constrained optimization applied to multiscale particle dynamics, where a fixed point algorithm is implemented to solve the first-order optimality system. This update scheme is inspired by the sweeping algorithm in \cite{albi2016mean}, and similar to the gradient descent method in \cite{Burger1}. The algorithm is coupled with pseudospectral methods, used to discretize space and time domains. This composition of methods offers an efficient and accurate solver for the class of problems discussed. To our knowledge, it is the first time that pseudospectral methods are used in the context of optimal control problems.