\documentclass[11pt, letterpaper]{article}
\usepackage[left=0.6 in,top=0.6in,right=0.6 in,bottom=0.6in]{geometry} % Document margins
\usepackage{ragged2e}
\usepackage[utf8]{inputenc}
\usepackage[english]{babel}
\setlength{\parindent}{0pt}


\begin{document}
	\thispagestyle{empty}
	
\RaggedLeft
Jonna Roden\\
School of Mathematics \\
The University of Edinburgh  \\
James Clerk Maxwell Building\\
Peter Guthrie Tait Road\\
{Edinburgh EH9 3FD}\\

	
\RaggedRight
Smith Institute\\
Willow Court\\
West Way\\
Minns Business Park\\
Oxford\\
OX2 0JB	\\	
\vspace{0.6 cm}
$11/05/2021$

\justify
\vspace{0.4 cm}
Dear Sir or Madam,\\
\\
\noindent
I would like to apply for the mathematics internship at the Smith Institute. I am a third year PhD student at the Maxwell Institute Graduate School for Analysis and its Applications, a joint programme by the Universities of Edinburgh and Heriot-Watt. I am working on modelling and optimization of particle dynamics, with a focus on the numerical implementation of efficient algorithms and on industrial applications.\\
\\
\noindent
I believe that applied and computational mathematics can have a significant impact on business solutions, through accurate modelling, efficient implementation of algorithms and effective application of optimization techniques. My career goal is to use my mathematical expertise to help companies work at their optimum, by supporting them in tackling business challenges and by optimizing processes to yield time, energy and financial savings. Working at the Smith Institute with a variety of clients perfectly aligns with this goal, by giving me the opportunity to tackle new challenges on a daily basis and to continually expand my skill set in the process. I am particularly impressed with the institute's focus on algorithm auditing, which is an important tool, given that data and algorithms increasingly inform business decisions across sectors.
\\
\\
\noindent
Key skills when working with industry are the ability to communicate mathematical ideas to a non-expert audience, to understand business challenges as a mathematical problem and to collaborate effectively in an interdisciplinary environment. Therefore, I prioritise science communication and outreach activities in my PhD training and I have organised several events with the aim of bringing together people from different career stages, across scientific fields and industry. I have sought out and worked on interdisciplinary projects, which moreover demonstrates my ability to quickly acquire necessary mathematical skills. As a PhD student in mathematics, organisation and time management, as well as the ability to work independently on a complex task, come naturally to me, and have been crucial in successfully working remotely under minimal supervision during the past year.  
I believe that my strong academic record, my proven ability to work independently and collaboratively on complex mathematical problems and my passion for industry and business applications of mathematics make me a great fit for your institute.
\\
\\
\noindent
I would be interested in starting an internship with the Smith Institute any time from September 2021 and I am flexible regarding the duration of the placement. If an earlier starting date would be preferred, I am happy to discuss this as well.
\\
\\
\noindent
I look forward to hearing from you.\\
\\
\noindent
Yours Sincerely,\\
Jonna Roden

 	
	
	
	
	
	
	
\end{document}