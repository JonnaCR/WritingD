\documentclass{resume} % Use the custom resume.cls style
\usepackage[left=0.4 in,top=0.4in,right=0.4 in,bottom=0.4in]{geometry} % Document margins
\usepackage{hyperref}

\newcommand{\tab}[1]{\hspace{.2667\textwidth}\rlap{#1}} 
\newcommand{\itab}[1]{\hspace{0em}\rlap{#1}}
\name{Jonna C Roden} % Your name
 % Your phone number and email

\begin{document}
School of Mathematics \hfill{J.C.Roden@sms.ed.ac.uk} \\
The University of Edinburgh  \hfill{07561078171}\\
James Clerk Maxwell Building\\
Peter Guthrie Tait Road\\
{Edinburgh EH9 3FD}

\begin{rSection}{Profile}
I am a PhD student in applied and computational mathematics working on modelling and optimization of industrial processes. A particular focus of my work lies on model simulation and the implementation of efficient and robust optimization algorithms in Matlab. My career goal is to bring my mathematical expertise from academia into industry, to help tackle the various challenges that we are faced with as a society today, while continuously growing my own skill set in the process. A crucial aspect of working with industry is the ability to explain mathematical results to a non-scientific audience, which is why I made science communication a priority in my PhD training. I have also co-organised several events which aim at building connections within the student community, to other scientific disciplines and to industry. I believe that innovative solutions to real-world problems are found through interdisciplinary collaboration and that effectively applied mathematics can have a significant impact on this process.
\vspace{0.7 cm}
\end{rSection}


	
\begin{rSection}{Education}
	{\bf PhD Applied and Computational Mathematics} \hfill{since 09/2018}\\
	{\it Maxwell Institute Graduate School in Analysis and its Applications (MIGSAA)}\\
	Joint PhD programme of the University of Edinburgh and Heriot-Watt University.
	\begin{itemize}
		\item[$\circ$] PhD topic: PDE-Constrained Optimization for Multiscale Particle Dynamics, with Industrial Applications.
		\item[$\circ$]	Modules taken include: Stochastic Processes, Introduction to Python Programming, Data Driven Modelling, Methods of Industrial Mathematics, Elliptic and Parabolic PDEs, Functional Analysis.
	\end{itemize}

	
{\bf BSc Mathematics} \hfill{09/2014 - 06/2018}	\\
{\it University of Strathclyde}\\	
	First Class Honours, with an average of 86\%. 
	\begin{itemize}
		\item[$\circ$] Modules taken include: Mathematical and Statistical Computing, Numerical Analysis, Mechanics of Rigid Bodies and Fluids, Finite Element Methods for Boundary Value Problems and Approximation, Mathematical Biology.
	\end{itemize}	

{\bf German Abitur} \hfill{08/2011 - 07/2014}\\
{\it Berufliches Gymnasium Gesundheit und Soziales}	
\begin{itemize}
	\item[$\circ$] Abitur Grade: 1.6 (equivalent to A Level grade A*AA)
	\item[$\circ$] Main Subjects: Psychology/Pedagogy, Mathematics, English.
\end{itemize}	
\vspace{0.4 cm}
\end{rSection}	
\begin{rSection}{Teaching Experience}
	{\bf University tutor (University of Edinburgh)} \hfill{since 09/2019}\\
	Modules include: Honours Algebra Skills (Computer Lab, using Sage), Linear Algebra, Several Variable Calculus and Differential Equations, Honours Complex Variables, Engineering Mathematics.\\
	\\
	{\bf Exam marker (University of Edinburgh)}\hfill{since 12/2019}\\
	Modules include: Linear Algebra, Several Variable Calculus and Differential Equations, Honours Complex Variables.
\end{rSection}

\begin{rSection}{Selected Projects}
	{\bf An Optimal Control Problem for a Marine Propulsion System} \hfill{02/2019 - 04/2019}\\
 	Modelling of a diesel-electric marine propulsion system, using deterministic and neural network models and optimization of the models for fuel efficiency. Collaboration with the industrial partner Duodrive and a team of PhD students. Implementation of the model and optimization routine in Matlab.\\

	{\bf BSc Thesis:}\hfill{01/2018 - 04/2018}\\
	{\bf The Method of Conformal Mappings Applied to Fluid Flow Around Obstacles} \\
	Gained in-depth understanding of the method of conformal mappings as well as concepts in aerodynamics and fluid mechanics involved and completed an extensive literature review on the topic. Presentation of the mathematical content to a general audience.\\
	
	{\bf Vertically Integrated Project (VIP)} \hfill{02/2016 - 05/2016}\\
	Selected to contribute to an interdisciplinary research project on antibiotic drug discovery. Used mathematical techniques and programming, such as clustering algorithms and statistical models. Worked in a team with Biology and Engineering students as well as with supervising professors.	Presented the joint result at a VIP conference. 
\end{rSection}
\begin{rSection}{Outreach Activities and Science Communication}
	{\bf PhD Alumni Talk Series} \hfill{10/2020 - 12/2020}\\
	Co-organised a series of talks by Mathematics PhD alumni on their career path after their PhD graduation, providing PhD students with an opportunity to network with alumni from their university.\\
	\\
	{\bf Outreach Event for BSc and MSc Students}\hfill{11/2020}\\
	Co-organised an outreach event for BSc and MSc Students, featuring talks by several current PhD students in Mathematics, as well as networking opportunities. \\
	\\
	{ \bf Student Research Article}\hfill{10/2020}\\
	Wrote an article describing my work and life as a PhD student to a non-mathematical, general audience. The article can be found at  \url{https://www.maths.ed.ac.uk/school-of-mathematics/news?nid=875} \\
	\\
	{ \bf Mathematics Communication and Public Engagement Workshop} \hfill{09/2020}\\
	Participated in a workshop by the distinguished science communicators Rob Eastaway and Ben Sparks to further my skills in public engagement and communication.\\
	\\
	{\bf Outreach Presentation to BSc Students} \hfill{10/2019}\\
	Gave a talk illustrating the advantages and challenges of being a PhD student in Mathematics, networking and answering students' questions about applying for and doing a PhD. 
\end{rSection}
\begin{rSection}{Leadership and Professional Experience}
	{\bf SIAM-IMA Student Chapter Committee Member} \hfill{since 06/2020}\\
	Organization of events to connect PhD students in mathematics with those from other disciplines and to industry, as well as outreach events for BSc and MSc students.\\
	\\
	{\bf University of Edinburgh PG Colloquium Organizer} \hfill{09/2019 - 08/2020}\\
	Organization of a seminar series where PhD students talk about their research or other interesting mathematical topics.\\
	\\
	{\bf Annual MIGSAA Colloquium Organizing Committee} \hfill{05/2019 - 09/2020}\\
	Organization of the annual MIGSAA Colloquium 2019, featuring research talks by three academics from the UK and Europe and an audience of 70 attendees.\\
	\\
\end{rSection}




\begin{rSection}{Awards}
	{\bf Travel Award}   \hfill{10/2019}	\\
	Awarded by the Autumn School on Optimal Control and Optimization with PDEs, Trier, Germany. \\
	\\
	{\bf PhD Funding, 4 Years} \hfill{09/2018}\\	
	Awarded by the Maxwell Institute Graduate School in Analysis and its Applications, a Centre for Doctoral Training funded by the UK Engineering and Physical Sciences Research Council (grant EP/L016508/01), the Scottish Funding Council, Heriot-Watt University and the University of Edinburgh.\\
	\\
	{\bf Astronomical Society of Glasgow Prize} \hfill{06/2018}\\ 
	Awarded for the most distinguished student in the final examinations for a BSc Honours or MSc degree in Mathematics or Physics.\\
	\\
	{\bf Member of the Dean’s List (Third Year)} \hfill{06/2017}\\
	List of students who have achieved a meritorious standard in the third year of the BSc Mathematics.
	
\end{rSection}



\begin{rSection}{Talks and Poster Presentations}
	New Directions in Classical Density Functional Theory Workshop (Contributed Talk) \hfill{05/2021}\\
	British Applied Mathematics Colloquium (Contributed Talk) \hfill{04/2021}\\
	LMS Scottish Numerical Methods Network Workshop on Multiscale Methods (Invited Talk)  \hfill{06/2020}\\
	SIAM-IMA Student Chapter Colloquium (Invited Talk)\hfill{02/2020}\\
	Autumn School on Optimal Control and Optimization with PDEs, Trier, Germany \hfill{10/2019}
	
	
\end{rSection}
\begin{rSection}{Publications}
	Aduamoah, M., Goddard, B.D., Pearson, J.W. \& Roden, J.C. PDE-Constrained Optimization Models and Pseudospectral Methods for Multiscale Particle Dynamics. Preprint at \url{https://arxiv.org/abs/2009.09850}(2020)
\end{rSection}	
	
	\begin{rSection}{Programming Skills and Languages}
		{\bf Programming}\\
		Highly proficient in Matlab. Knowledge of Python and R.\\
		\\
		{\bf Languages}\\
		Fluent in German and English. Basic knowledge of Spanish.
	\end{rSection}
\begin{rSection}{References}
	\begin{table}[h]
			\begin{tabular}{llll}
			\ &	Dr Ben Goddard & \ \ \ \ \ \ \ \ &{Dr John Pearson}\\
			\  &	Reader & \ \ \ \ \ \ \ \ &Reader \\
			\ &	University of Edinburgh & \ \ \ \ \ \ \ \ & University of Edinburgh\\
			\  &	School of Mathematics & \ \ \ \ \ \ \ \ &School of Mathematics\\
			\  &	B.Goddard@ed.ac.uk & \ \ \ \ \ \ \ \ &j.pearson@ed.ac.uk                        
			\end{tabular}
	\end{table}
 
	
\end{rSection}	
	
\end{document}	
	