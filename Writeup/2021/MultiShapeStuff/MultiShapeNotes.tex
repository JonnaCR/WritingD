
\section{Validation Tests Multishape}
(Note: all the things from this section are in PDECO/MultiShapeTesting in Matlab. The following files are particularly relevant:
PlottingNormalComparison, TablesExactSolutions, TableForwardExamples, TablesDiffIntetc and MultiShapeAssorted. The other files are called inside these wrappers. Plotting is currently commented out but burried in there somewhere...)

\subsection{Testing Differentiation, Interpolation, Convolution and Integration} \label{sec:ValidationDiffIntetc}
TablesDiffIntetc() in Matlab
	\begin{figure}[h]
	\centering
	\includegraphics[scale=0.35]{BoxSections.png}
	\caption{Different discretizations of the box (a - g).} 
	\label{F2}
\end{figure}
\begin{figure}[h]
	\centering
	\includegraphics[scale=0.35]{quad.png}
	\caption{Quadrilateral domain (q).} 
	\label{F3a}
\end{figure}
\begin{figure}[h]
	\centering
	\includegraphics[scale=0.35]{WedgeSections.png}
	\caption{Different discretizations of the wedge (h - k).} 
	\label{F4}
\end{figure}


We investigate the accuracy of differentiation, interpolation, integration and convolution comparing different discretizations of the box and a wedge. The different discretizations can be seen in Figures \ref{F2}, \ref{F3a} and \ref{F4}.
The operations are validated using the exact solution
\begin{align}\label{eq:exactsol1}
	\rho &= \exp( \alpha_1  t + \beta_1 y_1 + \beta_2 y_2),\\
	\nabla \rho &= [\beta_1 \rho, \beta_2 \rho],\notag \\
	\nabla^2 \rho &= \nabla \cdot \nabla \rho = (\beta_1^2 + \beta_2^2)\rho,\notag
\end{align}
where $	\beta_1 = 0.1 $, $\beta_2 = 0.1$, $\alpha_1 = -0.5$.


We test each of the operators $\Grad$, $\Div$ and $\Lap$ by computing the error between the numerical and the exact solution
\begin{align}\label{eq:ErrorMeasure1}
	\mathcal E_{\text{Abs}} (f)&= \max \max \abs{f_{num} - f_{ex}}\\
	\mathcal E_{\text{Rel}} (f)&= \frac{\mathcal E_{\text{Abs}}(f) }{ \max \max \abs{f_{ex}}}.
\end{align}
The $\Div$ operator is tested by taking the divergence of the exact solution for $\nabla \rho$ and comparing it to the exact solution for $\nabla^2 \rho$. Note that the exact $\nabla \rho$ has to be transformed into polar coordinates first. The solutions can be seen in Tables \ref{Tab:Grad}, \ref{Tab:div} and \ref{Tab:Lap}.

We investigate the functionality of the interpolation matrix by interpolating from the different multishapes onto a uniform grid. This grid is created by setting up a uniform rectangular grid which fully contains the multishape. Then we loop through the shapes and interpolate the test function $\rho$ \eqref{eq:exactsol1} onto the uniform points that lie inside each shape. In the end we discard the points that lie outside the multishape. This causes the final solution on the uniform grid to vary in size, depending on the discretization of the multishape. We test the functionality by comparing the interpolated function values to $\rho$ evaluated on the uniform grid points, using \eqref{eq:ErrorMeasure1}. The results can be seen in Table \ref{Tab:Interp}.

In a second test we interpolate a function $\rho$, \eqref{eq:exactsol1}, computed with $N =50$ on shapes (a) and (h) respectively onto a uniform grid, as described above. Then we compute \eqref{eq:exactsol1} on the multishapes (b), (f), and (g) with $N = 10$, $N = 20$ and $N = 30$. We interpolate this onto the same uniform grid as shape (a) using the physical interpolation function. The values of $\rho$ coming from (a) can be compared with those from (b), (f), and (g) on the uniform grid, using (a) as a reference solution. Similarly, the solution on shape (h) serves as reference solution for multishapes (j) and (k). The errors, computed using \eqref{eq:ErrorMeasure1} can be seen in Table \ref{Tab:InterpComp}.

Finally, we investigate what happens when we consider interpolation to a uniform grid with $N_1 \neq N_2$ for the wedge, when compared to the exact solution on the uniform grid. This is interesting, because when the solution is interpolated onto a uniform grid (in Cartesian coordinates), it is to be expected that more points are needed in the angular direction than in the radial direction, since in Cartesian coordinates points are more densely clustered in radial direction, as can be seen in Figure \ref{F4}. This is indeed what happens, as can be seen in Table \ref{Tab:InterpPol}. We get considerably better results for small $N_1$ than for small $N_2$ and vice versa. We could investigate whether this behaviour changes when we choose a shape with small angle and long radius instead.

We consider a similar approach for testing the convolution matrix. We compute the convolution matrix with $N = 50$ on shape (a) (or (h) respectively) and apply it to the function $\rho$ in equation \eqref{eq:exactsol1}. We then interpolate this onto the other shapes and compute the error \eqref{eq:ErrorMeasure1} with the convolution of $\rho$ on that shape. We use the value of the convolution on a single shape with $N = 50$ as the reference value for the multishape convolution. The results are displayed in Table \ref{Tab:Conv}.

For the integration vector, we compute the integral of $\rho$ on shape (a) and compare this to the integral of $\rho$ on multishapes (b), (f) and (g). We then compare the integral of $\rho$ on the single shape (h) to the one on discretizations (j) and (k). The errors computed with \eqref{eq:ErrorMeasure1} can be seen in Table \ref{Tab:Int}, where again the value of the integral on the single shape is the reference for the multishape integration.
\begin{table}
\centering
\begin{tabular}{ | c | c || c | c | c | c | c ||}
\hline
 & A.E. $ N=10$ & R.E. $N=10$ & A.E. $N = 20$ & R.E. $N = 20$ & A.E. $N=30$  & R.E. $N=30$ \\
\hline
\hline
 a & $\numprint{3.6637e-15}$ & $\numprint{4.0491e-14}$ & $\numprint{4.5158e-14}$ & $\numprint{4.9908e-13}$ & $\numprint{1.0987e-13}$ & $\numprint{1.2143e-12}$ \\
 b & $\numprint{2.5202e-14}$ & $\numprint{2.7853e-13}$ & $\numprint{1.3006e-13}$ & $\numprint{1.4374e-12}$ & $\numprint{3.9264e-13}$ & $\numprint{4.3394e-12}$ \\
 f & $\numprint{5.7149e-14}$ & $\numprint{6.3159e-13}$ & $\numprint{4.4410e-13}$ & $\numprint{4.9081e-12}$ & $\numprint{1.7717e-12}$ & $\numprint{1.9580e-11}$ \\
 h & $\numprint{3.1143e-5}$ & $\numprint{1.3588e-4}$ & $\numprint{1.3522e-11}$ & $\numprint{5.9203e-11}$ & $\numprint{4.0515e-13}$ & $\numprint{1.7705e-12}$ \\
 j & $\numprint{3.1143e-5}$ & $\numprint{1.3588e-4}$ & $\numprint{1.3522e-11}$ & $\numprint{5.9203e-11}$ & $\numprint{9.8901e-13}$ & $\numprint{4.3219e-12}$ \\
 k & $\numprint{9.8299e-8}$ & $\numprint{4.2888e-7}$ & $\numprint{3.1769e-13}$ & $\numprint{1.3866e-12}$ & $\numprint{8.6467e-13}$ & $\numprint{3.7732e-12}$ \\
\hline
\end{tabular}
\caption{Table $\Grad$}
\label{Tab:Grad}
\end{table}
\begin{table}
\centering
\begin{tabular}{ | c | c | c | c | c | c | c |}
\hline
 & \multicolumn{2}{c|}{$N = 10$}  & \multicolumn{2}{c|}{$N = 20$}  & \multicolumn{2}{c|}{$N = 30$} \\
\hline
 & $\mathcal E_{Abs}$ & $\mathcal E_{Rel}$ & $\mathcal E_{Abs}$ & $\mathcal E_{Rel}$ & $\mathcal E_{Abs}$  & $\mathcal E_{Rel}$ \\
\hline
 a & $\numprint{8.3267e-16}$ & $\numprint{4.6012e-14}$ & $\numprint{4.8798e-15}$ & $\numprint{2.6965e-13}$ & $\numprint{1.2278e-14}$ & $\numprint{6.7849e-13}$ \\
 b & $\numprint{2.9594e-15}$ & $\numprint{1.6353e-13}$ & $\numprint{1.3121e-14}$ & $\numprint{7.2507e-13}$ & $\numprint{4.0837e-14}$ & $\numprint{2.2566e-12}$ \\
 c & $\numprint{8.3614e-15}$ & $\numprint{4.6204e-13}$ & $\numprint{2.8682e-14}$ & $\numprint{1.5849e-12}$ & $\numprint{1.0649e-13}$ & $\numprint{5.8845e-12}$ \\
 d & $\numprint{7.9138e-15}$ & $\numprint{4.3731e-13}$ & $\numprint{4.2664e-14}$ & $\numprint{2.3575e-12}$ & $\numprint{9.6187e-14}$ & $\numprint{5.3152e-12}$ \\
 e & $\numprint{3.6846e-15}$ & $\numprint{2.0360e-13}$ & $\numprint{2.1975e-14}$ & $\numprint{1.2143e-12}$ & $\numprint{6.9682e-14}$ & $\numprint{3.8505e-12}$ \\
 f & $\numprint{9.7717e-15}$ & $\numprint{5.3997e-13}$ & $\numprint{6.4991e-14}$ & $\numprint{3.5913e-12}$ & $\numprint{1.3360e-13}$ & $\numprint{7.3824e-12}$ \\
 g & $\numprint{3.0878e-15}$ & $\numprint{1.7063e-13}$ & $\numprint{1.0623e-14}$ & $\numprint{5.8704e-13}$ & $\numprint{3.3662e-14}$ & $\numprint{1.8601e-12}$ \\
 h & $\numprint{5.2272e-5}$ & $\numprint{1.6127e-3}$ & $\numprint{3.4844e-11}$ & $\numprint{1.0758e-9}$ & $\numprint{5.3096e-14}$ & $\numprint{1.6387e-12}$ \\
 i & $\numprint{9.2692e-8}$ & $\numprint{2.8672e-6}$ & $\numprint{3.2897e-14}$ & $\numprint{1.0155e-12}$ & $\numprint{6.4497e-14}$ & $\numprint{1.9903e-12}$ \\
 j & $\numprint{5.2272e-5}$ & $\numprint{1.6127e-3}$ & $\numprint{3.4852e-11}$ & $\numprint{1.0761e-9}$ & $\numprint{1.3768e-13}$ & $\numprint{4.2490e-12}$ \\
 k & $\numprint{8.4307e-8}$ & $\numprint{2.6010e-6}$ & $\numprint{4.5634e-14}$ & $\numprint{1.4080e-12}$ & $\numprint{9.0605e-14}$ & $\numprint{2.7954e-12}$ \\
\hline
\end{tabular}
\caption{Table $\Div$}
\label{Tab:div}
\end{table}
\begin{table}
\centering
\begin{tabular}{ | c | c | c | c | c | c | c |}
\hline
 & \multicolumn{2}{c|}{$N = 10$}  & \multicolumn{2}{c|}{$N = 20$}  & \multicolumn{2}{c|}{$N = 30$} \\
\hline
 & $\mathcal E_{Abs}$ & $\mathcal E_{Rel}$ & $\mathcal E_{Abs}$ & $\mathcal E_{Rel}$ & $\mathcal E_{Abs}$  & $\mathcal E_{Rel}$ \\
\hline
 a & $\numprint{1.7055e-13}$ & $\numprint{9.4244e-12}$ & $\numprint{4.7951e-12}$ & $\numprint{2.6497e-10}$ & $\numprint{2.5029e-11}$ & $\numprint{1.3831e-9}$ \\
 b & $\numprint{1.3937e-12}$ & $\numprint{7.7015e-11}$ & $\numprint{4.4301e-11}$ & $\numprint{2.4480e-9}$ & $\numprint{1.2869e-10}$ & $\numprint{7.1114e-9}$ \\
 c & $\numprint{7.7744e-12}$ & $\numprint{4.2960e-10}$ & $\numprint{3.8083e-10}$ & $\numprint{2.1044e-8}$ & $\numprint{3.8902e-9}$ & $\numprint{2.1497e-7}$ \\
 d & $\numprint{1.4964e-11}$ & $\numprint{8.2688e-10}$ & $\numprint{5.5707e-10}$ & $\numprint{3.0783e-8}$ & $\numprint{1.6726e-9}$ & $\numprint{9.2426e-8}$ \\
 e & $\numprint{4.2647e-12}$ & $\numprint{2.3566e-10}$ & $\numprint{1.0580e-10}$ & $\numprint{5.8462e-9}$ & $\numprint{1.0035e-9}$ & $\numprint{5.5450e-8}$ \\
 f & $\numprint{1.4029e-11}$ & $\numprint{7.7521e-10}$ & $\numprint{5.8611e-10}$ & $\numprint{3.2388e-8}$ & $\numprint{3.0143e-9}$ & $\numprint{1.6657e-7}$ \\
 g & $\numprint{9.4664e-13}$ & $\numprint{5.2310e-11}$ & $\numprint{3.5819e-11}$ & $\numprint{1.9793e-9}$ & $\numprint{1.4403e-10}$ & $\numprint{7.9586e-9}$ \\
 h & $\numprint{5.3665e-4}$ & $\numprint{1.6556e-2}$ & $\numprint{1.0457e-9}$ & $\numprint{3.2287e-8}$ & $\numprint{1.7551e-10}$ & $\numprint{5.4167e-9}$ \\
 i & $\numprint{4.6482e-6}$ & $\numprint{1.4378e-4}$ & $\numprint{5.0420e-11}$ & $\numprint{1.5565e-9}$ & $\numprint{2.3952e-10}$ & $\numprint{7.3915e-9}$ \\
 j & $\numprint{5.3665e-4}$ & $\numprint{1.6556e-2}$ & $\numprint{1.0223e-9}$ & $\numprint{3.1565e-8}$ & $\numprint{7.2370e-10}$ & $\numprint{2.2335e-8}$ \\
 k & $\numprint{3.3996e-6}$ & $\numprint{1.0488e-4}$ & $\numprint{1.0291e-10}$ & $\numprint{3.1753e-9}$ & $\numprint{4.0443e-10}$ & $\numprint{1.2478e-8}$ \\
\hline
\end{tabular}
\caption{Table $\Lap$}
\label{Tab:Lap}
\end{table}
\begin{table}
\centering
\begin{tabular}{ | c | c | c | c | c | c | c |}
\hline
 & \multicolumn{2}{c|}{$N = 10$}  & \multicolumn{2}{c|}{$N = 20$}  & \multicolumn{2}{c|}{$N = 30$} \\
\hline
 & $\mathcal E_{Abs}$ & $\mathcal E_{Rel}$ & $\mathcal E_{Abs}$ & $\mathcal E_{Rel}$& $\mathcal E_{Abs}$  & $\mathcal E_{Rel}$ \\
\hline
 a & $\numprint{9.9920e-16}$ & $\numprint{1.1043e-15}$ & $\numprint{1.1102e-15}$ & $\numprint{1.2270e-15}$ & $\numprint{1.9984e-15}$ & $\numprint{2.2086e-15}$ \\
 b & $\numprint{8.8818e-16}$ & $\numprint{9.8159e-16}$ & $\numprint{1.5543e-15}$ & $\numprint{1.7178e-15}$ & $\numprint{1.8874e-15}$ & $\numprint{2.0859e-15}$ \\
 c & $\numprint{8.8818e-16}$ & $\numprint{9.8159e-16}$ & $\numprint{1.6653e-15}$ & $\numprint{1.8405e-15}$ & $\numprint{2.2204e-15}$ & $\numprint{2.4540e-15}$ \\
 d & $\numprint{8.8818e-16}$ & $\numprint{9.8159e-16}$ & $\numprint{1.9984e-15}$ & $\numprint{2.2086e-15}$ & $\numprint{2.6645e-15}$ & $\numprint{2.9448e-15}$ \\
 e & $\numprint{8.8818e-16}$ & $\numprint{9.8159e-16}$ & $\numprint{1.4433e-15}$ & $\numprint{1.5951e-15}$ & $\numprint{2.6645e-15}$ & $\numprint{2.9448e-15}$ \\
 f & $\numprint{1.2212e-15}$ & $\numprint{1.3497e-15}$ & $\numprint{1.6653e-15}$ & $\numprint{1.8405e-15}$ & $\numprint{2.8866e-15}$ & $\numprint{3.1902e-15}$ \\
 g & $\numprint{8.8818e-16}$ & $\numprint{9.8159e-16}$ & $\numprint{1.8874e-15}$ & $\numprint{2.0859e-15}$ & $\numprint{2.5535e-15}$ & $\numprint{2.8221e-15}$ \\
 h & $\numprint{4.6655e-6}$ & $\numprint{2.8922e-6}$ & $\numprint{7.9314e-13}$ & $\numprint{4.9000e-13}$ & $\numprint{2.6645e-15}$ & $\numprint{1.6450e-15}$ \\
 i & $\numprint{1.1935e-8}$ & $\numprint{7.3650e-9}$ & $\numprint{3.3307e-15}$ & $\numprint{2.0553e-15}$ & $\numprint{3.1086e-15}$ & $\numprint{1.9183e-15}$ \\
 j & $\numprint{4.6655e-6}$ & $\numprint{2.8922e-6}$ & $\numprint{7.9226e-13}$ & $\numprint{4.8945e-13}$ & $\numprint{4.2188e-15}$ & $\numprint{2.6047e-15}$ \\
 k & $\numprint{7.7730e-9}$ & $\numprint{4.8017e-9}$ & $\numprint{2.2204e-15}$ & $\numprint{1.3705e-15}$ & $\numprint{4.4409e-15}$ & $\numprint{2.7405e-15}$ \\
\hline
\end{tabular}
\caption{Table Interp}
\label{Tab:Interp}
\end{table}
\begin{table}
\centering
\begin{tabular}{ | c | c | c | c | c | c | c |}
\hline
 & \multicolumn{2}{c|}{$N = 10$}  & \multicolumn{2}{c|}{$N = 20$}  & \multicolumn{2}{c|}{$N = 30$} \\
\hline
 & $\mathcal E_{Abs}$ & $\mathcal E_{Rel}$ & $\mathcal E_{Abs}$ & $\mathcal E_{Rel}$ & $\mathcal E_{Abs}$  & $\mathcal E_{Rel}$ \\
\hline
 b & $\numprint{2.3315e-15}$ & $\numprint{2.5767e-15}$ & $\numprint{2.3315e-15}$ & $\numprint{2.5767e-15}$ & $\numprint{3.2196e-15}$ & $\numprint{3.5583e-15}$ \\
 c & $\numprint{2.4425e-15}$ & $\numprint{2.6994e-15}$ & $\numprint{2.8866e-15}$ & $\numprint{3.1902e-15}$ & $\numprint{2.5535e-15}$ & $\numprint{2.8221e-15}$ \\
 d & $\numprint{2.3315e-15}$ & $\numprint{2.5767e-15}$ & $\numprint{3.2196e-15}$ & $\numprint{3.5583e-15}$ & $\numprint{3.3307e-15}$ & $\numprint{3.6810e-15}$ \\
 e & $\numprint{2.3315e-15}$ & $\numprint{2.5767e-15}$ & $\numprint{2.6645e-15}$ & $\numprint{2.9448e-15}$ & $\numprint{2.8866e-15}$ & $\numprint{3.1902e-15}$ \\
 f & $\numprint{2.2204e-15}$ & $\numprint{2.4540e-15}$ & $\numprint{2.7756e-15}$ & $\numprint{3.0675e-15}$ & $\numprint{3.2196e-15}$ & $\numprint{3.5583e-15}$ \\
 g & $\numprint{2.6645e-15}$ & $\numprint{2.9448e-15}$ & $\numprint{2.9976e-15}$ & $\numprint{3.3129e-15}$ & $\numprint{3.6637e-15}$ & $\numprint{4.0491e-15}$ \\
 i & $\numprint{1.1886e-8}$ & $\numprint{7.3362e-9}$ & $\numprint{5.1070e-15}$ & $\numprint{3.1520e-15}$ & $\numprint{6.4393e-15}$ & $\numprint{3.9742e-15}$ \\
 j & $\numprint{4.8940e-6}$ & $\numprint{3.0205e-6}$ & $\numprint{8.1379e-13}$ & $\numprint{5.0226e-13}$ & $\numprint{4.6629e-15}$ & $\numprint{2.8779e-15}$ \\
 k & $\numprint{7.8923e-9}$ & $\numprint{4.8710e-9}$ & $\numprint{5.1070e-15}$ & $\numprint{3.1520e-15}$ & $\numprint{5.1070e-15}$ & $\numprint{3.1520e-15}$ \\
\hline
\end{tabular}
\caption{Table Interp Compare}
\label{Tab:InterpComp}
\end{table}
\begin{table}
\centering
\begin{tabular}{ | c | c | c | c | c | c | c |}
\hline
 & $N_1 = 20$ & $N_1 = 20$& $N_1 = 5$& $N_1 = 10$& $N_1 = 30$  & $N_1 = 20$ \\
\hline
 & $N_2 = 10$ & $ N_2 = 15$ & $ N_2 = 20$ & $ N_2 = 20$ & $ N_2 = 20$  & $ N_2 = 30$ \\
\hline
 h & $\numprint{2.8922e-6}$ & $\numprint{5.1117e-10}$ & $\numprint{1.5795e-9}$ & $\numprint{4.8973e-13}$ & $\numprint{4.9000e-13}$ & $\numprint{1.5080e-15}$ \\
 i & $\numprint{7.3650e-9}$ & $\numprint{1.6031e-14}$ & $\numprint{1.5807e-9}$ & $\numprint{1.2332e-15}$ & $\numprint{1.9183e-15}$ & $\numprint{2.4663e-15}$ \\
 j & $\numprint{2.8922e-6}$ & $\numprint{5.1117e-10}$ & $\numprint{5.1835e-11}$ & $\numprint{4.8959e-13}$ & $\numprint{4.8986e-13}$ & $\numprint{1.9192e-15}$ \\
 k & $\numprint{4.8017e-9}$ & $\numprint{7.6215e-14}$ & $\numprint{1.5808e-9}$ & $\numprint{1.2334e-15}$ & $\numprint{2.4668e-15}$ & $\numprint{2.0554e-15}$ \\
\hline
\end{tabular}
\caption{Table Comparing interpolation errors on wedges, when $N_1 \neq N_2$.}
\label{Tab:InterpPol}
\end{table}
\begin{table}
\centering
\begin{tabular}{ | c | c | c | c | c | c | c |}
\hline
 & \multicolumn{2}{c|}{$N = 10$}  & \multicolumn{2}{c|}{$N = 20$}  & \multicolumn{2}{c|}{$N = 30$} \\
\hline
 & $\mathcal E_{Abs}$ & $\mathcal E_{Rel}$ & $\mathcal E_{Abs}$ & $\mathcal E_{Rel}$ & $\mathcal E_{Abs}$  & $\mathcal E_{Rel}$ \\
\hline
 b & $\numprint{9.1924e-8}$ & $\numprint{5.5907e-8}$ & $\numprint{7.5495e-15}$ & $\numprint{4.5500e-15}$ & $\numprint{8.2157e-15}$ & $\numprint{4.9463e-15}$ \\
 c & $\numprint{9.2322e-8}$ & $\numprint{5.5989e-8}$ & $\numprint{8.8818e-15}$ & $\numprint{5.3498e-15}$ & $\numprint{1.3323e-14}$ & $\numprint{8.0197e-15}$ \\
 d & $\numprint{9.2351e-8}$ & $\numprint{5.6018e-8}$ & $\numprint{8.8818e-15}$ & $\numprint{5.3490e-15}$ & $\numprint{1.7097e-14}$ & $\numprint{1.0293e-14}$ \\
 e & $\numprint{1.8382e-8}$ & $\numprint{1.1094e-8}$ & $\numprint{9.1038e-15}$ & $\numprint{5.4844e-15}$ & $\numprint{1.3989e-14}$ & $\numprint{8.4207e-15}$ \\
 f & $\numprint{1.4109e-8}$ & $\numprint{8.5341e-9}$ & $\numprint{8.2157e-15}$ & $\numprint{4.9497e-15}$ & $\numprint{1.9984e-14}$ & $\numprint{1.2033e-14}$ \\
 g & $\numprint{1.9228e-10}$ & $\numprint{1.1579e-10}$ & $\numprint{9.7700e-15}$ & $\numprint{5.8815e-15}$ & $\numprint{1.5543e-14}$ & $\numprint{9.3564e-15}$ \\
 i & $\numprint{3.2840e-6}$ & $\numprint{1.3349e-6}$ & $\numprint{1.7055e-12}$ & $\numprint{6.9138e-13}$ & $\numprint{1.5987e-14}$ & $\numprint{6.4744e-15}$ \\
 j & $\numprint{1.6310e-3}$ & $\numprint{6.6333e-4}$ & $\numprint{5.0919e-8}$ & $\numprint{2.0616e-8}$ & $\numprint{8.9782e-12}$ & $\numprint{3.6342e-12}$ \\
 k & $\numprint{2.7196e-6}$ & $\numprint{1.1056e-6}$ & $\numprint{1.7668e-12}$ & $\numprint{7.1588e-13}$ & $\numprint{2.9310e-14}$ & $\numprint{1.1870e-14}$ \\
\hline
\end{tabular}
\caption{Table Conv}
\label{Tab:Conv}
\end{table}
\begin{table}
\centering
\begin{tabular}{ | c | c | c | c | c | c | c |}
\hline
 & \multicolumn{2}{c|}{$N = 10$}  & \multicolumn{2}{c|}{$N = 20$}  & \multicolumn{2}{c|}{$N = 30$} \\
\hline
 & $\mathcal E_{Abs}$ & $\mathcal E_{Rel}$ & $\mathcal E_{Abs}$ & $\mathcal E_{Rel}$ & $\mathcal E_{Abs}$  & $\mathcal E_{Rel}$ \\
\hline
 b & $\numprint{0.0000e+00}$ & $\numprint{0.0000e+00}$ & $\numprint{4.4409e-16}$ & $\numprint{1.4937e-16}$ & $\numprint{8.8818e-16}$ & $\numprint{2.9873e-16}$ \\
 c & $\numprint{0.0000e+00}$ & $\numprint{0.0000e+00}$ & $\numprint{4.4409e-16}$ & $\numprint{1.4937e-16}$ & $\numprint{1.3323e-15}$ & $\numprint{4.4810e-16}$ \\
 d & $\numprint{0.0000e+00}$ & $\numprint{0.0000e+00}$ & $\numprint{4.4409e-16}$ & $\numprint{1.4937e-16}$ & $\numprint{8.8818e-16}$ & $\numprint{2.9873e-16}$ \\
 e & $\numprint{0.0000e+00}$ & $\numprint{0.0000e+00}$ & $\numprint{8.8818e-16}$ & $\numprint{2.9873e-16}$ & $\numprint{1.3323e-15}$ & $\numprint{4.4810e-16}$ \\
 f & $\numprint{0.0000e+00}$ & $\numprint{0.0000e+00}$ & $\numprint{0.0000e+00}$ & $\numprint{0.0000e+00}$ & $\numprint{8.8818e-16}$ & $\numprint{2.9873e-16}$ \\
 g & $\numprint{0.0000e+00}$ & $\numprint{0.0000e+00}$ & $\numprint{4.4409e-16}$ & $\numprint{1.4937e-16}$ & $\numprint{8.8818e-16}$ & $\numprint{2.9873e-16}$ \\
 i & $\numprint{3.9603e-11}$ & $\numprint{6.1859e-12}$ & $\numprint{8.8818e-16}$ & $\numprint{1.3873e-16}$ & $\numprint{3.5527e-15}$ & $\numprint{5.5493e-16}$ \\
 j & $\numprint{2.4085e-8}$ & $\numprint{3.7621e-9}$ & $\numprint{1.7764e-15}$ & $\numprint{2.7746e-16}$ & $\numprint{2.6645e-15}$ & $\numprint{4.1620e-16}$ \\
 k & $\numprint{2.7334e-11}$ & $\numprint{4.2695e-12}$ & $\numprint{1.7764e-15}$ & $\numprint{2.7746e-16}$ & $\numprint{1.7764e-15}$ & $\numprint{2.7746e-16}$ \\
\hline
\end{tabular}
\caption{Table Int}
\label{Tab:Int}
\end{table}


\subsection{Exact Tests}
The solution to known testproblems with Dirichlet and no-flux boundary conditions are considered in this section. The errors are calculated as an $l_2$ error in space and an $l_\infty$ error in time.
\subsubsection{Exact Tests - Dirichlet Conditions}
(MS\_TestJonnaADExactDisectBoxes and MS\_TestJonnaADExactDisectBoxesPart2\\ and MS\_TestJonnaADExactDisectWedges)
Several examples are run, using exact solutions, to validate the multishape code. This is done using an exact solution to the advection diffusion equation on an infinite domain, so that Dirichlet boundary conditions can be applied, by matching the value of the exact solution on the boundary of the multishape.
The exact solution is \cite{Hutomo_2019}
\begin{align}\label{eq:DirichletExactSol}
	\rho &= \exp(\alpha t + \beta_1 y_1 + \beta_2 y_2)\\
	\mathbf v &= \left(\beta_1 - \frac{\alpha}{2 \beta_1} + p_1\exp(-\beta_1 y_1) , \beta_2 - \frac{\alpha}{2 \beta_2} + p_2\exp(-\beta_2 y_2) \right),\notag
\end{align}
where $\beta_1 = 0.1$, $\beta_2 = 0.1$, $\alpha = -0.5$, $p_1 = -1$ and $p_2 = 1$.
We compare the exact solution on a box of dimensions $[0,2] \times [0,2] $ with different discretizations of the box using multishape, signified by letters (a) to (g), see Figure \ref{F2}. Each of the shapes are discretized with $N = 10$, $N = 20$ and $N = 30$ points in each spatial direction, which means that the dissected box has more points in total than the original box. The ODE solver tolerances are $10^{-9}$. The solution can be seen in Figure \ref{F3}. The question is whether the results of the PDE on the box and the different discretizations of the box have a similar error when compared to the exact solution. The absolute and relative errors, $\mathcal E_{Abs}$ and $\mathcal E_{Rel}$, are measured in an $l_2$ norm in space and an $l_\infty$ norm in time and are displayed in Table \ref{Tab:DirichletEx1}. The errors for the non-rectangular quadrilateral, which is shown in Figure \ref{F3a}, are displayed on the row with the letter (q). 


	\begin{figure}[h]
		\centering
		\includegraphics[scale=0.35]{boxEx.png}
		\caption{Exact solution on the box.} 
		\label{F3}
	\end{figure}


The same test can be done for a wedge. Here, a single wedge and discretized versions are considered, denoted by (h) to (k), see Figure \ref{F4}. Table \ref{Tab:DirichletEx1} also shows the errors measured against the exact solution for different discretizations of the wedge, for $N = 10$, $N = 20$ and $N = 30$. The solution can be seen in Figure \ref{F5}.


\begin{figure}[h]
	\centering
	\includegraphics[scale=0.35]{wedgeEx.png}
	\caption{Exact solution on the wedge.} 
	\label{F5}
\end{figure}

Next the advection diffusion equation is solved on a multishape which is composed of four quadrilaterals, see Figure \ref{F6}.  The same is done for a second example involving a wedge, see Figure \ref{F7}.  The errors, as compared to the exact solution \eqref{eq:DirichletExactSol}, are displayed in Table \ref{Tab:DirichletEx1MS}. There, the first multishape is denoted ms1 and the second ms2.
\begin{figure}[h]
	\centering
	\includegraphics[scale=0.35]{example1.png}
	\caption{Example 1 multishape} 
	\label{F6}
\end{figure}


\begin{figure}[h]
	\centering
	\includegraphics[scale=0.35]{example2.png}
	\caption{Example 2 multishape} 
	\label{F7}
\end{figure}

\begin{table}
\centering
\begin{tabular}{ | c | c | c | c | c | c | c |}
\hline
 & \multicolumn{2}{c|}{$N = 10$}  & \multicolumn{2}{c|}{$N = 20$}  & \multicolumn{2}{c|}{$N = 30$} \\
\hline
 & $\mathcal E_{Abs}$ & $\mathcal E_{Rel}$ & $\mathcal E_{Abs}$ & $\mathcal E_{Rel}$ & $\mathcal E_{Abs}$  & $\mathcal E_{Rel}$ \\
\hline
 a & $\numprint{2.5869e-7}$ & $\numprint{1.0894e-9}$ & $\numprint{2.2063e-7}$ & $\numprint{1.1235e-9}$ & $\numprint{2.1913e-7}$ & $\numprint{1.1159e-9}$ \\
 b & $\numprint{3.4991e-7}$ & $\numprint{1.1308e-9}$ & $\numprint{3.2073e-7}$ & $\numprint{1.1377e-9}$ & $\numprint{3.1877e-7}$ & $\numprint{1.1307e-9}$ \\
 c & $\numprint{4.1386e-7}$ & $\numprint{1.1596e-9}$ & $\numprint{3.9107e-7}$ & $\numprint{1.1500e-9}$ & $\numprint{3.8858e-7}$ & $\numprint{1.1426e-9}$ \\
 d & $\numprint{4.2622e-7}$ & $\numprint{1.0268e-9}$ & $\numprint{3.9019e-7}$ & $\numprint{1.0026e-9}$ & $\numprint{3.8768e-7}$ & $\numprint{9.9616e-10}$ \\
 e & $\numprint{4.4595e-7}$ & $\numprint{1.0704e-9}$ & $\numprint{3.3852e-7}$ & $\numprint{9.8259e-10}$ & $\numprint{3.3718e-7}$ & $\numprint{1.0085e-9}$ \\
 f & $\numprint{4.1985e-7}$ & $\numprint{1.0275e-9}$ & $\numprint{4.2547e-7}$ & $\numprint{1.0412e-9}$ & $\numprint{4.2291e-7}$ & $\numprint{1.0350e-9}$ \\
 g & $\numprint{5.0161e-7}$ & $\numprint{1.1254e-9}$ & $\numprint{4.4470e-7}$ & $\numprint{1.1323e-9}$ & $\numprint{4.3978e-7}$ & $\numprint{1.1198e-9}$ \\
 q & $\numprint{2.4145e-7}$ & $\numprint{1.0389e-9}$ & $\numprint{2.2844e-7}$ & $\numprint{1.1160e-9}$ & $\numprint{2.2505e-7}$ & $\numprint{1.0995e-9}$ \\
 h & $\numprint{1.8507e-3}$ & $\numprint{4.4410e-6}$ & $\numprint{3.1141e-7}$ & $\numprint{7.4763e-10}$ & $\numprint{2.7613e-7}$ & $\numprint{7.5178e-10}$ \\
 i & $\numprint{3.4678e-6}$ & $\numprint{6.5183e-9}$ & $\numprint{3.9525e-7}$ & $\numprint{7.5983e-10}$ & $\numprint{3.8485e-7}$ & $\numprint{7.4376e-10}$ \\
 j & $\numprint{2.6220e-3}$ & $\numprint{4.4489e-6}$ & $\numprint{4.2335e-7}$ & $\numprint{7.4943e-10}$ & $\numprint{3.8922e-7}$ & $\numprint{7.5064e-10}$ \\
 k & $\numprint{2.6814e-6}$ & $\numprint{4.0877e-9}$ & $\numprint{4.4144e-7}$ & $\numprint{7.0896e-10}$ & $\numprint{4.3211e-7}$ & $\numprint{6.9682e-10}$ \\
\hline
\end{tabular}
\caption{Table Dirichlet Exact Solution 1}
\label{Tab:DirichletEx1}
\end{table}
\begin{table}
\centering
\begin{tabular}{ | c | c | c | c | c | c | c |}
\hline
 & \multicolumn{2}{c|}{$N = 10$}  & \multicolumn{2}{c|}{$N = 20$}  & \multicolumn{2}{c|}{$N = 30$} \\
\hline
 & $\mathcal E_{Abs}$ & $\mathcal E_{Rel}$ & $\mathcal E_{Abs}$ & $\mathcal E_{Rel}$ & $\mathcal E_{Abs}$  & $\mathcal E_{Rel}$ \\
\hline
 ms1 & $\numprint{5.9588e-7}$ & $\numprint{1.3030e-9}$ & $\numprint{6.1246e-7}$ & $\numprint{1.3259e-9}$ & $\numprint{6.0034e-7}$ & $\numprint{1.2997e-9}$ \\
 ms2 & $\numprint{1.6829e-3}$ & $\numprint{3.6591e-6}$ & $\numprint{3.7239e-7}$ & $\numprint{8.9375e-10}$ & $\numprint{3.7589e-7}$ & $\numprint{8.9532e-10}$ \\
\hline
\end{tabular}
\caption{Table Dirichlet Exact Solution 1 on two multishapes}
\label{Tab:DirichletEx1MS}
\end{table}

\subsubsection{Exact Tests - Dirichlet Conditions, Solution 2}

We solve an exact Dirichlet problem which does not have an exponential form but a quadratic form. In order to do this, we add on a source term $f$ to the advection-diffusion equation.
We define the exact solution as
\begin{align*}
	\rho &= t y_1^2 y_2^2,\\
	f &= y_1^2 y_2^2 - 2 y_1^2 t - 2 y_2^2 t + 2 y_1 y_2^2 t,
\end{align*}
and with a velocity field of strength one acting in the $y_1$ direction.
We run this on the box and the wedge discretizations (a) to (g) and the quadrilateral (q). The results are displayed in Table \ref{Tab:DirichletEx2}.



\begin{table}
\centering
\begin{tabular}{ | c | c | c | c | c | c | c |}
\hline
 & \multicolumn{2}{c|}{$N = 10$}  & \multicolumn{2}{c|}{$N = 20$}  & \multicolumn{2}{c|}{$N = 30$} \\
\hline
 & $\mathcal E_{Abs}$ & $\mathcal E_{Rel}$ & $\mathcal E_{Abs}$ & $\mathcal E_{Rel}$ & $\mathcal E_{Abs}$  & $\mathcal E_{Rel}$ \\
\hline
 a & $\numprint{2.2747e-13}$ & $\numprint{3.9799e-16}$ & $\numprint{3.9207e-13}$ & $\numprint{6.3023e-16}$ & $\numprint{4.7291e-13}$ & $\numprint{7.5978e-16}$ \\
 b & $\numprint{1.1317e-12}$ & $\numprint{1.1097e-15}$ & $\numprint{6.1443e-13}$ & $\numprint{5.9848e-16}$ & $\numprint{2.5646e-12}$ & $\numprint{2.5682e-15}$ \\
 c & $\numprint{7.0072e-13}$ & $\numprint{7.0033e-16}$ & $\numprint{1.4877e-12}$ & $\numprint{1.5185e-15}$ & $\numprint{2.0691e-12}$ & $\numprint{2.4102e-15}$ \\
 d & $\numprint{6.8357e-13}$ & $\numprint{5.1274e-16}$ & $\numprint{1.4899e-12}$ & $\numprint{2.9801e-15}$ & $\numprint{2.6521e-12}$ & $\numprint{2.0667e-15}$ \\
 e & $\numprint{2.2548e-12}$ & $\numprint{6.7534e-15}$ & $\numprint{5.5358e-13}$ & $\numprint{6.3464e-16}$ & $\numprint{8.8186e-13}$ & $\numprint{2.3179e-15}$ \\
 f & $\numprint{8.1603e-13}$ & $\numprint{1.5216e-15}$ & $\numprint{1.4019e-12}$ & $\numprint{8.7680e-16}$ & $\numprint{3.3732e-12}$ & $\numprint{1.8904e-15}$ \\
 g & $\numprint{4.9611e-13}$ & $\numprint{1.9913e-15}$ & $\numprint{5.0195e-12}$ & $\numprint{8.5065e-15}$ & $\numprint{1.4727e-12}$ & $\numprint{1.1481e-15}$ \\
 q & $\numprint{6.0563e-13}$ & $\numprint{1.7871e-15}$ & $\numprint{9.9468e-13}$ & $\numprint{4.3977e-15}$ & $\numprint{1.1245e-12}$ & $\numprint{3.2386e-15}$ \\
 h & $\numprint{9.2590e+00}$ & $\numprint{1.5820e-4}$ & $\numprint{3.6221e-6}$ & $\numprint{6.1886e-11}$ & $\numprint{5.9042e-11}$ & $\numprint{1.0457e-15}$ \\
 i & $\numprint{2.6170e-2}$ & $\numprint{3.1608e-7}$ & $\numprint{5.9610e-11}$ & $\numprint{7.4688e-16}$ & $\numprint{8.2340e-11}$ & $\numprint{1.0439e-15}$ \\
 j & $\numprint{1.3082e+01}$ & $\numprint{1.5808e-4}$ & $\numprint{5.1279e-6}$ & $\numprint{6.1963e-11}$ & $\numprint{2.1720e-10}$ & $\numprint{2.7449e-15}$ \\
 k & $\numprint{1.9298e-2}$ & $\numprint{2.0145e-7}$ & $\numprint{6.8410e-11}$ & $\numprint{7.4568e-16}$ & $\numprint{9.9529e-11}$ & $\numprint{1.0842e-15}$ \\
\hline
\end{tabular}
\caption{Table Dirichlet Exact Solution 2}
\label{Tab:DirichletEx2}
\end{table}

\subsubsection{Exact Tests - No-Flux Conditions}
We consider an exact solution on a box with no-flux boundary conditions for the advection diffusion equation. The solution is
\begin{align*}
	 \rho = 2 + \exp(-(\mu_1^2 + \mu_2^2)t)\cos(\mu_1 y_1)\cos(\mu_2 y_2),
\end{align*}
where $	\mu_1 = n\pi/L_1$, $\mu_2 = n\pi/L_2$ and $n = 2$, $L_1 = d - c$, $ L_2 = b - a$ for a domain $ [a,b]\times [c,d]$. We use the discretizations of the box displayed in Figure \ref{F2} with $N = 10$, $N= 20$ and $N = 30$. The exact solution for this problem can be seen in Figure \ref{F3b}. In Table \ref{Tab:NoFluxEx} we can see the errors made on the different discretizations. We can observe that the errors in multishapes (c), (d) and (f) are considerably larger than the errors for the other shapes. This is because of the complex discretization chosen in these multishapes. Due to this discretization, the standard averaging of the two normals at an intersection corner does not result in reasonable normals. This becomes an issue when computing no-flux boundary conditions, since this requires the normal information. The code library offers the option of overriding individual normals. We redefine the normals at the intersection corner to be the outward normals of the box. This improves the errors greatly as can be seen in Table \ref{Tab:NoFluxCorr}. The change in normals for these three multishapes is shown in Figure \ref{FNorm}.

\begin{figure}[h]
	\centering
	\includegraphics[scale=0.35]{boxExNoFlux.png}
	\caption{Exact solution on the box with no-flux boundary conditions.} 
	\label{F3b}
\end{figure}

\begin{figure}[h]
	\centering
	\includegraphics[scale=0.35]{NormalCorrected.png}
	\caption{Original, non-sensible normals are compared to new, customized normals.} 
	\label{FNorm}
\end{figure}

\begin{table}
\centering
\begin{tabular}{ | c | c | c | c | c | c | c |}
\hline
 & \multicolumn{2}{c|}{$N = 10$}  & \multicolumn{2}{c|}{$N = 20$}  & \multicolumn{2}{c|}{$N = 30$} \\
\hline
 & $\mathcal E_{Abs}$ & $\mathcal E_{Rel}$ & $\mathcal E_{Abs}$ & $\mathcal E_{Rel}$ & $\mathcal E_{Abs}$  & $\mathcal E_{Rel}$ \\
\hline
 a & $\numprint{1.0458e-2}$ & $\numprint{2.5230e-5}$ & $\numprint{2.8350e-7}$ & $\numprint{6.8623e-10}$ & $\numprint{2.6819e-7}$ & $\numprint{6.4918e-10}$ \\
 b & $\numprint{1.0178e-2}$ & $\numprint{1.7387e-5}$ & $\numprint{3.5488e-7}$ & $\numprint{6.0816e-10}$ & $\numprint{3.5538e-7}$ & $\numprint{6.0901e-10}$ \\
 c & $\numprint{7.1323e-2}$ & $\numprint{9.8994e-5}$ & $\numprint{7.0449e-3}$ & $\numprint{9.7842e-6}$ & $\numprint{2.6254e-3}$ & $\numprint{3.6463e-6}$ \\
 d & $\numprint{8.4906e-2}$ & $\numprint{1.0217e-4}$ & $\numprint{8.2570e-3}$ & $\numprint{9.9356e-6}$ & $\numprint{3.0338e-3}$ & $\numprint{3.6506e-6}$ \\
 e & $\numprint{5.6960e-3}$ & $\numprint{8.1264e-6}$ & $\numprint{4.3060e-7}$ & $\numprint{6.0669e-10}$ & $\numprint{4.3142e-7}$ & $\numprint{6.0784e-10}$ \\
 f & $\numprint{1.2934e-1}$ & $\numprint{1.5474e-4}$ & $\numprint{1.3465e-2}$ & $\numprint{1.6099e-5}$ & $\numprint{4.0715e-3}$ & $\numprint{4.8753e-6}$ \\
 g & $\numprint{8.7462e-6}$ & $\numprint{1.0753e-8}$ & $\numprint{5.1501e-7}$ & $\numprint{6.2332e-10}$ & $\numprint{5.1469e-7}$ & $\numprint{6.2294e-10}$ \\
\hline
\end{tabular}
\caption{Table No-Flux Exact Solution}
\label{Tab:NoFluxEx}
\end{table}
\begin{table}
\centering
\begin{tabular}{ | c | c | c | c | c | c | c |}
\hline
 & \multicolumn{2}{c|}{$N = 10$}  & \multicolumn{2}{c|}{$N = 20$}  & \multicolumn{2}{c|}{$N = 30$} \\
\hline
 & $\mathcal E_{Abs}$ & $\mathcal E_{Rel}$ & $\mathcal E_{Abs}$ & $\mathcal E_{Rel}$ & $\mathcal E_{Abs}$  & $\mathcal E_{Rel}$ \\
\hline
 c & $\numprint{1.8668e-2}$ & $\numprint{2.5911e-5}$ & $\numprint{4.5961e-7}$ & $\numprint{6.4023e-10}$ & $\numprint{4.5881e-7}$ & $\numprint{6.3911e-10}$ \\
 d & $\numprint{2.4277e-2}$ & $\numprint{2.9194e-5}$ & $\numprint{5.2874e-7}$ & $\numprint{6.3810e-10}$ & $\numprint{5.2815e-7}$ & $\numprint{6.3740e-10}$ \\
 f & $\numprint{2.1887e-3}$ & $\numprint{2.6634e-6}$ & $\numprint{5.3703e-7}$ & $\numprint{6.3964e-10}$ & $\numprint{5.3681e-7}$ & $\numprint{6.3938e-10}$ \\
\hline
\end{tabular}
\caption{Table No-Flux Exact Solution with Corrected Normal Vectors at Intersections}
\label{Tab:NoFluxCorr}
\end{table}
\subsection{Forward Problems on multishapes}
We first consider a forward problem on the different discretizations of the box, see Figure \ref{F2}. We compare the results of the discretized boxes with the result for box (a) with large $N$.
We choose the initial condition for $\rho$ to be
\begin{align*}
	\rho_0 = \exp(-2((y_1 - 0.7 )^2 + (y_2 - 0.2)^2)),
\end{align*}
and impose a constant flow of strength $0.8$ acting upward.
We choose $N = 10$, $N = 20$ and $N = 30$ as before. The errors are displayed in Table \ref{Tab:FWProbBox} and the result can be seen in Figure \ref{FFW1}. 
%
%\begin{table}
%	\caption{Errors (compared to whole box with $N = 50$) for different discretizations of the box}
%	\begin{tabular}{ ||c| c| c| c| c| c| c|| }
%		\hline
%		\hline
%		& b & c & d & e & f & g\\ 
%		\hline
%		Abs. Error, $N =20$& $0.0129$ & $0.0194$ & $0.0337$& $0.032$&  $ 0.0165$ & $0.0099$ \\  
%		Rel. Error, $N =20$& $0.001$& $0.0025$ & $0.0024$ & $0.0025$& $0.0019$  & $0.0007$ \\
%		Abs. Error, $N =30$& $0.0054$ & $0.0078$ & $0.0137$ &$0.0134$&$0.0064$& $0.0026$\\  
%		Rel. Error, $N =30$ & $0.0003$& $0.0007$ &$0.0006$ &$0.0007$&$0.0005$& $0.0001$\\
%		\hline
%		\hline
%	\end{tabular}
%	\label{Tab3:ErrorsFWBox}
%\end{table}
\begin{figure}[h]
	\centering
	\includegraphics[scale=0.35]{FWBox.png}
	\caption{Forward Example on box discretizations} 
	\label{FFW1}
\end{figure}


We follow the same idea, but now consider the wedge discretizations, see Figure \ref{F4}. We choose the initial condition for $\rho$ to be
\begin{align*}
	\rho_0 = \exp(-2((y_1 - 1.5 )^2 + (y_2 - 4.5)^2)),
\end{align*}
and impose a constant flow of strength $3$ acting from left to right, along the angular direction.
We choose $N = 20$ and $N = 30$. The errors are displayed in Table \ref{Tab:FWProbWedge} and the result can be seen in Figure \ref{FFW2}. 

%\begin{table}
%	\caption{Errors (compared to whole wedge (h) ) for different discretization (into two and three shapes) of the wedge}
%	\begin{tabular}{ ||c| c| c| c|| }
%		\hline
%		\hline
%		& i & j &k\\ 
%		\hline
%		Abs. Error, $N =20$& $0.0233 $ & $0.0750 $ & $0.0189 $ \\  
%		Rel. Error, $N =20$& $0.0026 $& $0.0068 $ &$0.0013 $ \\
%		Abs. Error, $N =30$& $0.0071 $ & $0.0316 $ & $0.0113 $  \\  
%		Rel. Error, $N =30$ & $0.0005 $& $0.0019 $ &$ 0.0005$  \\
%		\hline
%		\hline
%	\end{tabular}
%	\label{Tab4:ErrorsFWWedge}
%\end{table}
\begin{figure}[h]
	\centering
	\includegraphics[scale=0.35]{FWWedge.png}
	\caption{Forward Example on wedge discretizations} 
	\label{FFW2}
\end{figure}



Finally, two more complex multishape examples are considered.
The first of these examples is solving an advection diffusion problem on a multishape consisting of two quadrilaterals and two wedges, with constant velocity of strength ten. The initial condition for this problem is:
 \begin{align*}
 	\rho_0 = \exp( -2(y_1 -0.5)^2 - 2 (y_2 + 1)^2).
 \end{align*}
The result, evaluated for $N= 20$ on each shape, can be seen in Figure \ref{F8}.

\begin{figure}[h]
	\centering
	\includegraphics[scale=0.35]{ex1.png}
	\caption{Forward Problem 1, different colour scale for each plot to highlight particle mass location}
	\label{F8}
\end{figure}

In a second example, the velocity is of strength $5$ and the initial condition is:
 \begin{align*}
	\rho_0 = \exp( -2(y_1 -0.5)^2 - 2 (y_2 - 1.5)^2).
\end{align*}
The result, which is computed on a multishape made up of four quadrilaterals into a channel, can be seen in Figure \ref{F9}.

\begin{figure}[h]
	\centering
	\includegraphics[scale=0.35]{ex2.png}
	\caption{Forward Problem 2, different colour scale for each plot to highlight particle mass location}
	\label{F9}
\end{figure}

In Table \ref{Tab:Example12} the solution to these two examples is evaluated with $N = 30$ and compared to the solution evaluated with $N = 10$, $N = 15$ and $N = 20$, using the same error measure as before. We can see that the solution converges with the number of points. It is especially noticeable for Example 1, that $N = 10$ does not give a reliable solution at all, while $N = 15$ provides a solution that is reasonably close to the one with $N = 30$. The reason that Example 1 is much more sensitive to the number of points than Example 2 could be that it contains polar shapes. As discussed in Section \ref{sec:ValidationDiffIntetc}, for accurate interpolation of the wedge, the angular direction needs more points than the radial direction, and in particular was not very accurate with $N = 10$ in the angular direction.




\begin{table}
\centering
\begin{tabular}{ | c | c | c | c | c | c | c |}
\hline
 & \multicolumn{2}{c|}{$N = 10$}  & \multicolumn{2}{c|}{$N = 20$}  & \multicolumn{2}{c|}{$N = 30$} \\
\hline
 & $\mathcal E_{Abs}$ & $\mathcal E_{Rel}$ & $\mathcal E_{Abs}$ & $\mathcal E_{Rel}$ & $\mathcal E_{Abs}$  & $\mathcal E_{Rel}$ \\
\hline
 b & $\numprint{6.8019e-2}$ & $\numprint{1.7873e-2}$ & $\numprint{1.9441e-2}$ & $\numprint{2.4954e-3}$ & $\numprint{7.8355e-3}$ & $\numprint{6.6545e-4}$ \\
 c & $\numprint{4.5117e-2}$ & $\numprint{6.9628e-3}$ & $\numprint{1.2944e-2}$ & $\numprint{9.9990e-4}$ & $\numprint{5.3658e-3}$ & $\numprint{2.7644e-4}$ \\
 d & $\numprint{1.1864e-1}$ & $\numprint{1.7013e-2}$ & $\numprint{3.3678e-2}$ & $\numprint{2.3638e-3}$ & $\numprint{1.3708e-2}$ & $\numprint{6.3705e-4}$ \\
 e & $\numprint{1.2516e-1}$ & $\numprint{1.9815e-2}$ & $\numprint{3.1984e-2}$ & $\numprint{2.5133e-3}$ & $\numprint{1.3367e-2}$ & $\numprint{6.9858e-4}$ \\
 f & $\numprint{5.8413e-2}$ & $\numprint{1.4025e-2}$ & $\numprint{1.6467e-2}$ & $\numprint{1.9292e-3}$ & $\numprint{6.3541e-3}$ & $\numprint{4.9237e-4}$ \\
 g & $\numprint{4.0532e-2}$ & $\numprint{5.7827e-3}$ & $\numprint{9.9059e-3}$ & $\numprint{7.0226e-4}$ & $\numprint{2.5623e-3}$ & $\numprint{1.2085e-4}$ \\
\hline
\end{tabular}
\caption{Table Forward Problem on Box}
\label{Tab:FWProbBox}
\end{table}
\begin{table}
\centering
\begin{tabular}{ | c | c | c | c | c | c | c |}
\hline
 & \multicolumn{2}{c|}{$N = 10$}  & \multicolumn{2}{c|}{$N = 20$}  & \multicolumn{2}{c|}{$N = 30$} \\
\hline
 & $\mathcal E_{Abs}$ & $\mathcal E_{Rel}$ & $\mathcal E_{Abs}$ & $\mathcal E_{Rel}$ & $\mathcal E_{Abs}$  & $\mathcal E_{Rel}$ \\
\hline
 i & $\numprint{8.1092e-2}$ & $\numprint{1.8792e-2}$ & $\numprint{2.3281e-2}$ & $\numprint{2.6395e-3}$ & $\numprint{7.0628e-3}$ & $\numprint{5.3006e-4}$ \\
 j & $\numprint{2.8569e-1}$ & $\numprint{5.2137e-2}$ & $\numprint{7.5030e-2}$ & $\numprint{6.8191e-3}$ & $\numprint{3.1628e-2}$ & $\numprint{1.9130e-3}$ \\
 k & $\numprint{7.0422e-2}$ & $\numprint{9.7166e-3}$ & $\numprint{1.8884e-2}$ & $\numprint{1.2887e-3}$ & $\numprint{1.1284e-2}$ & $\numprint{5.1154e-4}$ \\
\hline
\end{tabular}
\caption{Table Forward Problem on Wedge}
\label{Tab:FWProbWedge}
\end{table}
\begin{table}
\centering
\begin{tabular}{ | c | c | c | c | c | c | c |}
\hline
 & \multicolumn{2}{c|}{$N = 10$}  & \multicolumn{2}{c|}{$N = 15$}  & \multicolumn{2}{c|}{$N = 20$} \\
\hline
 & $\mathcal E_{Abs}$ & $\mathcal E_{Rel}$ & $\mathcal E_{Abs}$ & $\mathcal E_{Rel}$ & $\mathcal E_{Abs}$  & $\mathcal E_{Rel}$ \\
\hline
 Ex1 & $\numprint{1.7380e+121}$ & $\numprint{1.3801e+121}$ & $\numprint{2.9942e-1}$ & $\numprint{5.0366e-2}$ & $\numprint{1.5166e-1}$ & $\numprint{1.7141e-2}$ \\
 Ex2 & $\numprint{3.1819e-2}$ & $\numprint{8.9836e-3}$ & $\numprint{1.4655e-2}$ & $\numprint{2.2034e-3}$ & $\numprint{9.7279e-3}$ & $\numprint{1.0914e-3}$ \\
\hline
\end{tabular}
\caption{Table Forward Examples 1 and 2}
\label{Tab:Example12}
\end{table}
