\documentclass[11pt, a4paper]{article}
%\usepackage{proj1}
\usepackage{natbib}
\usepackage{fancyhdr}  
\usepackage{subcaption}
\usepackage{caption}
\usepackage{graphicx}
\usepackage{numprint}
\usepackage{multirow}
\linespread{1.25} 
\setlength{\parindent}{0cm}
\graphicspath{{Images/}}
\usepackage{hyperref}
\usepackage{amsmath}
\usepackage{amsfonts}
\usepackage{amssymb}
\usepackage{amsthm}
\usepackage{mathtools}
\usepackage{commath}
\usepackage{bbm}

%\usepackage[sc,osf]{mathpazo}
\usepackage{subcaption}
\usepackage[a4paper, top=1in, left=1.0in, right=1.0in, bottom=1in, includehead, includefoot]{geometry} %Usually have top as 1in

\usepackage{listings}
\usepackage{color} %red, green, blue, yellow, cyan, magenta, black, white
\definecolor{mygreen}{RGB}{28,172,0} % color values Red, Green, Blue
\definecolor{mylilas}{RGB}{170,55,241}


\hypersetup{colorlinks,linkcolor={black},citecolor={blue},urlcolor={black}}
\usepackage{color}
\urlstyle{same}


\theoremstyle{definition}
\newtheorem{definition}{Definition}[section]

%\newcommand{\Sta}{\rho}
\newcommand{\adja}{q_a}
\newcommand{\adjb}{q_b}
\newcommand{\adjaB}{q_{a,\partial \Omega}}
\newcommand{\adjbB}{q_{b,\partial \Omega}}
%\newcommand{\Con}{u}
\newcommand{\ra}{\rho_a}
\newcommand{\rb}{\rho_b}
\newcommand{\w}{\mathbf{w}}
\newcommand{\Stav}{\mathbf{v}}
\newcommand{\Adja}{\mathbf{p}}
\newcommand{\Adjb}{q}
\newcommand{\Adjc}{{p}_{\partial \Sigma}}
\newcommand{\Con}{\mathbf{f}}
\newcommand{\n}{\mathbf{n}}
\newcommand{\h}{\mathbf{h}}
\newcommand{\K}{\mathbf{K}}


\pagenumbering{gobble}
\begin{document}
	\section*{Deriving Optimality Conditions that slightly differ from the usual ones}
	We consider time independent flow control, control through the external potential and a target that is only considered at the final time rather than over the whole time horizon.
	\section{Time independent flow control}
	We have the following OCP:
	\begin{align*}
		&J = \frac{1}{2}\int_0^T \int_\Omega (\rho - \widehat \rho)^2 dr dt + \frac{\beta}{2} \int_\Omega \w(r)^2 dr\\
		&\text{subject to:}\\
		&\frac{\partial \rho}{\partial t} = \nabla^2 \rho - \nabla \cdot (\rho \w(r))
	\end{align*}
	Then the Lagrangian is:
	\begin{align*}
		\mathcal{L}(\rho,\w, q) &=  \frac{1}{2}\int_0^T \int_\Omega (\rho - \widehat \rho)^2 dr dt + \frac{\beta}{2} \int_\Omega \w(r)^2 dr \\
		&- \int_0^T \int_\Omega q\frac{\partial \rho}{\partial t} - q\nabla^2 \rho + q\nabla \cdot (\rho \w) dr dt.
	\end{align*}
	And after integrating by parts (neglecting the BCs because they are unchanged from before):
	\begin{align*}
		\mathcal{L}(\rho,\w, q) &=  \frac{1}{2}\int_0^T \int_\Omega (\rho - \widehat \rho)^2 dr dt + \frac{\beta}{2} \int_\Omega \w(r)^2 dr \\
		&- \int_0^T \int_\Omega q\frac{\partial \rho}{\partial t} - q\nabla^2 \rho -  \rho \w \cdot \nabla q dr dt.
	\end{align*}
	Taking derivatives with respect to $\w$ gives:
	\begin{align*}
		\mathcal{L}_\w(\rho,\w, q)h &= \int_\Omega \beta \w(r) \cdot \h(r) dt + \int_0^T \int_\Omega \rho \h(r) \cdot \nabla q dr dt.
	\end{align*}
	Since $\w$ does not depend on $t$, neither does $\h$ and so this can be taken out of the time integral:
	\begin{align*}
		\mathcal{L}_\w(\rho,\w, q)h &= \int_\Omega \bigg( \beta \w(r) \cdot \h(r)  + \h(r) \cdot \int_0^T \rho  \nabla q dt \bigg) dr.
	\end{align*}
	Then we get:
	\begin{align*}
		\beta \w(r)  +  \int_0^T \rho  \nabla q dt = 0
	\end{align*}
	And finally:
	\begin{align*}
		\w(r) = - \frac{1}{\beta} \int_0^T \rho  \nabla q dt
	\end{align*}
	
	\section{$V_{ext}$ control}
	We have the following OCP:
	\begin{align*}
		&J = \frac{1}{2}\int_0^T \int_\Omega (\rho - \widehat \rho)^2 dr dt + \frac{\beta}{2} \int_0^T\int_\Omega V_{ext}^2 dr\\
		&\text{subject to:}\\
		&\frac{\partial \rho}{\partial t} = \nabla^2 \rho + \nabla \cdot (\rho \nabla V_{ext})
	\end{align*}
	The Lagrangian is:
	\begin{align*}
		\mathcal{L}(\rho, V_{ext}, q) &= \frac{1}{2}\int_0^T \int_\Omega (\rho - \widehat \rho)^2 dr dt + \frac{\beta}{2} \int_0^T\int_\Omega V_{ext}^2 dr\\
		&- \int_0^T \int_\Omega q\frac{\partial \rho}{\partial t} - q\nabla^2 \rho - q\nabla \cdot (\rho \nabla V_{ext}) dr dt.
	\end{align*}
	We need to integrate by parts twice to get the term in $V_{ext}$ into the necessary form:
	\begin{align*}
		\int_0^T \int_\Omega q \nabla \cdot (\rho \nabla V_{ext}) dr dt &= \int_0^T \int_{\partial \Omega} q \rho \nabla V_{ext} \cdot \n dr dt - \int_0^T \int_\Omega \rho \nabla V_{ext} \cdot \nabla q dr dt \\
		& = \int_0^T \int_{\partial \Omega} q \rho \nabla V_{ext} \cdot \n  - \rho V_{ext} \nabla q \cdot \n dr dt + \int_0^T \int_\Omega V_{ext} \nabla \cdot (\rho \nabla q) dr dt
	\end{align*}
	We will also have 
	\begin{align*}
		\int_0^T \int_\Omega q\nabla^2 \rho = \int_0^T \int_{\partial \Omega} q \nabla \rho \cdot \n- \rho \nabla q \cdot \n dr dt + \int_0^T \int_\Omega \rho \nabla^2 q dr dt
	\end{align*}
	\subsection{Boundary Conditions}
	And the boundary conditions:
	\begin{align*}
		\int_0^T \int_{\partial \Omega} q_{\partial \Omega}\nabla \rho \cdot \n + q_{\partial \Omega}\rho \nabla V_{ext} \cdot \n dr dt
	\end{align*}
	Combining these:
	\begin{align*}
		&\int_0^T \int_{\partial \Omega} q \rho \nabla V_{ext} \cdot \n  - \rho V_{ext} \nabla q \cdot \n + q \nabla \rho \cdot \n - \rho \nabla q \cdot \n + q_{\partial \Omega}\nabla \rho \cdot \n + q_{\partial \Omega}\rho \nabla V_{ext} \cdot \n dr dt
	\end{align*}
	During the derivation of the adjoint equation we have:

	\begin{align*}
		\int_0^T \int_{\partial \Omega} \n \cdot h \bigg(q  \nabla V_{ext} -  V_{ext} \nabla q -\nabla q + q_{\partial \Omega} \nabla V_{ext} \bigg) +
		\nabla h \cdot \n \bigg(q  + q_{\partial \Omega}  \bigg)dr dt
	\end{align*}
	Then from the $\nabla h$ terms we get $q_{\partial \Omega} = - q$ and so:
	\begin{align*}
		(q  \nabla V_{ext} -  V_{ext} \nabla q -\nabla q - q \nabla V_{ext}  ) \cdot \n = 0
	\end{align*}
	And therefore the new adjoint boundary condition is:
	\begin{align*}
		(1 + V_{ext})\frac{\partial q}{\partial n} = 0.
	\end{align*}

	\subsection{Gradient Equation}
	We take the derivative of the Lagrangian with respect to $V_{ext}$:
	\begin{align*}
		\mathcal{L}_{V_{ext}} (\rho, V_{ext},q)h &= \int_0^T \int_\Omega \beta V_{ext} h + \nabla \cdot (\rho \nabla q) h dr dt 
	\end{align*}
	Then we have:
	\begin{align*}
		\beta V_{ext}  + \nabla \cdot (\rho \nabla q)  = 0
	\end{align*}
	And finally 
	\begin{align*}
		V_{ext} = - \frac{1}{\beta} \nabla \cdot (\rho \nabla q).
	\end{align*}
	
	
	\section{Target at final time}
	We consider the following OCP, where $\widehat \rho$ does not depend on time; we are only interested in the target at the final time $T$. The problem reads:
	\begin{align*}
		&J = \frac{1}{2} \int_\Omega (\rho(T) - \widehat \rho)^2 dr  + \frac{\beta}{2} \int_0^T\int_\Omega \w^2 dr\\
		&\text{subject to:}\\
		&\frac{\partial \rho}{\partial t} = \nabla^2 \rho - \nabla \cdot (\rho \w)
	\end{align*}
	Then the Lagrangian is:
	\begin{align*}
		\mathcal{L}(\rho,\w, q) &= \frac{1}{2} \int_\Omega (\rho(T) - \widehat \rho)^2 dr  + \frac{\beta}{2} \int_0^T\int_\Omega \w^2 dr\\
		&- \int_0^T \int_\Omega q\frac{\partial \rho}{\partial t} - q\nabla^2 \rho + q\nabla \cdot (\rho \w) dr dt.
	\end{align*}
	From integrating by parts we get:
	\begin{align*}
	\int_0^T \int_\Omega	-q\frac{\partial \rho}{\partial t} dr dt= - \int_\Omega q(T) \rho(T) - q(0) \rho(0) dr +  \int_0^T \int_\Omega	\rho \frac{\partial q}{\partial t} dr dt
	\end{align*}
		\begin{align*}
		\mathcal{L}(\rho,\w, q) &= \frac{1}{2} \int_\Omega (\rho(T) - \widehat \rho)^2 dr  + \frac{\beta}{2} \int_0^T\int_\Omega \w^2 dr\\
		&- \int_\Omega q(T) \rho(T) - q(0) \rho(0) dr + \int_0^T \int_\Omega \rho \frac{\partial q}{\partial t}   + q\nabla^2 \rho - q\nabla \cdot (\rho \w) dr dt.
	\end{align*}
	Taking the derivative with respect to $\rho $ gives:
	\begin{align*}
		\mathcal{L}_\rho(\rho,\w, q)h &= \int_\Omega  (\rho(T) - \widehat \rho) h(T) dr - \int_\Omega q(T) h(T) dr\\
		&+ \int_0^T \int_\Omega h \frac{\partial q}{\partial t}   + q\nabla^2 h - q\nabla \cdot (h \w) dr dt.
	\end{align*}
	Considering the terms for $h(T)$, we get:
	\begin{align*}
		(\rho(T) - \widehat \rho) - q(T) = 0,
	\end{align*}
	and so 
	\begin{align*}
	q(T) = \rho(T) - \widehat \rho.
	\end{align*}
	The adjoint PDE remains unchanged, except for the fact that $ \rho - \widehat \rho$ does not enter the PDE anymore.




\end{document}