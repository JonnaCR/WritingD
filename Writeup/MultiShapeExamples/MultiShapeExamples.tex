\documentclass[11pt, a4paper]{article}
%\usepackage{proj1}
\usepackage{natbib}
\usepackage{fancyhdr}  
\usepackage{subcaption}
\usepackage{caption}
\usepackage{graphicx}
\usepackage{numprint}
\usepackage{multirow}
\linespread{1.25} 
\setlength{\parindent}{0cm}
\graphicspath{{Images/}}
\usepackage{hyperref}
\usepackage{amsmath}
\usepackage{amsfonts}
\usepackage{amssymb}
\usepackage{amsthm}
\usepackage{mathtools}
\usepackage{commath}
\usepackage{bbm}

%\usepackage[sc,osf]{mathpazo}
\usepackage{subcaption}
\usepackage[a4paper, top=1in, left=1.0in, right=1.0in, bottom=1in, includehead, includefoot]{geometry} %Usually have top as 1in

\usepackage{listings}
\usepackage{color} %red, green, blue, yellow, cyan, magenta, black, white
\definecolor{mygreen}{RGB}{28,172,0} % color values Red, Green, Blue
\definecolor{mylilas}{RGB}{170,55,241}


\hypersetup{colorlinks,linkcolor={black},citecolor={blue},urlcolor={black}}
\usepackage{color}
\urlstyle{same}


\theoremstyle{definition}
\newtheorem{definition}{Definition}[section]

%\newcommand{\Sta}{\rho}
\newcommand{\adja}{q_a}
\newcommand{\adjb}{q_b}
\newcommand{\adjaB}{q_{a,\partial \Omega}}
\newcommand{\adjbB}{q_{b,\partial \Omega}}
%\newcommand{\Con}{u}
\newcommand{\ra}{\rho_a}
\newcommand{\rb}{\rho_b}
\newcommand{\w}{\mathbf{w}}
\newcommand{\Stav}{\mathbf{v}}
\newcommand{\Adja}{\mathbf{p}}
\newcommand{\Adjb}{q}
\newcommand{\Adjc}{{p}_{\partial \Sigma}}
\newcommand{\Con}{\mathbf{f}}
\newcommand{\n}{\mathbf{n}}
\newcommand{\h}{\mathbf{h}}
\newcommand{\K}{\mathbf{K}}


\pagenumbering{gobble}
\begin{document}
We consider the following optimal control problem:
\begin{align*}
&\mathcal{J}(\rho, \w) = \frac{1}{2}|| \rho - \widehat \rho||^2_{L_2(\Sigma)} + \frac{\beta}{2}|| \w ||^2_{L_2(\Sigma)}\\
&\text{subject to:}\\
&\frac{\partial \rho}{\partial t} = \nabla^2 \rho - \nabla (\rho \w) + \kappa \nabla \int_\Omega \rho(r) \rho(r') \K(r,r') dr'.
\end{align*}

\section{Example 1}
The initial configuration for this example is:
\begin{align*}
&\rho_0 = exp(-2((y_1 - 0.5 )^2 + (y_2 + 0.5)^2))\\
&\w = \mathbf 0.
\end{align*}
The target $\widehat \rho$ is set by running a forward problem with the same $\rho_0$ but with constant velocity $\w = \mathbf{1}$.
The domain for this example is a quadrilateral and a wedge and can be seen in Figure \ref{Dom1}. The tolerances are $10^{-3}/ 10^{-7}$. Per shape 

\begin{figure}[h]
	\centering
	\includegraphics[scale=0.6]{Dom1.png}
	\caption{Domain Example 1} 
	\label{Dom1}
\end{figure}
Choosing $\kappa = -1$, we get $J_{FW} = 0.0251$, $J_{Opt} = 0.0020$. In Figure \ref{FEx1a} $\widehat \rho$ and corresponding $\w$ are plotted, while in Figure \ref{FEx1b}, the optimal result is displayed. 
\begin{figure}[h]
	\centering
	\includegraphics[scale=0.6]{FW1n1.png}
	\caption{Example 1, $\widehat \rho$, $\kappa = -1$} 
	\label{FEx1a}
\end{figure}
\begin{figure}[h]
	\centering
	\includegraphics[scale=0.6]{Opt1n1.png}
	\caption{Example 1, optimal $\rho$ and $\w$, $\kappa = -1$} 
	\label{FEx1b}
\end{figure}
Choosing $\kappa = 1$, we get $J_{FW} = 0.0176$, $J_{Opt} = 0.0020$. In Figure \ref{FEx1c} $\widehat \rho$ and corresponding $\w$ are plotted, while in Figure \ref{FEx1d}, the optimal result is displayed. It can be seen that while the attractive particles clump in the middle, the repulsive particles are clustered at the boundaries.
\begin{figure}[h]
	\centering
	\includegraphics[scale=0.6]{FW11.png}
	\caption{Example 1, $\widehat \rho$, $\kappa = 1$} 
	\label{FEx1c}
\end{figure}
\begin{figure}[h]
	\centering
	\includegraphics[scale=0.6]{Opt11.png}
	\caption{Example 1, optimal $\rho$, $\w$, $\kappa = 1$} 
	\label{FEx1d}
\end{figure}
(Note to self: this corresponds to Test10/ Example5a in code.)
\section{Example 2}
As initial configuration we choose:
\begin{align*}
&\rho_0 = exp(-2((y_1 - 0.5 )^2 + (y_2 + 0.5)^2))\\
&\w = \mathbf{0}.
\end{align*}	
The target is a forward problem run with the same $\rho_0$ but with a constant background flow $\w = \mathbf{5}$. We run the problem up to time $2$, as opposed to time $1$ as usual. This is more stable than increasing the strength of the background flow. The domain of the problem is a channel, see Figure \ref{Dom2}.
\begin{figure}[h]
	\centering
	\includegraphics[scale=0.6]{Dom2.png}
	\caption{Domain Example 2} 
	\label{Dom2}
\end{figure}


Choosing $\kappa = -1$, we get $J_{FW} =  0.4111$, $J_{Opt} =  0.0807$. In Figure \ref{FEx2a} $\widehat \rho$ and corresponding $\w$ are plotted, while in Figure \ref{FEx2b}, the optimal result is displayed. 
\begin{figure}[h]
	\centering
	\includegraphics[scale=0.3]{FW2n1.png}
	\caption{Example 2, $\widehat \rho$, $\kappa = -1$} 
	\label{FEx2a}
\end{figure}
\begin{figure}[h]
	\centering
	\includegraphics[scale=0.3]{Opt2n1.png}
	\caption{Example 2, optimal $\rho$ and $\w$, $\kappa = -1$} 
	\label{FEx2b}
\end{figure}
Choosing $\kappa = 1$, we get $J_{FW} =  0.3501$, $J_{Opt} =  0.0821$. In Figure \ref{FEx2c} $\widehat \rho$ and corresponding $\w$ are plotted, while in Figure \ref{FEx2d}, the optimal result is displayed. Again, the difference between attractive and repulsive interaction is clearly displayed by the clustering of the particles in the channel.
\begin{figure}[h]
	\centering
	\includegraphics[scale=0.3]{FW21.png}
	\caption{Example 2, $\widehat \rho$, $\kappa = 1$} 
	\label{FEx2c}
\end{figure}
\begin{figure}[h]
	\centering
	\includegraphics[scale=0.3]{Opt21.png}
	\caption{Example 2, optimal $\rho$ and $\w$, $\kappa = 1$} 
	\label{FEx2d}
\end{figure}







	
\section{Example 3}
As initial configuration we choose:
\begin{align*}
&\rho_0 = exp(-2((y_1 + 1)^2 + (y_2 + 0.3)^2))\\
&\w = \mathbf{0}.
\end{align*}	
The target is a forward problem run with the same $\rho_0$ but with a background flow of constant one, shaped along the domain. This and the domain of the problem can be seen in Figure \ref{Dom3}.
\begin{figure}[h]
	\centering
	\includegraphics[scale=0.4]{Dom3.png}
	\includegraphics[scale=0.4]{Flow3.png}
	\caption{Domain and Flow Example 3} 
	\label{Dom3}
\end{figure}	
	
\end{document}