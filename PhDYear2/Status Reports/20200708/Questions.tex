
\documentclass[11pt, a4paper]{article}
%\usepackage{proj1}
\usepackage{natbib}
\usepackage{fancyhdr}  
\usepackage{subcaption}
\usepackage{caption}
\usepackage{graphicx}
\linespread{1.25} 
\setlength{\parindent}{0cm}
\graphicspath{{Images/}}
\usepackage{hyperref}
\usepackage{amsmath}
\usepackage{amsfonts}
\usepackage{amssymb}
\usepackage{amsthm}
\usepackage{mathtools}
\usepackage{commath}
\usepackage[warning,debug,nosepfour,autolanguage]{numprint}

%\usepackage[sc,osf]{mathpazo}
\usepackage{subcaption}
\usepackage[a4paper, top=1in, left=1.0in, right=1.0in, bottom=1in, includehead, includefoot]{geometry} %Usually have top as 1in

\usepackage{listings}
\usepackage{color} %red, green, blue, yellow, cyan, magenta, black, white
\definecolor{mygreen}{RGB}{28,172,0} % color values Red, Green, Blue
\definecolor{mylilas}{RGB}{170,55,241}


\hypersetup{colorlinks,linkcolor={black},citecolor={blue},urlcolor={black}}
\usepackage{color}
\urlstyle{same}


\theoremstyle{definition}
\newtheorem{definition}{Definition}[section]

\title{Exact Solutions for the Full Problem \\with Force Control and with Flow Control}
\date{}
%\newcommand{\Sta}{\rho}
\newcommand{\Adj}{p}
\newcommand{\adj}{q}
%\newcommand{\Con}{u}
\newcommand{\Sta}{\rho}
\newcommand{\Stav}{\mathbf{v}}
\newcommand{\Adja}{\mathbf{p}_\Sigma}
\newcommand{\Adjb}{q}
\newcommand{\Adjc}{p_{\partial \Sigma}}
\newcommand{\Con}{\mathbf{f}}
\newcommand{\nor}{\mathbf{n}}

\pagenumbering{gobble}
\begin{document}
Following the meeting on 09/07/2020	
\section{Question 1 (John):}
Considering the OCP discussed yesterday, where we add a term into the cost functional (where $\Con$ is an imposed, fixed flow, $\mathbf{w}$ is the control flow):
	\begin{align}
	&\min_{\Sta,\Stav,\mathbf{w}} \quad \frac{1}{2}|| \Sta - \hat \Sta ||_{L_2(\Sigma)}^2 + \frac{\eta}{2}||\nabla \Stav||_{L_2(\Sigma)}^2 +\frac{\beta}{2}||\mathbf{w}||_{L_2(\Sigma)}^2\\
	&\text{subject to:} \notag\\
	& m \Sta \frac{\partial \Stav}{\partial t} = \bigg(- m \Sta (\Stav \cdot \nabla)\Stav - \Sta \nabla V^{ext} - \Sta \Con - \Sta \mathbf{w} - \nabla \Sta - m \gamma \Sta \Stav\bigg) \ \ \ \qquad \ \ \quad\text{in} \quad \Sigma \notag\\
	&\frac{\partial \Sta}{\partial t} = - \nabla \cdot (\Sta \Stav) \notag\\
	 \notag\\
	&\Sta \Stav \cdot \mathbf{n} =0	\qquad\text{on } \partial \Omega \notag
	\end{align}
	Considering the relevant part of the Lagrangian:
	\begin{align*}
	\mathcal{L}(\Sta,\Stav, \Adjb, \Adja, \mathbf{w}) = ... + \frac{\eta}{2}\int_0^T \int_{\Omega} \bigg( \nabla \Stav \bigg)^2 dr dt + ...
	\end{align*}
	In 1D this is:
	\begin{align*}
	\mathcal{L}(\Sta,v, \Adjb, p, w) = ... + \frac{\eta}{2}\int_0^T \int_{\Omega} \bigg(\frac{\partial v}{\partial x} \bigg)^2 dx dt + ... = ... + \int_0^T \int_{\Omega} \bigg(\frac{\partial v}{\partial x} \bigg)\bigg(\frac{\partial v}{\partial x} \bigg) dx dt + ...
	\end{align*}
	Then, taking the derivative and using the product rule:
	\begin{align*}
	\mathcal{L}_v(\Sta,v, \Adjb, p, w)h = ... +\frac{\eta}{2} \int_0^T \int_{\Omega} \bigg(\frac{\partial h}{\partial x} \bigg)\bigg(\frac{\partial v}{\partial x} \bigg) + \bigg(\frac{\partial v}{\partial x} \bigg)\bigg(\frac{\partial h}{\partial x} \bigg) dx dt + ...
	\end{align*}
	Integrating by parts gives:
	\begin{align*}
	\mathcal{L}_v(\Sta,v, \Adjb, p, w)h &= ... +\frac{\eta}{2} \bigg[ 
	 \int_0^T \int_{\partial\Omega}  h\bigg(\frac{\partial v}{\partial n} \bigg) + \bigg(\frac{\partial v}{\partial n} \bigg)h dx dt	 
	- \int_0^T \int_{\Omega} h\bigg(\frac{\partial^2 v}{\partial x^2} \bigg) + \bigg(\frac{\partial^2 v}{\partial x^2} \bigg)h dx dt  \bigg]+ ...\\
	&= ... +\eta \bigg[ 
	\int_0^T \int_{\partial\Omega} \bigg(\frac{\partial v}{\partial n} \bigg)h dx dt	 
	- \int_0^T \int_{\Omega} \bigg(\frac{\partial^2 v}{\partial x^2} \bigg)h dx dt  \bigg]+ ...
	\end{align*}
	So going through the derivation, we would get a boundary term $\frac{\partial v}{\partial x}$ and a term $\frac{\partial^2 v}{\partial x^2} $ in the second adjoint equation.
	In particular, from the boundary terms we get that the relationship between the adjoints $p$, $\Adjc$ and $q$ is:
	\begin{align*}
	& m v p +  \Adjc + \Adjb + \eta\frac{\partial v}{\partial n} =0
	\end{align*}
	And from the interior terms we get that the second adjoint equation is:
	\begin{align}
	 m\Sta \frac{\partial p}{\partial t}  &= -\eta \frac{\partial^2 v}{\partial x^2}    - m \frac{\partial \Sta}{\partial t} p  
	-\Sta\nabla \Adjb +m \Sta \frac{\partial v}{\partial x} p +m \gamma \Sta p \\
	&-m \Sta \bigg(v \frac{\partial}{\partial x}\bigg)p - m \Sta \frac{\partial v}{\partial x}p  - m (v  \frac{\partial \rho}{\partial x})p\ \qquad\qquad \qquad\text{in} \quad \Sigma \notag \\
	&p(T)=0. \notag
	\end{align}
	So we get the negative Laplacian for $v$. But in the meeting we said it will be $\Stav - \nabla \Stav$, i.e. $v - \frac{\partial^2 v}{\partial x^2}$. I do not get an additional term for $v$, so I either misunderstood something in the meeting or something in my derivation is wrong.
	(side note: the other equations are not listed because they're not affected by this change of the cost functional. They can be seen in the document I sent two weeks ago 'ThesisEndOfYear1', under Archer Optimality Conditions.)
	
\section{Question 2 (Ben):}
Considering this version of the problem instead
(where $\Con$ is an imposed flow, $\mathbf{w}$ is the control flow and there is the smoothing term $\eta \nabla^2 \Stav$ for $\Stav$ in the PDE):
\begin{align}
&\min_{\Sta,\Stav,\mathbf{w}} \quad \frac{1}{2}|| \Sta - \hat \Sta ||_{L_2(\Sigma)}^2  +\frac{\beta}{2}||\mathbf{w}||_{L_2(\Sigma)}^2\\
&\text{subject to:} \notag\\
& m \Sta \frac{\partial \Stav}{\partial t} = \bigg(- m \Sta (\Stav \cdot \nabla)\Stav - \Sta \nabla V^{ext} - \Sta \Con - \Sta \mathbf{w} - \nabla \Sta - m \gamma \Sta \Stav + \eta \nabla^2 \Stav \bigg) \ \ \ \qquad \ \ \quad\text{in} \quad \Sigma \notag\\
&\frac{\partial \Sta}{\partial t} = - \nabla \cdot (\Sta \Stav) \notag\\
\notag \\
&\Sta \Stav \cdot \mathbf{n} =0	\qquad\text{on } \partial \Omega \notag
\end{align}
We discussed yesterday that we can rewrite the PDE system in terms of $\rho = e^s$. However, in this case we get:
\begin{align*}
&\min_{\Sta,\Stav,\mathbf{w}} \quad \frac{1}{2}|| \Sta - \hat \Sta ||_{L_2(\Sigma)}^2  +\frac{\beta}{2}||\mathbf{w}||_{L_2(\Sigma)}^2\\
&\text{subject to:} \notag\\
&\frac{\partial \Stav}{\partial t} = \bigg(-  (\Stav \cdot \nabla)\Stav - \frac{1}{m} \nabla V^{ext} - \frac{1}{m} \Con - \frac{1}{m} \mathbf{w} - \frac{1}{m } s-  \gamma \Stav + \frac{1}{m}e^{-s}\eta \nabla^2 \Stav \bigg) \ \ \ \qquad \ \ \quad\text{in} \quad \Sigma \notag \\
&\frac{\partial \Sta}{\partial t} = - \nabla \cdot (\Sta \Stav) \notag \\
\notag \\
& \Stav \cdot \mathbf{n} =0	\qquad\text{on } \partial \Omega \notag
\end{align*} 
So then we have $e^{-s}$ in the problem and that's not really the point of the transformation, right? Equation (39) in the Archer paper doesn't multiply $\eta \nabla^2 \Stav$ by $\rho$ so I think doing that would change the meaning of the term?
\\
\\
The same question actually applies to the adjoint equation for the setup in question 1 (see Equation 2). In Setup 1 (Equation 1), the adjoint equation becomes a problem because of the extra Laplacian term from the cost functional. Since this term in Equation 2 doesn't have a $\rho$ term, we can't readily make the transformation there.\\
In Setup 2 (Equation 3) the adjoint equation isn't a problem for rewriting $\rho = e^s$, but the forward equation is.
Long story short: Does it make sense in either Setup (Equation 1 or Equation 3), to apply this transformation and keep the negative exponential $e^{-s}$ in the equation, or is it better to just leave the variable as $\rho$?

\section{Question 2.5}
I think I forgot to double check this yesterday (or I forgot that I did...): In the adjoint equations (e.g. Equation 2 above) I have $\frac{\partial \rho}{\partial t}$ terms. Is it correct to take the first differentiation matrix from the 'time line' TDC?
I have implemented it this way:\\
\\
Difft = this.TDC.ComputeDifferentiationMatrix();\\
Dt = Difft.Dy;\\
Dtrho = Dt*rho;\\
\\
and then interpolate as normal onto the correct time in the ODE solver.

\end{document}