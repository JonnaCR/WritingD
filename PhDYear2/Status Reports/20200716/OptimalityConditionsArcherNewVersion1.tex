
\subsubsection{PDE-Constrained Optimization Problem}
The domain is $\Sigma=\Omega \times [0,T]$. There are two state variables, the particle density $\Sta$ and the velocity $\Stav$. The control is a background flow term $\mathbf{w}$. 
\begin{align*}
&\min_{\Sta,\Stav,\mathbf{w} } \quad \frac{1}{2}||\Sta - \hat \Sta||_{L_2(\Sigma)}^2  +\frac{\beta}{2}||\mathbf{w}||_{L_2(\Sigma)}^2\\
&\text{subject to:}\\
&m \Sta \frac{\partial \Stav}{\partial t}= - m \Sta (\Stav \cdot \nabla)\Stav -\Sta \nabla V_{ext} -\Sta \Con +\Sta \mathbf{w} - \nabla \Sta - m\gamma \Sta \Stav + \eta \rho \nabla^2 \Stav \\
& \quad \quad -\int_\Omega \rho(r) \rho(r') \mathbf{K}(r,r')dr' \\
&\frac{\partial \Sta}{\partial t} + \nabla \cdot (\Sta \Stav)=0 \qquad\qquad \qquad\qquad\qquad\quad \quad\quad\qquad \qquad\qquad \qquad\qquad\qquad\quad \qquad\qquad\qquad\quad\ \ \text{in} \quad \Sigma\\
\\
&\Sta \Stav \cdot \mathbf{n} =0\qquad\qquad \qquad\qquad\qquad\qquad\qquad\qquad\qquad \qquad\qquad \qquad\qquad\qquad \qquad\qquad\qquad\quad \qquad \text{on} \quad \partial  \Omega\\
& \Sta(r,0)=\Sta_0\\
& \Stav(r,0)=\Stav_0.
\end{align*}
Here, we have (+++ I am not sure how to write this in terms of $\mathbf{K}(r,r')$+++):
\begin{align*}
\mathcal{F}[\Sta]=\int_\Omega  \bigg( V_{ext}\Sta + \Sta (\log \Sta -1) +  \frac{1}{2}\int_\Omega \Sta(r) \Sta(r')V_2(|r-r'|)dr' \bigg) dr.
\end{align*}

Then:
\begin{align*}
\rho \nabla \frac{\delta \mathcal{F}[\Sta]}{\delta \Sta} = \Sta \nabla V_{ext} + \nabla \Sta + \int_\Omega \Sta(r) \Sta(r') \nabla V_2(|r-r'|)dr',
\end{align*}
which matches Equation $(39)$ in Archer's paper.
\subsubsection*{The Lagrangian}
The Lagrangian for the above problem is:
\begin{align*}
&\mathcal{L}(\Sta,\Stav,\mathbf{w},\Adja,\Adjb,\Adjc) = \int_0^T \int_\Omega  \frac{1}{2}(\Sta - \hat \Sta)^2 drdt  +\int_0^T \int_\Omega  \frac{\beta}{2}\mathbf{w}^2 drdt\\
&+ \int_0^T \int_\Omega (m \Sta \frac{\partial \Stav}{\partial t} + m \Sta (\Stav \cdot \nabla)\Stav + \Sta \nabla V_{ext} + \Sta \Con - \Sta \mathbf{w} + \nabla \Sta + m\gamma \Sta \Stav  - \rho \eta \nabla^2 \Stav\\
&+\int_\Omega \rho(r) \rho(r') \mathbf{K}(r,r')dr') \cdot \Adja dr dt\\
& + \int_0^T \int_\Omega (\frac{\partial \Sta}{\partial t} + \nabla \cdot (\Sta \Stav)) \Adjb dr dt\\ 
& +\int_0^T \int_{\partial\Omega} \Sta \Stav \cdot \mathbf{n} \Adjc dr dt,
\end{align*}
where $\Adja$, $\Adjb$ and $\Adjc$ are Lagrange multipliers associated with the PDE for $\Stav$, the PDE for $\Sta$ and the boundary condition, respectively.


\subsubsection{Adjoint Equation 1}

The derivative of $\mathcal{L}$ with respect to $\Sta$ in some direction $h$ is, where ${h} \in C_0^\infty(\Sigma) $:
\begin{align*}
&\mathcal{L}_\Sta(\Sta,\Stav,\mathbf{w},\Adja,\Adjb,\Adjc)h = \int_0^T \int_\Omega  (\Sta - \hat \Sta)h drdt \\
&+ \int_0^T \int_\Omega (m h \frac{\partial \Stav}{\partial t}\cdot \Adja + m h( (\Stav \cdot \nabla)\Stav )\cdot \Adja+ h\nabla V_{ext}\cdot \Adja + h \Con \cdot  \Adja - h \mathbf{w} \cdot \Adja + \nabla h\cdot \Adja - \eta h \nabla^2 \Stav \cdot \Adja)  dr dt\\
&+ \int_0^T \int_\Omega (m\gamma h \Stav +\int_\Omega h(r) \rho(r') \mathbf{K}(r,r')dr'+\int_\Omega \rho(r) h(r') \mathbf{K}(r,r') dr') \cdot \Adja dr dt\\
& + \int_0^T \int_\Omega (\Adjb\frac{\partial h}{\partial t} + \Adjb\nabla \cdot (h \Stav))  dr dt +\int_0^T \int_{\partial\Omega} \Adjc h\Stav \cdot \mathbf{n}  dr dt,
\end{align*}
where the product rule is used to take the derivative of the interaction term.
Looking at different integral terms individually:
\begin{align*}
I_1= \int_0^T \int_\Omega \nabla h\cdot \Adja dr dt = \int_0^T \int_{\partial \Omega} h \Adja \cdot \mathbf{n} dr dt - \int_0^T \int_{\Omega} \nabla\cdot \Adja h dr dt
\end{align*}
 
\begin{align*}
I_2 = \int_0^T \int_\Omega \Adjb\frac{\partial h}{\partial t} dr dt = \int_\Omega h(T) \Adjb(T) dr dt - \int_0^T \int_\Omega  \frac{\partial \Adjb}{\partial t}h dr dt
\end{align*}
Note that ${h}(r,0)=0$, (in order to satisfy the condition for all admissible ${h}$) and so the initial condition vanishes from the above expression.
\begin{align*}
I_3= \int_0^T \int_\Omega \Adjb\nabla \cdot (h \Stav) dr dt = \int_0^T \int_{\partial \Omega} \Adjb \Stav \cdot \mathbf{n} h dr dt - \int_0^T \int_\Omega \nabla \Adjb \cdot \Stav h dr dt.
\end{align*}
Furthermore, we have:
\begin{align*}
I_{2B}&= \int_0^T \int_\Omega \bigg(\int_\Omega \rho(r) h(r') \mathbf{K}(r,r')dr'\bigg) \cdot \Adja(r) drdt\\
&=\int_0^T \int_\Omega \int_\Omega \rho(r) h(r') \mathbf{K}(r,r') \cdot \Adja(r) drdr'dt,
\end{align*}
swapping the order of integration. Then we have:
\begin{align*}
I_{2B}&= \int_0^T \int_\Omega  h(r')\bigg(\int_\Omega  \rho(r)\mathbf{K}(r,r')\cdot \Adja(r) dr \bigg)dr'dt,
\end{align*}
and relabelling $r \to r'$ and $r' \to r$ gives:
\begin{align*}
I_{2B}&= \int_0^T \int_\Omega  h(r)\bigg(\int_\Omega  \rho(r')\mathbf{K}(r',r) \cdot \Adja(r') dr' \bigg)drdt.
\end{align*}
If we assume that $\mathbf{K}(r',r) = - \mathbf{K}(r,r')$, we get:
\begin{align*}
I_{2B}&= -\int_0^T \int_\Omega  h(r)\bigg(\int_\Omega  \rho(r')\mathbf{K}(r,r') \cdot \Adja(r') dr' \bigg)drdt.
\end{align*}
Replacing $I_1, I_2, I_{2B}$ and $I_3$ in the derivative gives:
\begin{align*}
&\mathcal{L}_\Sta(\Sta,\Stav,\mathbf{w},\Adja,\Adjb,\Adjc)h = \int_\Omega h(T) \Adjb(T) dr dt  \\
&+ \int_0^T \int_\Omega \bigg( (\Sta - \hat \Sta) +m  \frac{\partial \Stav}{\partial t}\cdot \Adja + m  ((\Stav \cdot \nabla)\Stav )\cdot \Adja+ \nabla V_{ext}\cdot \Adja + \Con \cdot \Adja - \mathbf{w} \cdot \Adja \\
&- \eta \nabla^2 \Stav \cdot \Adja -\nabla\cdot \Adja  - \ \nabla \Adjb \cdot \Stav  -  \frac{\partial \Adjb}{\partial t} + m \gamma \Stav \cdot \Adja\bigg) h dr dt \\
&+ \int_0^T \int_\Omega  \bigg(\int_\Omega  \rho(r')(\Adja(r) - \Adja(r')) \cdot\mathbf{K}(r,r')   dr'  \bigg)hdr dt\\
&+\int_0^T \int_{\partial \Omega} ( \Adja \cdot \mathbf{n}  +  \Adjb \Stav \cdot \mathbf{n}   +\Adjc \Stav \cdot \mathbf{n})h  dr dt
\end{align*}
Setting $\mathcal{L}_\Sta(\Sta,\Stav,\Con,\Adja,\Adjb,\Adjc)h=0$, and restricting the admissible set of choices of $h$ to:
\begin{align*}
h&=0 \quad \text{on} \quad \partial \Omega\\
h(T)&=0.
\end{align*}
Then the derivative becomes:
\begin{align*}
 &\int_0^T \int_\Omega \bigg( (\Sta - \hat \Sta) +m  \frac{\partial \Stav}{\partial t}\cdot \Adja + m  ((\Stav \cdot \nabla)\Stav )\cdot \Adja+ \nabla V_{ext}\cdot \Adja + \Con \cdot \Adja - \mathbf{w} \cdot \Adja \\
 &- \eta \nabla^2 \Stav \cdot \Adja  -\nabla\cdot \Adja  - \ \nabla \Adjb \cdot \Stav  -  \frac{\partial \Adjb}{\partial t} +m \gamma \Stav \cdot \Adja \bigg) h dr dt \\
 &+ \int_0^T \int_\Omega \bigg(  \int_\Omega  \rho(r')(\Adja(r) - \Adja(r')) \cdot \mathbf{K}(r,r')   dr'  \bigg)hdr dt\\
 &=0.
\end{align*}
Since this has to hold for all $h \in C_0^\infty(\Sigma)$ and $C_0^\infty(\Sigma)$ is dense in $L_2(\Sigma)$, the first adjoint equation is derived as:
\begin{align}
& \frac{\partial \Adjb}{\partial t} = (\Sta - \hat \Sta) +m  \frac{\partial \Stav}{\partial t}\cdot \Adja + m ( (\Stav \cdot \nabla)\Stav) \cdot \Adja+ \nabla V_{ext}\cdot \Adja + \Con \cdot \Adja - \mathbf{w} \cdot \Adja  - \eta \nabla^2 \Stav \cdot \Adja \\
&-\nabla\cdot \Adja  -  \nabla \Adjb \cdot \Stav + m \gamma \Stav \cdot \Adja + \int_\Omega  \rho(r')( \Adja(r) - \Adja(r') ) \cdot\mathbf{K}(r,r')   dr' \qquad \text{in} \quad \Sigma \notag
\end{align}
Then, relaxing the conditions on $h$, such that $h(T) \neq 0$ is a permissible choice, gives:
\begin{align*}
\int_\Omega h(T) \Adjb(T) dr dt=0,
\end{align*}
and by the same density argument as above, this gives the final time condition for $\Adjb$:
\begin{align*}
\Adjb(T) = {0} .
\end{align*}
Finally, allowing $h \neq 0$ on $\partial\Omega$ result in:
\begin{align*}
\int_0^T \int_{\partial \Omega} ( \Adja \cdot \mathbf{n}  +  \Adjb \Stav \cdot \mathbf{n}   +\Adjc \Stav \cdot \mathbf{n})h  dr dt=0,
\end{align*}
and again by a density argument:
\begin{align*}
 \Adja \cdot \mathbf{n}  +  \Adjb \Stav \cdot \mathbf{n}   +\Adjc \Stav \cdot \mathbf{n} = 0\qquad \text{on} \quad \partial \Omega
\end{align*}
Since $\Stav \cdot \mathbf{n} =0$ on $ \partial\Omega$, the boundary condition reduces to:
\begin{align*}
\Adja \cdot \mathbf{n} = 0 \qquad \text{on} \quad \partial\Omega.
\end{align*}
Therefore, the first adjoint equation of this problem is:
\begin{align*}
& \frac{\partial \Adjb}{\partial t} = (\Sta - \hat \Sta) +m  \frac{\partial \Stav}{\partial t}\cdot \Adja + m ( (\Stav \cdot \nabla)\Stav) \cdot \Adja+ \nabla V_{ext}\cdot \Adja + \Con \cdot \Adja - \mathbf{w} \cdot \Adja  - \eta \nabla^2 \Stav \cdot \Adja \\
&-\nabla\cdot \Adja  -  \nabla \Adjb \cdot \Stav + m \gamma \Stav \cdot \Adja + \int_\Omega  \rho(r')(\Adja(r) - \Adja(r')) \cdot\mathbf{K}(r,r')   dr' \qquad \text{in} \quad \Sigma \\
& \Adja \cdot \mathbf{n} = 0 \qquad \text{on} \quad \partial\Omega\\
 &\Adjb(T) = {0} .
\end{align*}
Substituting $\frac{\partial \Stav}{\partial t}$ from the forward equations gives:
\begin{align*}
& \frac{\partial \Adjb}{\partial t} = (\Sta - \hat \Sta) +\bigg(- m (\Stav \cdot \nabla)\Stav -\nabla V_{ext} -\Con +\mathbf{w} - \frac{\nabla \Sta}{\Sta} - m\gamma \Stav + \eta  \nabla^2 \Stav -\int_\Omega  \rho(r') \mathbf{K}(r,r')dr' \bigg)\cdot \Adja         \\
&+ m ( (\Stav \cdot \nabla)\Stav) \cdot \Adja+ \nabla V_{ext}\cdot \Adja + \Con \cdot \Adja - \mathbf{w} \cdot \Adja  - \eta \nabla^2 \Stav \cdot \Adja \\
&-\nabla\cdot \Adja  -  \nabla \Adjb \cdot \Stav + m \gamma \Stav \cdot \Adja + \int_\Omega  \rho(r')(\Adja(r) - \Adja(r')) \cdot\mathbf{K}(r,r')  dr' \qquad \text{in} \quad \Sigma \\
& \Adja \cdot \mathbf{n} = 0 \qquad \text{on} \quad \partial \Omega\\
&\Adjb(T) = {0} .
\end{align*}
Cancelling terms and using that $\nabla \Sta = \Sta \nabla \ln \Sta$:
\begin{align*}
& \frac{\partial \Adjb}{\partial t} = (\Sta - \hat \Sta) - \nabla (\ln \Sta)\cdot \Adja -\nabla\cdot \Adja  -  \nabla \Adjb \cdot \Stav  - \int_\Omega  \rho(r')\Adja(r') \cdot\mathbf{K}(r,r')   dr' \qquad \text{in} \quad \Sigma \\
& \Adja \cdot \mathbf{n} = 0 \qquad \text{on} \quad \partial \Omega\\
&\Adjb(T) = {0} .
\end{align*}

\subsubsection{Adjoint Equation 2}
Taking the derivative of the above Lagrangian with respect to $\Stav$ in the direction $\mathbf{h} \in C_0^\infty(\Sigma)$, gives:
\begin{align*}
\mathcal{L}_\Stav(\Sta,\Stav,\mathbf{w},\Adja,\Adjb,\Adjc)\mathbf{h} &=  \int_0^T \int_\Omega (m \Sta \frac{\partial \mathbf{h} }{\partial t} + m \Sta (\mathbf{h} \cdot \nabla)\Stav + m \Sta (\Stav \cdot \nabla)\mathbf{h} + m \gamma \Sta \mathbf{h} - \eta \Sta \nabla^2 \mathbf{h}) \cdot \Adja dr dt\\
& + \int_0^T \int_\Omega ( \nabla \cdot (\Sta \mathbf{h})) \Adjb dr dt\\ 
& +\int_0^T \int_{\partial\Omega} \Sta \mathbf{h} \cdot \mathbf{n} \Adjc dr dt.
\end{align*}

Some of the terms are considered separately, as in the previous calculations:

\begin{align*}
I_4 &= \int_0^T \int_\Omega m \Sta \frac{\partial \mathbf{h} }{\partial t} \cdot \Adja dr dt \\
&= \int_\Omega m \Sta(T) \Adja(T) \cdot \mathbf{h}(T) dr dt - \int_0^T \int_\Omega  m\frac{\partial \Sta}{\partial t} \Adja \cdot \mathbf{h} dr dt - \int_0^T \int_\Omega m \Sta \frac{\partial \Adja}{\partial t} \cdot \mathbf{h} dr dt.
\end{align*}
Note that $\mathbf{h}(0)=\mathbf{0}$, in order to satisfy the conditions on $\mathbf{h}$, as before.
\begin{align*}
I_5= \int_0^T \int_\Omega \Adjb\nabla \cdot ( \Sta \mathbf{h}) dr dr = \int_0^T \int_{\partial \Omega} \Adjb \Sta  \mathbf{n}\cdot \mathbf{h} dr dt - \int_0^T \int_\Omega \Sta\nabla \Adjb \cdot  \mathbf{h} dr dt
\end{align*}

\begin{align*}
I_6 = \int_0^T \int_\Omega m \Sta ((\mathbf{h} \cdot \nabla)\Stav ) \cdot\Adja dr dt = \int_0^T \int_\Omega m \Sta ((\nabla \Stav)^\top\Adja) \cdot  \mathbf{h} dr dt
\end{align*}

\begin{align*}
I_7&=\int_0^T \int_\Omega m \Sta ((\Stav \cdot \nabla)\mathbf{h}) \cdot \Adja dr dt
= \int_0^T \int_{\partial \Omega} m \Sta (\Stav \cdot \mathbf{n})(\Adja \cdot \mathbf{h})dr dt \\
&- \int_0^T \int_\Omega (m \Sta ((\Stav \cdot \nabla)\Adja)\cdot \mathbf{h} + m \Sta (\nabla \cdot \Stav)(\Adja \cdot \mathbf{h}) + m (\Stav \cdot \nabla \Sta)(\Adja \cdot \mathbf{h}))drdt
\end{align*}
\begin{align*}
I_8 &= \int_0^T \int_\Omega \eta \rho \nabla^2 \mathbf{h} \cdot \Adja dr dt = \int_0^T \int_{\partial \Omega} \eta (\nabla \cdot \mathbf{h}) (\Sta \Adja \cdot \mathbf{n}) dr dt - \int_0^T \int_\Omega \bigg(\nabla \cdot (\Sta \Adja) \bigg) \bigg(\nabla \cdot \mathbf{h}\bigg) dr dt\\
& = \int_0^T \int_{\partial \Omega} \eta (\nabla \cdot \mathbf{h}) (\Sta \Adja \cdot \mathbf{n}) dr dt -  \int_0^T \int_{\partial \Omega} \eta\bigg(\nabla \cdot(\Sta \Adja)\bigg) (\mathbf{h} \cdot \mathbf{n}) dr dt + \int_0^T \int_\Omega \eta \nabla^2(\Sta \Adja) \cdot \mathbf{h} dr dt.
\end{align*}
Replacing the rewritten integrals gives:
\begin{align*}
&\mathcal{L}_\Stav(\Sta,\Stav,\mathbf{w},\Adja,\Adjb,\Adjc) \mathbf{h} = \int_\Omega m \Sta(T) \Adja(T) \cdot \mathbf{h}(T) dr dt\\
&+\int_0^T \int_\Omega 
\bigg( - \eta\nabla^2 (\Sta \Adja) - m \frac{\partial \Sta}{\partial t} \Adja  -  m\Sta \frac{\partial \Adja}{\partial t} +m \gamma \Sta \Adja\\
&-\Sta\nabla \Adjb +m \Sta (\nabla \Stav)^\top\Adja 
-m \Sta (\Stav \cdot \nabla)\Adja - m \Sta (\nabla \cdot \Stav)\Adja  - m (\Stav \cdot \nabla \Sta)\Adja  \bigg)\cdot  \mathbf{h} drdt\\
& +\int_0^T \int_{\partial\Omega} ( m \Sta (\Stav \cdot \Adja)+\Sta  \Adjc + \Adjb \Sta)  \mathbf{n}\cdot \mathbf{h} dr dt + \int_0^T \int_{\partial \Omega} \bigg(\eta\bigg(\nabla \cdot(\Sta \Adja)\bigg) (\mathbf{h} \cdot \mathbf{n}) -   \eta (\nabla \cdot \mathbf{h}) (\Sta \Adja \cdot \mathbf{n}) \bigg)dr dt\\
\end{align*}
Then, setting $\mathcal{L}_\Stav(\Sta,\Stav,\mathbf{w},\Adja,\Adjb,\Adjc) \mathbf{h}=\mathbf{0}$ and placing the restrictions on $\mathbf{h}$, as before:
\begin{align*}
\mathbf{h}&=\mathbf{0}, \ \ \nabla \cdot \mathbf{h} = 0 \quad \text{on} \quad \partial \Omega\\
\mathbf{h}(T)&=\mathbf{0},
\end{align*}
gives:
\begin{align*}
&\int_0^T \int_\Omega 
\bigg( - \eta\nabla^2 (\Sta \Adja) - m \frac{\partial \Sta}{\partial t} \Adja  -  m\Sta \frac{\partial \Adja}{\partial t} +m \gamma \Sta \Adja\\
&-\Sta\nabla \Adjb +m \Sta (\nabla \Stav)^\top\Adja 
-m \Sta (\Stav \cdot \nabla)\Adja - m \Sta (\nabla \cdot \Stav)\Adja  - m (\Stav \cdot \nabla \Sta)\Adja  \bigg)\cdot  \mathbf{h} drdt = 0
\end{align*}
Employing the density argument that $C_0^\infty(\Sigma)$ is dense in $L_2(\Sigma)$, which has to hold for all $\mathbf{h}\in C_0^\infty(\Sigma)$, results in:
\begin{align*}
  m\Sta \frac{\partial \Adja}{\partial t} =& - \eta\nabla^2 (\Sta \Adja) - m \frac{\partial \Sta}{\partial t} \Adja   +m \gamma \Sta \Adja-\Sta\nabla \Adjb +m \Sta (\nabla \Stav)^\top\Adja \\
&-m \Sta (\Stav \cdot \nabla)\Adja - m \Sta (\nabla \cdot \Stav)\Adja  - m (\Stav \cdot \nabla \Sta)\Adja  \ \qquad\qquad \text{in} \quad \Sigma.
\end{align*}
Then, relaxing the conditions on $\mathbf{h}$, so that $\mathbf{h}(T) \neq \mathbf{0} $ is permissible, gives
\begin{align*}
 \int_\Omega m \Sta(T) \Adja(T) \cdot \mathbf{h}(T) dr dt=0,
\end{align*}
and so, since $\Sta \neq 0$, this results in the final time condition for $\Adja$:
\begin{align}
\Adja(T)=\mathbf{0}.
\end{align}
Finally, relaxing the conditions on the boundary terms to choose $\mathbf{h}=\mathbf{0}$ and $\nabla \cdot \mathbf{h}  \neq 0$ on $\partial \Omega$ gives:
\begin{align*}
\int_0^T \int_{\partial \Omega} -   \eta (\nabla \cdot \mathbf{h}) (\Sta \Adja \cdot \mathbf{n})  dr dt = 0,
\end{align*}
which, by the same density argument as above, gives, since $\rho \neq 0$ by assumption:
\begin{align}
\label{CondAdj1}
- \eta  \rho \Adja  \cdot \mathbf{n}&= 0  \notag\\
 \Adja  \cdot \mathbf{n} &= 0 \quad \text{on} \quad \partial \Omega.
\end{align}
Then relaxing the final condition, such that $\mathbf{h} \neq 0$ on $\partial \Omega$, we get:
\begin{align*}
\int_0^T \int_{\partial\Omega} \bigg( m \Sta (\Stav \cdot \Adja)+\Sta  \Adjc + \Adjb \Sta + \eta\nabla \cdot(\Sta \Adja)\bigg) ( \mathbf{n}\cdot \mathbf{h})dr dt=0.
\end{align*}
By the same density argument as above, this results in:
\begin{align*}
&(  m \Sta (\Stav \cdot \Adja)+\Sta  \Adjc + \Adjb \Sta + \eta\nabla \cdot(\Sta \Adja)) \mathbf{n} =\mathbf{0}
\end{align*}
+++ Now not sure if any more simplification possible...+++
The second adjoint equation of the above problem is:
\begin{align*}
  m\Sta \frac{\partial \Adja}{\partial t} =& - \eta \nabla^2 (\Sta \Adja) - m \frac{\partial \Sta}{\partial t} \Adja   +m \gamma \Sta \Adja-\Sta\nabla \Adjb +m \Sta (\nabla \Stav)^\top\Adja \\
&-m \Sta (\Stav \cdot \nabla)\Adja - m \Sta (\nabla \cdot \Stav)\Adja  - m (\Stav \cdot \nabla \Sta)\Adja  \ \qquad\qquad \text{in} \quad \Sigma\\
&\Adja \cdot \mathbf{n} = 0 \qquad \qquad \qquad \qquad \qquad \qquad \qquad \qquad \qquad \quad \text{on} \quad \partial \Omega \\
&\Adja(T)=\mathbf{0}
\end{align*}
Finally, substituting the expression for $\frac{\partial \Sta}{\partial t} $ from the forward equations gives:
\begin{align*}
m\Sta \frac{\partial \Adja}{\partial t} =& - \eta \nabla^2 (\Sta \Adja) + m (\nabla(\Sta \Stav))\Adja   +m \gamma \Sta \Adja-\Sta\nabla \Adjb +m \Sta (\nabla \Stav)^\top\Adja \\
&-m \Sta (\Stav \cdot \nabla)\Adja - m \Sta (\nabla \cdot \Stav)\Adja  - m (\Stav \cdot \nabla \Sta)\Adja  \ \qquad\qquad \qquad \text{in} \quad \Sigma\\
&\Adja \cdot \mathbf{n} = 0 \qquad \qquad \qquad \qquad \qquad \qquad \qquad \qquad \qquad \qquad \quad \text{on} \quad \partial \Omega \\
&\Adja(T)=\mathbf{0}
\end{align*}
And therefore, multiplying out the new term and cancelling, we get:
\begin{align*}
m\Sta \frac{\partial \Adja}{\partial t} =& - \eta \nabla^2 (\Sta \Adja)   +m \gamma \Sta \Adja-\Sta\nabla \Adjb +m \Sta (\nabla \Stav)^\top\Adja -m \Sta (\Stav \cdot \nabla)\Adja   \ \qquad\qquad \  \text{in} \quad \Sigma\\
&\Adja \cdot \mathbf{n} = 0 \qquad \qquad \qquad \qquad \qquad \qquad \qquad \qquad \qquad \qquad \qquad \qquad \quad \text{on} \quad \partial \Omega \\
&\Adja(T)=\mathbf{0}
\end{align*}
\subsubsection{The Gradient Equation}
Taking the derivative of the Lagrangian with respect to $\Con$, in the direction $\mathbf{h} \in C_0^\infty(\Sigma)$, gives:
\begin{align*}
\mathcal{L}_{\mathbf{w}}(\Sta,\Stav,\mathbf{w},\Adja,\Adjb,\Adjc) \mathbf{h}= \int_0^T \int_\Omega \beta \mathbf{w} \cdot \mathbf{h} dr dt - \int_0^T \int_\Omega \Sta \Adja \cdot \mathbf{h} dr dt \\
= \int_0^T \int_\Omega ( \beta \mathbf{w} - \Sta \Adja) \cdot \mathbf{h} dr dt.
\end{align*}
Employing the same density argument for the permissible $\mathbf{h}$ gives the gradient equation of the problem:
\begin{align*}
 \mathbf{w} = \frac{1}{\beta} \Sta \Adja \quad \text{in} \quad \Sigma \quad \text{and on} \quad \partial\Omega.
\end{align*}

\subsubsection{Rewriting the equations for implementation}
We employ the transformation $\rho = e^s$, so that $s = \ln \rho$. This is in order to ensure that $\rho$ remains positive, which is a natural condition for the particle density to satisfy. For now, neglect interaction term.
\\
\\
The forward equations become:
\begin{align}
& \frac{\partial \Stav}{\partial t}= -  (\Stav \cdot \nabla)\Stav - \frac{1}{m} \nabla V_{ext} -\frac{1}{m}\Con +\frac{1}{m} \mathbf{w} - \frac{1}{m} \nabla s - \gamma \Stav +  \frac{\eta}{m} \nabla^2 \Stav \label{eqn:INreducedv1}\\
&\quad \quad  -\int_\Omega e^{s(r')} \mathbf{K}(r,r')dr'\\
 &\frac{\partial s}{\partial t} = - \Stav \cdot \nabla s - \nabla \cdot \Stav \label{eqn:INreducedrho1} .
\end{align}
Here, we only divided the first equation by $m \Sta$ and used the fact that $\nabla \Sta = \Sta \nabla \ln \Sta$.\\
\\
The first adjoint equation does not change much. It was: 
\begin{align*}
& \frac{\partial \Adjb}{\partial t} = (\Sta - \hat \Sta) - \nabla (\ln \Sta)\cdot \Adja -\nabla\cdot \Adja  -  \nabla \Adjb \cdot \Stav  - \int_\Omega  \rho(r')\Adja(r') \cdot\mathbf{K}(r,r')   dr' \qquad \text{in} \quad \Sigma 
\end{align*}
and becomes:
\begin{align*}
& \frac{\partial \Adjb}{\partial t} = (e^s - \hat \Sta) - \nabla s \cdot \Adja -\nabla\cdot \Adja  -  \nabla \Adjb \cdot \Stav  - \int_\Omega  e^{s(r')}\Adja(r') \cdot\mathbf{K}(r,r')   dr' \qquad \text{in} \quad \Sigma. 
\end{align*}
\\
The second adjoint equation was:
\begin{align*}
m\Sta \frac{\partial \Adja}{\partial t} =& - \eta \nabla^2 (\Sta \Adja)   +m \gamma \Sta \Adja-\Sta\nabla \Adjb +m \Sta (\nabla \Stav)^\top\Adja -m \Sta (\Stav \cdot \nabla)\Adja   \ \qquad\qquad \  \text{in} \quad \Sigma
\end{align*}
Rewriting the first term in the new variable $s$ gives:
\begin{align*}
- \eta \nabla^2 (\rho \Adja) &= - e^s \bigg(2 \eta \nabla \Adja \cdot \nabla s + \eta \Adja \cdot (\nabla s)^2 + \eta \Adja \cdot \nabla^2 s + \eta \nabla^2 \Adja    \bigg)
\end{align*}
And therefore the new adjoint equation is:
\begin{align*}
\frac{\partial \Adja}{\partial t} =& - \frac{2\eta}{m} \nabla \Adja \cdot \nabla s - \frac{\eta}{m} \Adja \cdot (\nabla s)^2 - \frac{\eta}{m}\Adja \cdot \nabla^2 s - \frac{\eta}{m} \nabla^2 \Adja \\
&+\gamma \Adja-\Sta\nabla \Adjb + (\nabla \Stav)^\top\Adja -  (\Stav \cdot \nabla)\Adja   \ \qquad\qquad \  \text{in} \quad \Sigma
\end{align*}
\\
\\
Finally, in both adjoints, time is reversed due to the negative Laplacian term and the final time conditions, using $\tau = T-t$. 
The first adjoint equation becomes:
\begin{align*}
& \frac{\partial \Adjb}{\partial \tau} = - (e^s - \hat \Sta) + \nabla s \cdot \Adja +\nabla\cdot \Adja  +  \nabla \Adjb \cdot \Stav  + \int_\Omega  e^{s(r')}\Adja(r') \cdot\mathbf{K}(r,r')   dr' \qquad \text{in} \quad \Sigma. 
\end{align*}
The second adjoint equation gives:
\begin{align*}
\frac{\partial \Adja}{\partial \tau} =& \frac{2\eta}{m} \nabla \Adja \cdot \nabla s + \frac{\eta}{m} \Adja \cdot (\nabla s)^2 + \frac{\eta}{m}\Adja \cdot \nabla^2 s + \frac{\eta}{m} \nabla^2 \Adja \\
&-\gamma \Adja+\Sta\nabla \Adjb - (\nabla \Stav)^\top\Adja+  (\Stav \cdot \nabla)\Adja   \ \qquad\qquad \  \text{in} \quad \Sigma
\end{align*}