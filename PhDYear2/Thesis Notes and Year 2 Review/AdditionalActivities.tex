\documentclass[11pt, a4paper]{article}
%\usepackage{proj1}
\usepackage{natbib}
\usepackage{fancyhdr}  
\usepackage{subcaption}
\usepackage{caption}
\usepackage{graphicx}
\linespread{1.25} 
\setlength{\parindent}{0cm}
\graphicspath{{Images/}}
\usepackage{hyperref}
\usepackage{amsmath}
\usepackage{amsfonts}
\usepackage{amssymb}
\usepackage{amsthm}
\usepackage{mathtools}
\usepackage{commath}
\usepackage{bbm}

%\usepackage[sc,osf]{mathpazo}
\usepackage{subcaption}
\usepackage[a4paper, top=1in, left=1.0in, right=1.0in, bottom=1in, includehead, includefoot]{geometry} %Usually have top as 1in

\usepackage{listings}
\usepackage{color} %red, green, blue, yellow, cyan, magenta, black, white
\definecolor{mygreen}{RGB}{28,172,0} % color values Red, Green, Blue
\definecolor{mylilas}{RGB}{170,55,241}


\hypersetup{colorlinks,linkcolor={black},citecolor={blue},urlcolor={black}}
\usepackage{color}
\urlstyle{same}


\theoremstyle{definition}
\newtheorem{definition}{Definition}[section]

%\newcommand{\Sta}{\rho}
\newcommand{\Adj}{p}
\newcommand{\adj}{q}
%\newcommand{\Con}{u}
\newcommand{\Sta}{\rho}
\newcommand{\Stav}{\mathbf{v}}
\newcommand{\Adja}{\mathbf{p}_\Sigma}
\newcommand{\Adjb}{q}
\newcommand{\Adjc}{p_{\partial \Sigma}}
\newcommand{\Con}{\mathbf{f}}
\newcommand{\nor}{\mathbf{n}}



\title{End Of Year Report - Other Activities}
\author{Jonna C. Roden\\ \\Supervision by Dr Ben Goddard and Dr John Pearson\\ \\ \vspace{0.5cm} MIGSAA}
\date{\today}


\pagenumbering{gobble}
\begin{document}
\maketitle

	
\section{Teaching}
I have been teaching four courses over the past two semesters. These include 'Introduction to Linear Algebra' (Y1), 'Several Variable Calculus and Differential Equations' (Y2), 'Honours Algebra (Skills)' (Y3) and 'Honours Complex Variables' (Y3).\\
The first and second year classes were in small workshop groups, 'Honours Algebra (Skills)' was taught in a computer lab and focussed on the algebra software 'Sage'. 'Honours Complex Variables' was an open tutorial with half of the student cohort and several tutors in the same room at each session.\\
Furthermore, I participated in the markfests for 'Introduction to Linear Algebra' and 'Several Variable Calculus and Differential Equations'. I have also marked a question (remotely due to Covid-19) for the 'Honours Complex Variables' final exam.
\section{Classes and Autumn School}
In the first semester of this year, I have attended the MSc course on Stochastic Analysis, a mini-course on Industrial Mathematics and an Autumn School on Optimal Control and Optimization with PDEs. In the second semester I have taken the advanced MIGSAA course on Numerical Analysis of Partial Differential Equations. In total I have been awarded 45 credits.
\section{Generic Skills}
I have completed seven generic skills activities this year, as stated in the logbook. I have been a co-organiser of the MIGSAA Annual Colloquium in September 2019. I have taken two semesters of Spanish classes via the University's short courses. Furthermore, I have been a Mentor for the MAC-MIGs students. I have organised the PG Colloquium for the past two semesters, and, collaboratively with Heriot-Watt, organised an online PhD seminar during the summer, the Maxwell PhD Seminar.
Lastly, I got two generic skills credits for the tutoring and marking activities mentioned in the teaching section above.
\section{Citizenship}
I have attended most ACM seminars this year, some analysis and optimization seminars in UoE and some mathematical biology meetings in HW. I have also attended most of the sessions of the MAC-MIGs Modelling course. I have, as organiser, attended all but one PG Colloquium sessions, as well as all Maxwell PhD Seminar sessions.
I have given a talk at the SIAM-IMA Student Chapter PhD Colloquium, presented a poster at the Annual MIGSAA Colloquium, and presented my work at a 'This is what we do' session to the MAC-MIGs students. I was supposed to give a talk about my work at the BAMC in Glasgow in April, which has been postponed to 2021 due to the current situation. I have given a talk at the LMS Scottish Numerical Methods Network workshop on Multiscale Methods in June. In terms of outreach work, I have held a 10 minute talk about PhD life to Y4 and Y5 undergraduate UoE students and I co-facilitated a lunch meeting, organised by the Piscopia Initiative, aimed at female undergraduate students in order to enhance postgraduate applications from female students. While this type of event was organised for different Schottish universities, the one I attended has been at the University of Glasgow.
Finally, as the organiser of the PG Colloquium I have also been an active part in the UoE PG Committee, such as by helping to organise the departmental christmas party. Currently we are preparing strategies on how to welcome the new PhD cohort in Edinburgh, as well as working on offering different social activities for the existing PhD community to decrease impact of Covid-19 measures on the social life of all PhD students.

\section{Outlook}
Outside my research project I am planning to get involved in the following activities during the next year.
I will take at least one more class for credits. I may attend a second class for credits or attendance only, depending on my other commitments. In terms of knowledge enhancement, I would like to focus on catching up on PDE Theory/Analysis, as well as on expanding my coding skills, by learning Python or a similar language. ++ maybe more optimization techniques too -- goes with career stuff\\
I hope to increase my involvement in conferences and workshops, sharing my work by giving talks, presenting posters and networking with different people. I will continue to attend seminars and workshops on different topics. I am hoping that, additionally, I will take part in an academic/ industry focussed study group.\\
I want to invest some of my additional time in career planning and skills development. This can mean taking workshops, coding classes or by seeking out networking opportunities.\\ 	
I am now a member of the SIAM-IMA Student Chapter Committee in Edinburgh and we are planning to host a series of events with different focus, such as outreach, industry, social events and PhD research talks. 
I also hope to be involved with some projects of the Piscopia Initiative as well as with the PG Committee in a more casual function. \\


 

	
	
\end{document}