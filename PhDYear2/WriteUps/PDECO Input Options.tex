\documentclass[11pt, a4paper]{article}
%\usepackage{proj1}
\usepackage{natbib}
\usepackage{fancyhdr}  
\usepackage{subcaption}
\usepackage{caption}
\usepackage{graphicx}
\linespread{1.25} 
\setlength{\parindent}{0cm}
\graphicspath{{Images/}}
\usepackage{hyperref}
\usepackage{amsmath}
\usepackage{amsfonts}
\usepackage{amssymb}
\usepackage{amsthm}
\usepackage{mathtools}
\usepackage{commath}

%\usepackage[sc,osf]{mathpazo}
\usepackage{subcaption}
\usepackage[a4paper, top=1in, left=1.0in, right=1.0in, bottom=1in, includehead, includefoot]{geometry} %Usually have top as 1in

\usepackage{listings}
\usepackage{color} %red, green, blue, yellow, cyan, magenta, black, white
\definecolor{mygreen}{RGB}{28,172,0} % color values Red, Green, Blue
\definecolor{mylilas}{RGB}{170,55,241}


\hypersetup{colorlinks,linkcolor={black},citecolor={blue},urlcolor={black}}
\usepackage{color}
\urlstyle{same}


\theoremstyle{definition}
\newtheorem{definition}{Definition}[section]

\title{Exact Solutions for the Full Problem \\with Force Control and with Flow Control}
\date{}
\newcommand{\Sta}{\rho}
\newcommand{\Adj}{p}
\newcommand{\Con}{u}

\pagenumbering{gobble}
\begin{document}
	
\section*{PDECO - Input Options}
A summary of all possible input options for PDECO. The options are called 'opts'. This has three substructures 'optsPhys','optsNum' and 'optsPlot'. These are detailed in the following.

\section{optsPhys}	
There are three sub-structures in this structure. They are 'ProbSpecs', 'DataIn' and 'Params'.
\subsection{optsPhys.optsPhysFW}
The structure 'optsPhys.optsPhysFW' contains the following substructures: 'ProbSpecs', 'DataInput' and 'Params'.

\subsubsection{ProbSpecs}
The structure 'ProbSpecs' contains all problem specifications that are needed to specify a given forward problem to solve. The input options are summarized in the table below.
   \begin{center}
	\begin{tabular}{ |c| c | }
		\hline
		Input Name & 'BCFunStr' \\ 
		\hline
		Description & Determines boundary conditions\\ 
		\hline 
		Options & 'ComputeDirichletBCs', 'ComputeNeumannBCs', 'ComputeMixedBCs' \\
		& takes inputs: 'rho', 'rhoflux', 'bound', 'normal', 'this'. \\
		& parameter 'eps' for Dirichlet Contribution in MixedBCs  (See 'Params') \\
		\hline
		\hline
		Input Name & 'PDERHSStr' \\ 
		\hline
		Description & Determines which PDE is solved,\\ 
		& all can include interaction term, switched on with 'gamma' (see 'Params')\\
		\hline 
		Options 1D & 'D\_Force': Diffusion, Force Control \\
		&  AD\_Force: Advection Diffusion, Force Control (inc. Vext, Force) \\
		&  AD\_Forcefl: Advection Diffusion, Force Control (inc. Vext, Force, wFlow) \\
		&  AD\_Flow: Advection Diffusion, Flow Control (only Flow term) \\
		&  AD\_Flowf: Advection Diffusion, Flow Control (Flow term and Force) \\
		&  AD\_FlowfVext: Advection Diffusion, Flow Control (Flow term, Force, Vext) \\
		\hline
		Options 2D & These remain the same for 1D and 2D\\
		\hline
        Note: & To include a new PDE see separate table below.\\
	    \hline		
	\end{tabular}
\end{center}


   \begin{center}
	\begin{tabular}{ |c| c | }
		\hline
		Input Name & 'ComputeNorm' \\ 
		\hline
		Description & Specifies in which norm errors are computed within the solver. \\
		& Takes function name as input. Needs to have syntax ComputeNorm(fNew,fOld,SInt,TInt).\\ 
		&  (SInt: Space Integration\ TInt: Time Integration)\\
		\hline 
		Options & 'ComputeRelL2LinfNorm': Relative Norm, L2 space, Linf time.  \\
		& 'ComputeRelPWNorm': Relative Norm, Pointwise errors. \\
		& 'ComputeL1Norm': Absolute Norm, L1 space and time. \\
		\hline
	\end{tabular}
   \end{center}	



\subsubsection*{Naming conventions for adding PDEs}	
Two things to be aware of: \\
- if your gradient equation is different than force/flow control, you need to update this in the code\\
- generally, your PDERHSFlag needs to be added inside the code too\\
\\
The naming convention for adding new PDEs into the code is:\\
- for a forward problem (also the forward problem during the optimization -- it'll be the same both times)\\
'ComputeFW' + string to specify the problem.
- for the adjoint equation\\
'ComputeOptP' + the SAME string that specifies the FW problem\\
slightly differs in 2D -- see table below
\begin{center}
	\begin{tabular}{ |c| c | }
\hline
 & Adding new PDEs into PDECO: \\
 \hline
1D FW (\& rho Opt )& Naming convention of the function: 'ComputeFW'\&'PDERHSStr':\\ 
Input arguments:& this, RHSInput \\
\hline
1D Adjoint Equation & Naming convention of the function: 'ComputeOptp'\& 'PDERHSStr'. \\
Input arguments:& this, pRHSInput\\
\hline
2D FW (\& rho Opt ) & Naming convention of the function: 'ComputeFW2D'\&'PDERHSStr'\\
Input arguments: & this, RHSInput\\ 
\hline
2D Adjoint Equation & Naming convention of the function:'ComputeOptp2D'\& 'PDERHSStr'\\
Input arguments:& this, pRHSInput\\ 
\hline
	\end{tabular}
\end{center}	
\subsubsection{optsPhysFW.DataInput}	
This structure includes all the input data to the solver. It can take the input as matrices or as function name. For each variable, the code checks whether the variable name exists as input. If it does, this will be used in the code, and interpolated where necessary. If it does not exist, then the field is created and filled by the given function. If all variables are specified by the function, then the only input in 'DataIn' is the function name! The below table illustrates the options. \\

	
 \begin{center}
	\begin{tabular}{ |c| c | c|}
		\hline
		Input Name & Input Description & Options \\ 
		\hline
		testFun & Function Name of input function & e.g. 'AD\_Flow\_Neumann\_Exact'. \\
		& & Needs all the below variables in output structure. \\
	 	& & Set variable to zero if not needed. \\ 
	 	DataRecompConv & Decides whether to check for datastorage & true/ false\\
	 	& results for the convolution matrix &\\
	 	DataRecompFW & Decides whether to check for datastorage & true/ false\\
	 	& results for the forward problem &\\
		\hline
		Optional: & & \\
		\hline
		rhoIC & Initial Condition for rho & Input vector ($1 \times N$) or not existent.\\
		pIC & Final Time Condition for p & Input vector ($1 \times N$) or not existent.\\ 
		wFlow & Flow Control term & Input Matrix ($n \times N$) or not existent. \\
		wForce & Force Control term & Input Matrix ($n \times N$) or not existent. \\
		Force & Additional Force term  & Input Matrix ($n \times N$) or not existent. \\
		Flow & Additional Flow  term & Input Matrix ($n \times N$) or not existent. \\
		Vext & External Potential term & Input Matrix ($n \times N$) or not existent. \\
		\hline
	\end{tabular}
 \end{center}		
	
	Note that the 'DataRecomp...' flags are there to determine whether to recompute your function (true) or whether to find a result in the datastorage folder (false). If nothing is specified, the default is 'false'.
\subsubsection{optsPhysFW.Params}
Here, any relevant parameters for the FW problem are specified. 

	\begin{center}
		\begin{tabular}{ |c| c | }
			\hline
			Input Name & Description \\ 
			\hline
			beta & Regularization parameter in optimization \\
			gamma & Magnitude of particle interaction term. \\ 
			D0 & Diffusion Coefficient.\\
		    eps & Contribution of Dirichlet term to Mixed BCs. \\
			other & Additional input structure to 'testFun'. \\
			\hline
		\end{tabular}
	\end{center}
\subsection{optsPhys.optsPhysOpt}
In this structure we have the inputs that are only necessary for the optimization but not for the forward solution. The optimization problem still gets all the inputs for the forward problem.
The structures are: 'OptSolver', and 'OptDataInput'.


\subsubsection{OptSolver}
All things that are needed to specify the optimization solver are specified here. 
\begin{center}
	\begin{tabular}{ | c | c |}
		\hline
		Input Name & 'SolverFlag' \\ 
		\hline
		Description & Choosing the solver for Optimization\\ 
		\hline 
		Options & 'fsolve': Inbuilt MatLab solver and with Multiple Shooting. \\
		& 'Picard': Picard update and Multiple Shooting \\
		& 'FixPt': Picard update/ Fixed Point iteration, no shooting \\
		\hline
		\hline
		Input Name & 'AdaSolverStr' \\ 
		\hline
		Description & Option to make Picard or Fixed Point Algorithm adaptive.\\ 
		\hline 
		Options & Input is a function name, Function input 'Err' and 'lambda'.\\
		& 'Adaptive': will change 'lambda' (see 'Params') to be adaptive\\
		& '[  ]', or exclusion of this option will leave 'lambda' static. \\
		\hline
     \end{tabular}
	\end{center}
Other input:
\begin{center}
	\begin{tabular}{ | c | c |}
		\hline
		lambda & Mixing rate of old and new solution in 'Picard' and 'FixPt' solvers.\\
		\hline
		BCFunOptStr & In case the adjoint has different BCs  from rho (such as in non-zero Dirichlet case)\\
		\hline
		OptsTols & Tolerances for the optimization solver (structure - see below).\\
		\hline
	\end{tabular}
\end{center}
These are the tolerances for the different optimization solvers.
\begin{center}
	\begin{tabular}{ | c | c | c |}
		\hline
		Location of usage & Input Name  & Description  \\ 
		\hline
		fsolve & 'OptsTols.FunTol' & Function Tolerance  \\
		fsolve & 'OptsTols.OptiTol' & Optimality Tolerance  \\ 
		fsolve & 'OptsTols.StepTol' & Stepsize Tolerance  \\ 
		\hline 
		Picard, FixPt & 'OptsTols.ConsTol' & Consistency Tolerance  \\ 
		& & Consistency condition for convergence.\\
		\hline
	\end{tabular}
\end{center}

\subsubsection*{OptDataInput}
The DataRecompOpt option determines whether datastorage checks for an existing solution or whether it recomputes the solution.
\begin{center}
	\begin{tabular}{ | c | c | c |}
		\hline
		Input Name & Description  & Input Format  \\ 
		\hline
		OptirhoIG & Initial guess for Optimization & Input Matrix ($n \times N$), not existent, \\
		& & or 'rhoFW' to call forward result as IG\\
		\hline
		OptipIG &  Initial guess for Optimization & Input Matrix ($n \times N$) or not existent. \\
		\hline
		DataRecompOpt & whether to recompute data & true/false\\
		\hline
	\end{tabular}
\end{center}



\subsection{optsPhys.V2Num/optsNum.V2Num}
'V2Num' is part of both 'optsPhys' and 'optsNum'. It contains the Kernel function for the particle interaction term and takes additional parameters.
\begin{center}
	\begin{tabular}{ |c| c | }
		\hline
		Input Name &  Description \\ 
		\hline
		'V2' & Takes a function name, e.g. 'ComputeGaussian'. \\
		 & This function takes the inputs 'V2Num' and 'y', the points.\\ 
		'alpha' & A parameter for the function. \\
		\hline 
	\end{tabular}
\end{center}	
	
\section{optsNum}	

This structure has four (five) substructures: 'PhysArea', 'PlotArea'. 'TimeArea' and 'Tols'	(and 'V2Num' - see above).

\subsection{optsNum.PhysArea}
Specifies the physical spacial domain on which computations should be carried out.
\begin{center}
	\begin{tabular}{ |c| c | c| c|}
		\hline
		Dimension &  Shape & Number of space points & Boundary \\ 
		\hline
		 1D & 'SpectralLine' & 'N' & 'yMin', 'yMax'\\
		 2D & 'Box' & 'N' (input as e.g. $[50,50]$) & 'y1Min', 'y1Max', 'y2Min', 'y2Max' \\ 
		\hline 
	\end{tabular}
\end{center}
\subsection{optsNum.PlotArea}
Specifies plotting points and area. 
\begin{center}
	\begin{tabular}{ |c| c | c| }
		\hline
		Dimension & Number of plotting points & Boundary \\ 
		\hline
		1D   & 'N' & 'yMin', 'yMax'\\
		2D & 'N1', 'N2'  & 'y1Min', 'y1Max', 'y2Min', 'y2Max' \\ 
		\hline 
	\end{tabular}
\end{center}	
\subsection{optsNum.TimeArea}
Specifies the time domain on which computations should be carried out.
\begin{center}
	\begin{tabular}{ | c | c | c |}
		\hline
		 Boundary left & Boundary right & Number of time points  \\ 
		\hline
	    't0' & 'TMax' & 'n'\\
		\hline 
	\end{tabular}
\end{center}	
	
\subsection{optsNum.Tols}
Solver tolerances etc.
\begin{center}
	\begin{tabular}{ | c | c | c |}
		\hline
	     Location of usage & Input Name  & Description  \\ 
		\hline
		ODE solver & 'AbsTol' & Absolute Tolerance  \\
		ODE solver & 'RelTol' & Relative Tolerance  \\ 
		\hline
	\end{tabular}
\end{center}
	
\section{optsPlot}
This structure is independent of the forward or optimization computation. It contains the following:

\begin{center}
	\begin{tabular}{ | c | c |}
		\hline
         'plotTimes' & plotting times the solution is interpolated on\\
		\hline
		 'VarStruct' & structure that specifies the variables to be plotted \\
		 & and the location on 'subplot'. up to 8 locations specifiable. \\
		 'VarStruct' Inputs: & A string followed by a number to specify the location\\
		 Strings & 'rhoIC', 'pIC', 'OptirhoIG', 'OptipIG', 'rhoFW', 'rhoOpt', 'pOpt'\\
		 & 'wFW', 'wOpt', 'rhoHat', 'Force', Flow', 'Vext'\\
		 \hline
	\end{tabular}
\end{center}
There is in principle also an option to supply a matrix to be plotted. Then you can't specify the location on the plot but instead give a structure 'VecOpts' containing 'VecOpts.titleName' (title of plot), 'Vecopts.yLabel' (yLabel of plot).\\
Example:\\
VarStr = struct('OptirhoIG',1,'OptipIG',2,M,VecOpts);\\
where M is the matrix you are interested in plotting.
	
\end{document}