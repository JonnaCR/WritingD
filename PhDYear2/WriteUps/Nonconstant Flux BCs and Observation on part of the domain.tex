\documentclass[11pt, a4paper]{article}
%\usepackage{proj1}
\usepackage{natbib}
\usepackage{fancyhdr}  
\usepackage{subcaption}
\usepackage{caption}
\usepackage{graphicx}
\linespread{1.25} 
\setlength{\parindent}{0cm}
\graphicspath{{Images/}}
\usepackage{hyperref}
\usepackage{amsmath}
\usepackage{amsfonts}
\usepackage{amssymb}
\usepackage{amsthm}
\usepackage{mathtools}
\usepackage{commath}
\usepackage{bbm}
%\usepackage[sc,osf]{mathpazo}
\usepackage{subcaption}
\usepackage[a4paper, top=1in, left=1.0in, right=1.0in, bottom=1in, includehead, includefoot]{geometry} %Usually have top as 1in

\usepackage{listings}
\usepackage{color} %red, green, blue, yellow, cyan, magenta, black, white
\definecolor{mygreen}{RGB}{28,172,0} % color values Red, Green, Blue
\definecolor{mylilas}{RGB}{170,55,241}


\hypersetup{colorlinks,linkcolor={black},citecolor={blue},urlcolor={black}}
\usepackage{color}
\urlstyle{same}

\newcommand{\Sta}{\rho}
\newcommand{\Stav}{\mathbf{v}}
\newcommand{\Adja}{{p}_Q}
\newcommand{\Adjb}{q}
\newcommand{\Adjc}{p_{\partial Q}}
\newcommand{\Con}{{f}}
\newcommand{\nor}{\mathbf{n}}

\theoremstyle{definition}
\newtheorem{definition}{Definition}[section]


\pagenumbering{gobble}
\begin{document}
	
	\lstset{language=Matlab,%
		%basicstyle=\color{red},
		breaklines=true,%
		morekeywords={matlab2tikz},
		keywordstyle=\color{blue},%
		morekeywords=[2]{1}, keywordstyle=[2]{\color{black}},
		identifierstyle=\color{black},%
		stringstyle=\color{mylilas},
		commentstyle=\color{mygreen},%
		showstringspaces=false,%without this there will be a symbol in the places where there is a space
		numbers=left,%
		numberstyle={\tiny \color{black}},% size of the numbers
		numbersep=9pt, % this defines how far the numbers are from the text
		emph=[1]{for,end,break},emphstyle=[1]\color{blue}, %some words to emphasise
		%emph=[2]{word1,word2}, emphstyle=[2]{style},    
		basicstyle=\footnotesize\ttfamily,
	}

\section*{Observation on a Part of the Domain and Non-Constant Flux Boundary Conditions}


\begin{figure}[h]
	\includegraphics[scale=0.8]{observation.png}
	\caption{Domain of Interest}
	\label{Observation1}
\end{figure}

The problem of interest is of the form:
\begin{align*}
&\min_{\Sta, \Con} \quad \frac{1}{2}|| \Sta -\hat \Sta||^2_{L_2( Q_{Ob})} + \frac{\beta}{2}|| \Con||^2_{L_2(Q)}\\
\text{subject to:}\\
&\partial_t \rho = \nabla^2 \rho - \nabla \cdot (\rho \mathbf{w}_{Flow}) +\nabla \cdot (\rho \nabla V_{ext}) + \nabla \cdot \int_\Omega \rho(r) \rho(r') \nabla V_2(|r-r'|) dr' + \Con \quad  \quad\text{in} \quad Q,\notag\\
& \rho = \rho_0 \quad \text{at} \quad t=0 \notag\\
& - \mathbf{j} \cdot \nor = \mathbbm{1}_{\partial \Omega_L}( C_{L1}  + C_{L2}\Sta) +\mathbbm{1}_{\partial \Omega_R} ( C_{R1}  + C_{R2}\Sta) +\mathbbm{1}_{\partial \Omega_I} 0, \quad  \quad\text{on} \quad \partial Q, 
\end{align*}
where $C_{L1}, C_{L2}, C_{R1}$, $C_{R2}$ are constants and $\mathbbm{1}$ is the indicator function of the set (the parts of the boundary) of interest.
Furthermore, $\mathbf{j}$ satisfies:
\begin{align*}
\mathbf{j}=\nabla \rho - (\rho \mathbf{w}_{Flow}) +(\rho \nabla V_{ext}) +  \int_\Omega \rho(r) \rho(r') \nabla V_2(|r-r'|) dr'.
\end{align*}
Moreover, let $\hat \Sta$ be defined such that:
\begin{align*}
\hat \Sta = \mathbbm{1}_{ \Omega_{Ob1}} \tilde \Sta  +\mathbbm{1}_{ \Omega_{Ob2}} 0.
\end{align*}

\subsection*{The Lagrangian}
The Lagrangian is of the form:
\begin{align*}
\mathcal{L}(\Sta,\Con,\Adja,\Adjc ) &=\frac{1}{2} \int_0^T \int_{\Omega_{Ob}} (\Sta - \hat \Sta)^2 dr dt + \frac{\beta}{2}\int_0^T \int_\Omega \Con^2 drdt \\
&+ \int_0^T \int_\Omega \bigg( \partial_t \rho - \nabla^2 \rho + \nabla \cdot (\rho \mathbf{w}_{Flow}) -\nabla \cdot (\rho \nabla V_{ext}) + \nabla \cdot \int_\Omega \rho(r) \rho(r') \nabla V_2(|r-r'|) -f \bigg) \Adja dr dt\\
&+ \int_0^T \int_{\partial \Omega} \bigg(  \bigg(-\nabla \rho+ (\rho \mathbf{w}_{Flow}) -(\rho \nabla V_{ext}) -  \int_\Omega \rho(r) \rho(r') \nabla V_2(|r-r'|) dr' \bigg)\cdot \nor\\
&  -\mathbbm{1}_{\partial \Omega_L}( C_{L1}  + C_{L2}\Sta) -\mathbbm{1}_{\partial \Omega_R} ( C_{R1}  + C_{R2}\Sta) -\mathbbm{1}_{\partial \Omega_I} 0 \bigg) \Adjc dr dt.
\end{align*}

\subsection*{The Adjoint Equation}
The derivative of $\mathcal{L}$ with respect to $\rho$ is, as taken from the extended project:
\begin{align*}
&\mathcal{L}_\rho (\rho,\mathbf{w},\Adja,\Adjc)h=
\int_\Omega h(T) \Adja(T) dr\\
&+ \int_0^T \int_\Omega \bigg( \mathbbm{1}_{ \Omega_{Ob}} (\rho- \hat{\rho})  - \partial_t \Adja  - \nabla \Adja \cdot \mathbf{w}_{Flow}  - \nabla^2 \Adja \notag 
+  \nabla \Adja \cdot \nabla V_{ext}  \notag \\
&+ \int_\Omega (\nabla  \Adja(r)+\nabla  \Adja(r')) \rho(r') \nabla V_2(|r-r'|) dr'+ \int_{\partial \Omega} ( \Adjc(r') - \Adja(r')) \rho(r')   \frac{\partial V_2(|r-r'|)}{\partial n} dr' \bigg) h dr dt \\
&+  \int_0^T\int_{\partial \Omega}  \bigg(
\bigg(\frac{\partial \Adja }{\partial n} + \Adja  \mathbf{w} \cdot \mathbf{n} - \Adjc \mathbf{w}_{Flow} \cdot \mathbf{n}  +  \Adjc \dfrac{\partial V_{ext}}{\partial n} - \Adja \frac{\partial V_{ext}}{\partial n} + ( \Adjc - \Adja)  \int_\Omega \rho(r') \frac{\partial V_2(|r-r'|)}{\partial n} dr'\\
& -\mathbbm{1}_{\partial \Omega_L} C_{L2} \Adjc   -\mathbbm{1}_{\partial \Omega_R} C_{R2} \Adjc \bigg)h + \bigg( \Adjc- \Adja \bigg) \frac{\partial h}{\partial n} \bigg) dr dt =0.
\end{align*}
Then, from appropriate analysis we find that:
\begin{align*}
\Adjc = \Adja,
\end{align*}
and therefore we get:
\begin{align*}
\mathbbm{1}_{\Omega_{Ob}}(\rho- \hat{\rho})   - \partial_t  \Adja  - \nabla \Adja \cdot \mathbf{w}_{Flow}  - \nabla^2 \Adja \notag 
+  \nabla \Adja \cdot \nabla V_{ext}  \notag \\
+ \int_\Omega (\nabla  \Adja(r)+\nabla  \Adja(r')) \rho(r') \nabla V_2(|r-r'|) dr' &=0, \quad \text{in} \quad Q, \\
\frac{\partial \Adja }{\partial n}  -\mathbbm{1}_{\partial \Omega_L} C_{L2} \Adja   -\mathbbm{1}_{\partial \Omega_R} C_{R2} \Adja&=0, \quad \text{on} \quad \partial Q.
\end{align*}
In particular, this is:
\begin{align*}
\mathbbm{1}_{\Omega_{Ob1}}(\rho- \hat{\rho}) +\mathbbm{1}_{\Omega_{Ob2}}\rho  - \partial_t  \Adja  - \nabla \Adja \cdot \mathbf{w}_{Flow}  - \nabla^2 \Adja \notag 
+  \nabla \Adja \cdot \nabla V_{ext}  \notag \\
+ \int_\Omega (\nabla  \Adja(r)+\nabla  \Adja(r')) \rho(r') \nabla V_2(|r-r'|) dr' &=0, \quad \text{in} \quad Q, \\
\frac{\partial \Adja }{\partial n}  -\mathbbm{1}_{\partial \Omega_L} C_{L2} \Adja   -\mathbbm{1}_{\partial \Omega_R} C_{R2} \Adja&=0, \quad \text{on} \quad \partial Q.
\end{align*}














\end{document}