\documentclass[11pt, a4paper]{article}
%\usepackage{proj1}
\usepackage{natbib}
\usepackage{fancyhdr}  
\usepackage{subcaption}
\usepackage{caption}
\usepackage{graphicx}
\linespread{1.25} 
\setlength{\parindent}{0cm}
\graphicspath{{Images/}}
\usepackage{hyperref}
\usepackage{amsmath}
\usepackage{amsfonts}
\usepackage{amssymb}
\usepackage{amsthm}
\usepackage{mathtools}
\usepackage{commath}

%\usepackage[sc,osf]{mathpazo}
\usepackage{subcaption}
\usepackage[a4paper, top=1in, left=1.0in, right=1.0in, bottom=1in, includehead, includefoot]{geometry} %Usually have top as 1in

\usepackage{listings}
\usepackage{color} %red, green, blue, yellow, cyan, magenta, black, white
\definecolor{mygreen}{RGB}{28,172,0} % color values Red, Green, Blue
\definecolor{mylilas}{RGB}{170,55,241}


\hypersetup{colorlinks,linkcolor={black},citecolor={blue},urlcolor={black}}
\usepackage{color}
\urlstyle{same}


\theoremstyle{definition}
\newtheorem{definition}{Definition}[section]

\title{Exact Solutions for the Full Problem \\with Force Control and with Flow Control}
\date{}
\newcommand{\Sta}{\rho}
\newcommand{\Adj}{p}
\newcommand{\Con}{u}

\pagenumbering{gobble}
\begin{document}
	\section{Some bits and pieces that are not part of the paper.}
	
	
	+++ Two additional nonlinear control problems below, may be deleted? +++
	\section{Neumann boundary conditions, Symmetric Example 1}
	Consider the following symmetric setup:
	\begin{align*}
	\widehat \rho &= \frac{1}{2}(1-t) + t\frac{1}{4}(\cos(\pi y)+2)\\
	\rho_{0} &= \frac{1}{2}\\
	q_{T} &= 0\\
	\vec{w} &= 0\\
	f &=0\\
	V_{ext} &=0
	\end{align*}
	Table \ref{TabNFlowAddEx1} summarizes the results for this example.The attractive interaction term causes $\rho$ to move towards the centre of the domain. Since $\widehat \rho$ is also centred in the domain, $J_{uc}$ is small for $\gamma =-1$ in comparison to the problems with $\gamma =0$ and $\gamma =1$. This example illustrates that the particle interaction term can have a significant impact on the optimization problem considered. 
	
	\begin{table}
		\begin{tabular}{ ||c|| c | c |c | c ||}
			\hline
			$\beta$ / $\gamma$ & $10^{-3}$  & $10^{-1}$  & $10$ & $10^3$ \\ 
			\hline 
			& $J_{uc} = 0.0041$ & $J_{uc} = 0.0041$  & $J_{uc} = 0.0041$ & $J_{uc} = 0.0041$\\ 
			$-1$ & $J_c = 0.0002$ & $J_c = 0.0033$ & $J_c = 0.0040$ & $J_c = 0.0041$\\ 
			& Iter. $= 607$ & Iter. $= 637$  & Iter. $= 311$ & Iter. $= 1$\\ 
			\hline
			& $J_{uc} = 0.0104$ & $J_{uc} = 0.0104$  & $J_{uc} = 0.0104$& $J_{uc} = 0.0104$\\
			$0$  & $J_c = 0.0005$ & $J_c = 0.0086$  & $J_c = 0.0104$ & $J_c = 0.0104$\\ 
			& Iter. $= 635$ & Iter. $= 671$  & Iter. $= 340$ & Iter. $= 1$\\ 
			\hline
			& $J_{uc} = 0.0195$ & $J_{uc} = 0.0195$  & $J_{uc} = 0.0195$ & $J_{uc} = 0.0195$\\
			$1$  & $J_c = 0.0011$ & $J_c = 0.0164$  & $J_c = 0.0195$ & $J_c = 0.0195$\\ 
			& Iter. $= 656$ & Iter. $= 696$  & Iter. $= 356$ & Iter. $= 1$\\ 
			\hline 
		\end{tabular}
		\caption{}
		\label{TabNFlowAddEx1}
	\end{table}
	
	
	\section{Neumann boundary conditions, Symmetric Example 2}
	Consider the following symmetric setup, which is the opposite of the first symmetric example:
	\begin{align*}
	\widehat \rho &= \frac{1}{2}(1-t) + t\frac{1}{4}(-\cos(\pi y)+2)\\
	\rho_{0} &= \frac{1}{2}\\
	q_{T} &= 0\\
	\vec{w} &= 0\\
	f &=0\\
	V_{ext} &=0
	\end{align*}
	This example can be compared to the Symmetric Example 1. Here, the desired state is having $\rho$ clustered at both boundaries, which is similar to the effect of the repulsive interaction term $\gamma = 1$. Therefore, for this choice of interaction term, the value of the cost functional $J_{uc}$ is smaller than the one for $\gamma = 0$ and $\gamma = -1$. This is the opposite to the observation made in the Symmetric Example 1, which is to be expected, given the two choices of desired state.
	
	\begin{table}
		\begin{tabular}{ ||c|| c | c |c | c ||}
			\hline
			$\beta$ / $\gamma$ & $10^{-3}$  & $10^{-1}$  & $10$ & $10^3$ \\ 
			\hline 
			& $J_{uc} = 0.0209$ & $J_{uc} = 0.0209$  & $J_{uc} = 0.0209$ & $J_{uc} = 0.0209$\\ 
			$-1$ & $J_c = 0.0009$ & $J_c = 0.0168$ & $J_c = 0.0209$ & $J_c = 0.0209$\\ 
			& Iter. $= 646$ & Iter. $= 691$  & Iter. $= 379$ & Iter. $= 1$\\ 
			\hline
			& $J_{uc} = 0.0104$ & $J_{uc} = 0.0104$  & $J_{uc} = 0.0104$& $J_{uc} = 0.0104$\\
			$0$  & $J_c = 0.0005$ & $J_c = 0.0086$  & $J_c = 0.0104$ & $J_c = 0.0104$\\ 
			& Iter. $= 635$ & Iter. $= 671$  & Iter. $= 340$ & Iter. $= 1$\\ 
			\hline
			& $J_{uc} = 0.0047$ & $J_{uc} = 0.0047$  & $J_{uc} = 0.0047$ & $J_{uc} = 0.0047$\\
			$1$  & $J_c = 0.00034$ & $J_c = 0.0040$  & $J_c = 0.0047$ & $J_c = 0.0047$\\ 
			& Iter. $= 623$ & Iter. $= 651$  & Iter. $= 297$ & Iter. $= 1$\\ 
			\hline 
		\end{tabular}
		\caption{}
		\label{TabNFlowAddEx2}
	\end{table}
	
	
	
\end{document}