
\subsubsection{PDE-Constrained Optimization Problem}
In the following we consider an optimal control problem constrained by \eqref{eqn:INeqns1}. 
The domain is $\Sigma=\Omega \times [0,T]$. As described in the previous section, there are two state variables, the particle density $\Sta$ and the velocity $\Stav$. The control is applied through a background flow term $\mathbf{w}$ and the desired state is denoted by $\hat \rho$. 
\begin{align*}
&\min_{\Sta,\Stav,\mathbf{w} } \mathcal J(\Sta,\Stav) \coloneqq \quad \frac{1}{2}||\Sta - \hat \Sta||_{L_2(\Sigma)}^2  +\frac{\beta}{2}||\mathbf{w}||_{L_2(\Sigma)}^2\\
&\text{subject to:}\\
& \frac{\partial \Stav}{\partial t}= -  (\Stav \cdot \nabla)\Stav - \gamma  \Stav + \frac{\eta}{m} \nabla^2 \Stav  -\frac{1}{m}\Con +\frac{1}{m}\mathbf{w} -\frac{1}{m} \nabla V_{ext} - \frac{1}{m}\nabla \ln \Sta  -\frac{1}{m}\int_\Omega \rho(r') \mathbf{K}(r,r')dr' \\
&\frac{\partial \Sta}{\partial t} + \nabla \cdot (\Sta \Stav)=0 \qquad\qquad \qquad\qquad\qquad\quad \quad\quad\qquad \qquad\qquad \qquad\qquad\qquad\quad \qquad\quad\ \ \text{in} \quad \Sigma\\
\\
&\Sta \Stav \cdot \mathbf{n} =0\qquad\qquad \qquad\qquad\qquad\qquad\qquad\qquad\qquad \qquad\qquad \qquad\qquad\qquad \qquad\qquad\quad  \text{on} \quad \partial  \Omega\\
& \Sta(r,0)=\Sta_0\\
& \Stav(r,0)=\Stav_0.
\end{align*}

\subsubsection*{The Lagrangian}
The Lagrangian for the above problem is:
\begin{align*}
&\mathcal{L}(\Sta,\Stav,\mathbf{w},\Adja,\Adjb,\Adjc) = \int_0^T \int_\Omega  \frac{1}{2}(\Sta - \hat \Sta)^2 drdt  +\int_0^T \int_\Omega  \frac{\beta}{2}\mathbf{w}^2 drdt\\
&+ \int_0^T \int_\Omega (\frac{\partial \Stav}{\partial t}+  (\Stav \cdot \nabla)\Stav + \gamma  \Stav - \frac{\eta}{m} \nabla^2 \Stav  +\frac{1}{m}\Con -\frac{1}{m}\mathbf{w} +\frac{1}{m} \nabla V_{ext} + \frac{1}{m}\nabla \ln \Sta  \\
&\quad \quad +\frac{1}{m}\int_\Omega \rho(r') \mathbf{K}(r,r')dr') \cdot \Adja dr dt\\
& + \int_0^T \int_\Omega (\frac{\partial \Sta}{\partial t} + \nabla \cdot (\Sta \Stav)) \Adjb dr dt\\ 
& +\int_0^T \int_{\partial\Omega} \Sta \Stav \cdot \mathbf{n} \Adjc dr dt,
\end{align*}
where $\Adja$, $\Adjb$ and $\Adjc$ are Lagrange multipliers associated with the PDE for $\Stav$, the PDE for $\Sta$ and the boundary condition, respectively.


\subsubsection{Adjoint Equation 1}

The derivative of $\mathcal{L}$ with respect to $\Sta$ in some direction $h$ is taken where ${h} \in C_0^\infty(\Sigma) $.
First, the derivative of $f(\rho) = \nabla \ln \rho $ needs to be treated separately. We consider the Fr\'echet derivative $Df(h) = \frac{\delta f}{\delta \rho}h$. We find that $\frac{\delta f}{\delta \rho} = \nabla \frac{1}{\rho}$ and so the Fr\'echet derivative becomes $Df(h) = \left(\nabla \frac{1}{\rho}\right)h$. \\
Then we get:
\begin{align*}
&\mathcal{L}_\Sta(\Sta,\Stav,\mathbf{w},\Adja,\Adjb,\Adjc)h = \int_0^T \int_\Omega  (\Sta - \hat \Sta)h drdt \\
&+ \int_0^T \int_\Omega \bigg( \frac{1}{m}\nabla \bigg(\frac{1}{\rho}\bigg)\cdot \Adja h\bigg)  dr dt + \int_0^T \int_\Omega \bigg(\int_\Omega h(r') \mathbf{K}(r,r') dr'\bigg) \cdot \Adja dr dt\\
& + \int_0^T \int_\Omega \bigg(\Adjb\frac{\partial h}{\partial t} + \Adjb\nabla \cdot (h \Stav)\bigg)  dr dt +\int_0^T \int_{\partial\Omega} \Adjc h\Stav \cdot \mathbf{n}  dr dt,
\end{align*}
where the product rule is used to take the derivative of the interaction term.
Looking at different integral terms individually:

\begin{align*}
I_1 = \int_0^T \int_\Omega \Adjb\frac{\partial h}{\partial t} dr dt = \int_\Omega h(T) \Adjb(T) dr dt - \int_0^T \int_\Omega  \frac{\partial \Adjb}{\partial t}h dr dt
\end{align*}
Note that ${h}(r,0)=0$, (in order to satisfy the condition for all admissible ${h}$) and so the initial condition vanishes from the above expression.
\begin{align*}
I_2= \int_0^T \int_\Omega \Adjb\nabla \cdot (h \Stav) dr dt = \int_0^T \int_{\partial \Omega} \Adjb \Stav \cdot \mathbf{n} h dr dt - \int_0^T \int_\Omega \nabla \Adjb \cdot \Stav h dr dt.
\end{align*}
Furthermore, we have:
\begin{align*}
I_{3}&= \int_0^T \int_\Omega \bigg(\int_\Omega  h(r') \mathbf{K}(r,r')dr'\bigg) \cdot \Adja(r) drdt\\
&=\int_0^T \int_\Omega \int_\Omega h(r') \mathbf{K}(r,r') \cdot \Adja(r) drdr'dt,
\end{align*}
swapping the order of integration. Then we have:
\begin{align*}
I_{3}&= \int_0^T \int_\Omega  h(r')\bigg(\int_\Omega  \mathbf{K}(r,r')\cdot \Adja(r) dr \bigg)dr'dt,
\end{align*}
and relabelling $r \to r'$ and $r' \to r$ gives:
\begin{align*}
I_{3}&= \int_0^T \int_\Omega  h(r)\bigg(\int_\Omega  \mathbf{K}(r',r) \cdot \Adja(r') dr' \bigg)drdt.
\end{align*}
If we assume that $\mathbf{K}(r',r) = - \mathbf{K}(r,r')$, we get:
\begin{align*}
I_{3}&= -\int_0^T \int_\Omega  h(r)\bigg(\int_\Omega  \mathbf{K}(r,r') \cdot \Adja(r') dr' \bigg)drdt.
\end{align*}
Replacing $I_1, I_2$ and $I_3$ in the derivative gives:
\begin{align*}
&\mathcal{L}_\Sta(\Sta,\Stav,\mathbf{w},\Adja,\Adjb,\Adjc)h = \int_\Omega h(T) \Adjb(T) dr dt  \\
&+ \int_0^T \int_\Omega \bigg( (\Sta - \hat \Sta) - \frac{\partial \Adjb}{\partial t} +\frac{1}{m}\nabla \bigg(\frac{1}{\rho}\bigg)\cdot \Adja -  \int_\Omega  \Adja(r') \cdot\mathbf{K}(r,r')   dr'  \bigg)hdr dt\\
&+\int_0^T \int_{\partial \Omega} \bigg(  \Adjb \Stav \cdot \mathbf{n}   +\Adjc \Stav \cdot \mathbf{n}\bigg)h  dr dt
\end{align*}
Setting $\mathcal{L}_\Sta(\Sta,\Stav,\Con,\Adja,\Adjb,\Adjc)h=0$, and restricting the admissible set of choices of $h$ to:
\begin{align*}
h&=0 \quad \text{on} \quad \partial \Omega\\
h(T)&=0.
\end{align*}
Then the derivative becomes:
\begin{align*}
 &\int_0^T \int_\Omega \bigg((\Sta - \hat \Sta) - \frac{\partial \Adjb}{\partial t} +\frac{1}{m}\nabla \bigg(\frac{1}{\rho}\bigg)\cdot \Adja -  \int_\Omega  \Adja(r') \cdot\mathbf{K}(r,r')   dr'  \bigg)hdr dt =0.
\end{align*}
Since this has to hold for all $h \in C_0^\infty(\Sigma)$ and $C_0^\infty(\Sigma)$ is dense in $L_2(\Sigma)$, the first adjoint equation is derived as:
\begin{align}
&(\Sta - \hat \Sta) - \frac{\partial \Adjb}{\partial t} - \nabla \Adjb \cdot \Stav +\frac{1}{m}\nabla \bigg(\frac{1}{\rho}\bigg)\cdot \Adja  -  \int_\Omega  \Adja(r') \cdot\mathbf{K}(r,r')   dr'  \qquad \text{in} \quad \Sigma \notag
\end{align}
Then, relaxing the conditions on $h$, such that $h(T) \neq 0$ is a permissible choice, gives:
\begin{align*}
\int_\Omega h(T) \Adjb(T) dr dt=0,
\end{align*}
and by the same density argument as above, this gives the final time condition for $\Adjb$:
\begin{align*}
\Adjb(T) = {0} .
\end{align*}
Finally, allowing $h \neq 0$ on $\partial\Omega$ result in:
\begin{align*}
\int_0^T \int_{\partial \Omega} \bigg(  \Adjb \Stav \cdot \mathbf{n}   +\Adjc \Stav \cdot \mathbf{n}\bigg)h  dr dt=0,
\end{align*}
and again by a density argument:
\begin{align*}
  \Adjb \Stav \cdot \mathbf{n}   +\Adjc \Stav \cdot \mathbf{n} = 0\qquad \text{on} \quad \partial \Omega
\end{align*}
However, since $\Stav \cdot \mathbf{n} =0$ on $ \partial\Omega$, this term vanishes.
Therefore, the first adjoint equation of this problem is:
\begin{align*}
&  \frac{\partial \Adjb}{\partial t}  = (\Sta - \hat \Sta)- \nabla \Adjb \cdot \Stav +\frac{1}{m}\nabla \bigg(\frac{1}{\rho}\bigg)\cdot \Adja -  \int_\Omega  \Adja(r') \cdot\mathbf{K}(r,r')   dr'  \qquad \text{in} \quad \Sigma \\
 &\Adjb(T) = {0} .
\end{align*}


\subsubsection{Adjoint Equation 2}

Taking the derivative of the above Lagrangian with respect to $\Stav$ in the direction $\mathbf{h} \in C_0^\infty(\Sigma)$, gives:
\begin{align*}
\mathcal{L}_\Stav(\Sta,\Stav,\mathbf{w},\Adja,\Adjb,\Adjc)\mathbf{h} &=  \int_0^T \int_\Omega ( \frac{\partial \mathbf{h} }{\partial t} +  (\mathbf{h} \cdot \nabla)\Stav +  (\Stav \cdot \nabla)\mathbf{h} +  \gamma \mathbf{h} - \frac{\eta}{m}\nabla^2 \mathbf{h}) \cdot \Adja dr dt\\
& + \int_0^T \int_\Omega ( \nabla \cdot (\Sta \mathbf{h})) \Adjb dr dt\\ 
& +\int_0^T \int_{\partial\Omega} \Sta \mathbf{h} \cdot \mathbf{n} \Adjc dr dt.
\end{align*}

Some of the terms are considered separately, as in the previous calculations:

\begin{align*}
I_4 &= \int_0^T \int_\Omega  \frac{\partial \mathbf{h} }{\partial t} \cdot \Adja dr dt \\
&= \int_\Omega \Adja(T) \cdot \mathbf{h}(T) dr dt  - \int_0^T \int_\Omega \frac{\partial \Adja}{\partial t} \cdot \mathbf{h} dr dt.
\end{align*}
Note that $\mathbf{h}(0)=\mathbf{0}$, in order to satisfy the conditions on $\mathbf{h}$, as before.
\begin{align*}
I_5= \int_0^T \int_\Omega \Adjb\nabla \cdot ( \Sta \mathbf{h}) dr dr = \int_0^T \int_{\partial \Omega} \Adjb \Sta  \mathbf{n}\cdot \mathbf{h} dr dt - \int_0^T \int_\Omega \Sta\nabla \Adjb \cdot  \mathbf{h} dr dt
\end{align*}

\begin{align*}
I_6 = \int_0^T \int_\Omega  ((\mathbf{h} \cdot \nabla)\Stav ) \cdot\Adja dr dt = \int_0^T \int_\Omega  ((\nabla \Stav)^\top\Adja) \cdot  \mathbf{h} dr dt
\end{align*}

\begin{align*}
I_7&=\int_0^T \int_\Omega ((\Stav \cdot \nabla)\mathbf{h}) \cdot \Adja dr dt
= \int_0^T \int_{\partial \Omega}(\Stav \cdot \mathbf{n})(\Adja \cdot \mathbf{h})dr dt \\
&- \int_0^T \int_\Omega ( ((\Stav \cdot \nabla)\Adja)\cdot \mathbf{h} +(\nabla \cdot \Stav)(\Adja \cdot \mathbf{h}) )drdt
\end{align*}
\begin{align*}
I_8 &= \int_0^T \int_\Omega \frac{\eta}{m} \nabla^2 \mathbf{h} \cdot \Adja dr dt =
\int_0^T \int_\Omega \frac{\eta}{m} \bigg( ( \nabla^2 (\Adja) ) \cdot \mathbf{h} + \nabla \cdot \bigg( \nabla ( \Adja \cdot \mathbf{h} ) - 2 (\nabla\Adja)^\top \mathbf{h} \bigg) \bigg) dr dt\\
&= \int_0^T \int_{\partial \Omega} \bigg(\frac{\eta}{m}  \nabla ( \Adja \cdot \mathbf{h} ) - \frac{2\eta}{m}  (\nabla \Adja)^\top \mathbf{h} \bigg) \cdot \mathbf{n} dr dt + \int_0^T \int_\Omega\frac{\eta}{m}  ( \nabla^2 \Adja ) \cdot \mathbf{h} dr dt\\
&=\int_0^T \int_{\partial \Omega} \bigg(\frac{\eta}{m} (\nabla \Adja)^\top \mathbf{h} + \frac{\eta}{m}(\nabla \mathbf{h})^\top \Adja - \frac{2\eta}{m}  (\nabla \Adja)^\top \mathbf{h} \bigg) \cdot \mathbf{n} dr dt + \int_0^T \int_\Omega\frac{\eta}{m}  ( \nabla^2 \Adja ) \cdot \mathbf{h} dr dt \\
&=\int_0^T \int_{\partial \Omega} \bigg(\frac{\eta}{m}(\nabla \mathbf{h})^\top \Adja - \frac{\eta}{m}  (\nabla \Adja)^\top \mathbf{h} \bigg) \cdot \mathbf{n} dr dt + \int_0^T \int_\Omega\frac{\eta}{m}  ( \nabla^2 \Adja ) \cdot \mathbf{h} dr dt
\end{align*}

Replacing the rewritten integrals gives:
\begin{align*}
&\mathcal{L}_\Stav(\Sta,\Stav,\mathbf{w},\Adja,\Adjb,\Adjc) \mathbf{h} = \int_\Omega\Adja(T) \cdot \mathbf{h}(T) dr dt\\
&+\int_0^T \int_\Omega 
\bigg( - \frac{\eta}{m} \nabla^2 \Adja -   \frac{\partial \Adja}{\partial t} + \gamma\Adja-\Sta\nabla \Adjb +(\nabla \Stav)^\top\Adja 
- (\Stav \cdot \nabla)\Adja -  (\nabla \cdot \Stav)\Adja   \bigg)\cdot  \mathbf{h} drdt\\
& +\int_0^T \int_{\partial\Omega} (\Stav \cdot \mathbf{n})(\Adja \cdot \mathbf{h}) +(\Sta  \Adjc + \Adjb \Sta)  (\mathbf{n}\cdot \mathbf{h}) dr dt + \int_0^T \int_{\partial \Omega}  \bigg( \frac{\eta}{m}  (\nabla \Adja)^\top \mathbf{h}  - \frac{\eta}{m}(\nabla \mathbf{h})^\top \Adja \bigg) \cdot \mathbf{n} dr dt\\
\end{align*}
Then, setting $\mathcal{L}_\Stav(\Sta,\Stav,\mathbf{w},\Adja,\Adjb,\Adjc) \mathbf{h}=\mathbf{0}$ and placing the restrictions on $\mathbf{h}$, as before:
\begin{align*}
\mathbf{h}&=\mathbf{0}, \ \ \nabla \mathbf{h} = 0 \quad \text{on} \quad \partial \Omega\\
\mathbf{h}(T)&=\mathbf{0},
\end{align*}
gives:
\begin{align*}
&\int_0^T \int_\Omega 
\bigg( - \frac{\eta}{m} \nabla^2 \Adja -   \frac{\partial \Adja}{\partial t} + \gamma\Adja-\Sta\nabla \Adjb +(\nabla \Stav)^\top\Adja 
- (\Stav \cdot \nabla)\Adja -  (\nabla \cdot \Stav)\Adja    \bigg)\cdot  \mathbf{h} drdt = 0
\end{align*}
Employing the density argument that $C_0^\infty(\Sigma)$ is dense in $L_2(\Sigma)$, which has to hold for all $\mathbf{h}\in C_0^\infty(\Sigma)$, results in:
\begin{align*}
   \frac{\partial \Adja}{\partial t} =  - \frac{\eta}{m} \nabla^2 \Adja  + \gamma\Adja-\Sta\nabla \Adjb +(\nabla \Stav)^\top\Adja 
- (\Stav \cdot \nabla)\Adja -  (\nabla \cdot \Stav)\Adja     \ \qquad\qquad \text{in} \quad \Sigma.
\end{align*}
Then, relaxing the conditions on $\mathbf{h}$, so that $\mathbf{h}(T) \neq \mathbf{0} $ is permissible, gives
\begin{align*}
 \int_\Omega m \Sta(T) \Adja(T) \cdot \mathbf{h}(T) dr dt=0,
\end{align*}
and so, since $\Sta \neq 0$, this results in the final time condition for $\Adja$:
\begin{align}
\Adja(T)=\mathbf{0}.
\end{align}
Finally, relaxing the conditions on the boundary terms to choose $\mathbf{h}=\mathbf{0}$ and $\nabla \mathbf{h}  \neq 0$ on $\partial \Omega$ gives:
\begin{align*}
\int_0^T \int_{\partial \Omega} - \frac{\eta}{m} (\nabla \mathbf{h})^\top \Adja \cdot \mathbf{n} dr dt = 0,
\end{align*}
which, by the same density argument as above, gives, since $\eta \neq 0$ by assumption:
\begin{align}
\label{CondAdj1}
- \frac{\eta}{m}  \Adja  \cdot \mathbf{n}&= 0  \notag\\
 \Adja  \cdot \mathbf{n} &= 0 \quad \text{on} \quad \partial \Omega.
\end{align}
Then relaxing the final condition, such that $\mathbf{h} \neq 0$ on $\partial \Omega$, we get:
\begin{align*}
\int_0^T \int_{\partial\Omega} (\Stav \cdot \mathbf{n})(\Adja \cdot \mathbf{h}) +(\Sta  \Adjc + \Adjb \Sta)  (\mathbf{n}\cdot \mathbf{h}) +\frac{\eta}{m}  (\nabla \Adja)^\top ( \mathbf{n}\cdot \mathbf{h})dr dt=0.
\end{align*}
By the same density argument as above, this results in:
\begin{align*}
&(\Stav \cdot \mathbf{n})\Adja  +(\Sta  \Adjc + \Adjb \Sta)\mathbf{n}+\frac{\eta}{m}  (\nabla \Adja)^\top \mathbf{n} =\mathbf{0}
\end{align*}
And since $\Stav \cdot \mathbf{n} = \mathbf{0}$ on $\partial \Omega$, we have the following relationship between the Lagrange multipliers:
\begin{align*}
\bigg(\Sta  \Adjc + \Adjb \Sta+\frac{\eta}{m}  (\nabla \Adja)^\top \bigg) \mathbf{n} =\mathbf{0}
\end{align*}
+++ Now not sure if any more simplification possible...+ Also. I am not sure if the condition is $\mathbf{h} = \mathbf{0}$ on $\partial \Omega$, or if it should be $\mathbf{h} \cdot \mathbf{n}= \mathbf{0}$+++
The second adjoint equation of the above problem is:
\begin{align*}
&\frac{\partial \Adja}{\partial t} =  - \frac{\eta}{m} \nabla^2 \Adja  + \gamma\Adja-\Sta\nabla \Adjb +(\nabla \Stav)^\top\Adja 
- (\Stav \cdot \nabla)\Adja -  (\nabla \cdot \Stav)\Adja     \ \qquad\qquad \text{in} \quad \Sigma\\
&\Adja \cdot \mathbf{n} = 0 \qquad \qquad \qquad \qquad \qquad \qquad \qquad \qquad \qquad \quad \text{on} \quad \partial \Omega \\
&\Adja(T)=\mathbf{0}
\end{align*}
\subsubsection{The Gradient Equation}
Taking the derivative of the Lagrangian with respect to $\Con$, in the direction $\mathbf{h} \in C_0^\infty(\Sigma)$, gives:
\begin{align*}
\mathcal{L}_{\mathbf{w}}(\Sta,\Stav,\mathbf{w},\Adja,\Adjb,\Adjc) \mathbf{h}= \int_0^T \int_\Omega \beta \mathbf{w} \cdot \mathbf{h} dr dt - \int_0^T \int_\Omega \frac{1}{m} \Adja \cdot \mathbf{h} dr dt \\
= \int_0^T \int_\Omega ( \beta \mathbf{w} - \frac{1}{m}\Adja) \cdot \mathbf{h} dr dt.
\end{align*}
Employing the same density argument for the permissible $\mathbf{h}$ gives the gradient equation of the problem:
\begin{align*}
 \mathbf{w} = \frac{1}{\beta m} \Adja \quad \text{in} \quad \Sigma \quad \text{and on} \quad \partial\Omega.
\end{align*}

\subsubsection{Rewriting the equations for implementation}
We employ the transformation $\rho = e^s$, so that $s = \ln \rho$. This is in order to ensure that $\rho$ remains positive, which is a natural condition for the particle density to satisfy. For now, neglect interaction term.
\\
\\
The forward equations become:
\begin{align}
& \frac{\partial \Stav}{\partial t}= -  (\Stav \cdot \nabla)\Stav - \frac{1}{m} \nabla V_{ext} -\frac{1}{m}\Con +\frac{1}{m} \mathbf{w} - \frac{1}{m} \nabla s - \gamma \Stav +  \frac{\eta}{m} \nabla^2 \Stav \label{eqn:INreducedv1}\\
&\quad \quad  -\int_\Omega e^{s(r')} \mathbf{K}(r,r')dr'\\
 &\frac{\partial s}{\partial t} = - \Stav \cdot \nabla s - \nabla \cdot \Stav \label{eqn:INreducedrho1} .
\end{align}
Here, we only divided the first equation by $m \Sta$ and used the fact that $\nabla \Sta = \Sta \nabla \ln \Sta$.\\
\\
The first adjoint equation does not change much. It was: 
\begin{align*}
&  \frac{\partial \Adjb}{\partial t}  = (\Sta - \hat \Sta)- \nabla \Adjb \cdot \Stav +\frac{1}{m}\nabla \bigg(\frac{1}{\rho}\bigg)\cdot \Adja -  \int_\Omega  \Adja(r') \cdot\mathbf{K}(r,r')   dr' 
\end{align*}
and becomes:
\begin{align*}
&  \frac{\partial \Adjb}{\partial t}  = (\Sta - \hat \Sta)- \nabla \Adjb \cdot \Stav +\frac{1}{m}\nabla e^{-s}\cdot \Adja -  \int_\Omega  \Adja(r') \cdot\mathbf{K}(r,r')   dr' 
\end{align*}
\\
The second adjoint equation was:
\begin{align*}
&\frac{\partial \Adja}{\partial t} =  - \frac{\eta}{m} \nabla^2 \Adja  + \gamma\Adja-\Sta\nabla \Adjb +(\nabla \Stav)^\top\Adja 
- (\Stav \cdot \nabla)\Adja -  (\nabla \cdot \Stav)\Adja   
\end{align*}
And therefore the new adjoint equation is:
\begin{align*}
&\frac{\partial \Adja}{\partial t} =  - \frac{\eta}{m} \nabla^2 \Adja  + \gamma\Adja-e^s\nabla \Adjb +(\nabla \Stav)^\top\Adja 
- (\Stav \cdot \nabla)\Adja -  (\nabla \cdot \Stav)\Adja   
\end{align*}

Finally, in both adjoints, time is reversed due to the negative Laplacian term and the final time conditions, using $\tau = T-t$. 
The first adjoint equation becomes:
\begin{align*}
\frac{\partial \Adjb}{\partial \tau}  = - (e^s - \hat \Sta)+ \nabla \Adjb \cdot \Stav -\frac{1}{m}\nabla e^{-s}\cdot \Adja  +  \int_\Omega  \Adja(r') \cdot\mathbf{K}(r,r')   dr' 
\end{align*}
The second adjoint equation gives:
\begin{align*}
&\frac{\partial \Adja}{\partial \tau} =   \frac{\eta}{m} \nabla^2 \Adja  - \gamma\Adja-e^s\nabla \Adjb -(\nabla \Stav)^\top\Adja 
+ (\Stav \cdot \nabla)\Adja +  (\nabla \cdot \Stav)\Adja  
\end{align*}