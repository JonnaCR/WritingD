
Example 1 in Section \ref{sec:Examples1d} is considered to compare the computational time taken of the fixed point algorithm and the inbuilt Matlab function \texttt{fsolve}. Note that the comparison is slightly impacted by the fact that convergence is measured differently in these two numerical methods. However, a general comparison can be made regarding the efficiency of the two approaches.
We choose $n=20$, $N=30$, the ODE solver tolerance is set to be $10^{-8}$, the optimality tolerance is $10^{-4}$ and $\beta = 10^{-3}$. 
As can be seen in Table \ref{TabA3:Prob1}, the running time of the fixed point algorithm is considerably faster than for \texttt{fsolve}, while the resulting values of the cost functional remain the same. This can be confirmed by comparing the number of function evaluations for each method, which is an important measure when dealing with large systems, such as the two-dimensional problems discussed in this paper, since each iteration is costly for large problems. The differences in $\rho$ and $\adj$ are broadly in line with the optimality tolerance set, however the control differs more because $\vec{w}$ is updated using the optimal values of $\rho$ and $\adj$. 
%
%\begin{table}
\begin{tabular}{ | c | c || c | c | c ||}
\hline
\multicolumn{2}{|c||}{} & Fixed Point & \texttt{fsolve} & Difference   \\
\hline
\hline
 & $\mathcal{J}_{uc}$ & $\numprint{0.0438}$ & $\numprint{0.0438}$ &   \\
 & $\mathcal{J}_{c}$ & $\numprint{0.0011}$ & $\numprint{0.0011}$ &   \\
 & \texttt{Iter} (\texttt{funcEval}) & $\numprint{670}$ ($\numprint{670}$)  & $\numprint{38}$ ($\numprint{31959}$)  &   \\
$\kappa =-1$ & Time taken (s) & $\numprint{2.4939e+2}$ & $\numprint{9.1546e+3}$ &   \\
 & $\mathcal{E}_{\rho_{Diff}}$ & & &$\numprint{1.1348e-3}$  \\
 & $\mathcal{E}_{\adj_{Diff}}$ & & &$\numprint{7.2742e-5}$  \\
 & $\mathcal{E}_{\vec{w}_{Diff}}$ & & & $\numprint{7.6725e-2}$  \\
\hline
 & $\mathcal{J}_{uc}$ & $\numprint{0.0434}$ & $\numprint{0.0434}$ &   \\
 & $\mathcal{J}_{c}$ & $\numprint{0.0020}$ & $\numprint{0.0020}$ &   \\
 & \texttt{Iter} (\texttt{funcEval}) & $\numprint{654}$ ($\numprint{654}$)  & $\numprint{38}$ ($\numprint{34239}$)  &   \\
$\kappa =1$ & Time taken (s) & $\numprint{3.3794e+2}$ & $\numprint{1.0167e+4}$ &   \\
 & $\mathcal{E}_{\rho_{Diff}}$ & & &$\numprint{3.0610e-4}$  \\
 & $\mathcal{E}_{\adj_{Diff}}$ & & &$\numprint{4.8701e-5}$  \\
 & $\mathcal{E}_{\vec{w}_{Diff}}$ & & & $\numprint{8.9056e-3}$  \\
\hline
\end{tabular}
\caption{Comparison of the outputs of the fixed point method, with those obtained using \texttt{fsolve}.}
\label{TabA3:Prob1}
\end{table} %\label{TabA3:Prob1}
