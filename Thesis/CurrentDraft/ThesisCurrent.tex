\documentclass[11pt, a4paper]{article}
%\usepackage{proj1}
\usepackage{natbib}
\usepackage{fancyhdr}  
\usepackage{subcaption}
\usepackage{caption}
\usepackage{graphicx}
\usepackage{numprint}
\usepackage{multirow}
\linespread{1.25} 
\setlength{\parindent}{0cm}
\graphicspath{{Images/}}
\usepackage{hyperref}
\usepackage{amsmath}
\usepackage{amsfonts}
\usepackage{amssymb}
\usepackage{amsthm}
\usepackage{mathtools}
\usepackage{commath}
\usepackage{bbm}

%\usepackage[sc,osf]{mathpazo}
\usepackage{subcaption}
\usepackage[a4paper, top=1in, left=1.0in, right=1.0in, bottom=1in, includehead, includefoot]{geometry} %Usually have top as 1in

\usepackage{listings}
\usepackage{color} %red, green, blue, yellow, cyan, magenta, black, white
\definecolor{mygreen}{RGB}{28,172,0} % color values Red, Green, Blue
\definecolor{mylilas}{RGB}{170,55,241}


\hypersetup{colorlinks,linkcolor={black},citecolor={blue},urlcolor={black}}
\usepackage{color}
\urlstyle{same}


\theoremstyle{definition}
\newtheorem{definition}{Definition}[section]

\newcommand{\adja}{q_a}
\newcommand{\adjb}{q_b}
\newcommand{\adjaB}{q_{a,\partial \Omega}}
\newcommand{\adjbB}{q_{b,\partial \Omega}}
\newcommand{\adjB}{q_{\partial \Omega}}
\newcommand{\Adja}{\mathbf{p}}
\newcommand{\Adjb}{q}
\newcommand{\adj}{q}
\newcommand{\Adjc}{{q}_{\partial \Omega}}
\newcommand{\ra}{\rho_a}
\newcommand{\rb}{\rho_b}
\newcommand{\w}{\mathbf{w}}
\newcommand{\f}{\mathbf{f}}
\newcommand{\ve}{\mathbf{v}}
\newcommand{\n}{\mathbf{n}}
\newcommand{\h}{\mathbf{h}}
\newcommand{\K}{\mathbf{K}}
\newcommand{\hr}{\widehat \rho}



\title{{\huge PDE-Constrained Optimization \\for Multiscale Particle Dynamics} \\ with Industrial Applications}
\author{Jonna C. Roden\\ \\Supervision by Dr Ben Goddard and Dr John Pearson\\ \vspace{0.5cm} Maxwell Institute Graduate School for Analysis and its Applications}
\date{\today}


\pagenumbering{gobble}
\begin{document}
	\maketitle
\begin{abstract}
+ Later +
	
\end{abstract}

\newpage
%\section*{Acknowledgements}
%+ Later +
%\newpage
\pagenumbering{Roman} 
\tableofcontents
%\newpage
%\listoffigures
%\listoftables
\newpage
\pagenumbering{arabic} 

	
	
	\section{Introduction}
	
	\section{DFT/ Statistical Mechanics Background}
	
	\section{DDFT Background}
		\begin{itemize}
			\item Some general background, link to DFT stuff
			\item Derivation from particle equations
			\item Highlight limitations of models
			\item Focus on overdamped equations
		\end{itemize}

	\section{PDE-Constrained Optimization}
		\begin{itemize}
			\item General Intro to the class of problems and a general problem setup
			\item Link to other areas?
			\item Literature review of Mean-Field PDECO and link between the two fields.
			\item Here may be a good point to go into the mean-field convergence stuf of the microscopic OCP to ours a little more.
		\end{itemize}
	
	\subsection{Optimality Conditions}
		\begin{itemize}
			\item Maybe here or above: discuss how to solve OCPs in different ways.
			\item Frechet stuff.
			\item Optimality conditions for the overdamped equations with Dirichlet and no-flux BCs
			\item Find a general notation to make future derivations 'additive'.
		\end{itemize}

	\section{Numerical Methods}
	\subsection{Pseudospectral Methods and 2DChebClass}
		\begin{itemize}
			\item Pseudospectral Methods and Spectral Element Methods
			\item How shapes work
			\item How multishape works
			\item Validation in appendix??
			\item Convergence results for pseudospectral methods.
			\item Comparison with FEM and FDM
		\end{itemize}
	\subsection{Optimization Methods}
		\begin{itemize}
			\item Fixed Point (+ Armijo Extension)
			\item Newton-Krylov
			\item Picard Multiple Shooting
			\item fsolve Multiple Shooting 
			\item fsolve algorithms in Matlab
		\end{itemize}
	
	\section{Validation of the Numerical Methods}
	\begin{itemize}
		\item Error measures
		\item Validation of 2DChebClass stuff
		\item Exact Solutions in 1D/ 2D/ (3D?!)
		\item Validation against fsolve (and comparison of the different solvers)
		\item Perturbing w
		\item Other things -- see notes
	\end{itemize}

	\section{Numerical Experiments}
		\begin{itemize}
			\item Only examples with overdamped equations
			\item Maybe box and multishape?
			\item Paper examples (and others?)
		\end{itemize}

    \section{Sedimentation}
	\subsection{Background DDFT}
		\begin{itemize}
			\item Derivation of the extra term from SPT, explain FMT, etc. 
			\item A little literature review
			\item Comparison of simulations to Archer?
		\end{itemize}
	\subsection{OCP}
		\begin{itemize}
			\item State possible OCPs
			\item Derive Optimality Conditions
			\item Here add Periodic BCs and time-independent control
			\item In the future maybe also the things on curl free control
		\end{itemize}
	\subsection{Numerical Examples}
		\begin{itemize}
			\item Sedimentation examples with clustering etc in box
			\item Cool multishape sedimentation examples
			\item Comparison with overdamped result possible
			\item Constriction examples possible, too
			\item Periodic vs no-flux investigation
			\item Time-dependent vs time-independent control 
		\end{itemize}

	\section{Inertial Equations}
	\subsection{Background DDFT}
		\begin{itemize}
			\item Derivation of the inertial equtions from microscopic dynamics
			\item Limitations of the model, smoothing term for velocity
			\item Link to overdamped equations
			\item A little literature review on this type of model
		\end{itemize}
	
	\subsection{OCP}
		\begin{itemize}
			\item Statement of OCPs 
			\item Derivation of optimality conditions
			\item Subdomain Observation, Boundary control, non-constant flux?
		\end{itemize}
	\subsection{Numerical Examples}
		\begin{itemize}
			\item 1D and 2D in box
			\item Comparison with overdamped model
			\item Some cool multishape examples?
			\item If subdomain and boundary stuff in this section, then those examples here too.
			\item Do we get in and outflow to work?
		\end{itemize}

	\section{Multiple Species}
	\subsection{Background DDFT}
	Quick literature review and illustrate coupling of equations.
	\subsection{OCP}
	\begin{itemize}
		\item State OCP
		\item Derive overdamped optimality conditions
		\item Extend to inertial case (notation such that this is straightforward)
	\end{itemize}
	\subsection{Numerical Examples}
	\begin{itemize}
		\item Example of overdamped OCP
		\item Example of inertial one -- comparison/agreement
		\item Some cool mulitshape example with two species
	\end{itemize}


	\section{Industry application 1: Sedimentation}
	\section{Industry application 2: Inerital multiple species}
	\pagebreak	
	\bibliography{GeneralBib}
	\bibliographystyle{unsrt}
	
	
\end{document}