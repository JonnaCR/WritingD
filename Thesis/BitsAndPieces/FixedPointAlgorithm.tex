In this section we describe the fixed point algorithm, which is an efficient and stable optimization method for the optimal control problems considered above. 
We denote the discretized versions of the variables $\rho$, $\adj$ and $\mathbf{w}$ with $P$, $Q$ and $W$, respectively. Each of these matrices is of the form $A = [\boldsymbol{a_0}, \boldsymbol{a_1}, ... ,\boldsymbol{a_n}]$, where the vectors $\boldsymbol{a_k}$ represent the solutions at the discretized times $k \in 0,1,....,n$, where $n$ is the number of time points. In particular, the first column of $P$, denoted by $\boldsymbol{\rho_0}$, corresponds to the initial condition $\rho(r,0)$. If the spatial domain is one-dimensional, $P$, $Q$ and $W$ are of size $N \times (n + 1)$, where $N$ is the number of spatial points. In the two-dimensional case, $P$ and $Q$ are of size $(N_1N_2) \times (n + 1)$, where $N_1$ is the number of spatial points in the direction of $x_1$ and $N_2$ the points along the $x_2$ axis. Generally, $N_1 = N_2$. The discretized control $W$ for linear control problems is also $(N_1N_2) \times (n + 1)$ dimensional, while it is $(2N_1N_2) \times (n + 1)$ dimensional for nonlinear control problems. This is due to the fact that the control is represented by a vector field, when applied nonlinearly.
\\
\\
The optimization algorithm is initialized with a guess for the control, $W^{(0)}$. Then, in each iteration, denoted by $i$, the following steps are computed:
\vspace{0.1cm}
\begin{enumerate}
	\item Starting with a guess for the control $W^{(i)}$ as input variable, the corresponding state $P^{(i)}$ is found by solving the forward equation.
	\item In a next step, the value of the adjoint, $Q^{(i)}$, is established by computing the adjoint equation, using $W^{(i)}$ and $P^{(i)}$ as inputs. Since $P^{(i)}$ contains the solution for all discretized times $k \in 0,1,...,n$, this circumvents issues resulting from the non-local coupling in time, resulting from reversing time in the adjoint equation. As illustrated in the same section, time is reversed in the adjoint equation, so that the result is a matrix $\tilde{Q}^{(i)} =  [\boldsymbol{\adj_n},\boldsymbol{\adj_{n-1}}, ..., \boldsymbol{\adj_1} ]$. The columns of $\tilde{Q}^{(i)}$ are permuted to obtain the solution  $Q^{(i)}$.
	\item The gradient equation is solved, given the solutions $P^{(i)}$ and $Q^{(i)}$. This results in a new value for the control, $W^{(i)}_g$.
	\item  The convergence of the optimization scheme is measured by computing the error between $W^{(i)}$ and $W^{(i)}_{g}$. The error measure, $\mathcal{E}$, is discussed in detail in Section \ref{sec:ErrorMeasure}. 
	\begin{itemize}
		\item  If this error is lower than a set tolerance, the optimality system is self-consistent. This implies that the solution triplet ($\bar{P},\bar{W},\bar{Q}$) solves the (discretized) optimality system, and is therefore an optimal solution to the PDE-constrained optimization problem of interest.
		\item If the error is above the optimality tolerance, Step 5 is executed.
	\end{itemize}
	\item Finally, the update $W^{(i+1)}$ is a linear combination of the current guess $W^{(i)}$, and the value obtained in Step 3, $W^{(i)}_{g}$, employing a mixing rate $\lambda \in [0,1]$:
	\begin{align*}
	W^{(i+1)} = (1-\lambda)W^{(i)} + \lambda W^{(i)}_{g}.
	\end{align*}
	The guess for the control is updated from $W^{(i)} $ to $W^{(i+1)} $ and Steps 1-5 are repeated until the method converges. 
\end{enumerate}
\vspace{0.3cm}
The update scheme in Step 5, with mixing rate $\lambda$, is known to stabilise fixed point methods, see \cite{Roth1}. Typical values of $\lambda$, which provide stable convergence, lie between $0.1$ and $0.001$. Throughout this paper, $\lambda =0.01$, unless stated otherwise. This mixing scheme is similar to the updating scheme presented in~\cite{Burger1}. 
Note that, while the solutions $P^{(i)}$ and $Q^{(i)}$ change in each iteration, the initial condition $\boldsymbol{\rho_0}$ and final time condition $\boldsymbol{\adj_n}$ remain unchanged throughout the process. Therefore, the only variable inducing a change in the solution is $W^{(i)}$.